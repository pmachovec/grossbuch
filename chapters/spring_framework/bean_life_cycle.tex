\index{Bean life cycle methods}
\index{Spring bean life cycle methods}
\index{Initialization method}
\index{Destroy method}
In~a~bean class you~can~define a~method that will be~executed after the~bean is~prepared by~a~\hyperref[springcontainrer]{container} (initialization, dependencies and~values injection, internal Spring processing). This method is~called \textit{initialization method}. Similarly you~can~define a~method that will be~executed before the~\hyperref[springcontainrer]{container} is shut~down (the~\textit{context.close()} is~called). This method is~called \textit{destroy method}.

Each~bean can~have only one initialization and~one destroy method. Methods can~have any \hyperref[accessmodifiers]{access modifier} and~any return value. There~is no~use for~it, therefore, it's~convenient to~make methods private and~with \textit{void} return value. Methods can't have any~parameters.

Initialization methods executions depend on~bean initialization points\,--\,they can~be~executed in~various moments. On~the~other hand, destroy methods executions of~all~beans created~by the~same \hyperref[springcontainrer]{container} are~performed in~the~same moment\,--\,when the~\hyperref[springcontainrer]{container} is shut~down.

\warning Destroy methods work only for~\hyperref[singletondp]{singletons}. For~\hyperref[prototypedp]{prototypes} they can~be defined and~configured without any~problem, but~they are never executed.\\

\noindent To~set an~initialization method for a~bean put the~method's name to~the~attribute \textit{init--method} of~the~element \textit{bean}. To~set a~destroy method put the~method's name to~the~attribute \textit{destroy--method} of~the~element \textit{bean}.
\newpage

\example
\begin{lstlisting}[language=XML, title={Configuration XML}]
    <?xml version="1.0" encoding="UTF-8"?>
    <beans ...>
      <bean id="(*\tikzmarknodebf{lc1xml1singbeanid}{singletonBeanId}[ForestGreen]*)" class="package.subfolder.CommonClass" init-method="(*\tikzmarknodebf{lc1xml1singinit}{initMethod}[ForestGreen]*)" destroy-method="(*\tikzmarknodebf{lc1xml1singdestroy}{destroyMethod}[ForestGreen]*)">
          ...
      </bean>
      <bean id="(*\tikzmarknodebf{lx1xml1protbeanid}{prototypeBeanId}[ForestGreen]*)" class="package.subfolder.CommonClass" scope="prototype" init-method="(*\tikzmarknodebf{lc1xml1protinit}{initMethod}[ForestGreen]*)" destroy-method="(*\tikzmarknodebf{lc1xml1protdestroy}{destroyMethod}[ForestGreen]*)">
        ...
      </bean>
    </beans>
\end{lstlisting}
\begin{lstlisting}[language=Java, title={Common class for singleton and prototype beans}]
    private void (*\tikzmarknodebf{lc1java1init}{initMethod}*)() {
        ...DO SOMETHING...
    }

    private void (*\tikzmarknodebf{lc1java1destroy}{destroyMethod}*)() {
        ...DO SOMETHING...
    }
\end{lstlisting}
\begin{lstlisting}[language=Java, title={Usage}]
    ClassPathXmlApplicationContext context = new ClassPathXmlApplicationContext("configurationFile.xml");

    CommonClassInterface singletonInstance1 = context.getBean("(*\tikzmarknodebf{lc1java1singbeanid1}{singletonBeanId}[ForestGreen]*)", CommonClassInterface.class);
    CommonClassInterface singletonInstance2 = context.getBean("(*\tikzmarknodebf{lc1java1singbeanid2}{singletonBeanId}[ForestGreen]*)", CommonClassInterface.class);

    CommonClassInterface prototypeInstance1 = context.getBean("(*\tikzmarknodebf{lc1java1protbeanid1}{prototypeBeanId}[ForestGreen]*)", CommonClassInterface.class);
    CommonClassInterface prototypeInstance2 = context.getBean("(*\tikzmarknodebf{lc1java1protbeanid2}{prototypeBeanId}[ForestGreen]*)", CommonClassInterface.class);

    context.close();
\end{lstlisting}
\begin{tikzpicture}[remember picture, overlay]
    \drawarrow{[xshift=-6mm] lc1xml1singbeanid.south}{[xshift=-9mm] lc1java1singbeanid1.north}
    \drawarrow{[xshift=-6mm] lc1xml1singbeanid.south}{[xshift=-3mm] lc1java1singbeanid2.north}
    \drawarrow{[xshift=6mm] lx1xml1protbeanid.south}{[xshift=9mm] lc1java1protbeanid1}[red]
    \drawarrow{[xshift=6mm] lx1xml1protbeanid.south}{[xshift=3mm] lc1java1protbeanid2}[red]
    \drawarrow{lc1xml1singinit}{[xshift=-6mm] lc1java1init.north}[green]
    \drawarrow{lc1xml1protinit}{[xshift=-6mm] lc1java1init.north}[green]
    \drawarrow{lc1xml1singdestroy}{[xshift=-6mm] lc1java1destroy.north}[blue]
    \drawarrow{lc1xml1protdestroy}{[xshift=-6mm] lc1java1destroy.north}[blue]
\end{tikzpicture}

\noindent The~initialization method is~called three times\,--\,once for~the~singleton instance and~twice for~prototype instances. The~destroy method is~called only once\,--\,for~the~singleton instance, although configured even for~prototype beans. Destroy methods of~prototype beans don't work.