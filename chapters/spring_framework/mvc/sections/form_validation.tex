\newsection{Form Validation}
The~\hyperref[hibernate]{Hibernate} project, which is more known for~its~JPA, also provides tools for~Spring~MVC form validations.
It~consists~of special \hyperref[javaannotation]{annotations} that can~be added to~POJO class fields and~with specific \hyperref[parameterargument]{parameters} restrict the~field content and~define an~error message.
For~example, you~can restrict a~String field to~contain only alphanumeric characters by~a~regular expression, or~you~can~restrict a~number field range by~specifying minimum and~maximum~etc.
When such restriction isn't followed in~the~corresponding field input (a~user inserts a~non--matching value), an~error can~be detected on~an~instance of~the~class \mbit{BindingResult}, which is given to~the~form--processing controller method as~another \hyperref[parameterargument]{argument}.
A~proper reaction can~then be carried~out.

Typical use case is to~return to~the~form page and~display some warning with the~error message defined earlier.
When the~model attribute of~the~form--processing controller method has~another annotation \mbit{@Valid}, it's~possible to~detect the~invalid field, retrieve the~error message, insert it to~a~special element (which is hidden by~default) and~make this element visible.
Everything in~the~form template code.
\newpage

\examplenonl[string field validation allowing ASCII letters only]
\enlargethispage{10mm}
\thispagestyle{empty}
\begin{lstlisting}[language=Java, title={POJO class with restricted String field}]
public class (*\tmnbf{smvc6java1pojo}{PojoClass}*) {
    @@>@Pattern<@@(regexp = "[a-zA-Z]+", message = "ASCII letters only")
    private String pojoField;

    public String get(*\tmnbf{smvc6java1field1}{PojoField}*)() {
        return dropdownField;
    }

    public void set(*\tmnbf{smvc6java1field2}{PojoField}*)(String pojoField) {
        this.pojoField = pojoField;
    }
}
\end{lstlisting}
\begin{lstlisting}[language=XML, title={HTML form \tmnbf{smvc6xml1form}{\textit{theForm}}\textit{.html} setting the field}]
    <?xml version="1.0" encoding="UTF-8"?>
    <!DOCTYPE html>
    <html xmlns:th="http://www.thymeleaf.org">
      ...
      <body>
      <form th:object="(*\Sc*){(*\tmnbf{smvc6xml1mapping}{mappingId}[ForestGreen]*)}" ...>
        <input type="text" th:field="*{(*\tmnbf{smvc6xml1field1}{pojoField}[ForestGreen]*)}">
      </form>
      <div th:if="(*\Sc*){#fields.hasErrors('(*\tmnbf{smvc6xml1field2}{pojoField}[ForestGreen]*)')}" th:errors="*{(*\tmnbf{smvc6xml1field3}{pojoField}[ForestGreen]*)}"/>
      <!-- The div is invisible by default, and would still be even if it had some content -->
      </body>
    </html>
\end{lstlisting}
\begin{lstlisting}[language=Java, title={Controller processing the form}]
    @>@Controller
    @@>@RequestMapping<@@("...")
    public class ControllerClass {
        @@>@RequestMapping<@@("...")
        public String processForm(@@>@Valid @ModelAttribute<@@("(*\tmnbf{smvc6java2mapping}{mappingId}[ForestGreen]*)") (*\tmnbf{smvc6java2pojo}{PojoClass}*) pojo, BindingResult bindingResult) {
            if (bindingResult.hasErrors()) {
                return "(*\tmnbf{smvc6java2form}{theForm}[ForestGreen]*)";
            } else {
                ...DO SOMETHING ELSE...
            }
        }
    }
\end{lstlisting}
\begin{tikzpicture}[remember picture, overlay]
    \drawarrow{smvc6java1pojo}{smvc6java2pojo}
    \drawarrow{[xshift=-3mm] smvc6java1field1.south east}{smvc6xml1field1.north}[red]
    \drawarrow{[xshift=-3mm] smvc6java1field2.south east}{smvc6xml1field1.north}[red]
    \drawarrow{smvc6xml1field1.south}{smvc6xml1field2.north}[red]
    \drawarrow{smvc6xml1field1.south}{smvc6xml1field3.north}[red]
    \drawarrow{[xshift=3mm] smvc6xml1form.south west}{smvc6java2form.north}[green]
    \drawarrow{[xshift=-3mm] smvc6xml1mapping.south east}{smvc6java2mapping.north}[blue]
\end{tikzpicture}
\newpage

\noindent When the~form is~submitted with a~"wrong" value in~the~field, it's~detected in~the~\mbit{BindingResult} instance and~the~controller method returns the~form (like a~method initially displaying the~form).
In~the~form the~error on~the~field is~detected and~the~(initially invisible) \textit{div} is~made visible and~the~error message from the~POJO field is~set as~the~\textit{div} content.

\newsubsection{Custom Validation Annotations}
Form validation doesn't have to~use only annotations provided by~Hibernate.
It's~possible to~create \hyperref[javacustomannotations]{custom annotations} for~validation.
The~annotation class target must~be \mbit{FIELD} and~the~retention must~be \mbit{RUNTIME}.
The~class must~be also annotated with \mbitq{@Constraint}, which has~a~\hyperref[parameterargument]{parameter} \mbit{validatedBy}, which gets one~or~more references to~validator classes (there's rarely more than~one, see~further how~to~create~it).
The~class can~contain additional parameter methods, but~three are~compulsory:
\begin{itemize}
    \itembfd{message} string, the~error message, used even in~Hibernate
    \itembfd{groups} array of~class references, groups for~related constraints
    \itembfd{payload} array of~class references extending the~\mbitq{Payload} interface, custom details about validation failures (severity level, error code etc.)
\end{itemize}
\noindent It~isn't necessary to~interact with \mbit{groups} or~\mbit{payload} in~any~way.
They just need to~be~\hyperref[declarationdefinition]{defined}.
It's~OK to~simply set their default values to~empty arrays.

At~least one validator class must~be created for~a~validation annotation.
This class must implement the~interface \mbitq{ConstraintValidator}, which~is a~\hyperref[javagenerics]{generic type} taking two references.
The~first one is the~validation annotation class and~the~second one is~the~type of~the~validated field (most likely, but~not~necessarily, string).
The~interface enforces two~methods:
\begin{itemize}
    \itembfd{initialize} void, instance of~the~annotation class as~a~parameter, serves for~accessing annotation parameter values
    \itembfd{isValid} boolean, validated field value and~constraint validator context as~parameters, should return \mbit{false} when the~field value is~supposed to~be~invalid, otherwise \mbit{true}, field value parameter name (the~first~one) doesn't have to~correspond a~variable in~POJO or~a~field~ID in~a~form
\end{itemize}
The~annotation can~then~be applied to~a~POJO field.
The~validator class isn't specified anywhere else.
\newpage

\example
\begin{lstlisting}[language=Java, title={Custom validation annotation class}]
    @@>@Constraint<@@(validatedBy = (*\tmnbf{smvc7annotvalid}{CustomValidator}*))
    @@>@Retention<@@(RetentionPolicy.RUNTIME)
    @@>@Target<@@(ElementType.FIELD)
    public @interface (*\tmnbf{smvc7annot}{CustomAnnotation}*) {
        String[] (*\tmnbf{smvc7annotarray}{stringArray}*)() default {"simpleArrayMember"};
        String (*\tmnbf{smvc7annotstring}{simpleString}*)() default "simpleStringValue";
        String message() default "Something is wrong";
        Class<?> groups() default {};
        Class<? extends Payload> payload() default {};
    }
\end{lstlisting}
\begin{lstlisting}[language=Java, title={Custom validator class, validated field must contain either a~value from the~array or~must~be equal to~the~simple string}]
    public class (*\tmnbf{smvc7valid}{CustomValidator}*) implements ConstraintValidator<(*\tmnbf{smvc7validannot1}{CustomAnnotation}*), (*\tmnbf{smvc7validstringtype1}{String}*)> {
        private String[] stringArray;
        private String simpleString;

        @>@Override
        public void initialize((*\tmnbf{smvc7validannot2}{CustomAnnotation}*) customAnnotation) {
            stringArray = customAnnotation.(*\tmnbf{smvc7validarray}{stringArray}*)();
            simpleString = customAnnotation.(*\tmnbf{smvc7validstring}{simpleString}*)();
        }

        @>@Override
        public boolean isValid((*\tmnbf{smvc7validstringtype2}{String}*) validatedFieldValue, ConstraintValidatorContext context) {
            if (!Arrays.asList(stringArray).contains(validatedFieldValue) && (!simpleString.equals(validatedFieldValue)) {
                return false;
            }

            return false;
        }
    }
\end{lstlisting}
\begin{tikzpicture}[remember picture, overlay]
    \drawarrow{smvc7annotvalid.south}{smvc7valid.north}
    \drawarrow{smvc7annot.south}{[xshift=-6mm] smvc7validannot1.north}[red]
    \drawarrow{smvc7annot.south}{[xshift=-6mm] smvc7validannot2.north}[red]
    \drawarrow{smvc7annotarray.south}{[xshift=3mm] smvc7validarray.north}[green]
    \drawarrow{smvc7annotstring.south}{[xshift=-3mm] smvc7validstring.north}[blue]
    \drawarrow{smvc7validstringtype1.south}{smvc7validstringtype2.north}[Magenta]
\end{tikzpicture}
\noindent Usage in~a~POJO class is~the~same as~when using Hibernate validation annotations.
\newpage
