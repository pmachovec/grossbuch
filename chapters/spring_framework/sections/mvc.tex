\newsection{Spring MVC}
\index{Spring MVC}
\index{Model--view--controller}
\index{MVC}
\index{Front controller}
\index{DispatchServlet}
Spring provides a~set of~tools to~create \hyperref[webserviceapplication]{web applications} based on~the~\hyperref[mvc]{\textit{model--view--controller}} architectural pattern.
The~client--server communication leverages the~original \hyperref[servlet]{servlet} technology, therefore, it~uses the~same terminology.
Server side classes are~called \textit{servlets}, data coming from clients are~\textit{parameters} wrapped in~\textit{requests} and~data going back to~clients are~set as~\textit{attributes} to~\textit{responses}.
Applications are~suitable to~be~deployed on~\hyperref[tomcat]{Tomcat} and,~of~course, they consist of~three basic MVC components.

\begin{itemize}
    \item \textbf{Model} is~represented by~an~instance of~the~class \textit{Model} from the~package \textitquoted{org.springframework.ui}. Any~object can~be inserted to~the~model instance as~an~attribute (\textitquoted{model.addAttribute(OBJECT,"attributeName")}).
          At~the~beginning it's empty and~must~be filled with attributes by~the~controller.
    \item \textbf{View} is~represented by~a~template engine, which gets the~filled model, inserts attributes to~an~HTML page template and~puts a~final HTML page to~the~response.
          The~basic template engine for~Spring is "good" old \hyperref[jsp]{JSP}, but~nowadays safer technologies can~be used.
          In~following examples \href{https://www.thymeleaf.org/}{Thymeleaf} will~be~used.
  Implementation logic of~template engines exists in~a~form of~\hyperref[springinversionofcontrol]{Spring beans}, for~which a~configuration must~be~set.
    \item \textbf{Controller} consists of~the~Java implementation to~be~run on~the~server.
          It~gets request parameters, processes them and~inserts attributes to~the~model.
          There~must~be at~least one controller class with at~least one controller method.
          Opposite to~old \hyperref[servlet]{servlets} in Spring~MVC URLs are~set for~these methods, not~for~whole classes.
\end{itemize}

\newsubsection{Basic Request Processing}
\begin{enumerate}[label=\arabic*)]
    \item Client triggers an~action sending a~request to~the~server.
    \item The~request comes to~a~\textit{front controller}. There's a~class called \textit{DispatchServlet} behind this controller.
          This class is~fully provided by~Spring, you~only configure URL for~it \hyperref[servleturl]{like for~standard servlets}.
    \item The~front controller creates a~model instance and~forwards it together with the~request to~a~custom controller, which is a~part of~a~concrete application.
    \item The~custom controller triggers the~server side logic and~inserts attributes to~the~model.
          At~the~end of~the~logic processing it~forwards the~filled model to~a~template engine.
    \item Template engine replaces placeholders in a~HTML page template with attributes from the~model and~sends the~final HTML page in~the~response to~the~client.
\end{enumerate}
\newpage

\begin{figure}[ht]
    \centering
    \begin{tikzpicture}
        \node [draw, minimum width=30mm, minimum height=20mm, line width=.3mm] (client) at (0, 0) {};
        \node [bob, minimum size=10mm, monitor] (innerClinet) at (0,0.2) {Client};
        \node [draw, minimum width=30mm, minimum height=20mm, line width=.3mm] (dispatchServlet) at (8, 4) {\fontsize{11}{11} \selectfont DispatchServlet};
        \node [draw, minimum width=30mm, minimum height=20mm, line width=.3mm] (controller) at (8, 0) {Controller};
        \node [draw, minimum width=30mm, minimum height=20mm, line width=.3mm, align=center] (templateEngine) at (8, -4) {View\\{\fontsize{10}{10} \selectfont Template Engine}};
        \node [draw, line width=.3mm] (model1) at (9, 2.4) {Model};
        \node [draw, line width=.3mm] (model1) at (9, -1.6) {Model};

        \diagramarrow[.3mm]{client.north east}{dispatchServlet.south west}[Request][above=3mm]
        \diagramarrow{dispatchServlet.south}{controller.north}
        \diagramarrow{controller.south}{templateEngine.north}
        \diagramarrow[.3mm]{templateEngine.north west}{client.south east}[Response][below=3mm]
    \end{tikzpicture}
    \captit{Spring MVC schema}
\end{figure}

\newsubsection{Application Deployment on~Tomcat}
There's no difference from the~\hyperref[tomcatdeployment]{standard Tomcat deployment}.
A~whole Spring~MVC application is~located in~a~folder under the~Tomcat's \mboxtextit{webapps} folder, configuration is~stored in~the~\mboxtextitquoted{WEB-INF/web.xml} file, Java classes are~located in~\mboxtextitquoted{WEB-INF/classes} and~third party jars are~located in~\mboxtextitquoted{WEB-INF/lib}.
HTML templates can~be located anywhere in~the~application folder.

Additionally there's the~Spring configuration XML file.
This~can~be also located anywhere as~it's~referenced in~\mboxtextit{web.xml}, but~it's~common practice to~put~it directly to~the~\mboxtextit{WEB-INF} folder, i.e.,~to~the~same level as~\mboxtextit{web.xml}.
This file contains, beside others, the~configuration of~\hyperref[springinversionofcontrol]{Spring beans} of~the~chosen template engine.
When using Thymeleaf, you~must configure three beans\,--\,template resolver, template engine and~view resolver.
It~isn't important to~know what bean does what, just remember they must~be configured, the~template engine bean uses the~template resolver bean, and~the~view resolver bean uses the~template engine bean.
For~the~template resolver bean you~must also configure prefix and~suffix.
Prefix is a~location (relative to~the~application folder) where HTML templates are~located and~suffix is~the~extension of~your HTML templates (e.g.,~\mboxtextitquoted{.html}).
See~the~example below for~details.

\warning There can~be only one prefix and~suffix. You~must therefore have all templates in~one folder (it~doesn't affect web browser paths, that's configured in~controllers, see~further). Also, all~templates must have the~same extension.
\newpage

\begin{lstlisting}[language=XML, title=Example of a~\textit{web.xml} file with a~\textit{DispatchServlet} configuration]
    <?xml version="1.0" encoding="UTF-8"?>
    <web-app ...>
      <display-name>APPLICATION NAME</display-name>
      <absolute-ordering/>

      <servlet>
        <servlet-name>dispatcher</servlet-name>
        <servlet-class>org.springframework.web.(*servlet*).DispatcherServlet </servlet-class>
        <init-param>
          <param-name>contextConfigLocation</param-name>
          <param-value>/WEB-INF/configurationFile.xml</param-value>
        </init-param>
        <load-on-startup>1</load-on-startup>
      </servlet>

      <servlet-mapping>
        <servlet-name>dispatcher</servlet-name>
        <url-pattern>/</url-pattern>
      </servlet-mapping>
    </web-app>
\end{lstlisting}

\noindent Controllers are~classes marked by~the~annotation \mboxtextit{@Controller} from the~package \mboxtextitquoted{org.springframework.stereotype}.
The~annotation extends the~\mboxtextit{@Component} annotation, therefore, controller classes are~visible by~\hyperref[iocannotations]{component scanning}.
Classes generally don't~extend any other class or~implement an~interface (they~can if~needed).

Controller methods are~marked by~the~annotation \mboxtextit{@RequestMapping} from the~package \mboxtextitquoted{org.springframework.web.bind.annotation}.
This annotation gets as~a~\hyperref[parameterargument]{parameter} a~string that defines the~URL by~which the~method is~accessed.
For~proper functionality this string must begin with a~slash.
The~annotation can~be used even without the~parameter, it's~then equivalent to~specifying an~empty string.

One~controller method can~get as~\hyperref[parameterargument]{parameters} a~whole request, separate request \hyperref[jspattributeparameter]{parameters} and~the~model instance.
\hyperref[parameterargument]{Arguments} of~controller methods are~variable, they~can~get any~combination of~those objects.
When using separate request parameters, each corresponding method parameter must~be marked by~the~annotation \mboxtextit{@RequestParam} from the~package \mboxtextitquoted{org.springframework.web.bind.annotation}.
This annotation gets as~a~\hyperref[parameterargument]{parameter} a~string that must correspond to~the~key of~the~parameter in~the~request.
Methods can~have any name, but~they must return a~string that must~be the~name of~the~template file to~which the~model should~be forwarded.
The~name is onle the~plain file name, without any folder path and~even without an~extension. That's defined in~the~Spring configuration.

Values are~written to~the~model instance as~\hyperref[jspattributeparameter]{attribues} and~can~be accessed in~the~HTML template by~the~engine expression language.
Any~object can~be written to~the~model.
Thymeleaf expression language provides rich syntax for~various object types processing.
\newpage

\begin{lstlisting}[language=XML, title={Example of a~Spring XML with Thymeleaf configuration}]
    <?xml version="1.0" encoding="UTF-8"?>
    <beans ...>
      <!-- Main package configuration, expected to be located in WEB-INF/classes -->
      <context:component-scan base-package="somepackage.subpackage"/>

      <!-- Formatting and validation support -->
      <mvc:annotation-driven/>

      <!-- Template engine configuration -->
      <bean id="templateResolver" class="org.thymeleaf.spring4.templateresolver. SpringResourceTemplateResolver">
        <property name="prefix" value="/FOLDER_WITH_TEMPLATES"/>
        <property name="suffix" value=".html"/>
        <property name="characterEncoding" value="UTF-8"/>
      </bean>
      <bean id="templateEngine" class="org.thymeleaf.spring4.SpringTemplateEngine">
        <property name="templateResolver" ref="templateResolver"/>
        <property name="enableSpringELCompiler" value="true"/>
      </bean>
      <bean class="org.thymeleaf.spring4.view.ThymeleafViewResolver">
        <property name="templateEngine" ref="templateEngine"/>
        <property name="characterEncoding" value="UTF-8"/>
      </bean>
    </beans>
\end{lstlisting}

\warning There is a~slash at~the~beginning of~the~prefix value and~the~suffix value contains the~dot!

\note Even controller classes can~have the~\mboxtextit{@RequestMapping} annotation with some string (beginning with a~slash) defining an~URL.
URLs defined by~the~annotation for~methods are~then relative to~the~URL of~the~class.
\newpage

\example[a~controller class and~a~template file]
\begin{lstlisting}[language=Java, title={Controller class}]
    package somepackage.subpackage;

    @>@Controller
    @@>@RequestMapping<@@("/classUrl")
    public class ControllerClass {
        @@>@RequestMapping<@@("/methodUrl")
        public String controllerMethod(HttpServletRequest wholeRequest, @@>@RequestParam<@@("paramKey") String singleParameter, Model model) {
            // Reading the parameter from the request
            String parameterFromRequest = wholeRequest.getParameter("paramKey");

            // Writing boolean to the model
            model.addAttribute("(*\tikzmarknodebf{smvc1java1attr}{modelAttribute}[ForestGreen]*)", parameterFromRequest.equals(singleParameter));

            // Forwarding the model to a template
            return "(*\tikzmarknodebf{smvc1java1templ}{templateFile}[ForestGreen]*)";
        }
    }
\end{lstlisting}
\begin{lstlisting}[language=XML, title={Thymeleaf template file called \tikzmarknodebf{smvc1xml1templ}{\textit{templateFile}}\textit{.html}}]
    <?xml version="1.0" encoding="UTF-8"?>
    <!DOCTYPE html>
    <html xmlns:th="http://www.thymeleaf.org">
      ...
      <body>
        <div th:text="'Result: ' + (*\SSS*){(*\tikzmarknodebf{smvc1xml1attr}{modelAttribute}[ForestGreen]*)}"></div>
      </body>
    </html>
\end{lstlisting}
\begin{tikzpicture}[remember picture, overlay]
    \drawarrow{smvc1java1attr}{smvc1xml1attr}
    \drawarrow{smvc1java1templ}{smvc1xml1templ}[red]
\end{tikzpicture}

\noindent When a~request with a~\hyperref[jspattributeparameter]{parameter} with~a~key \mboxtextit{paramkey} is~sent to~the~url \mboxtextitquoted{HOST:PORT/appFolder/classUrl/methodUrl}, a~response containing a~web page with a~simple text \textitquoted{Result: true} is~returned and~displayed in~the~browser (the~parameter retrieved from the~request is always the~same as~the~one retrieved separately).

\warning HTML forms are~directed to~controller methods by~the~attribute \mboxtextit{action}.
When a~form is~submitted, it~replaces the~last part of~the~URL in~the~browser address bar (everything after the~last slash) with the~value of~the~attribute.
This can~result in~unwanted behavior when the~form is~accessed over a~method with the~\mboxtextit{RequestMapping} annotation without parameters.
\newpage

\example[problems with a~form without request mapping path]
\begin{lstlisting}[language=XML, title={Simple HTML form \tikzmarknodebf{smvc2xml1form}{\textit{theForm}}\textit{.html}}]
    <?xml version="1.0" encoding="UTF-8"?>
    <!DOCTYPE html>
    <html>
      ...
      <body>
        <form action="(*\tikzmarknodebf{smvc2xml1meth}{methodUrl}[ForestGreen]*)" method="post">
          <input type="text" name="(*\tikzmarknodebf{smvc2xml1param}{paramKey}[ForestGreen]*)">
        </form>
      </body>
    </html>
\end{lstlisting}
\begin{lstlisting}[language=Java, title={Controller class}]
    package somepackage.subpackage;

    @>@Controller
    @@>@RequestMapping<@@("/classUrl")
    public class ControllerClass {
        @@>@RequestMapping<@@ // Request mapping without a path
        public String showForm() {
            return "(*\tikzmarknodebf{smvc2java1form}{theForm}[ForestGreen]*)"
        }

        @@>@RequestMapping<@@("/(*\tikzmarknodebf{smvc2java1meth}{methodUrl}[ForestGreen]*)")
        public String processForm(@@>@RequestParam<@@("(*\tikzmarknodebf{smvc2java1param}{paramKey}[ForestGreen]*)") String singleParameter,
                                  Model model) {
            ...DO SOMETHING WITH THE PARAMETER...
            return "templateFile";
        }
    }
\end{lstlisting}
\begin{tikzpicture}[remember picture, overlay]
    \drawarrow{smvc2xml1form}{smvc2java1form}
    \drawarrow{smvc2xml1meth}{smvc2java1meth}[red]
    \drawarrow{smvc2xml1param}{smvc2java1param}[green]
\end{tikzpicture}

\noindent When \mboxtextitquoted{HOST:PORT/appFolder/classUrl} is~accessed in~a~browser, it~displays the~form.
When the~form is~submitted, it~replaces the~last part of~the~address\,--\,\mboxtextit{classUrl}\,--\,with the~value from~the~\mboxtextit{action} attribute\,--\,\mboxtextit{methodUrl}.
Therefore, the~request is~forwarded to~\mboxtextitquoted{HOST:PORT/appFolder/methodUrl}, but~the~correct method address is \mboxtextitquoted{HOST:PORT/appFolder/\textbf{classUrl/}methodUrl}.
The~solution is either to~add \mboxtextitquoted{classUrl/} to~the~\mboxtextit{action} attribute or~to~better organize request mapping paths.
\newpage

\newsubsection{UTF-8 encoding}
As~many other applications, Spring~MVC doesn't like non--ASCII characters by~default.
To~make it properly handling the~UTF-8 encoding you~must add the~character encoding property element into the~template resolver and~Thymeleaf view resolver bean configurations (shown in~an~example above) and~also configure a~filter in~the~\mboxtextit{web.xml} file.
If~you~have more \mboxtextit{filter} elements there, the~one for~encoding must always go as~first.
On~the~other hand, it~doesn't matter if the~filter comes before or~after servlet configurations.

\begin{lstlisting}[language=XML, title=Example of a~\textit{web.xml} file with encoding filter configuration]
    <?xml version="1.0" encoding="UTF-8"?>
    <web-app ...>
      <display-name>APPLICATION NAME</display-name>
      ...
      <filter>
        <filter-name>encodingFilter</filter-name>
        <filter-class>org.springframework.web.filter.CharacterEncodingFilter </filter-class>
        <init-param>
          <param-name>encoding</param-name>
          <param-value>UTF-8</param-value>
        </init-param>
        <init-param>
          <param-name>forceEncoding</param-name>
          <param-value>true</param-value>
        </init-param>
      </filter>
      <filter-mapping>
        <filter-name>encodingFilter</filter-name>
        <url-pattern>/*</url-pattern>
      </filter-mapping>
    </web-app>
\end{lstlisting}
\newpage

\newsubsection{Binding Form Values to Model}
So~far when a~form was~submitted, its values\,--\,request parameters\,--\,were extracted manually one by~one.
But~there's a~possibility to~automatically encapsulate them into \hyperref[pojo]{POJOs}, which are~available as~the~model attributes.
I.e.,~instead of reading request parameters separately and~inserting them to~some POJOs for~further processing (e.g.,~storing to~a~database), you~get those POJOs directly.
\newline

\noindent \textbf{Inputs with values provided by users:}\\
\noindent This is a~simple use case.
In~a~form there's a~"free" value input field (text, number, email) where a~user can~type any~value matching the~field type (absolutely anything to~a~text field, any~number to~a~number field etc.).
The~input has~a~Thymeleaf attribute \mboxtextit{th:field} used for~mapping to~a~POJO field.
For~such~input a~field with a~getter and~a~setter with matching names are~created in~the~desired POJO class and~Spring simply assigns the~typed value to~the~field.
Yes,~there's a~possibility of~type mismatch (e.g.,~text inserted to~an~integer field) and~the~code must~be prepared for~it.
Usual trick is~to~have POJO fields all String and~convert them to~other types in~try--catch blocks deeper in~the~program.
The~POJO must~be already present in~the~model when the~form is~submitted.
Because of~this the~POJO must~be created and~inserted to~the~model in~the~controller method responsible for~displaying the~form.
\newline

\noindent \textbf{Inputs with fixed values:}\\
\noindent For~dropdowns the~principle is completely the~same as~for~"free" inputs.
There's a~field with a~getter and~a~setter with names matching the~\mboxtextit{th:field} attribute of~the~\mboxtextit{select} element in~the~form and~a~selected value (content of~the~\textit{value} attribute of~the~\textit{option} element) is~assigned to~the~field.

For~inputs without an~enclosing parent element, like radios and~checkboxes, there's a~slight difference.
The~identifying attribute (which appears in~each input element), is~the~original HTML \mboxtextit{name}, not~the~Thymeleaf \mboxtextit{th:field}.
Getters and~setters of~POJO fields must match that~one.

Inputs with fixed values can~be generated from a~collection in~the~code (map, list,\dots) or~even from a~\hyperref[properties]{\mboxtextit{properties}} file.
A~\mboxtextit{properties} file must~be included in~the~Spring configuration, retrieved in~the~controller and~added to~the~model as~a~separate attribute.
Iteration over the~collection members is~performed by~the~Thymeleaf attribute \mboxtextit{th:each}.
For~dropdowns this attribute is~assigned to~a~single \mboxtextit{option} element.
For~radios and~checkboxes there must~be a~single \mboxtextit{input} element enclosed in~a~special Thymeleaf element \mboxtextit{th:block} and~this element gets the~\mboxtextit{th:each} attribute.

It's~also possible to~bind inputs enabling to~set multiple values, like checkboxes and~multiselects.
For~such inputs POJO fields must~be arrays, not~simple strings.
\newpage

\examplenonl[binding text input to \hyperref[pojo]{POJO} String field]
\enlargethispage{20mm}
\thispagestyle{empty}
\begin{lstlisting}[language=Java, title={POJO class with one String field}]
    public class (*\tikzmarknodebf{smvc3java1pojo}{PojoClass}*) {
        private String pojoField;

        public String get(*\tikzmarknodebf{smvc3java1field1}{PojoField}*)() {
            return pojoField;
        }

        public String set(*\tikzmarknodebf{smvc3java1field2}{PojoField}*)(String pojoField) {
            this.pojoField = pojoField;
        }
    }
\end{lstlisting}
\begin{lstlisting}[language=XML, title={HTML form setting the field}]
    <?xml version="1.0" encoding="UTF-8"?>
    <!DOCTYPE html>
    <html xmlns:th="http://www.thymeleaf.org">
      ...
      <body>
        <form method="post" th:object="(*\SSS*){(*\tikzmarknodebf{smvc3xml1mapping}{mappingId}[ForestGreen]*)}" th:action="@{(*\tikzmarknodebf{smvc3xml1class}{/classUrl}[ForestGreen]\tikzmarknodebf{smvc3xml1meth}{/methodUrl}[ForestGreen]*)}">
          <input type="text" th:field="*{(*\tikzmarknodebf{smvc3xml1field}{pojoField}[ForestGreen]*)}">
        </form>
      </body>
    </html>
\end{lstlisting}
\begin{lstlisting}[language=Java, title={Controller displaying and processing the form}]
    @>@Controller
    @@>@RequestMapping<@@("(*\tikzmarknodebf{smvc3java2class}{/classUrl}[ForestGreen]*)")
    public class ControllerClass {
        @>@RequestMapping
        public String showForm(Model model) {
            // Adding the POJO to model
            model.addAttribute("(*\tikzmarknodebf{smvc3java2mapping1}{mappingId}[ForestGreen]*)", new (*\tikzmarknodebf{smvc3java2pojo1}{PojoClass}*)())
            return "theForm"
        }
    
        @@>@RequestMapping<@@("(*\tikzmarknodebf{smvc3java2meth}{/methodUrl}[ForestGreen]*)")
        public String processForm(@@>@ModelAttribute<@@("(*\tikzmarknodebf{smvc3java2mapping2}{mappingId}[ForestGreen]*)") (*\tikzmarknodebf{smvc3java2pojo2}{PojoClass}*) pojo) {
            String pojoField = pojo.getPojoField();
            ...DO SOMETHING WITH THE POJO FIELD...
            return "...SOMETHING...";
        }
    }
\end{lstlisting}
\begin{tikzpicture}[remember picture, overlay]
    \drawarrow{smvc3java1pojo.east}{[xshift=3mm] smvc3java2pojo1.north}[black][.3][bend left=18mm]
    \drawarrow{smvc3java1pojo.east}{smvc3java2pojo2.north}[black][.3][bend left=18mm]
    \drawarrow{smvc3java1field1.south}{[xshift=-3mm] smvc3xml1field.north}[red]
    \drawarrow{smvc3java1field2.south}{[xshift=-3mm] smvc3xml1field.north}[red]
    \drawarrow{[xshift=3mm] smvc3xml1mapping.south}{smvc3java2mapping1.north}[green]
    \drawarrow{[xshift=3mm] smvc3xml1mapping.south}{smvc3java2mapping2.north}[green]
    \drawarrow{smvc3xml1class}{smvc3java2class}[blue]
    \drawarrow{smvc3xml1meth}{smvc3java2meth}[Magenta]
\end{tikzpicture}
\newpage

\examplenonl[binding dropdown created from map]
\begin{lstlisting}[language=Java, title={POJO class with one String field}]
    public class PojoClass {
        private String dropdownField;
        private LinkedHashMap<String, String> dropdownMap;

        public PojoClass() { // Constructor
            ...dropdownMap initialization...
        }

        public LinkedHashMap<String, String> get(*\tikzmarknodebf{smvc4java1map}{DropdownMap}*)() {
            return dropdownMap;
        }

        public String get(*\tikzmarknodebf{smvc4java1field1}{DropdownField}*)() {
            return dropdownField;
        }

        public void set(*\tikzmarknodebf{smvc4java1field2}{DropdownField}*)(String dropdownField) {
            this.dropdownField = dropdownField;
        }
    }
\end{lstlisting}
\begin{lstlisting}[language=XML, title={HTML form setting the field}]
    <?xml version="1.0" encoding="UTF-8"?>
    <!DOCTYPE html>
    <html xmlns:th="http://www.thymeleaf.org">
      ...
      <body>
        <form ...>
          <select th:field="*{(*\tikzmarknodebf{smvc4xml1field}{dropdownField}[ForestGreen]*)}">
            <option th:each="mapEntry: (*\SSS*){mappingId.(*\tikzmarknodebf{smvc4xml1map}{dropdownMap}[ForestGreen]*).entrySet()}" th:value="${mapEntry.key}" th:text="${mapEntry.value}"/>
          </select>
        </form>
      </body>
    </html>
\end{lstlisting}
\begin{tikzpicture}[remember picture, overlay]
    \drawarrow{smvc4java1map}{smvc4xml1map}
    \drawarrow{smvc4java1field1.south}{smvc4xml1field.north}[red]
    \drawarrow{smvc4java1field2.south}{smvc4xml1field.north}[red]
\end{tikzpicture}

\notenonl Controller would be the same as in the previous example.
\newpage

\example[binding checkbox created from a \textit{properties} file]
\enlargethispage{10mm}
\begin{lstlisting}[language=Java, title={POJO class with one array field}]
    public class PojoClass {
        private String[] checkboxField;

        public String get(*\tikzmarknodebf{smvc5java1field1}{CheckboxField}*)() {
            return dropdownField;
        }

        public void set(*\tikzmarknodebf{smvc5java1field2}{CheckboxField}*)(String[] checkboxField) {
            this.checkboxField = checkboxField;
        }
    }
\end{lstlisting}
\begin{lstlisting}[language=XML, title={HTML form setting the field}]
    <?xml version="1.0" encoding="UTF-8"?>
    <!DOCTYPE html>
    <html xmlns:th="http://www.thymeleaf.org">
      ...
      <body>
        <form ...>
          <th:block th:each="mapEntry: (*\SSS*){mappingId.(*\tikzmarknodebf{smvc5xml1map}{checkboxMap}[ForestGreen]*).entrySet()}">
            <input type="checkbox" name="(*\tikzmarknodebf{smvc5xml1field}{checkboxField}[ForestGreen]*)" th:value="(*\SSS*){mapEntry.key}"><span th:text="(*\SSS*){mapEntry.value}"/><br/>
          </select>
        </form>
      </body>
    </html>
\end{lstlisting}
\begin{lstlisting}[language=Java, title={Controller converting \textit{properties} file to a map}]
    @>@Controller
    @@>@RequestMapping<@@("...")
    public class ControllerClass {
        @@>@Value<@@("#{(*\tikzmarknodebf{smvc5java2props}{propertiesFileId}[ForestGreen]*)}") // Parsing the properties file
        private val mapFromPropertiesFile: LinkedHashMap<String, String>;

        @>@RequestMapping
        public String showForm(Model model) {
            ...
            model.addAttribute("(*\tikzmarknodebf{smvc5java2map}{checkboxMap}[ForestGreen]*)", mapFromPropertiesFile)
            ...
        }
    }
\end{lstlisting}
\begin{tikzpicture}[remember picture, overlay, shift={(current page.south west)}]
    \coordinate(smvc5java2propsdummy) at (smvc5java2props |-,0);
    \drawarrow{smvc5java1field1.south}{smvc5xml1field.north}
    \drawarrow{smvc5java1field2.south}{smvc5xml1field.north}
    \drawarrow{smvc5xml1map}{smvc5java2map}[red]
    \drawarrow[stealth-]{[xshift=9mm] smvc5java2props.south}{[xshift=9mm] smvc5java2propsdummy}[green]
\end{tikzpicture}
\newpage

\begin{lstlisting}[language=XML, title={Spring XML configuration}]
    <?xml version="1.0" encoding="UTF-8"?>
    <beans ...>
      ...
      <util:properties id="(*\tikzmarknodebf{smvc5xml2props}{propertiesFileId}[ForestGreen]*)" location="WEB-INF/fileName.properties"/>
      ...
    </beans>
\end{lstlisting}
\begin{tikzpicture}[remember picture, overlay, shift={(current page.north west)}]
    \coordinate(smvc5xml2propsdummy) at (smvc5xml2props |-,0);
    \drawarrow[-stealth]{[xshift=-10.9mm] smvc5xml2propsdummy}{[xshift=-10.9mm] smvc5xml2props.north}[green]
\end{tikzpicture}

\newsubsection{Form Validation}
The~\hyperref[hibernate]{Hibernate} project, which is more known for~its~JPA, also provides tools for~Spring~MVC form validations.
It~consists~of special annotations that can~be added to~POJO class fields and~with specific \hyperref[parameterargument]{parameters} restrict the~field content and~define an~error message.
For~example, you~can restrict a~String field to~contain only alphanumeric characters by~a~regular expression, you~can~restrict a~number field range by~specifying minimum and~maximum~etc.
When such a~restriction isn't followed in~the~corresponding field input (a~user inserts a~non--matching value), an~error can~be detected on~an~instance of~the~class \mboxtextit{BindingResult}, which is given to~the~form--processing controller method as~another \hyperref[parameterargument]{argument}.
A~proper reaction can~then be carried~out.

Typical use case is to~return to~the~form page and~display some warning with the~error message defined earlier.
When the~model attribute of~the~form--processing controller method has~another annotation \mboxtextit{@Valid}, it's~possible to~detect the~invalid field, retrieve the~error message, insert it to~a~special element (which is hidden by~default) and~make this element visible.
Everything in~the~form template code.
\newpage

\examplenonl[string field validation allowing ASCII letters only]
\enlargethispage{10mm}
\thispagestyle{empty}
\begin{lstlisting}[language=Java, title={POJO class with restricted String field}]
    public class (*\tikzmarknodebf{smvc6java1pojo}{PojoClass}*) {
        @@>@Pattern<@@(regexp = "[a-zA-Z]+", message = "ASCII letters only")
        private String pojoField;

        public String get(*\tikzmarknodebf{smvc6java1field1}{PojoField}*)() {
            return dropdownField;
        }

        public void set(*\tikzmarknodebf{smvc6java1field2}{PojoField}*)(String pojoField) {
            this.pojoField = pojoField;
        }
    }
\end{lstlisting}
\begin{lstlisting}[language=XML, title={HTML form \tikzmarknodebf{smvc6xml1form}{\textit{theForm}}\textit{.html} setting the field}]
    <?xml version="1.0" encoding="UTF-8"?>
    <!DOCTYPE html>
    <html xmlns:th="http://www.thymeleaf.org">
      ...
      <body>
        <form th:object="(*\SSS*){(*\tikzmarknodebf{smvc6xml1mapping}{mappingId}[ForestGreen]*)}" ...>
          <input type="text" th:field="*{(*\tikzmarknodebf{smvc6xml1field1}{pojoField}[ForestGreen]*)}">
        </form>
        <div th:if="(*\SSS*){#fields.hasErrors('(*\tikzmarknodebf{smvc6xml1field2}{pojoField}[ForestGreen]*)')}" th:errors="*{(*\tikzmarknodebf{smvc6xml1field3}{pojoField}[ForestGreen]*)}"/>
        <!-- The div is invisible by default, and would still be even if it had some content -->
      </body>
    </html>
\end{lstlisting}
\begin{lstlisting}[language=Java, title={Controller processing the form}]
    @>@Controller
    @@>@RequestMapping<@@("...")
    public class ControllerClass {
        @@>@RequestMapping<@@("...")
        public String processForm(@@>@Valid @ModelAttribute<@@("(*\tikzmarknodebf{smvc6java2mapping}{mappingId}[ForestGreen]*)") (*\tikzmarknodebf{smvc6java2pojo}{PojoClass}*) pojo, BindingResult bindingResult) {
            if (bindingResult.hasErrors()) {
                return "(*\tikzmarknodebf{smvc6java2form}{theForm}[ForestGreen]*)";
            } else {
                ...DO SOMETHING ELSE...
            }
        }
    }
\end{lstlisting}
\begin{tikzpicture}[remember picture, overlay]
    \drawarrow{smvc6java1pojo}{smvc6java2pojo}
    \drawarrow{[xshift=-3mm] smvc6java1field1.south east}{smvc6xml1field1.north}[red]
    \drawarrow{[xshift=-3mm] smvc6java1field2.south east}{smvc6xml1field1.north}[red]
    \drawarrow{smvc6xml1field1.south}{smvc6xml1field2.north}[red]
    \drawarrow{smvc6xml1field1.south}{smvc6xml1field3.north}[red]
    \drawarrow{[xshift=3mm] smvc6xml1form.south west}{smvc6java2form.north}[green]
    \drawarrow{[xshift=-3mm] smvc6xml1mapping.south east}{smvc6java2mapping.north}[blue]
\end{tikzpicture}
\newpage

\noindent When the~form is~submitted with a~"wrong" value in~the~field, it's~detected in~the~\mboxtextit{BindingResult} instance and~the~controller method returns the~form (like a~method initially displaying the~form).
In~the~form the~error on~the~field is~detected and~the~(initially invisible) \textit{div} is~made visible and~the~error message from the~POJO field is~set as~the~\textit{div} content.

\newsubsection{Custom Validation Annotations}
\todo

\newsubsection{Request Preprocessing}
\todo
