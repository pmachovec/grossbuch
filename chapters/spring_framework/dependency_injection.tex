\index{Dependency injection}
\index{Dependency}
\label{dependencyinjection}
Spring bean classes usually contain \hyperref[variablefieldproperty]{fields} of~some interface types. \hyperref[springcontainrer]{Spring container} assigns instances of~classes implementing those interfaces to~those fields. And~this assignment\,--\,injecting\,--\,is the~dependency injection.

Instances stored in fields are~called \textit{dependencies}. They mostly serve for~encapsulating configurable dynamic behavior of~Spring beans. The~configuration telling which bean should have assigned which dependency is a~part~of the~XML configuration file.

\newline\note Dependencies are~also Spring beans.

\newsubsection{Constructor Injection}
\index{Constructor injection}
\label{constructorinjection}
One~of basic approaches to~the~dependency injection is to~set the~dependency via the~wanted class constructor. The~constructor must accept a~parameter of a~type of~an~interface, which is~implemented by the~dependency class. The~dependency class instance\,--\,another bean\,--\, is~specified and~injected to~the~wanted bean in~the~XML configuration file by~the~element \textit{constructor--arg}.
\newpage

\example
\begin{lstlisting}[language=XML, title={Configuration XML}]
    <?xml version="1.0" encoding="UTF-8"?>
    <beans ...>
      <bean id="(*\tikzmarknodebf{di1xml1depbeanid}{dependencyBeanId}[ForestGreen]*)" class="package.subfolder.DependencyClass"/>
      <bean id="(*\tikzmarknodebf{di1xml1beanid}{wantedBeanId}[ForestGreen]*)" class="package.subfolder.(*\tikzmarknodebf{di1xml1class}{WantedClass}[ForestGreen]*)">
        <constructor-arg ref="(*\tikzmarknodebf{di1xml1depbeanid2}{dependencyBeanId}[ForestGreen]*)"/>
      </bean>
    </beans>
\end{lstlisting}
\begin{lstlisting}[language=Java, title={Wanted class with the constructor}]
    private DependencyInterface dependencyField;

    public (*\tikzmarknodebf{di1java1class}{WantedClass}*)(DependencyInterface dependencyValue) {
        dependencyField = dependencyValue;
    }

    public WHATEVER (*\tikzmarknodebf{di1java1usedep}{useDependency}*)() {
        ...DO SOMETHING WITH THE DEPENDENCY FIELD...
    }
\end{lstlisting}
\begin{lstlisting}[language=Java, title={Usage}]
    ClassPathXmlApplicationContext context = new ClassPathXmlApplicationContext("configurationFile.xml");
    WantedClassInterface wantedClassInstance = context.getBean("(*\tikzmarknodebf{di1java2beanid}{wantedBeanId}[ForestGreen]*)", WantedClassInterface.class);
    wantedClassInstance.(*\tikzmarknodebf{di1java2usedep}{useDependency}*)();
\end{lstlisting}
\begin{tikzpicture}[remember picture, overlay]
    \drawarrow{di1xml1depbeanid}{di1xml1depbeanid2}
    \drawarrow{di1xml1beanid}{[xshift=-3mm] di1java2beanid.north}[red]
    \drawarrow{di1xml1class}{di1java1class}[green]
    \drawarrow{[xshift=3mm] di1java1usedep.south}{di1java2usedep}[blue]
\end{tikzpicture}

\newsubsection{Setter Injection}
\index{Setter injection}
\label{setterinjection}
Another basic approach to~the~dependency injection is to~set the~dependency via a~setter method defined in~the~wanted class. The~wanted class must have a~zero--parameter constructor. It's~safer to~always define~it, even~if there is no other constructor (and~therefore the~zero--parameter one is~implicit). There~is nothing wrong with keeping the~constructor body empty. The~injection is~similar to~the~\hyperref[constructorinjection]{constructor} approach. The~dependency class instance\,--\,another bean\,--\, is~specified and~injected to~the~wanted bean in~the~XML configuration file by~the~element \textit{property}.

\newline\warning There is one crucial restriction to~follow. The~\textit{property} element must contain the~attribute \textit{name} with some value, let's say \textit{someName}. The~name of~the~setter method in the~wanted class must then be exactly \textit{setSomeName}. The~name of~the~field variable doesn't matter.
\newpage

\example
\begin{lstlisting}[language=XML, title={Configuration XML}]
    <?xml version="1.0" encoding="UTF-8"?>
    <beans ...>
      <bean id="(*\tikzmarknodebf{di2xml1depbeanid}{dependencyBeanId}[ForestGreen]*)" class="package.subfolder.DependencyClass"/>
      <bean id="(*\tikzmarknodebf{di2xml1beanid}{wantedBeanId}[ForestGreen]*)" class="package.subfolder.(*\tikzmarknodebf{di2xml1class}{WantedClass}[ForestGreen]*)">
        <property name="(*\tikzmarknodebf{di2xml1depfield}{dependencyField}[ForestGreen]*)" ref="(*\tikzmarknodebf{di2xml1depbeanid2}{dependencyBeanId}[ForestGreen]*)"/>
      </bean>
    </beans>
\end{lstlisting}
\begin{lstlisting}[language=Java, title={Wanted class with the zero--parameter constructor and the setter method}]
    private DependencyInterface dependencyField;

    public (*\tikzmarknodebf{di2java1class}{WantedClass}*)() {}

    public void set(*\tikzmarknodebf{di2java1depfield}{DependencyField}*)(DependencyInterface dependencyValue) {
        dependencyField = dependencyValue;
    }

    public WHATEVER (*\tikzmarknodebf{di2java1usedep}{useDependency}*)() {
        ...DO SOMETHING WITH THE DEPENDENCY FIELD...
    }
\end{lstlisting}
\begin{lstlisting}[language=Java, title={Usage}]
    ClassPathXmlApplicationContext context = new ClassPathXmlApplicationContext("configurationFile.xml");
    WantedClassInterface wantedClassInstance = context.getBean("(*\tikzmarknodebf{di2java2beanid}{wantedBeanId}[ForestGreen]*)", WantedClassInterface.class);
    wantedClassInstance.(*\tikzmarknodebf{di2java2usedep}{useDependency}*)();
\end{lstlisting}
\begin{tikzpicture}[remember picture, overlay]
    \drawarrow{di2xml1depbeanid}{di2xml1depbeanid2}
    \drawarrow{[xshift=-3mm] di2xml1beanid.north}{di2java2beanid}[red]
    \drawarrow{di2xml1class}{di2java1class}[green]
    \drawarrow{di2xml1depfield}{di2java1depfield}[blue]
    \drawarrow{[xshift=3mm] di2java1usedep.south}{di2java2usedep}[yellow][.5]
\end{tikzpicture}

\newsubsection{Autowiring}
\index{Autowired}
\label{autowiring}
Similarly to~the~\hyperref[inversionofcontrol]{inversion of~control} even the~dependency injection can~be~handled with \hyperref[javaannotation]{Java annotations}. A~Spring functionality called \textit{autowiring} automatically detects a~dependency bean class that can~be~injected to~a~field in~the~main bean class. A~\hyperref[constructorinjection]{constructor} or~a~\hyperref[setterinjection]{setter method} performing the~injection must be annotated with the~annotation \textit{Autowired}, which processed in~runtime. The~dependency bean class must be~annotated with the~annotation \textit{Component} in~the~zero--parameter form (no~bean~ID specified). When properly combined with the~\hyperref[iocannotations]{annotation--configured} \hyperref[inversionofcontrol]{inversion of~control}, the~configuration XML file contains only the~component scanning configuration.
\newpage

\example[constructor injection]
\label{constructorinjectionautowire}
\begin{lstlisting}[language=XML, title={Configuration XML}]
    <?xml version="1.0" encoding="UTF-8"?>
    <beans ...
           xmlns:context="http://www.springframework.org/schema/context"
           ...>
      <(*\textcolor{blue}{context:}*)component-scan base-package="package.subfolder"/>
    </beans>
\end{lstlisting}
\begin{lstlisting}[language=Java, title={Wanted class with the constructor}]
    @>@Component
    public class (*\tikzmarknodebf{di3java1class}{WantedClass}*) implements (*\tikzmarknodebf{di3java1interface}{WantedClassInterface}*) {
        private (*\tikzmarknodebf{di3java1depinterface}{DependencyInterface}*) dependencyField;

        @>@Autowired
        public (*\tikzmarknodebf{di3java1class2}{WantedClass}*)((*\tikzmarknodebf{di3java1depinterface2}{DependencyInterface}*) dependencyValue) {
            dependencyField = dependencyValue;
        }

        public WHATEVER (*\tikzmarknodebf{di3java1usedep}{useDependency}*)() {
            ...DO SOMETHING WITH THE DEPENDENCY FIELD...
        }
    }
\end{lstlisting}
\begin{lstlisting}[language=Java, title={Dependency class}]
    @>@Component
    public class DependencyClass implements (*\tikzmarknodebf{di3java2depinterface}{DependencyInterface}*) {
        ...
    }
\end{lstlisting}
\begin{lstlisting}[language=Java, title={Usage}]
    ClassPathXmlApplicationContext context = new ClassPathXmlApplicationContext("configurationFile.xml");
    (*\tikzmarknodebf{di3java3interface}{WantedClassInterface}*) wantedClassInstance = context.getBean("(*\tikzmarknodebf{di3java3beanid}{wantedClass}[ForestGreen]*)", (*\tikzmarknodebf{di3java3interface2}{WantedClassInterface}*).class);
    wantedClassInstance.(*\tikzmarknodebf{di3java3usedep}{useDependency}*)();
\end{lstlisting}
\begin{tikzpicture}[remember picture, overlay]
    \drawarrow{di3java1class.south}{di3java1class2}
    \drawarrow{di3java1class.south}{di3java3beanid}
    \drawarrow{di3java1interface.south}{di3java3interface}[red]
    \drawarrow{di3java1interface.south}{di3java3interface2}[red]
    \drawarrow{di3java1depinterface.south}{di3java1depinterface2}[green]
    \drawarrow{di3java1depinterface.south}{di3java2depinterface}[green]
    \drawarrow{di3java1usedep}{di3java3usedep}[blue]
\end{tikzpicture}
\newpage

\example[setter injection]
\label{setterinjectionautowire}
\begin{lstlisting}[language=XML, title={Configuration XML}]
    <?xml version="1.0" encoding="UTF-8"?>
    <beans ...
           xmlns:context="http://www.springframework.org/schema/context"
           ...>
      <(*\textcolor{blue}{context:}*)component-scan base-package="package.subfolder"/>
    </beans>
\end{lstlisting}
\begin{lstlisting}[language=Java, title={Wanted class with the zero--parameter constructor and the setter method}]
    @>@Component
    public class (*\tikzmarknodebf{di4java1class}{WantedClass}*) implements (*\tikzmarknodebf{di4java1interface}{WantedClassInterface}*) {
       private (*\tikzmarknodebf{di4java1depinterface}{DependencyInterface}*) dependencyField;

       public (*\tikzmarknodebf{di4java1class2}{WantedClass}*)() {}

       @>@Autowired
       public void setDependencyField((*\tikzmarknodebf{di4java1depinterface2}{DependencyInterface}*) dependencyValue) {
          dependencyField = dependencyValue;
       }

       public WHATEVER (*\tikzmarknodebf{di4java1usedep}{useDependency}*)() {
          ...DO SOMETHING WITH THE DEPENDENCY FIELD...
       }
    }
\end{lstlisting}
\begin{lstlisting}[language=Java, title={Dependency class}]
    @>@Component
    public class DependencyClass implements (*\tikzmarknodebf{di4java2depinterface}{DependencyInterface}*) {
        ...
    }
\end{lstlisting}
\begin{lstlisting}[language=Java, title={Usage}]
    ClassPathXmlApplicationContext context = new ClassPathXmlApplicationContext("configurationFile.xml");
    (*\tikzmarknodebf{di4java3interface}{WantedClassInterface}*) wantedClassInstance = context.getBean("(*\tikzmarknodebf{di4java3beanid}{wantedClass}[ForestGreen]*)", (*\tikzmarknodebf{di4java3interface2}{WantedClassInterface}*).class);
    wantedClassInstance.(*\tikzmarknodebf{di4java3usedep}{useDependency}*)();
\end{lstlisting}
\begin{tikzpicture}[remember picture, overlay]
    \drawarrow{di4java1class.south}{di4java1class2}
    \drawarrow{di4java1class.south}{di4java3beanid}
    \drawarrow{di4java1interface.south}{di4java3interface}[red]
    \drawarrow{di4java1interface.south}{di4java3interface2}[red]
    \drawarrow{di4java1depinterface.south}{di4java1depinterface2}[green]
    \drawarrow{di4java1depinterface.south}{di4java2depinterface}[green]
    \drawarrow{di4java1usedep}{di4java3usedep}[blue]
\end{tikzpicture}

\note Opposite to~the~old \hyperref[setterinjection]{setter injection} with~XML configuration file the~name of~an~autowired setter method isn't restricted in~any~way. The~Spring's autowiring is~directed by~the~type of~its parameter. I.e.,~in~the~previous example the~name of~the~setter method could~be even literally \textit{absoluteNonsense} and~the~dependency injection would~be still working.
\newpage

\newsubsection{Field Injection}
\index{Field injection}
\label{fieldinjection}
The~\hyperref[autowiring]{autowiring} annotation can~be~used directly to~dependency \hyperref[variablefieldproperty]{fields} (class fields only, not~local variables inside methods). These fields then don't need any~explicit setting, autowiring finds, creates and~assigns a~correct dependency class instance automatically. It's~like the~\hyperref[setterinjection]{setter injection} (i.e.,~with zero--parameter constructor), but~without \textit{set} methods. The~Spring implementation behind this "magic" uses \hyperref[reflection]{Java reflection}.

\newline\example
\begin{lstlisting}[language=XML, title={Configuration XML}]
    <?xml version="1.0" encoding="UTF-8"?>
    <beans ...
           xmlns:context="http://www.springframework.org/schema/context"
           ...>
      <(*\textcolor{blue}{context:}*)component-scan base-package="package.subfolder"/>
    </beans>
\end{lstlisting}
\begin{lstlisting}[language=Java, title={Wanted class with the zero--parameter constructor}]
    @>@Component
    public class (*\tikzmarknodebf{di5java1class}{WantedClass}*) implements (*\tikzmarknodebf{di5java1interface}{WantedClassInterface}*) {
        @>@Autowired
        private (*\tikzmarknodebf{di5java1depinterface}{DependencyInterface}*) dependencyField;

        public (*\tikzmarknodebf{di5java1class2}{WantedClass}*)() {}

        public WHATEVER (*\tikzmarknodebf{di5java1usedep}{useDependency}*)() {
            ...DO SOMETHING WITH THE DEPENDENCY FIELD...
        }
    }
\end{lstlisting}
\begin{lstlisting}[language=Java, title={Dependency class}]
    @>@Component
    public class DependencyClass implements (*\tikzmarknodebf{di5java2depinterface}{DependencyInterface}*) {
        ...
    }
\end{lstlisting}
\begin{lstlisting}[language=Java, title={Usage}]
    ClassPathXmlApplicationContext context = new ClassPathXmlApplicationContext("configurationFile.xml");
    (*\tikzmarknodebf{di5java3interface}{WantedClassInterface}*) wantedClassInstance = context.getBean("(*\tikzmarknodebf{di5java3beanid}{wantedClass}[ForestGreen]*)", (*\tikzmarknodebf{di5java3interface2}{WantedClassInterface}*).class);
    wantedClassInstance.(*\tikzmarknodebf{di5java3usedep}{useDependency}*)();
\end{lstlisting}
\begin{tikzpicture}[remember picture, overlay]
    \drawarrow{di5java1class.south}{di5java1class2}
    \drawarrow{di5java1class.south}{di5java3beanid}
    \drawarrow{di5java1interface.south}{di5java3interface}[red]
    \drawarrow{di5java1interface.south}{di5java3interface2}[red]
    \drawarrow{di5java1depinterface}{di5java2depinterface}[green]
    \drawarrow{di5java1usedep}{di5java3usedep}[blue]
\end{tikzpicture}
\newpage

\newsubsection{Qualifiers}
\index{Qualifier}
So~far there has~always been exactly one implementation of~the~dependency interface and~for~\hyperref[autowiring]{autowiring} it~was clear what class instance inject to~the~dependency field. But~when there are more dependency classes matching the~dependency interface, running (not~building) the~program fails complaining about not~unique bean. To~solve this problem you~must provide the~dependency bean~ID as~an~argument of~the~annotation \textit{Qualifier}. When using \hyperref[constructorinjectionautowire]{constructor injection} or~\hyperref[setterinjectionautowire]{setter injection}, the~annotation is~applied to~the~argument of~the~constructor or~the~setter method. When using \hyperref[fieldinjection]{field injection}, the~annotation is~applied directly to~the~field variable.

\newline\example[constructor injection]
\enlargethispage{20mm}
\thispagestyle{empty}
\begin{lstlisting}[language=XML, title={Configuration XML}]
    <?xml version="1.0" encoding="UTF-8"?>
    <beans ...
           xmlns:context="http://www.springframework.org/schema/context"
           ...>
      <(*\textcolor{blue}{context:}*)component-scan base-package="package.subfolder"/>
    </beans>
\end{lstlisting}
\begin{lstlisting}[language=Java, title={Wanted class with the constructor}]
    @>@Component
    public class (*\tikzmarknodebf{di6java1class}{WantedClass}*) implements (*\tikzmarknodebf{di6java1interface}{WantedClassInterface}*) {
        private (*\tikzmarknodebf{di6java1depinterface}{DependencyInterface}*) dependencyField;

        @>@Autowired
        public (*\tikzmarknodebf{di6java1class2}{WantedClass}*)(@@>@Qualifier<@@("(*\tikzmarknodebf{di6java1depbeanid}{dependencyClass}[ForestGreen]*)") (*\tikzmarknodebf{di6java1depinterface2}{DependencyInterface}*) dependencyValue) {
            dependencyField = dependencyValue;
        }

        public WHATEVER (*\tikzmarknodebf{di6java1usedep}{useDependency}*)() {
            ...DO SOMETHING WITH THE DEPENDENCY FIELD...
        }
    }
\end{lstlisting}
\begin{lstlisting}[language=Java, title={Dependency class}]
    @>@Component
    public class (*\tikzmarknodebf{di6java2depclass}{DependencyClass}*) implements (*\tikzmarknodebf{di6java2depinterface}{DependencyInterface}*) {
        ...
    }
\end{lstlisting}
\begin{lstlisting}[language=Java, title={Usage}]
    ClassPathXmlApplicationContext context = new ClassPathXmlApplicationContext("configurationFile.xml");
    (*\tikzmarknodebf{di6java3interface}{WantedClassInterface}*) wantedClassInstance = context.getBean("(*\tikzmarknodebf{di6java3beanid}{wantedClass}[ForestGreen]*)", (*\tikzmarknodebf{di6java3interface2}{WantedClassInterface}*).class);
    wantedClassInstance.(*\tikzmarknodebf{di6java3usedep}{useDependency}*)();
\end{lstlisting}
\begin{tikzpicture}[remember picture, overlay]
    \drawarrow{di6java1class.south}{di6java1class2}
    \drawarrow{di6java1class.south}{di6java3beanid}
    \drawarrow{di6java1interface.south}{di6java3interface}[red]
    \drawarrow{di6java1interface.south}{di6java3interface2}[red]
    \drawarrow{di6java1depinterface.south}{[xshift=-6mm] di6java1depinterface2.north east}[green]
    \drawarrow{di6java1depinterface.south}{di6java2depinterface.north west}[green]
    \drawarrow{di6java1depbeanid}{di6java2depclass.north east}[blue]
    \drawarrow{di6java1usedep}{di6java3usedep}[Magenta]
\end{tikzpicture}
\newpage

\example[setter injection]
\begin{lstlisting}[language=XML, title={Configuration XML}]
    <?xml version="1.0" encoding="UTF-8"?>
    <beans ...
           xmlns:context="http://www.springframework.org/schema/context"
           ...>
      <(*\textcolor{blue}{context:}*)component-scan base-package="package.subfolder"/>
    </beans>
\end{lstlisting}
\begin{lstlisting}[language=Java, title={Wanted class with the zero--parameter constructor and the setter method}]
    @>@Component
    public class (*\tikzmarknodebf{di7java1class}{WantedClass}*) implements (*\tikzmarknodebf{di7java1interface}{WantedClassInterface}*) {
       private (*\tikzmarknodebf{di7java1depinterface}{DependencyInterface}*) dependencyField;

       public (*\tikzmarknodebf{di7java1class2}{WantedClass}*)() {}

       @>@Autowired
       public void setDependencyField(@@>@Qualifier<@@("(*\tikzmarknodebf{di7java1depbeanid}{dependencyClass}[ForestGreen]*)") (*\tikzmarknodebf{di7java1depinterface2}{DependencyInterface}*) dependencyValue) {
          dependencyField = dependencyValue;
       }

       public WHATEVER (*\tikzmarknodebf{di7java1usedep}{useDependency}*)() {
          ...DO SOMETHING WITH THE DEPENDENCY FIELD...
       }
    }
\end{lstlisting}
\begin{lstlisting}[language=Java, title={Dependency class}]
    @>@Component
    public class (*\tikzmarknodebf{di7java2depclass}{DependencyClass}*) implements (*\tikzmarknodebf{di7java2depinterface}{DependencyInterface}*) {
        ...
    }
\end{lstlisting}
\begin{lstlisting}[language=Java, title={Usage}]
    ClassPathXmlApplicationContext context = new ClassPathXmlApplicationContext("configurationFile.xml");
    (*\tikzmarknodebf{di7java3interface}{WantedClassInterface}*) wantedClassInstance = context.getBean("(*\tikzmarknodebf{di7java3beanid}{wantedClass}[ForestGreen]*)", (*\tikzmarknodebf{di7java3interface2}{WantedClassInterface}*).class);
    wantedClassInstance.(*\tikzmarknodebf{di7java3usedep}{useDependency}*)();
\end{lstlisting}
\begin{tikzpicture}[remember picture, overlay]
    \drawarrow{di7java1class.south}{di7java1class2}
    \drawarrow{di7java1class.south}{[xshift=-6mm] di7java3beanid.north}
    \drawarrow{di7java1interface.south}{di7java3interface}[red]
    \drawarrow{di7java1interface.south}{di7java3interface2}[red]
    \drawarrow{di7java1depinterface.south}{[xshift=-6mm] di7java1depinterface2.north east}[green]
    \drawarrow{di7java1depinterface.south}{di7java2depinterface.north west}[green]
    \drawarrow{di7java1depbeanid}{di7java2depclass.north east}[blue]
    \drawarrow{di7java1usedep}{di7java3usedep}[Magenta]
\end{tikzpicture}
\newpage

\example[field injection]
\begin{lstlisting}[language=XML, title={Configuration XML}]
    <?xml version="1.0" encoding="UTF-8"?>
    <beans ...
           xmlns:context="http://www.springframework.org/schema/context"
           ...>
      <(*\textcolor{blue}{context:}*)component-scan base-package="package.subfolder"/>
    </beans>
\end{lstlisting}
\begin{lstlisting}[language=Java, title={Wanted class with the zero--parameter constructor}]
    @>@Component
    public class (*\tikzmarknodebf{di8java1class}{WantedClass}*) implements (*\tikzmarknodebf{di8java1interface}{WantedClassInterface}*) {
        @>@Autowired
        @@>@Qualifier<@@("(*\tikzmarknodebf{di8java1depbeanid}{dependencyClass}[ForestGreen]*)")
        private (*\tikzmarknodebf{di8java1depinterface}{DependencyInterface}*) dependencyField;

        public (*\tikzmarknodebf{di8java1class2}{WantedClass}*)() {}

        public WHATEVER (*\tikzmarknodebf{di8java1usedep}{useDependency}*)() {
            ...DO SOMETHING WITH THE DEPENDENCY FIELD...
        }
    }
\end{lstlisting}
\begin{lstlisting}[language=Java, title={Dependency class}]
    @>@Component
    public class (*\tikzmarknodebf{di8java2depclass}{DependencyClass}*) implements (*\tikzmarknodebf{di8java2depinterface}{DependencyInterface}*) {
        ...
    }
\end{lstlisting}
\begin{lstlisting}[language=Java, title={Usage}]
    ClassPathXmlApplicationContext context = new ClassPathXmlApplicationContext("configurationFile.xml");
    (*\tikzmarknodebf{di8java3interface}{WantedClassInterface}*) wantedClassInstance = context.getBean("(*\tikzmarknodebf{di8java3beanid}{wantedClass}[ForestGreen]*)", (*\tikzmarknodebf{di8java3interface2}{WantedClassInterface}*).class);
    wantedClassInstance.(*\tikzmarknodebf{di8java3usedep}{useDependency}*)();
\end{lstlisting}
\begin{tikzpicture}[remember picture, overlay]
    \drawarrow{di8java1class.south}{di8java1class2}
    \drawarrow{di8java1class.south}{di8java3beanid}
    \drawarrow{di8java1interface.south}{di8java3interface}[red]
    \drawarrow{di8java1interface.south}{di8java3interface2}[red]
    \drawarrow{[xshift=-6mm] di8java1depinterface.south east}{[xshift=6mm] di8java2depinterface.north west}[green]
    \drawarrow{di8java1depbeanid}{di8java2depclass}[blue]
    \drawarrow{di8java1usedep}{di8java3usedep}[Magenta]
\end{tikzpicture}
\newpage

\newsubsection{Injecting literal values}
\label{injectingliteralvalues}
You~can inject even values of primitive types and~strings directly using the~XML configuration file. The~approach is the~same, but~instead~of using the~\textit{ref} attribute in~the~\textit{property} element you use the~\textit{value} attribute and~set directly the~value. You~can do it both for~\hyperref[constructorinjection]{constructor} and~\hyperref[setterinjection]{setter} injection.

\newline\warning You~can't use \hyperref[autowiring]{autowiring} for~this. Autowiring is~usable only when the~injected value is~an~instance of~an~explicitly defined Spring bean with the~\textit{Component} annotation. Nevertheless, it's~possible to~parse \textit{.properties} files with annotations, see~\hyperref[readingpropertiesannotations]{further}.

\newline\example[constructor injection]
\begin{lstlisting}[language=XML, title={Configuration XML}]
    <?xml version="1.0" encoding="UTF-8"?>
    <beans ...>
      <bean id="(*\tikzmarknodebf{di9xml1beanid}{wantedBeanId}[ForestGreen]*)" class="package.subfolder.(*\tikzmarknodebf{di9xml1class}{WantedClass}[ForestGreen]*)">
        <constructor-arg value="someValue"/>
      </bean>
    </beans>
\end{lstlisting}
\begin{lstlisting}[language=Java, title={Wanted class with the constructor}]
    private String concreteValueField;

    public (*\tikzmarknodebf{di9java1class}{WantedClass}*)(String concreteValue) {
        concreteValueField = concreteValue;
    }

    public WHATEVER (*\tikzmarknodebf{di9java1usevalue}{useConcreteValue}*)() {
        ...DO SOMETHING WITH THE CONCRETE VALUE FIELD...
    }
\end{lstlisting}
\begin{lstlisting}[language=Java, title={Usage}]
    ClassPathXmlApplicationContext context = new ClassPathXmlApplicationContext("configurationFile.xml");
    WantedClassInterface wantedClassInstance = context.getBean("(*\tikzmarknodebf{di9java2beanid}{wantedBeanId}[ForestGreen]*)", WantedClassInterface.class);
    wantedClassInstance.(*\tikzmarknodebf{di9java2usevalue}{useConcreteValue}*)();
\end{lstlisting}
\begin{tikzpicture}[remember picture, overlay]
    \drawarrow{di9xml1beanid}{di9java2beanid}
    \drawarrow{di9xml1class}{di9java1class}[red]
    \drawarrow{di9java1usevalue}{di9java2usevalue}[green]
\end{tikzpicture}
\newpage

\example[setter injection]
\begin{lstlisting}[language=XML, title={Configuration XML}]
    <?xml version="1.0" encoding="UTF-8"?>
    <beans ...>
      <bean id="(*\tikzmarknodebf{di10xml1beanid}{wantedBeanId}[ForestGreen]*)" class="package.subfolder.(*\tikzmarknodebf{di10xml1class}{WantedClass}[ForestGreen]*)">
        <property name="(*\tikzmarknodebf{di10xml1valuefield}{concreteValueField}[ForestGreen]*)" value="someValue"/>
      </bean>
    </beans>
\end{lstlisting}
\begin{lstlisting}[language=Java, title={Wanted class with the zero--parameter constructor and the setter method}]
    private String concreteValueField;

    public (*\tikzmarknodebf{di10java1class}{WantedClass}*)() {}

    public void set(*\tikzmarknodebf{di10java1valuefield}{ConcreteValueField}*)(String concreteValue) {
        concreteValueField = concreteValue;
    }

    public WHATEVER (*\tikzmarknodebf{di10java1usevalue}{useConcreteValue}*)() {
        ...DO SOMETHING WITH THE CONCRETE VALUE FIELD...
    }
\end{lstlisting}
\begin{lstlisting}[language=Java, title={Usage}]
    ClassPathXmlApplicationContext context = new ClassPathXmlApplicationContext("configurationFile.xml");
    WantedClassInterface wantedClassInstance = context.getBean("(*\tikzmarknodebf{di10java2beanid}{wantedBeanId}[ForestGreen]*)", WantedClassInterface.class);
    wantedClassInstance.(*\tikzmarknodebf{di10java2usevalue}{useConcreteValue}*)();
\end{lstlisting}
\begin{tikzpicture}[remember picture, overlay]
    \drawarrow{di10xml1beanid}{di10java2beanid}
    \drawarrow{di10xml1class}{di10java1class}[red]
    \drawarrow{di10xml1valuefield}{di10java1valuefield}[green]
    \drawarrow{di10java1usevalue}{di10java2usevalue}[blue]
\end{tikzpicture}
