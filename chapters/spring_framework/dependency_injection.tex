\index{Dependency injection}
\index{Dependency}
\label{dependencyinjection}
Spring bean classes usually contain \hyperref[variablefieldproperty]{fields} of~some interface types. \hyperref[springcontainrer]{Spring container} assigns instances of~classes implementing those interfaces to~those fields. And~this assignment\,--\,injecting\,--\,is the~dependency injection.

Instances stored in fields are~called \textit{dependencies}. They mostly serve for~encapsulating configurable dynamic behavior of~Spring beans. The~configuration telling which bean should have assigned which dependency is a~part~of the~XML configuration file.

\note Dependencies are~also Spring beans.

\newsubsection{Constructor Injection}
\index{Constructor injection}
\label{constructorinjection}
One~of basic approaches to~the~dependency injection is to~set the~dependency via the~wanted class constructor. The~constructor must accept a~parameter of a~type of~an~interface, which is~implemented by the~dependency class. The~dependency class instance\,--\,another bean\,--\, is~specified and~injected to~the~wanted bean in~the~XML configuration file by~the~element \textit{constructor--arg}.
\newpage

\example
\begin{lstlisting}[language=XML, title={Configuration XML}]
    <?xml version="1.0" encoding="UTF-8"?>
    <beans ...>
      <bean id="(*\tikzmarknodebf{di1xml1depbeanid}{dependencyBeanId}[ForestGreen]*)" class="package.subfolder.DependencyClass"/>
      <bean id="(*\tikzmarknodebf{di1xml1beanid}{wantedBeanId}[ForestGreen]*)" class="package.subfolder.(*\tikzmarknodebf{di1xml1class}{WantedClass}[ForestGreen]*)">
        <constructor-arg ref="(*\tikzmarknodebf{di1xml1depbeanid2}{dependencyBeanId}[ForestGreen]*)"/>
      </bean>
    </beans>
\end{lstlisting}
\begin{lstlisting}[language=Java, title={Wanted class with the constructor}]
    private DependencyInterface dependencyField;

    public (*\tikzmarknodebf{di1java1class}{WantedClass}*)(DependencyInterface dependencyValue) {
        dependencyField = dependencyValue;
    }

    public WHATEVER (*\tikzmarknodebf{di1java1usedep}{useDependency}[ForestGreen]*)() {
        ...DO SOMETHING WITH THE DEPENDENCY FIELD...
    }
\end{lstlisting}
\begin{lstlisting}[language=Java, title={Usage}]
    ClassPathXmlApplicationContext context = new ClassPathXmlApplicationContext("configurationFile.xml");
    WantedClassInterface wantedClassInstance = context.getBean("(*\tikzmarknodebf{di1java2beanid}{wantedBeanId}[ForestGreen]*)", WantedClassInterface.class);
    wantedClassInstance.(*\tikzmarknodebf{di1java2usedep}{useDependency}*)();
\end{lstlisting}
\begin{tikzpicture}[remember picture, overlay]
    \drawarrow{di1xml1depbeanid}{di1xml1depbeanid2}
    \drawarrow{di1xml1beanid}{[xshift=-3mm] di1java2beanid.north}[red]
    \drawarrow{di1xml1class}{di1java1class}[green]
    \drawarrow{[xshift=3mm] di1java1usedep.south}{di1java2usedep}[blue]
\end{tikzpicture}

\newsubsection{Setter Injection}
\index{Setter injection}
\label{setterinjection}
Another basic approach to~the~dependency injection is to~set the~dependency via a~setter method defined in~the~wanted class. The~wanted class must have a~zero--parameter constructor. It's~safer to~always define~it, even~if there is no other constructor (and~therefore the~zero--parameter one is~implicit). There~is nothing wrong with keeping the~constructor body empty. The~injection is~similar to~the~\hyperref[constructorinjection]{constructor} approach. The~dependency class instance\,--\,another bean\,--\, is~specified and~injected to~the~wanted bean in~the~XML configuration file by~the~element \textit{property}.

\warning There is one crucial restriction to~follow. The~\textit{property} element must contain the~attribute \textit{name} with some value, let's say \textit{someName}. The~name of~the~setter method in the~wanted class must then be exactly \textit{setSomeName}. The~name of~the~field variable doesn't matter.
\newpage

\example
\begin{lstlisting}[language=XML, title={Configuration XML}]
    <?xml version="1.0" encoding="UTF-8"?>
    <beans ...>
      <bean id="(*\tikzmarknodebf{di2xml1depbeanid}{dependencyBeanId}[ForestGreen]*)" class="package.subfolder.DependencyClass"/>
      <bean id="(*\tikzmarknodebf{di2xml1beanid}{wantedBeanId}[ForestGreen]*)" class="package.subfolder.(*\tikzmarknodebf{di2xml1class}{WantedClass}[ForestGreen]*)">
        <property name="(*\tikzmarknodebf{di2xml1depfield}{dependencyField}[ForestGreen]*)" ref="(*\tikzmarknodebf{di2xml1depbeanid2}{dependencyBeanId}[ForestGreen]*)"/>
      </bean>
    </beans>
\end{lstlisting}
\begin{lstlisting}[language=Java, title={Wanted class with the zero--parameter constructor and the setter method}]
    private DependencyInterface dependencyField;

    public (*\tikzmarknodebf{di2java1class}{WantedClass}*)() {}

    public void set(*\tikzmarknodebf{di2java1depfield}{DependencyField}*)(DependencyInterface dependencyValue) {
        dependencyField = dependencyValue;
    }

    public WHATEVER (*\tikzmarknodebf{di2java1usedep}{useDependency}*)() {
        ...DO SOMETHING WITH THE DEPENDENCY FIELD...
    }
\end{lstlisting}
\begin{lstlisting}[language=Java, title={Usage}]
    ClassPathXmlApplicationContext context = new ClassPathXmlApplicationContext("configurationFile.xml");
    WantedClassInterface wantedClassInstance = context.getBean("(*\tikzmarknodebf{di2java2beanid}{wantedBeanId}[ForestGreen]*)", WantedClassInterface.class);
    wantedClassInstance.(*\tikzmarknodebf{di2java2usedep}{useDependency}[ForestGreen]*)();
\end{lstlisting}
\begin{tikzpicture}[remember picture, overlay]
    \drawarrow{di2xml1depbeanid}{di2xml1depbeanid2}
    \drawarrow{[xshift=-3mm] di2xml1beanid.north}{di2java2beanid}[red]
    \drawarrow{di2xml1class}{di2java1class}[green]
    \drawarrow{di2xml1depfield}{di2java1depfield}[blue]
    \drawarrow{[xshift=3mm] di2java1usedep.south}{di2java2usedep}[yellow][.5]
\end{tikzpicture}

\newsubsection{Injecting literal values}
\label{injectingliteralvalues}
You~can inject even values of primitive types and~strings directly. The~approach is the~same, but~instead~of using the~\textit{ref} attribute in~the~\textit{property} element you use the~\textit{value} attribute and~set directly the~value. You~can do it both for~\hyperref[constructorinjection]{constructor} and~\hyperref[setterinjection]{setter} injection.\\
\newpage

\example[constructor injection]
\begin{lstlisting}[language=XML, title={Configuration XML}]
    <?xml version="1.0" encoding="UTF-8"?>
    <beans ...>
      <bean id="(*\tikzmarknodebf{di3xml1beanid}{wantedBeanId}[ForestGreen]*)" class="package.subfolder.(*\tikzmarknodebf{di3xml1class}{WantedClass}[ForestGreen]*)">
        <constructor-arg value="someValue"/>
      </bean>
    </beans>
\end{lstlisting}
\begin{lstlisting}[language=Java, title={Wanted class with the constructor}]
    private String concreteValueField;

    public (*\tikzmarknodebf{di3java1class}{WantedClass}*)(String concreteValue) {
        concreteValueField = concreteValue;
    }

    public WHATEVER (*\tikzmarknodebf{di3java1usevalue}{useConcreteValue}*)() {
        ...DO SOMETHING WITH THE CONCRETE VALUE FIELD...
    }
\end{lstlisting}
\begin{lstlisting}[language=Java, title={Usage}]
    ClassPathXmlApplicationContext context = new ClassPathXmlApplicationContext("configurationFile.xml");
    WantedClassInterface wantedClassInstance = context.getBean("(*\tikzmarknodebf{di3java2beanid}{wantedBeanId}[ForestGreen]*)", WantedClassInterface.class);
    wantedClassInstance.(*\tikzmarknodebf{di3java2usevalue}{useConcreteValue}*)();
\end{lstlisting}
\begin{tikzpicture}[remember picture, overlay]
    \drawarrow{di3xml1beanid}{di3java2beanid}
    \drawarrow{di3xml1class}{di3java1class}[red]
    \drawarrow{di3java1usevalue}{di3java2usevalue}[green]
\end{tikzpicture}
\newpage

\example[setter injection]
\begin{lstlisting}[language=XML, title={Configuration XML}]
    <?xml version="1.0" encoding="UTF-8"?>
    <beans ...>
      <bean id="(*\tikzmarknodebf{di4xml1beanid}{wantedBeanId}[ForestGreen]*)" class="package.subfolder.(*\tikzmarknodebf{di4xml1class}{WantedClass}[ForestGreen]*)">
        <property name="(*\tikzmarknodebf{di4xml1valuefield}{concreteValueField}[ForestGreen]*)" value="someValue"/>
      </bean>
    </beans>
\end{lstlisting}
\begin{lstlisting}[language=Java, title={Wanted class with the zero--parameter constructor and the setter method}]
    private String concreteValueField;

    public (*\tikzmarknodebf{di4java1class}{WantedClass}*)() {}

    public void set(*\tikzmarknodebf{di4java1valuefield}{ConcreteValueField}*)(String concreteValue) {
        concreteValueField = concreteValue;
    }

    public WHATEVER (*\tikzmarknodebf{di4java1usevalue}{useConcreteValue}*)() {
        ...DO SOMETHING WITH THE CONCRETE VALUE FIELD...
    }
\end{lstlisting}
\begin{lstlisting}[language=Java, title={Usage}]
    ClassPathXmlApplicationContext context = new ClassPathXmlApplicationContext("configurationFile.xml");
    WantedClassInterface wantedClassInstance = context.getBean("(*\tikzmarknodebf{di4java2beanid}{wantedBeanId}[ForestGreen]*)", WantedClassInterface.class);
    wantedClassInstance.(*\tikzmarknodebf{di4java2usevalue}{useConcreteValue}*)();
\end{lstlisting}
\begin{tikzpicture}[remember picture, overlay]
    \drawarrow{di4xml1beanid}{di4java2beanid}
    \drawarrow{di4xml1class}{di4java1class}[red]
    \drawarrow{di4xml1valuefield}{di4java1valuefield}[green]
    \drawarrow{di4java1usevalue}{di4java2usevalue}[blue]
\end{tikzpicture}