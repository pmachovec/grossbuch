\newsection{Bean Life Cycle Methods}
\index{Bean life cycle methods}
\index{Spring bean life cycle methods}
\index{Initialization method}
\index{Destroy method}
In~a~bean class you~can~define a~method that will be executed after the~bean is~prepared by~a~\hyperref[springcontainrer]{container} (initialization, dependencies and~values injection, internal Spring processing).
This method is~called \textit{initialization method}.
Similarly you~can~define a~method that will be executed before the~\hyperref[springcontainrer]{container} is shut~down (the~\textit{context.close()} is~called).
This method is~called \textit{destroy method}.

Each~bean can~have only one initialization and~one destroy method.
Methods can~have any \hyperref[javaaccessmodifiers]{access modifier} and~any return value.
There~is no~use for~it, therefore, it's~convenient to~make methods private and~with \textit{void} return value.
Methods can't have any~parameters.

Initialization methods executions depend on~bean initialization points -- they can~be executed in~various moments.
On~the~other hand, destroy methods executions of~all~beans created~by the~same \hyperref[springcontainrer]{container} are~performed in~the~same moment -- when the~\hyperref[springcontainrer]{container} is shut~down.

\warning Destroy methods work only for~\hyperref[singletondp]{singletons}.
For~\hyperref[prototypedp]{prototypes} they can~be defined and~configured without any~problem, but~they are never executed.\\

\noindent To~set an~initialization method for a~bean put the~method's name to~the~attribute \textit{init-method} of~the~element \textit{bean}.
To~set a~destroy method put the~method's name to~the~attribute \textit{destroy-method} of~the~element \textit{bean}.
\newpage

\example
%! language = TEXT
\begin{lstlisting}[language=XML, title={Configuration XML}]
    <?xml version="1.0" encoding="UTF-8"?>
    <beans ...>
      <bean id="(*\tmnbf{lc1xml1singbeanid}{singletonBeanId}[ForestGreen]*)" class="somepackage.subpackage.CommonClass" init-method="(*\tmnbf{lc1xml1singinit}{initMethod}[ForestGreen]*)" destroy-method="(*\tmnbf{lc1xml1singdestroy}{destroyMethod}[ForestGreen]*)">
          ...
      </bean>
      <bean id="(*\tmnbf{lc1xml1protbeanid}{prototypeBeanId}[ForestGreen]*)" class="somepackage.subpackage.CommonClass" scope="prototype" init-method="(*\tmnbf{lc1xml1protinit}{initMethod}[ForestGreen]*)" destroy-method="(*\tmnbf{lc1xml1protdestroy}{destroyMethod}[ForestGreen]*)">
        ...
      </bean>
    </beans>
\end{lstlisting}
%! language = TEXT
\begin{lstlisting}[language=Java, title={Snippet of a common bean class}]
    private void (*\tmnbf{lc1java1init}{initMethod}*)() {
        ...DO SOMETHING...
    }

    private void (*\tmnbf{lc1java1destroy}{destroyMethod}*)() {
        ...DO SOMETHING...
    }
\end{lstlisting}
%! language = TEXT
\begin{lstlisting}[language=Java, title={Usage}]
    ClassPathXmlApplicationContext context = new ClassPathXmlApplicationContext("configurationFile.xml");

    CommonClassInterface singletonInstance1 = context.getBean("(*\tmnbf{lc1java2singbeanid1}{singletonBeanId}[ForestGreen]*)", CommonClassInterface.class);
    CommonClassInterface singletonInstance2 = context.getBean("(*\tmnbf{lc1java2singbeanid2}{singletonBeanId}[ForestGreen]*)", CommonClassInterface.class);

    CommonClassInterface prototypeInstance1 = context.getBean("(*\tmnbf{lc1java2protbeanid1}{prototypeBeanId}[ForestGreen]*)", CommonClassInterface.class);
    CommonClassInterface prototypeInstance2 = context.getBean("(*\tmnbf{lc1java2protbeanid2}{prototypeBeanId}[ForestGreen]*)", CommonClassInterface.class);

    context.close();
\end{lstlisting}
\begin{tikzpicture}[remember picture, overlay]
    \drawarrow{[xshift=-6mm] lc1xml1singbeanid.south}{[xshift=-9mm] lc1java2singbeanid1.north}
    \drawarrow{[xshift=-6mm] lc1xml1singbeanid.south}{[xshift=-3mm] lc1java2singbeanid2.north}
    \drawarrow{[xshift=6mm] lc1xml1protbeanid.south}{[xshift=9mm] lc1java2protbeanid1}[red]
    \drawarrow{[xshift=6mm] lc1xml1protbeanid.south}{[xshift=3mm] lc1java2protbeanid2}[red]
    \drawarrow{lc1xml1singinit}{[xshift=-6mm] lc1java1init.north}[green]
    \drawarrow{lc1xml1protinit}{[xshift=-6mm] lc1java1init.north}[green]
    \drawarrow{lc1xml1singdestroy}{[xshift=-6mm] lc1java1destroy.north}[blue]
    \drawarrow{lc1xml1protdestroy}{[xshift=-6mm] lc1java1destroy.north}[blue]
\end{tikzpicture}

\noindent The~initialization method is~called three times --~once for~the~singleton instance and~twice for~prototype instances.
The~destroy method is~called only once --~for~the~singleton instance, although configured even for~prototype beans.
Destroy methods of~prototype beans don't work.

\newsubsection{Using Java Annotations}
\label{beanlifecycleannotations}
To~make a~method to~be the~\textit{initialization} method annotate it with the~\textit{PostConstruct} \hyperref[javaannotation]{annotation}.
To~make a~method to~be the~\textit{destroy} method annotate it with the~\textit{PreDestroy} \hyperref[javaannotation]{annotation}.

\warning Mentioned annotations belong to~the~\textit{annotation} package from the~\hyperref[javax]{\textit{javax}} extension set.
From Java~9 on no~\hyperref[javax]{\textit{javax}} package is a~part of~the~default Java.
You~must download it or~load it as~a~dependency.

\enlargethispage{10mm}
\example
%! language = TEXT
\begin{lstlisting}[language=Java, title={Singleton bean class}]
    import javax.annotations.*;

    @@>@Component<@@("(*\tmnbf{lc2java1singbeanid}{singletonBeanId}[ForestGreen]*)")
    public class SingletonClass implements SingletonClassInterface {
        ...
        @>@PostConstruct
        private void initMethod() {
            ...DO SOMETHING...
        }

        @>@PreDestroy
        private void destroyMethod() {
            ...DO SOMETHING...
        }
    }
\end{lstlisting}
%! language = TEXT
\begin{lstlisting}[language=Java, title={Prototype bean class}]
    import javax.annotations.*;

    @@>@Component<@@("(*\tmnbf{lc2java2protbeanid}{prototypeBeanId}[ForestGreen]*)")
    @@>@Scope<@@("prototype")
    public class PrototypeClass implements PrototypeClassInterface {
        ...
        @>@PostConstruct
        private void initMethod() {
            ...DO SOMETHING...
        }

        @>@PreDestroy
        private void destroyMethod() {
            ...DO SOMETHING...
        }
    }
\end{lstlisting}
\begin{tikzpicture}[remember picture, overlay, shift={(current page.south west)}]
    \coordinate(lc2java1singbeaniddummy) at (lc2java1singbeanid |-,0);
    \coordinate(lc2java2protbeaniddummy) at (lc2java2protbeanid |-,0);
    \drawarrow[stealth-]{[xshift=-9mm] lc2java1singbeanid.south}{[xshift=-9mm] lc2java1singbeaniddummy}
    \drawarrow[stealth-]{[xshift=9mm] lc2java2protbeanid.south}{[xshift=9mm] lc2java2protbeaniddummy}[red]
\end{tikzpicture}
\newpage
%! language = TEXT
\begin{lstlisting}[language=Java, title={Usage}]
    ClassPathXmlApplicationContext context = new ClassPathXmlApplicationContext("configurationFile.xml");

    CommonClassInterface singletonInstance1 = context.getBean("(*\tmnbf{lc2java3singbeanid1}{singletonBeanId}[ForestGreen]*)", SingletonClassInterface.class);
    CommonClassInterface singletonInstance2 = context.getBean("(*\tmnbf{lc2java3singbeanid2}{singletonBeanId}[ForestGreen]*)", SingletonClassInterface.class);

    CommonClassInterface prototypeInstance1 = context.getBean("(*\tmnbf{lc2java3protbeanid1}{prototypeBeanId}[ForestGreen]*)", PrototypeClassInterface.class);
    CommonClassInterface prototypeInstance2 = context.getBean("(*\tmnbf{lc2java3protbeanid2}{prototypeBeanId}[ForestGreen]*)", PrototypeClassInterface.class);

    context.close();
\end{lstlisting}
\begin{tikzpicture}[remember picture, overlay, shift={(current page.north west)}]
    \coordinate(lc2java3singbeaniddummy1) at (lc2java1singbeaniddummy |-, 0);
    \coordinate(lc2java3singbeaniddummy2) at (lc2java1singbeaniddummy |-, -30mm);
    \coordinate(lc2java3protbeaniddummy1) at (lc2java2protbeaniddummy |-, 0);
    \coordinate(lc2java3protbeaniddummy2) at (lc2java2protbeaniddummy |-, -30mm);
    \drawarrow[-]{[xshift=-9mm] lc2java3singbeaniddummy1}{[xshift=-9mm,yshift=-2mm] lc2java3singbeaniddummy2}
    \drawarrow[-stealth]{[xshift=-9.3mm] lc2java3singbeaniddummy2}{[xshift=3mm] lc2java3singbeanid1.north west}
    \drawarrow[-stealth]{[xshift=-9.1mm] lc2java3singbeaniddummy2}{lc2java3singbeanid2.north west}
    \drawarrow[-]{[xshift=9mm] lc2java3protbeaniddummy1}{[xshift=9mm, yshift=-2mm] lc2java3protbeaniddummy2}[red]
    \drawarrow[-stealth]{[xshift=9mm] lc2java3protbeaniddummy2}{[xshift=-3mm] lc2java3protbeanid1.north}[red]
    \drawarrow[-stealth]{[xshift=8.8mm] lc2java3protbeaniddummy2}{[xshift=6mm] lc2java3protbeanid2.north}[red]
    \node at ([xshift=-9mm] lc2java1singbeaniddummy |-, -31mm)[circle, fill, inner sep=1.5pt]{};
    \node at ([xshift=9mm] lc2java1singbeaniddummy |-, -31mm)[circle, fill, red, inner sep=1.5pt]{};
\end{tikzpicture}

\noindent Again the~initialization method is~called three times -- once for~the~singleton instance and~twice for~prototype instances.
The~destroy method is~called only once --~for~the~singleton instance.

\note There's one significant difference between annotation--based and~XML--based bean life cycle methods  One~bean class can~contain more than one annotated \textit{initialization} or~\textit{destroy} method.
When the~time of~their execution comes (bean instantiation, context closing), all~such methods are~triggered.
\textit{Initialization} methods are~triggered in~the~order reversed from the~order of~methods definition, \textit{destroy} methods are~called in~the~order of~methods definition.
All~the~other behavior (no~parameters allowed, \textit{destroy methods not~working for~\hyperref[prototypedp]{prototypes},~\dots}) and~conventions (\hyperref[javaaccessmodifiers]{access modifier} \textit{private}, no~return value,~\dots) are~the~same as~for~XML--based bean life cycle methods.

\newsubsection{Configuration with Java Code (No XML)}
When you define your own bean classes, you~should use \textit{PostConstruct} and~\textit{PreDestroy} (as~described in~the~\hyperref[beanlifecycleannotations]{annotation configuration section}) even with Java code configuration.
But~you can configure life cycle methods even in~configuration classes.
That's useful when you can't access bean classes source code.

Life cycle methods can~be configured by~specifying their names as~string arguments of~the~\textit{Bean} annotation.
Initialization methods are~specified by~the~\mbit{initMethod} parameter, destroy methods are~specified by~the~\textit{destroyMethod} parameter.
With this approach you can configure only one initialization method and~one destroy method per~bean (same as~with XML configuration).
\newpage

\example[multiple life cycle methods]
%! language = TEXT
\begin{lstlisting}[language=Java, title={Singleton bean class with multiple life cycle methods}]
    import javax.annotations.*;

    @@>@Component<@@("(*\tmnbf{lc4java1singbeanid}{singletonBeanId}[ForestGreen]*)")
    public class SingletonClass implements SingletonClassInterface {
        ...
        @>@PostConstruct
        private void firstInit() {
            ...DO SOMETHING...
        }

        @>@PostConstruct
        private void secondInit() {
            ...DO SOMETHING...
        }

        @>@PreDestroy
        private void firstDestroy() {
            ...DO SOMETHING...
        }

        @>@PreDestroy
        private void secondDestroy() {
            ...DO SOMETHING...
        }
    }
\end{lstlisting}
%! language = TEXT
\begin{lstlisting}[language=Java, title={Usage}]
    ClassPathXmlApplicationContext context = new ClassPathXmlApplicationContext("configurationFile.xml");

    CommonClassInterface singletonInstance = context.getBean("(*\tmnbf{lc4java2singbeanid}{singletonBeanId}[ForestGreen]*)", SingletonClassInterface.class);

    context.close();
\end{lstlisting}
\begin{tikzpicture}[remember picture, overlay]
    \drawarrow{lc4java1singbeanid}{lc4java2singbeanid}
\end{tikzpicture}

\noindent On~the~bean first initialization (it's~singleton) the~\textit{secondInit} method is~called first, then the~\textit{firstInit} method is~called.
On~the~context closing the~\textit{firstDestroy} method is~called first, then the~\textit{secondInit} method is~called.
If~the~bean scope was~set to~prototype, \textit{destroy} methods wouldn't be triggered.
\newpage

\example[life cycle methods configuration in a configuration class]
%! language = TEXT
\begin{lstlisting}[language=Java, title={Configuration class}]@>@Configuration
    @>@Configuration
    public class (*\tmnbf{lc5java1configclass}{ConfigurationClass}*) {
        @@>@Bean<@@(initMethod = "(*\tmnbf{lc5java1singinit}{initMethod}[ForestGreen]*)", destroyMethod = "(*\tmnbf{lc5java1singdestroy}{destroyMethod}[ForestGreen]*)")
        public CommonClassInterface (*\tmnbf{lc5java11singbeanid}{singletonBeanMethod}*)() {
            return new CommonClass(...);
        }

        @@>@Bean<@@(initMethod = "(*\tmnbf{lc5java1protinit}{initMethod}[ForestGreen]*)", destroyMethod = "(*\tmnbf{lc5java1protdestroy}{destroyMethod}[ForestGreen]*)")
        @@>@Scope<@@("prototype")
        public CommonClassInterface (*\tmnbf{lc5java11protbeanid}{prototypeBeanMethod}*)() {
            return new CommonClass(...);
        }
    }
\end{lstlisting}
%! language = TEXT
\begin{lstlisting}[language=Java, title={Snippet of a common bean class}]
    private void (*\tmnbf{lc5java2init}{initMethod}*)() {
        ...DO SOMETHING...
    }

    private void (*\tmnbf{lc5java2destroy}{destroyMethod}*)() {
        ...DO SOMETHING...
    }
\end{lstlisting}
%! language = TEXT
\begin{lstlisting}[language=Java, title={Usage}]
    ClassPathXmlApplicationContext context = new AnnotationConfigApplicationContext((*\tmnbf{lc5java3configclass}{ConfigurationClass}*).class);

    CommonClassInterface singletonInstance1 = context.getBean("(*\tmnbf{lc5java3singbeanid1}{singletonBeanMethod}[ForestGreen]*)", CommonClassInterface.class);
    CommonClassInterface singletonInstance2 = context.getBean("(*\tmnbf{lc5java3singbeanid2}{singletonBeanMethod}[ForestGreen]*)", CommonClassInterface.class);

    CommonClassInterface prototypeInstance1 = context.getBean("(*\tmnbf{lc5java3protbeanid1}{prototypeBeanMethod}[ForestGreen]*)", CommonClassInterface.class);
    CommonClassInterface prototypeInstance2 = context.getBean("(*\tmnbf{lc5java3protbeanid2}{prototypeBeanMethod}[ForestGreen]*)", CommonClassInterface.class);

    context.close();
\end{lstlisting}
\begin{tikzpicture}[remember picture, overlay]
    \drawarrow{lc5java1configclass.south}{lc5java3configclass.north}[black][.3][bend right=18mm]
    \drawarrow{[xshift=-6mm] lc5java11singbeanid.south}{[xshift=-6mm] lc5java3singbeanid1.north}[red]
    \drawarrow{[xshift=-6mm] lc5java11singbeanid.south}{[xshift=-3mm] lc5java3singbeanid2.north}[red]
    \drawarrow{[xshift=-6mm] lc5java11protbeanid.south}{[xshift=3mm] lc5java3protbeanid1.north}[green]
    \drawarrow{[xshift=-6mm] lc5java11protbeanid.south}{[xshift=6mm] lc5java3protbeanid2.north}[green]
    \drawarrow{lc5java1singinit}{lc5java2init.north}[blue]
    \drawarrow{lc5java1protinit}{lc5java2init.north}[blue]
    \drawarrow{lc5java1singdestroy}{lc5java2destroy.north}[yellow][.5]
    \drawarrow{lc5java1protdestroy}{lc5java2destroy.north}[yellow][.5]
\end{tikzpicture}

\noindent And~again the~initialization method is~called three times -- once for~the~singleton instance and~twice for~prototype instances.
The~destroy method is~called only once --~for~the~singleton instance.
