\newsection{Dependency Injection}
\index{Dependency injection}
\label{springdependencyinjection}
Spring bean classes can~contain some fields.
The~\hyperref[springcontainrer]{Spring container} can~automatically assign instances of~classes matching those fields types.
I.e., the~\hyperref[springcontainrer]{Spring container} can~perform an~automated \hyperref[dependencyinjection]{dependency injection}.

Basically it must be configured what field should get what class instance, this configuration is a~part of~the~XML configuration file.
With the~\hyperref[autowiring]{autowiring feature} (see~further) it's~even possible to~completely avoid the~configuration if~the~correct class for~the~field can~be identified unambiguously.

\note Dependencies injected by~Spring dependency injection must~be also Spring beans.

\newsubsection{Constructor Injection}
\index{Constructor injection}
\label{constructorinjection}
One~of~basic approaches to~the~dependency injection is to~set the~dependency via the~wanted class constructor.
The~constructor must accept a~parameter of a~type of~an~interface, which is~implemented by the~dependency class.
The~dependency class instance -- another bean --  is~specified and~injected to~the~wanted bean in~the~XML configuration file by~the~element \textit{constructor-arg}.

\newsubsection{Setter Injection}
\index{Setter injection}
\label{setterinjection}
Another basic approach to~the~dependency injection is to~set the~dependency via a~setter method defined in~the~wanted class.
The~wanted class must have a~zero--parameter constructor.
It's~safer to~always define~it, even~if there is no other constructor (and~therefore the~zero--parameter one is~implicit).
There~is nothing wrong with keeping the~constructor body empty.
The~injection is~similar to~the~\hyperref[constructorinjection]{constructor} approach.
The~dependency class instance -- another bean --  is~specified and~injected to~the~wanted bean in~the~XML configuration file by~the~element \textit{property}.

\warning There is one crucial restriction to~follow.
The~\textit{property} element must contain the~attribute \textit{name} with some value, let's say \textit{someName}.
The~name of~the~setter method in the~wanted class must then be exactly \textit{setSomeName}.
The~name of~the~field variable doesn't matter.

\warning Dependency fields in~the~main bean class can't~be declared as~final when using the~setter injection.
Final fields must be defined no~later than in~the~constructor, and~that can't happen with the~setter injection.

\emtwen
\newsubsection{Injecting literal values}
\label{injectingliteralvalues}
You~can inject even values of primitive types and~strings directly using the~XML configuration file.
The~approach is the~same, but~instead of~using the~\textit{ref} attribute in~the~\textit{property} element you use the~\textit{value} attribute and~set directly the~value.
You~can do it both for~\hyperref[constructorinjection]{constructor} and~\hyperref[setterinjection]{setter} injection.

\warning If~you~mismatch injected value type (e.g.,~inject letters to~an~integer field), the~program will remain compilable.
It~will~fail only at~runtime.
\newpage

\examplenonl[constructor injection]
%! language = TEXT
\begin{lstlisting}[language=XML, title={Configuration XML}]
    <?xml version="1.0" encoding="UTF-8"?>
    <beans ...>
      <bean id="(*\tmnbf{di1xml1depbeanid}{dependencyBeanId}[ForestGreen]*)" class="(*\tmnbf{di1xml1deppackage}{deppackage.depsubpackage}[ForestGreen]*).(*\tmnbf{di1xml1depclass}{DependencyClass}[ForestGreen]*)"/>
      <bean id="(*\tmnbf{di1xml1beanid}{wantedBeanId}[ForestGreen]*)" class="(*\tmnbf{di1xml1package}{somepackage.subpackage}[ForestGreen]*).(*\tmnbf{di1xml1class}{WantedClass}[ForestGreen]*)">
        <constructor-arg ref="(*\tmnbf{di1xml1depbeanid2}{dependencyBeanId}[ForestGreen]*)"/>
      </bean>
    </beans>
\end{lstlisting}
%! language = TEXT
\begin{lstlisting}[language=Java, title={Dependency class}]
    package (*\tmnbf{di1java1deppackage}{deppackage.depsubpackage}*);

    public class (*\tmnbf{di1java1depclass}{DependencyClass}*) implements (*\tmnbf{di1java1depinterface}{DependencyInterface}*) {
        ...
    }
\end{lstlisting}
%! language = TEXT
\begin{lstlisting}[language=Java, title={Wanted class with the constructor}]
    package (*\tmnbf{di1java2package}{somepackage.subpackage}*);

    public class (*\tmnbf{di1java2class}{WantedClass}*) implements (*\tmnbf{di1java2interface}{WantedClassInterface}*) {
        private (*\tmnbf{di1java2depinterface}{DependencyInterface}*) dependencyField;

        public (*\tmnbf{di1java2class2}{WantedClass}*)((*\tmnbf{di1java2depinterface2}{DependencyInterface}*) dependencyValue) {
            dependencyField = dependencyValue;
        }

        public WHATEVER (*\tmnbf{di1java2usedep}{useDependency}*)() {
            ...DO SOMETHING WITH THE DEPENDENCY FIELD...
        }
    }
\end{lstlisting}
%! language = TEXT
\begin{lstlisting}[language=Java, title={Usage}]
    ClassPathXmlApplicationContext context = new ClassPathXmlApplicationContext("configurationFile.xml");
    (*\tmnbf{di1java3interface}{WantedClassInterface}*) wantedClassInstance = context.getBean("(*\tmnbf{di1java3beanid}{wantedBeanId}[ForestGreen]*)", (*\tmnbf{di1java3interface2}{WantedClassInterface}*).class);
    wantedClassInstance.(*\tmnbf{di1java3usedep}{useDependency}*)();
\end{lstlisting}
\begin{tikzpicture}[remember picture, overlay]
    \drawarrow{di1xml1depbeanid}{di1xml1depbeanid2}
    \drawarrow{[xshift=-3mm] di1xml1deppackage.south}{di1java1deppackage}[red]
    \drawarrow{di1xml1depclass}{di1java1depclass}[green]
    \drawarrow{di1xml1beanid.south}{[xshift=-6mm]di1java3beanid.north}[blue][.3][bend right]
    \drawarrow{[xshift=3mm] di1xml1package.south}{di1java2package}[Magenta]
    \drawarrow{di1xml1class}{di1java2class}[yellow][.5]
    \drawarrow{di1java2class}{di1java2class2}[yellow][.5]
    \drawarrow{di1java1depinterface.south}{di1java2depinterface}
    \drawarrow{di1java1depinterface.south}{di1java2depinterface2}
    \drawarrow{di1java2interface.south}{di1java3interface}[red]
    \drawarrow{di1java2interface.south}{di1java3interface2}[red]
    \drawarrow{[xshift=-6mm] di1java2usedep.south}{di1java3usedep}[green]
\end{tikzpicture}

\noindent Note that the~dependency class is~not~explicitly created anywhere -- no~constructor nor~\textit{context.getBean} is~called for~it.
\newpage

\examplenonl[setter injection]
%! language = TEXT
\begin{lstlisting}[language=XML, title={Configuration XML}]
    <?xml version="1.0" encoding="UTF-8"?>
    <beans ...>
      <bean id="(*\tmnbf{di2xml1depbeanid}{dependencyBeanId}[ForestGreen]*)" class="(*\tmnbf{di2xml1deppackage}{deppackage.depsubpackage}[ForestGreen]*).(*\tmnbf{di2xml1depclass}{DependencyClass}[ForestGreen]*)"/>
      <bean id="(*\tmnbf{di2xml1beanid}{wantedBeanId}[ForestGreen]*)" class="(*\tmnbf{di2xml1package}{somepackage.subpackage}[ForestGreen]*).(*\tmnbf{di2xml1class}{WantedClass}[ForestGreen]*)">
        <property name="(*\tmnbf{di2xml1depfield}{dependencyField}[ForestGreen]*)" ref="(*\tmnbf{di2xml1depbeanid2}{dependencyBeanId}[ForestGreen]*)"/>
      </bean>
    </beans>
\end{lstlisting}
%! language = TEXT
\begin{lstlisting}[language=Java, title={Dependency class}]
    package (*\tmnbf{di2java1deppackage}{deppackage.depsubpackage}*);

    public class (*\tmnbf{di2java1depclass}{DependencyClass}*) implements (*\tmnbf{di2java1depinterface}{DependencyInterface}*) {
        ...
    }
\end{lstlisting}
%! language = TEXT
\begin{lstlisting}[language=Java, title={Wanted class with the zero--parameter constructor and the setter method}]
    package (*\tmnbf{di2java2package}{somepackage.subpackage}*);

    public class (*\tmnbf{di2java2class}{WantedClass}*) implements (*\tmnbf{di2java2interface}{WantedClassInterface}*) {
        private (*\tmnbf{di2java2depinterface}{DependencyInterface}*) dependencyField;

        public (*\tmnbf{di2java2class2}{WantedClass}*)() {}

        public void set(*\tmnbf{di2java2depfield}{DependencyField}*)((*\tmnbf{di2java2depinterface2}{DependencyInterface}*) dependencyValue) {
            dependencyField = dependencyValue;
        }

        public WHATEVER (*\tmnbf{di2java2usedep}{useDependency}*)() {
            ...DO SOMETHING WITH THE DEPENDENCY FIELD...
        }
    }
\end{lstlisting}
%! language = TEXT
\begin{lstlisting}[language=Java, title={Usage}]
    ClassPathXmlApplicationContext context = new ClassPathXmlApplicationContext("configurationFile.xml");
    (*\tmnbf{di2java3interface}{WantedClassInterface}*) wantedClassInstance = context.getBean("(*\tmnbf{di2java3beanid}{wantedBeanId}[ForestGreen]*)", (*\tmnbf{di2java3interface2}{WantedClassInterface}*).class);
    wantedClassInstance.(*\tmnbf{di2java3usedep}{useDependency}*)();
\end{lstlisting}
\begin{tikzpicture}[remember picture, overlay]
    \drawarrow{di2xml1depbeanid2.north}{di2xml1depbeanid.east}[black][.3][bend right=18mm]
    \drawarrow{di2xml1deppackage}{di2java1deppackage}[red]
    \drawarrow{[xshift=-9mm] di2xml1depclass.south}{di2java1depclass}[green]
    \drawarrow{di2xml1beanid.south}{[xshift=-6mm] di2java3beanid.north}[blue][.3][bend right]
    \drawarrow{di2xml1package.south}{di2java2package}[Magenta]
    \drawarrow{di2xml1class}{di2java2class}[yellow][.5]
    \drawarrow{di2java2class}{di2java2class2}[yellow][.5]
    \drawarrow{di2xml1depfield.south}{[xshift=3mm] di2java2depfield.north}
    \drawarrow{di2java1depinterface.south}{[xshift=-3mm] di2java2depinterface.north east}[red]
    \drawarrow{di2java1depinterface.south}{[xshift=6mm] di2java2depinterface2.north}[red]
    \drawarrow{di2java2interface.south}{[xshift=-6mm] di2java3interface.north}[green][.3][bend right=12mm]
    \drawarrow{di2java2interface.south}{di2java3interface2}[green]
    \drawarrow{di2java2usedep.south}{di2java3usedep}[blue]
\end{tikzpicture}
\newpage

\example[injecting literal values with a constructor]
%! language = TEXT
\begin{lstlisting}[language=XML, title={Configuration XML}]
    <?xml version="1.0" encoding="UTF-8"?>
    <beans ...>
      <bean id="(*\tmnbf{di3xml1beanid}{wantedBeanId}[ForestGreen]*)" class="(*\tmnbf{di3xml1package}{somepackage.subpackage}[ForestGreen]*).(*\tmnbf{di3xml1class}{WantedClass}[ForestGreen]*)">
        <constructor-arg value="someValue"/>
      </bean>
    </beans>
\end{lstlisting}
%! language = TEXT
\begin{lstlisting}[language=Java, title={Wanted class with the constructor}]
    package (*\tmnbf{di3java1package}{somepackage.subpackage}*);

    public class (*\tmnbf{di3java1class}{WantedClass}*) implements (*\tmnbf{di3java1interface}{WantedClassInterface}*) {
        private String concreteValueField;

        public (*\tmnbf{di3java1class2}{WantedClass}*)(String concreteValue) {
            concreteValueField = concreteValue;
        }

        public WHATEVER (*\tmnbf{di3java1usevalue}{useConcreteValue}*)() {
            ...DO SOMETHING WITH THE CONCRETE VALUE FIELD...
        }
    }
\end{lstlisting}
%! language = TEXT
\begin{lstlisting}[language=Java, title={Usage}]
    ClassPathXmlApplicationContext context = new ClassPathXmlApplicationContext("configurationFile.xml");
    (*\tmnbf{di3java2interface}{WantedClassInterface}*) wantedClassInstance = context.getBean("(*\tmnbf{di3java2beanid}{wantedBeanId}[ForestGreen]*)", (*\tmnbf{di3java2interface2}{WantedClassInterface}*).class);
    wantedClassInstance.(*\tmnbf{di3java2usevalue}{useConcreteValue}*)();
\end{lstlisting}
\begin{tikzpicture}[remember picture, overlay]
    \drawarrow{di3xml1beanid.south}{di3java2beanid.north}[black][.3][bend right]
    \drawarrow{di3xml1package}{di3java1package}[red]
    \drawarrow{di3xml1class}{di3java1class}[green]
    \drawarrow{di3java1class}{di3java1class2}[green]
    \drawarrow{di3java1interface.south}{di3java2interface}[blue]
    \drawarrow{di3java1interface.south}{di3java2interface2}[blue]
    \drawarrow{[xshift=-6mm] di3java1usevalue.south}{di3java2usevalue}[Magenta]
\end{tikzpicture}
\newpage

\examplenonl[injecting literal values with a setter method]
%! language = TEXT
\begin{lstlisting}[language=XML, title={Configuration XML}]
    <?xml version="1.0" encoding="UTF-8"?>
    <beans ...>
      <bean id="(*\tmnbf{di4xml1beanid}{wantedBeanId}[ForestGreen]*)" class="(*\tmnbf{di4xml1package}{somepackage.subpackage}[ForestGreen]*).(*\tmnbf{di4xml1class}{WantedClass}[ForestGreen]*)">
        <property name="(*\tmnbf{di4xml1valuefield}{concreteValueField}[ForestGreen]*)" value="someValue"/>
      </bean>
    </beans>
\end{lstlisting}
%! language = TEXT
\begin{lstlisting}[language=Java, title={Wanted class with the zero--parameter constructor and the setter method}]
    package (*\tmnbf{di4java1package}{somepackage.subpackage}*);

    public class (*\tmnbf{di4java1class}{WantedClass}*) implements (*\tmnbf{di4java1interface}{WantedClassInterface}*) {
        private String concreteValueField;

        public (*\tmnbf{di4java1class2}{WantedClass}*)() {}

        public void set(*\tmnbf{di4java1valuefield}{ConcreteValueField}*)(String concreteValue) {
            concreteValueField = concreteValue;
        }

        public WHATEVER (*\tmnbf{di4java1usevalue}{useConcreteValue}*)() {
            ...DO SOMETHING WITH THE CONCRETE VALUE FIELD...
        }
    }
\end{lstlisting}
%! language = TEXT
\begin{lstlisting}[language=Java, title={Usage}]
    ClassPathXmlApplicationContext context = new ClassPathXmlApplicationContext("configurationFile.xml");
    (*\tmnbf{di4java2interface}{WantedClassInterface}*) wantedClassInstance = context.getBean("(*\tmnbf{di4java2beanid}{wantedBeanId}[ForestGreen]*)", (*\tmnbf{di4java2interface2}{WantedClassInterface}*).class);
    wantedClassInstance.(*\tmnbf{di4java2usevalue}{useConcreteValue}*)();
\end{lstlisting}
\begin{tikzpicture}[remember picture, overlay]
    \drawarrow{di4xml1beanid}{di4java2beanid}[black][.3][bend right]
    \drawarrow{di4xml1package}{di4java1package}[red]
    \drawarrow{di4xml1class}{di4java1class}[green]
    \drawarrow{di4java1class}{di4java1class2}[green]
    \drawarrow{di4xml1valuefield}{di4java1valuefield}[blue]
    \drawarrow{di4java1interface.south}{di4java2interface}[Magenta][.3][bend right=12mm]
    \drawarrow{di4java1interface.south}{di4java2interface2}[Magenta]
    \drawarrow{di4java1usevalue}{di4java2usevalue}[yellow][.5]
\end{tikzpicture}
\newpage

\newsubsection{Autowiring}
\index{Autowired}
\label{autowiring}
Similarly to~the~\hyperref[springinversionofcontrol]{inversion of~control} even the~\hyperref[springdependencyinjection]{dependency injection} can~be handled with \hyperref[javaannotation]{Java annotations}.
A~Spring functionality called \textit{autowiring} automatically detects a~dependency bean class that can~be injected to~a~field in~the~main bean class.
A~\hyperref[constructorinjection]{constructor} or~a~\hyperref[setterinjection]{setter method} performing the~injection must be annotated with the~annotation \textit{Autowired}, which processed in~runtime.
The~dependency bean class must be annotated with the~annotation \textit{Component} in~the~zero--parameter form (no~bean~ID specified).
When properly combined with the~\hyperref[iocannotations]{annotation--configured} \hyperref[springinversionofcontrol]{inversion of~control}, the~configuration XML file contains only the~component scanning configuration.

\note Opposite to~the~standard \hyperref[setterinjection]{setter injection} with~XML configuration file the~name of~an~autowired setter method isn't restricted in~any~way.
The~Spring's autowiring is~directed by~the~type of~its parameter.
I.e.,~in~the~following example the~name of~the~setter method could~be even literally \textit{absoluteNonsense} and~the~dependency injection would~be still working.

\warning You~can't use autowiring for~\hyperref[injectingliteralvalues]{injecting literal values}.
Autowiring is~usable only when the~injected value is~an~instance of~an~explicitly defined Spring bean with the~\textit{Component} annotation.
Nevertheless, it's~possible to~parse \textit{properties} files with annotations, see~\hyperref[readingpropertiesannotations]{further}.

\newsubsection{Field Injection}
\index{Field injection}
\label{fieldinjection}
The~\hyperref[autowiring]{autowiring} annotation can~be used directly to~dependency fields (class fields only, not~local variables inside methods).
These fields then don't need any~explicit setting, autowiring finds, creates and~assigns a~correct dependency class instance automatically.
It's~like the~\hyperref[setterinjection]{setter injection} (i.e.,~with zero--parameter constructor), but~without \textit{set} methods.
The~Spring implementation behind this "magic" uses \hyperref[reflection]{Java reflection}.

\newsubsection{Qualifiers}
\index{Qualifier}
When there are more dependency classes matching the~dependency interface, it~isn't clear for~autowiring what class instance inject to~the~dependency field and~running (not~building) the~program fails complaining about not~unique bean.
To~solve this problem you~must provide the~dependency bean~ID as~an~argument of~the~annotation \textit{Qualifier}.
When using \hyperref[constructorinjectionautowire]{constructor injection} or~\hyperref[setterinjectionautowire]{setter injection}, the~annotation is~applied to~the~argument of~the~constructor or~the~setter method.
When using field injection, the~annotation is~applied directly to~the~field variable.
\newpage

\examplenonl[autowired constructor injection]
\label{constructorinjectionautowire}
%! language = TEXT
\begin{lstlisting}[language=XML, title={Configuration XML}]
    <?xml version="1.0" encoding="UTF-8"?>
    <beans ...
           xmlns:context="http://www.springframework.org/schema/context"
           ...>
      <context:component-scan base-package="(*\tmnbf{di5xml1package}{somepackage.subpackage}[ForestGreen]*)"/>
    </beans>
\end{lstlisting}
%! language = TEXT
\begin{lstlisting}[language=Java, title={Dependency class}]
    package deppackage.depsubpackage;

    @>@Component
    public class DependencyClass implements (*\tmnbf{di5java1depinterface}{DependencyInterface}*) {
        ...
    }
\end{lstlisting}
%! language = TEXT
\begin{lstlisting}[language=Java, title={Wanted class with the constructor}]
    package (*\tmnbf{di5java2package}{somepackage.subpackage}*);

    @>@Component
    public class (*\tmnbf{di5java2class}{WantedClass}*) implements (*\tmnbf{di5java2interface}{WantedClassInterface}*) {
        private (*\tmnbf{di5java2depinterface}{DependencyInterface}*) dependencyField;

        @>@Autowired
        public (*\tmnbf{di5java2class2}{WantedClass}*)()((*\tmnbf{di5java2depinterface2}{DependencyInterface}*) dependencyValue) {
            dependencyField = dependencyValue;
        }

        public WHATEVER (*\tmnbf{di5java2usedep}{useDependency}*)() {
            ...DO SOMETHING WITH THE DEPENDENCY FIELD...
        }
    }
\end{lstlisting}
%! language = TEXT
\begin{lstlisting}[language=Java, title={Usage}]
    ClassPathXmlApplicationContext context = new ClassPathXmlApplicationContext("configurationFile.xml");
    (*\tmnbf{di5java3interface}{WantedClassInterface}*) wantedClassInstance = context.getBean("(*\tmnbf{di5java3beanid}{wantedClass}[ForestGreen]*)", (*\tmnbf{di5java3interface2}{WantedClassInterface}*).class);
    wantedClassInstance.(*\tmnbf{di5java3usedep}{useDependency}*)();
\end{lstlisting}
\begin{tikzpicture}[remember picture, overlay]
    \drawarrow{di5xml1package}{di5java2package}
    \drawarrow{di5java1depinterface.south}{di5java2depinterface}[red]
    \drawarrow{di5java1depinterface.south}{di5java2depinterface2}[red]
    \drawarrow{di5java2class.south}{di5java2class2}[green]
    \drawarrow{di5java2class.south}{[xshift=-3mm] di5java3beanid.north}[green]
    \drawarrow{di5java2interface.south}{di5java3interface}[blue]
    \drawarrow{di5java2interface.south}{[xshift=3mm] di5java3interface2.north}[blue]
    \drawarrow{[xshift=-3mm] di5java2usedep.south}{di5java3usedep}[Magenta]
\end{tikzpicture}
\newpage

\examplenonl[autowired setter injection]
\label{setterinjectionautowire}
%! language = TEXT
\begin{lstlisting}[language=XML, title={Configuration XML}]
    <?xml version="1.0" encoding="UTF-8"?>
    <beans ...
           xmlns:context="http://www.springframework.org/schema/context"
           ...>
      <context:component-scan base-package="(*\tmnbf{di6xml1package}{somepackage.subpackage}[ForestGreen]*)"/>
    </beans>
\end{lstlisting}
%! language = TEXT
\begin{lstlisting}[language=Java, title={Dependency class}]
    package deppackage.depsubpackage;

    @>@Component
    public class DependencyClass implements (*\tmnbf{di6java1depinterface}{DependencyInterface}*) {
        ...
    }
\end{lstlisting}
%! language = TEXT
\begin{lstlisting}[language=Java, title={Wanted class with the zero--parameter constructor and the setter method}]
    package (*\tmnbf{di6java2package}{somepackage.subpackage}*);

    @>@Component
    public class (*\tmnbf{di6java2class}{WantedClass}*) implements (*\tmnbf{di6java2interface}{WantedClassInterface}*) {
        private (*\tmnbf{di6java2depinterface}{DependencyInterface}*) dependencyField;

        public (*\tmnbf{di6java2class2}{WantedClass}*)() {}

        @>@Autowired
        public void setDependencyField((*\tmnbf{di6java2depinterface2}{DependencyInterface}*) dependencyValue){
            dependencyField = dependencyValue;
        }

        public WHATEVER (*\tmnbf{di6java2usedep}{useDependency}*)() {
            ...DO SOMETHING WITH THE DEPENDENCY FIELD...
        }
    }
\end{lstlisting}
%! language = TEXT
\begin{lstlisting}[language=Java, title={Usage}]
    ClassPathXmlApplicationContext context = new ClassPathXmlApplicationContext("configurationFile.xml");
    (*\tmnbf{di6java3interface}{WantedClassInterface}*) wantedClassInstance = context.getBean("(*\tmnbf{di6java3beanid}{wantedClass}[ForestGreen]*)", (*\tmnbf{di6java3interface2}{WantedClassInterface}*).class);
    wantedClassInstance.(*\tmnbf{di6java3usedep}{useDependency}*)();
\end{lstlisting}
\begin{tikzpicture}[remember picture, overlay]
    \drawarrow{di6xml1package}{di6java2package}
    \drawarrow{di6java1depinterface.south}{di6java2depinterface}[red]
    \drawarrow{di6java1depinterface.south}{[xshift=6mm] di6java2depinterface2.north}[red]
    \drawarrow{di6java2class.south}{di6java2class2}[green]
    \drawarrow{di6java2class.south}{[xshift=-3mm] di6java3beanid.north}[green]
    \drawarrow{di6java2interface.south}{di6java3interface}[blue][.3][bend right=12mm]
    \drawarrow{di6java2interface.south}{di6java3interface2}[blue]
    \drawarrow{di6java2usedep}{di6java3usedep}[Magenta]
\end{tikzpicture}
\newpage

\examplenonl[autowired field injection]
%! language = TEXT
\begin{lstlisting}[language=XML, title={Configuration XML}]
    <?xml version="1.0" encoding="UTF-8"?>
    <beans ...
           xmlns:context="http://www.springframework.org/schema/context"
           ...>
      <context:component-scan base-package="(*\tmnbf{di7xml1package}{somepackage.subpackage}[ForestGreen]*)"/>
    </beans>
\end{lstlisting}
%! language = TEXT
\begin{lstlisting}[language=Java, title={Dependency class}]
    package deppackage.depsubpackage;

    @>@Component
    public class DependencyClass implements (*\tmnbf{di7java1depinterface}{DependencyInterface}*) {
        ...
    }
\end{lstlisting}
%! language = TEXT
\begin{lstlisting}[language=Java, title={Wanted class with the zero--parameter constructor}]
    package (*\tmnbf{di7java2package}{somepackage.subpackage}*);

    @>@Component
    public class (*\tmnbf{di7java2class}{WantedClass}*) implements (*\tmnbf{di7java2interface}{WantedClassInterface}*) {
        @>@Autowired
        private (*\tmnbf{di7java2depinterface}{DependencyInterface}*) dependencyField;

        public (*\tmnbf{di7java2class2}{WantedClass}*)() {}

        public WHATEVER (*\tmnbf{di7java2usedep}{useDependency}*)() {
            ...DO SOMETHING WITH THE DEPENDENCY FIELD...
        }
    }
\end{lstlisting}
%! language = TEXT
\begin{lstlisting}[language=Java, title={Usage}]
    ClassPathXmlApplicationContext context = new ClassPathXmlApplicationContext("configurationFile.xml");
    (*\tmnbf{di7java3interface}{WantedClassInterface}*) wantedClassInstance = context.getBean("(*\tmnbf{di7java3beanid}{wantedClass}[ForestGreen]*)", (*\tmnbf{di7java3interface2}{WantedClassInterface}*).class);
    wantedClassInstance.(*\tmnbf{di7java3usedep}{useDependency}*)();
\end{lstlisting}
\begin{tikzpicture}[remember picture, overlay]
    \drawarrow{di7xml1package}{di7java2package}
    \drawarrow{di7java1depinterface}{di7java2depinterface}[red]
    \drawarrow{di7java2class.south}{di7java2class2}[green]
    \drawarrow{di7java2class.south}{[xshift=-6mm]di7java3beanid.north}[green]
    \drawarrow{di7java2interface.south}{di7java3interface}[blue]
    \drawarrow{di7java2interface.south}{di7java3interface2}[blue]
    \drawarrow{di7java2usedep}{di7java3usedep}[Magenta]
\end{tikzpicture}
\newpage

\examplenonl[autowired constructor injection with a qualifier]
%! language = TEXT
\begin{lstlisting}[language=XML, title={Configuration XML}]
    <?xml version="1.0" encoding="UTF-8"?>
    <beans ...
           xmlns:context="http://www.springframework.org/schema/context"
           ...>
      <context:component-scan base-package="(*\tmnbf{di8xml1package}{somepackage.subpackage}[ForestGreen]*)"/>
    </beans>
\end{lstlisting}
%! language = TEXT
\begin{lstlisting}[language=Java, title={Dependency class}]
    package deppackage.depsubpackage;

    @>@Component
    public class (*\tmnbf{di8java1depclass}{DependencyClass}*) implements (*\tmnbf{di8java1depinterface}{DependencyInterface}*) {
        ...
    }
\end{lstlisting}
%! language = TEXT
\begin{lstlisting}[language=Java, title={Wanted class with the constructor}]
    package (*\tmnbf{di8java2package}{somepackage.subpackage}*);

    @>@Component
    public class (*\tmnbf{di8java2class}{WantedClass}*) implements (*\tmnbf{di8java2interface}{WantedClassInterface}*) {
        private (*\tmnbf{di8java2depinterface}{DependencyInterface}*) dependencyField;

        @>@Autowired
        public (*\tmnbf{di8java2class2}{WantedClass}*)(@@>@Qualifier<@@("(*\tmnbf{di8java2depbeanid}{dependencyClass}[ForestGreen]*)") (*\tmnbf{di8java2depinterface2}{DependencyInterface}*) dependencyValue) {
            dependencyField = dependencyValue;
        }

        public WHATEVER (*\tmnbf{di8java2usedep}{useDependency}*)() {
            ...DO SOMETHING WITH THE DEPENDENCY FIELD...
        }
    }
\end{lstlisting}
%! language = TEXT
\begin{lstlisting}[language=Java, title={Usage}]
    ClassPathXmlApplicationContext context = new ClassPathXmlApplicationContext("configurationFile.xml");
    (*\tmnbf{di8java3interface}{WantedClassInterface}*) wantedClassInstance = context.getBean("(*\tmnbf{di8java3beanid}{wantedClass}[ForestGreen]*)", (*\tmnbf{di8java3interface2}{WantedClassInterface}*).class);
    wantedClassInstance.(*\tmnbf{di8java3usedep}{useDependency}*)();
\end{lstlisting}
\begin{tikzpicture}[remember picture, overlay]
    \drawarrow{di8xml1package}{di8java2package}
    \drawarrow{di8java1depclass}{di8java2depbeanid}[red]
    \drawarrow{di8java1depinterface.south}{di8java2depinterface}[green]
    \drawarrow{di8java1depinterface.south}{[xshift=-3mm] di8java2depinterface2.north east}[green]
    \drawarrow{di8java2class.south}{di8java2class2}[blue]
    \drawarrow{di8java2class.south}{[xshift=-6mm] di8java3beanid.north}[blue]
    \drawarrow{di8java2interface.south}{di8java3interface}[Magenta][.3][bend right=12mm]
    \drawarrow{di8java2interface.south}{di8java3interface2}[Magenta]
    \drawarrow{di8java2usedep}{di8java3usedep}[yellow][.5]
\end{tikzpicture}
\newpage

\examplenonl[autowired setter injection with a qualifier]
%! language = TEXT
\begin{lstlisting}[language=XML, title={Configuration XML}]
    <?xml version="1.0" encoding="UTF-8"?>
    <beans ...
           xmlns:context="http://www.springframework.org/schema/context"
           ...>
      <context:component-scan base-package="(*\tmnbf{di9xml1package}{somepackage.subpackage}[ForestGreen]*)"/>
    </beans>
\end{lstlisting}
%! language = TEXT
\begin{lstlisting}[language=Java, title={Dependency class}]
    package deppackage.depsubpackage;

    @>@Component
    public class (*\tmnbf{di9java1depclass}{DependencyClass}*) implements (*\tmnbf{di9java1depinterface}{DependencyInterface}*) {
        ...
    }
\end{lstlisting}
%! language = TEXT
\begin{lstlisting}[language=Java, title={Wanted class with the zero--parameter constructor and the setter method}]
    package (*\tmnbf{di9java2package}{somepackage.subpackage}*);

    @>@Component
    public class (*\tmnbf{di9java2class}{WantedClass}*) implements (*\tmnbf{di9java2interface}{WantedClassInterface}*) {
        private (*\tmnbf{di9java2depinterface}{DependencyInterface}*) dependencyField;

        public (*\tmnbf{di9java2class2}{WantedClass}*)() {}

        @>@Autowired
        public void setDependencyField(@@>@Qualifier<@@("(*\tmnbf{di9java2depbeanid}{dependencyClass}[ForestGreen]*)") (*\tmnbf{di9java2depinterface2}{DependencyInterface}*) dependencyValue) {
            dependencyField = dependencyValue;
        }

        public WHATEVER (*\tmnbf{di9java2usedep}{useDependency}*)() {
            ...DO SOMETHING WITH THE DEPENDENCY FIELD...
        }
    }
\end{lstlisting}
%! language = TEXT
\begin{lstlisting}[language=Java, title={Usage}]
    ClassPathXmlApplicationContext context = new ClassPathXmlApplicationContext("configurationFile.xml");
    (*\tmnbf{di9java3interface}{WantedClassInterface}*) wantedClassInstance = context.getBean("(*\tmnbf{di9java3beanid}{wantedClass}[ForestGreen]*)", (*\tmnbf{di9java3interface2}{WantedClassInterface}*).class);
    wantedClassInstance.(*\tmnbf{di9java3usedep}{useDependency}*)();
\end{lstlisting}
\begin{tikzpicture}[remember picture, overlay]
    \drawarrow{di9xml1package}{di9java2package}
    \drawarrow{di9java1depclass}{di9java2depbeanid}[red]
    \drawarrow{di9java1depinterface.south}{di9java2depinterface}[green]
    \drawarrow{di9java1depinterface.south}{[xshift=-6mm] di9java2depinterface2.north east}[green]
    \drawarrow{di9java2class.south}{di9java2class2}[blue]
    \drawarrow{di9java2class.south}{[xshift=-6mm] di9java3beanid.north}[blue]
    \drawarrow{di9java2interface.south}{di9java3interface}[Magenta][.3][bend right=12mm]
    \drawarrow{di9java2interface.south}{di9java3interface2}[Magenta]
    \drawarrow{di9java2usedep}{di9java3usedep}[yellow][.5]
\end{tikzpicture}
\newpage

\examplenonl[autowired field injection with a qualifier]
%! language = TEXT
\begin{lstlisting}[language=XML, title={Configuration XML}]
    <?xml version="1.0" encoding="UTF-8"?>
    <beans ...
           xmlns:context="http://www.springframework.org/schema/context"
           ...>
      <context:component-scan base-package="(*\tmnbf{di10xml1package}{somepackage.subpackage}[ForestGreen]*)"/>
    </beans>
\end{lstlisting}
%! language = TEXT
\begin{lstlisting}[language=Java, title={Dependency class}]
    package deppackage.depsubpackage;

    @>@Component
    public class (*\tmnbf{di10java1depclass}{DependencyClass}*) implements (*\tmnbf{di10java1depinterface}{DependencyInterface}*) {
        ...
    }
\end{lstlisting}
%! language = TEXT
\begin{lstlisting}[language=Java, title={Wanted class with the zero--parameter constructor}]
    package (*\tmnbf{di10java2package}{somepackage.subpackage}*);

    @>@Component
    public class (*\tmnbf{di10java2class}{WantedClass}*) implements (*\tmnbf{di10java2interface}{WantedClassInterface}*) {
        @>@Autowired
        @@>@Qualifier<@@("(*\tmnbf{di10java2depbeanid}{dependencyClass}[ForestGreen]*)")
        private (*\tmnbf{di10java2depinterface}{DependencyInterface}*) dependencyField;

        public (*\tmnbf{di10java2class2}{WantedClass}*)() {}

        public WHATEVER (*\tmnbf{di10java2usedep}{useDependency}*)() {
            ...DO SOMETHING WITH THE DEPENDENCY FIELD...
        }
    }
\end{lstlisting}
%! language = TEXT
\begin{lstlisting}[language=Java, title={Usage}]
    ClassPathXmlApplicationContext context = new ClassPathXmlApplicationContext("configurationFile.xml");
    (*\tmnbf{di10java3interface}{WantedClassInterface}*) wantedClassInstance = context.getBean("(*\tmnbf{di10java3beanid}{wantedClass}[ForestGreen]*)", (*\tmnbf{di10java3interface2}{WantedClassInterface}*).class);
    wantedClassInstance.(*\tmnbf{di10java3usedep}{useDependency}*)();
\end{lstlisting}
\begin{tikzpicture}[remember picture, overlay]
    \drawarrow{di10xml1package}{[xshift=-6mm] di10java2package.north}
    \drawarrow{di10java1depclass}{[xshift=-3mm] di10java2depbeanid.north}[red]
    \drawarrow{di10java1depinterface.south}{[xshift=6mm] di10java2depinterface.north}[green]
    \drawarrow{di10java2class.south}{di10java2class2}[blue]
    \drawarrow{di10java2class.south}{[xshift=-6mm] di10java3beanid.north}[blue]
    \drawarrow{di10java2interface.south}{[xshift=3mm] di10java3interface.north}[Magenta]
    \drawarrow{di10java2interface.south}{di10java3interface2}[Magenta]
    \drawarrow{[xshift=-3mm] di10java2usedep.south}{di10java3usedep}[yellow][.5]
\end{tikzpicture}
\newpage

\newsubsection{Configuration with Java Code (No XML)}
Similarly as for~the~\hyperref[springinversionofcontrol]{inversion of~control} even for~the~\hyperref[springdependencyinjection]{dependency injection} the~configuration XML file can~be completely omitted and~replaced by a~special configuration class with the~\textit{Configuration} annotation.

When you use the~\textit{ComponentScan} annotation for~the~configuration class (see~the~example in~the~\hyperref[iocnoxml]{corresponding section of~the~inversion of~control}), it's~equivalent to~using the~XML configuration file with the~\textit{component-scan} element.
You~can~combine such configuration class with the~\hyperref[autowiring]{autowiring}.

You~can~also explicitly specify bean classes and~injection relationships among beans (equivalent of~explicit dependency injection configuration in~the~configuration XML file).
You~must instantiate the~dependency bean class in~a~method with the~\textit{Bean} annotation (as~by the~\hyperref[iocnoxml]{inversion of~control}) and~then use this method as~a~parameter of~the~main wanted bean ID class constructor or~setter method inside its \textit{Bean}--annotated method.
When you use the~setter approach, you~can't perform everything in~one step.
First, you must instantiate the~wanted class with the~zero--parameter constructor and~then call the~setter method on~the~instance.
Also, the~class instance is better not~to~be of~the~interface type, because interfaces generally don't contain setter declarations.

\warning It~may seem~OK (and~even working) to~omit the~\textit{Bean}--annotated method for~the~dependency bean class instantiation and~use directly the~constructor of~the~class when configuring the~injection.
However, don't do~that.
Beans behavior, scopes and~life cycles wouldn't work correctly.
Always create the~\textit{Bean}--annotated method for~the~dependency bean class instantiation and~call it when instantiating the~main wanted bean class.

\note Java configuration enables injecting literal values, and~opposite to~the~XML configuration ever with type check.
Simply use literal values as~arguments of~the~bean class constructor or~setter methods.
\newpage

\emten
\example[Java code configuration -- constructor dependency injection]
%! language = TEXT
\begin{lstlisting}[language=Java, title={Configuration class}]
    package confpackage.confsubpackage;

    @>@Configuration
    public class (*\tmnbf{di11java1configclass}{ConfigurationClass}*) {
        @>@Bean
        public (*\tmnbf{di11java1depinterface}{DependencyInterface}*) (*\tmnbf{di11java1depbeanid}{dependencyBeanMethod}*)() {
            return new (*\tmnbf{di11java1depclass}{DependencyClass}*)(...);
        }

        @>@Bean
        public (*\tmnbf{di11java1interface}{WantedClassInterface}*) (*\tmnbf{di11java1beanid}{wantedBeanMethod}*)() {
            return new (*\tmnbf{di11java1class}{WantedClass}*)((*\tmnbf{di11java1depbeanid2}{dependencyBeanMethod}*)());
        }
    }
\end{lstlisting}
%! language = TEXT
\begin{lstlisting}[language=Java, title={Dependency class}]
    package deppackage.depsubpackage;

    public class (*\tmnbf{di11java2depclass}{DependencyClass}*) implements (*\tmnbf{di11java2depinterface}{DependencyInterface}*) {
        ...
    }
\end{lstlisting}
%! language = TEXT
\begin{lstlisting}[language=Java, title={Wanted class with the constructor}]
    package somepackage.subpackage;

    public class (*\tmnbf{di11java3class}{WantedClass}*) implements (*\tmnbf{di11java3interface}{WantedClassInterface}*) {
        private (*\tmnbf{di11java3depinterface}{DependencyInterface}*) dependencyField;

        public (*\tmnbf{di11java3class2}{WantedClass}*)((*\tmnbf{di11java3depinterface2}{DependencyInterface}*) dependencyValue) {
            dependencyField = dependencyValue;
        }

        public WHATEVER (*\tmnbf{di11java3usedep}{useDependency}*)() {
            ...DO SOMETHING WITH THE DEPENDENCY FIELD...
        }
    }
\end{lstlisting}
%! language = TEXT
\begin{lstlisting}[language=Java, title={Usage}]
    AnnotationConfigApplicationContext context = new AnnotationConfigApplicationContext((*\tmnbf{di11java4configclass}{ConfigurationClass}*).class);
    (*\tmnbf{di11java4interface}{WantedClassInterface}*) wantedClassInstance = context.getBean("(*\tmnbf{di11java4beanid}{wantedBeanMethod}[ForestGreen]*)", (*\tmnbf{di11java4interface2}{WantedClassInterface}*).class);
    wantedClassInstance.(*\tmnbf{di11java4usedep}{useDependency}*)();
\end{lstlisting}
\begin{tikzpicture}[remember picture, overlay]
    \drawarrow{di11java1configclass}{[xshift=6mm] di11java4configclass.north}
    \drawarrow{[xshift=-6mm] di11java1depinterface.south}{di11java2depinterface}[red]
    \drawarrow{[xshift=-6mm] di11java1depinterface.south}{di11java3depinterface}[red]
    \drawarrow{[xshift=-6mm] di11java1depinterface.south}{[xshift=-6mm] di11java3depinterface2.north}[red]
    \drawarrow{di11java1depbeanid}{di11java1depbeanid2}[green]
    \drawarrow{[xshift=3mm] di11java1depclass.south}{[xshift=6mm] di11java2depclass.north}[blue]
    \drawarrow{[xshift=-3mm] di11java1interface.south}{di11java3interface}[yellow][.5]
    \drawarrow{[xshift=-3mm] di11java1interface.south}{di11java4interface}[yellow][.5]
    \drawarrow{[xshift=-3mm] di11java1interface.south}{di11java4interface2}[yellow][.5]
    \drawarrow{di11java1class}{di11java3class}[green]
    \drawarrow{di11java3class}{di11java3class2}[green]
    \drawarrow{[xshift=-3mm] di11java1beanid.south}{[xshift=3mm] di11java4beanid.north}[blue]
    \drawarrow{di11java3usedep}{di11java4usedep}[Magenta]
\end{tikzpicture}
\newpage

\newgeometry{top=15mm, left=25mm, right=25mm}
\emtwen
\example[Java code configuration -- setter dependency injection]
%! language = TEXT
\begin{lstlisting}[language=Java, title={Configuration class}]
    package confpackage.confsubpackage;

    @>@Configuration
    public class (*\tmnbf{di12java1configclass}{ConfigurationClass}*) {
        @>@Bean
        public (*\tmnbf{di12java1depinterface}{DependencyInterface}*) (*\tmnbf{di12java1depbeanid}{dependencyBeanMethod}*)() {
            return new (*\tmnbf{di12java1depclass}{DependencyClass}*)(...);
        }

        @>@Bean
        public (*\tmnbf{di12java1interface}{WantedClassInterface}*) (*\tmnbf{di12java1beanid}{wantedBeanMethod}*)() {
            (*\tmnbf{di12java1class}{WantedClass}*) wantedClassInstance = new (*\tmnbf{di12java1class2}{WantedClass}*)();
            wantedClassInstance.(*\tmnbf{di12java1valuefield}{setDependencyField}*)((*\tmnbf{di12java1depbeanid2}{dependencyBeanMethod}*)());
            return wantedClassInstance;
        }
    }
\end{lstlisting}
%! language = TEXT
\begin{lstlisting}[language=Java, title={Dependency class}]
    package deppackage.depsubpackage;

    public class (*\tmnbf{di12java2depclass}{DependencyClass}*) implements (*\tmnbf{di12java2depinterface}{DependencyInterface}*) {
        ...
    }
\end{lstlisting}
%! language = TEXT
\begin{lstlisting}[language=Java, title={Wanted class with the zero--parameter constructor and the setter method}]
    package somepackage.subpackage;

    public class (*\tmnbf{di12java3class}{WantedClass}*) implements (*\tmnbf{di12java3interface}{WantedClassInterface}*) {
        private (*\tmnbf{di12java3depinterface}{DependencyInterface}*) dependencyField;

        public (*\tmnbf{di12java3class2}{WantedClass}*)() {}

        public void (*\tmnbf{di12java3valuefield}{setDependencyField}*)((*\tmnbf{di12java3depinterface2}{DependencyInterface}*) dependencyField){
            dependencyField = dependencyValue;
        }

        public WHATEVER (*\tmnbf{di12java3usedep}{useDependency}*)() {
            ...DO SOMETHING WITH THE DEPENDENCY FIELD...
        }
    }
\end{lstlisting}
%! language = TEXT
\begin{lstlisting}[language=Java, title={Usage}]
    AnnotationConfigApplicationContext context = new AnnotationConfigApplicationContext((*\tmnbf{di12java4configclass}{ConfigurationClass}*).class);
    (*\tmnbf{di12java4interface}{WantedClassInterface}*) wantedClassInstance = context.getBean("(*\tmnbf{di12java4beanid}{wantedBeanMethod}[ForestGreen]*)", (*\tmnbf{di12java4interface2}{WantedClassInterface}*).class);
    wantedClassInstance.(*\tmnbf{di12java4usedep}{useDependency}*)();
\end{lstlisting}
\begin{tikzpicture}[remember picture, overlay]
    \drawarrow{di12java1configclass}{[xshift=3mm] di12java4configclass.north}
    \drawarrow{[xshift=-3mm] di12java1depinterface.south}{di12java2depinterface}[red]
    \drawarrow{[xshift=-3mm] di12java1depinterface.south}{di12java3depinterface}[red]
    \drawarrow{[xshift=-3mm] di12java1depinterface.south}{di12java3depinterface2}[red]
    \drawarrow{di12java1depbeanid}{di12java1depbeanid2}[YellowOrange]
    \drawarrow{di12java1depclass.south}{[xshift=3mm] di12java2depclass.north}[blue]
    \drawarrow{di12java1interface.south}{di12java3interface}[yellow][.5]
    \drawarrow{di12java1interface.south}{di12java4interface}[yellow][.5]
    \drawarrow{di12java1interface.south}{[xshift=-3mm] di12java4interface2.north}[yellow][.5]
    \drawarrow{di12java1class2}{di12java3class.north}[green]
    \drawarrow{di12java1class}{di12java3class.north}[green]
    \drawarrow{di12java3class}{di12java3class2}[green]
    \drawarrow{di12java1valuefield}{di12java3valuefield}[YellowOrange]
    \drawarrow{di12java1beanid.south}{[xshift=3mm] di12java4beanid.north}[blue]
    \drawarrow{di12java3usedep}{di12java4usedep}[Magenta]
\end{tikzpicture}
\restoregeometry
\newpage

\example[Java code configuration -- constructor field injection]
%! language = TEXT
\begin{lstlisting}[language=Java, title={Configuration class}]
    package confpackage.confsubpackage;

    @>@Configuration
    public class (*\tmnbf{di13java1configclass}{ConfigurationClass}*) {
        @>@Bean
        public (*\tmnbf{di13java1interface}{WantedClassInterface}*) (*\tmnbf{di13java1beanid}{wantedBeanMethod}*)() {
            return new (*\tmnbf{di13java1class}{WantedClass}*)("someValue");
        }
    }
\end{lstlisting}
%! language = TEXT
\begin{lstlisting}[language=Java, title={Wanted class with the constructor}]
    package somepackage.subpackage;

    public class (*\tmnbf{di13java2class}{WantedClass}*) implements (*\tmnbf{di13java2interface}{WantedClassInterface}*) {
        private String concreteValueField;

        public (*\tmnbf{di13java2class2}{WantedClass}*)(String concreteValue) {
            concreteValueField = concreteValue;
        }

        public WHATEVER (*\tmnbf{di13java2usevalue}{useConcreteValue}*)() {
            ...DO SOMETHING WITH THE CONCRETE VALUE FIELD...
        }
    }
\end{lstlisting}
%! language = TEXT
\begin{lstlisting}[language=Java, title={Usage}]
    AnnotationConfigApplicationContext context = new AnnotationConfigApplicationContext((*\tmnbf{di13java3configclass}{ConfigurationClass}*).class);
    (*\tmnbf{di13java3interface}{WantedClassInterface}*) wantedClassInstance = context.getBean("(*\tmnbf{di13java3beanid}{wantedBeanMethod}[ForestGreen]*)", (*\tmnbf{di13java3interface2}{WantedClassInterface}*).class);
    wantedClassInstance.(*\tmnbf{di13java3usevalue}{useConcreteValue}*)();
\end{lstlisting}
\begin{tikzpicture}[remember picture, overlay]
    \drawarrow{di13java1configclass}{[xshift=3mm] di13java3configclass.north}
    \drawarrow{[xshift=-6mm] di13java1interface.south}{di13java2interface}[red]
    \drawarrow{[xshift=-6mm] di13java1interface.south}{di13java3interface}[red]
    \drawarrow{[xshift=-6mm] di13java1interface.south}{[xshift=-3mm] di13java3interface2.north}[red]
    \drawarrow{di13java1beanid}{[xshift=3mm] di13java3beanid.north}[green]
    \drawarrow{di13java1class}{di13java2class}[blue]
    \drawarrow{di13java2class}{di13java2class2}[blue]
    \drawarrow{di13java2usevalue}{[xshift=-6mm] di13java3usevalue.north}[Magenta]
\end{tikzpicture}
\newpage

\example[Java code configuration -- setter field injection]
%! language = TEXT
\begin{lstlisting}[language=Java, title={Configuration class}]
    package confpackage.confsubpackage;

    @>@Configuration
    public class (*\tmnbf{di14java1configclass}{ConfigurationClass}*) {
        @>@Bean
        public (*\tmnbf{di14java1interface}{WantedClassInterface}*) (*\tmnbf{di14java1beanid}{wantedBeanMethod}*)() {
            (*\tmnbf{di14java1class}{WantedClass}*) wantedClassInstance = new (*\tmnbf{di14java1class2}{WantedClass}*)();
            wantedClassInstance.(*\tmnbf{di14java1valuefield}{setConcreteValueField}*)("someValue");
            return wantedClassInstance;
        }
    }
\end{lstlisting}
%! language = TEXT
\begin{lstlisting}[language=Java, title={Wanted class with the zero--parameter constructor and the setter method}]
    package somepackage.subpackage;

    public class (*\tmnbf{di14java2class}{WantedClass}*) implements (*\tmnbf{di14java2interface}{WantedClassInterface}*) {
        private String concreteValueField;

        public (*\tmnbf{di14java2class2}{WantedClass}*)() {}

        public void (*\tmnbf{di14java2valuefield}{setConcreteValueField}*)(String concreteValue) {
            concreteValueField = concreteValue;
        }

        public WHATEVER (*\tmnbf{di14java2usevalue}{useConcreteValue}*)() {
            ...DO SOMETHING WITH THE CONCRETE VALUE FIELD...
        }
    }
\end{lstlisting}
%! language = TEXT
\begin{lstlisting}[language=Java, title={Usage}]
    AnnotationConfigApplicationContext context = new AnnotationConfigApplicationContext((*\tmnbf{di14java3configclass}{ConfigurationClass}*).class);
    (*\tmnbf{di14java3interface}{WantedClassInterface}*) wantedClassInstance = context.getBean("(*\tmnbf{di14java3beanid}{wantedBeanMethod}[ForestGreen]*)", (*\tmnbf{di14java3interface2}{WantedClassInterface}*).class);
    wantedClassInstance.(*\tmnbf{di14java3usevalue}{useConcreteValue}*)();
\end{lstlisting}
\begin{tikzpicture}[remember picture, overlay]
    \drawarrow{di14java1configclass}{[xshift=3mm] di14java3configclass.north}
    \drawarrow{[xshift=-6mm] di14java1interface.south}{di14java2interface}[red]
    \drawarrow{[xshift=-6mm] di14java1interface.south}{di14java3interface}[red]
    \drawarrow{[xshift=-6mm] di14java1interface.south}{[xshift=-3mm] di14java3interface2.north}[red]
    \drawarrow{di14java1beanid}{[xshift=3mm] di14java3beanid.north}[green]
    \drawarrow{di14java1class}{di14java2class}[blue]
    \drawarrow{di14java2class}{di14java2class2}[blue]
    \drawarrow{[xshift=-3mm] di14java1valuefield.south}{di14java2valuefield}[yellow][.5]
    \drawarrow{di14java2usevalue}{[xshift=-6mm] di14java3usevalue.north}[Magenta]
\end{tikzpicture}
\newpage
