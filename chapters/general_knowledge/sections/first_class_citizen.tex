\newsection{First Class Citizen}
\index{First Class}
\label{firstclass}
It's~a~type of~an~entity in~a~programming language that~supports some general operations typically available for~entities in~that~language.
Those~general operations always depend on~a~concrete language, but~typically they~include assigning to~a~variable, passing to~a~\hyperref[subroutine]{subroutine} as~argument and~returning from~a~subroutine.
Sometimes, inserting to~a~\hyperref[datastructure]{data structure} and~equality testing capability is also mentioned.
Denotations \mbitqls{first;;;class;;;type}[,] \mbitqls{first;;;class;;;object}[,] \mbitqls{first;;;class;;;entity} and~\mbitqls{first;;;class;;;value} can~also occur.

For~example, numbers are~almost always first class citizens.
Arrays and~strings aren't first class citizens in~low level languages as~C or~Fortran.
And~mathematical operators aren't first class citizens even in~higher level languages like Java, C++ or~C\#.

In~some languages, like in~Kotlin or~JavaScript, their~\hyperref[subroutine]{subroutines} are~considered to~be first class citizens.
