\newsection{Package Manager}
\index{Package manager}
\index{Package management system}
\label{packagemanager}
A~package manager, or~package management system, is~a~software for~simplification and~automation of~computer programs installation, configuration, upgrading and~deletion.
Programs are distributed in~packages, which contain the~program software itself packaged in~an~archive file (zip,~tar,\dots) and~metadata like program name, description or~dependencies to~other programs.
Package managers are~usually connected with package repositories available over the~internet.

Package managers are typical for~\hyperref[linux]{Linux} \hyperref[os]{operating systems}, for~example Debian or~Ubuntu use the~popular~\hyperref[linuxapt]{APT}.
Oracle Linux has the~\href{http://yum.baseurl.org/}{Yum}.
But~there~are also some package managers available for~Windows, like \href{https://www.nuget.org/}{NuGet} or~\href{https://chocolatey.org/}{Chocolatey}.
There~is also the~\hyperref[powershellpackagemanagement]{PackageManagement} (formerly OneGet), which connects more Windows package managers to~one application.
Some people say that the~PackageManagement is a~package manager manager.

Not~only \hyperref[os]{operating systems} have package managers.
Some programming languages or~platforms also have their package managers.
For~example, Python has \href{https://pypi.org/}{PyPI} and~Node.js has~\hyperref[npm]{NPM}.
Gradle or~Maven with a~connection to~Maven repositories can~also be considered as~package managers (for~Java).
