\newsection{Multitasking}
\index{Multitasking}
\index{Context switch}
\index{Concurrency}
\index{Concurrent programming}
\index{Multiprogramming}
\label{multitasking}
It's~a~capability of~an~\hyperref[os]{operating system} to~process multiple tasks at~the~same time.
All~modern operating systems support multitasking.
For~example, in~Windows, you~can~have opened web browser downloading something, a~running music player and~many other stuff in~one~moment.
Plus there's always a~bunch of~hidden processes running in~the~background.
For~comparison, in~old \mbox{MS DOS} it~was~possible to~always run only one application.

The~parallel task processing is typically only illusional.
Individual processes are just very quickly swapped in~the~\hyperref[processorcpucore]{CPU}, creating an~illusion of~them running simultaneously.
The~process swipe is called \textit{context switch} and~the~whole illusional parallelism is called \textit{concurrency} or~\textit{concurrent programming}.

Nowadays, \hyperref[processorcpucore]{CPUs} contain multiple \hyperref[processorcpucore]{cores} and~some computers even contain multiple CPUs, which~makes the~parallel task processing really happening.
This real parallel processing is called \textit{multiprogramming}.
However, the~number of~available CPUs is almost never high enough to~cover all parallel processes, even multiprogramming computers perform swaps of~processes on~individual~CPUs.
\newline

\newsubsection{Preemption}
\index{Preemption}
\index{Preemptive}
\index{Non--preemptive}
\label{preemption}
Multitasking is \mbit{preemptive}, when~the~operating system solely decides which~process runs in~which~time, i.e,~which~process gets the~\hyperref[processorcpucore]{CPU}.
Processes can't~do anything about the~decision of~the~OS\@.
Nowadays, multitasking is mostly preemptive.

In~\mbit{non--preemptive} multitasking, it's a~responsibility of~a~running process to~stop in~a~short time interval and~give the~CPU control back to~the~OS\@.
The~OS then~decides about the~next process to~run.
Some mid--90s operating systems were working non--preemptively, for~example, \hyperref[32bvs64b]{32--bit} Windows~95 \mbox{or Mac OS} (that's something different \mbox{than today's} macOS).

In~general, preemptive multitasking is more stable.
When a~running process gets stuck in~preemptive multitasking, nothing serious happens, because the~OS will~take the~CPU from~it in~a~short time.
But~when a~running process gets stuck in~\mbox{non--preemptive} multitasking, it~never fulfills the~responsibility of~returning the~CPU control to~the~OS and~blocks the~whole~OS\@.
On~the~other hand, preemptive multitasking requires much~more complex operating systems and~is much~more expensive.
\newpage

\newsubsection{Multithreading}
\index{Multithreading}
\label{multithreading}
It's~a~technique of~writing individual programs so that processing of~multiple parts of~one program is~done in~parallel.
General multitasking is performed by~\hyperref[os]{OS} automatically, without~programmers somehow actively preparing their programs to~that.
In~the~past, when~a~programmer wanted to~use multitasking on~purpose, he~had to~write multiple programs and~make them communicate with~each other.
However, process communication is generally hard to~manage, therefore, mainstream programming languages enable to~separate one program, and~therefore its run as~a~process, to~multiple sections, denoted as~\mbit{threads}, which~can~be processed in~the~multitasking way when~the~whole program is~run.
Multithreading is preemptive in~general --~threads are managed by~\hyperref[os]{OS}.
\newline

\newsubsection{Concurrent VS Parallel VS Asynchronous}
\index{Parallel}
\index{Asynchronous}
\label{concurrentparallelasynchronous}
\begin{itemize}
    \itembf{Concurrent} was already described above.
            It means running multiple tasks at~the~same time.
            The~concurrent run is typically only illusional, the~illusion is achieved by very quick swaps of~tasks in~the~\hyperref[processorcpucore]{CPU}.
    \itembf{Parallel} is a~synonym for~\textit{concurrent}, its~meaning is the~same.
    \itembf{Asynchronous} means postponing execution of~actual task and~continuing processing following task.
            I.e.,~the~postponed code is~executed at~a~different position than~it's~written.
            It's~something like cutting~out a~task from~the~running thread.
            The~postponed task can~be run concurrently (in~another thread) or~wait until the~original thread picks it up again.
\end{itemize}

\note Although asynchronous behavior is~possible to~reach even in~a~single--threaded environment with~absolutely no~parallelism, asynchronicity is still considered to~be~multitasking.
\newline
