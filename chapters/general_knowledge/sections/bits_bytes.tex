\newsection{Bits and Bytes}
\index{Bit}
\index{Byte}
\label{bitsbytes}
A~\textbf{bit}, denoted~by lowercase letter~\itq{b}, is the~smallest unit of~computer \hyperref[systemmemory]{memory} or~\hyperref[harddiskdrive]{hard disk} capable of storing an~information.
One~bit can~hold only two values -- \textit{0} or~\textit{1}, also denoted as~\textit{false} or~\textit{true}.

A~\textbf{byte}, denoted~by capital letter~\itq{B}, is a~group of~eight bits in~computer \hyperref[systemmemory]{memory} or~\hyperref[harddiskdrive]{hard disk} with a~unique address.
That means the~computer can~work (read/write) with those eight bits only together.
It~can't use each bit separately.
One~byte can~hold 256 different values (8~bits $\Rightarrow$ 2~values in~8~positions $\Rightarrow$ 2\textsuperscript{8} = 256~combinations).

No~matter how small information is needed to~be~stored to~memory or~hard disk, it~always consumes at~least one byte.
Even if it's a~simple \mbox{1--bit} \mbit{true/false} value.
In~order to~achieve faster memory/hard drive access, some computers can~even group bytes.
In~such computers, even the~simplest \mbit{true/false} value consumes multiple bytes.

\enlargethispage{20mm}
\thispagestyle{empty}
\newsubsection{Hexadecimal Representation of Byte Contents}
A~byte always holds a~number ranging from~0 to~255.
Telling that the~number represents something else, for~example, a~character, is~a~work of~\hyperref[datatypes]{data types}.

Writing byte content as~a~number in~decimal system is quite unusual (it~would~be too~easy).
More often you~can~encounter the~binary representation (see~above), but~that consumes a~lot of~space.
The~most widely used byte contents representation is as~a~hexadecimal number, and~always with two digits.
For~example, the~value \itq{11} is~written as~\mbitq{00001011} in~the~binary representation and~as~\mbitq{0B} in~the~hexadecimal representation.
When writing multiple bytes, spaces should~be used between individual pairs of~digits (e.g.,~\mbitq{0B AD 13}), but~that isn't always the~case.

The~purpose of~the~hexadecimal representation is its maximal usefulness.
For~one byte only two character spaces are~always used and~it~uses all digits of~the~hexadecimal system.
The~highest number consisting~of two digits in~the~hexadecimal system is~\itq{FF}, i.e.,~255 in~the~decimal system, i.e.,~the~highest possible value in~one~byte.
\newpage

\newsubsection{Multiples of~Bytes}
\index{Kilobyte}
\index{Megabyte}
\index{Gigabyte}
\index{Kibibyte}
\index{Mebibyte}
\index{Gibibyte}
\warningnonl Higher numbers of~bytes are~denoted by standard decimal prefixes \mbitq{kilo}, \mbitq{mega}, \mbitq{giga}~etc.
However, when talking about bytes, these prefixes usually maliciously don't denote multiplication by a~power of~10 to~3, 6, 9 etc., but~2 to~10, 20, 30~etc.

For~example, kilobyte doesn't mean 1,000 bytes (10\textsuperscript{3}), but~1,024 bytes (2\textsuperscript{10}).
Similarly, megabyte means 1,048,576 bytes (2\textsuperscript{20}), gigabyte means 1,073,741,824 bytes (2\textsuperscript{30})~etc.

This is wrong in~pure terminology.
\textit{Kilo} means one thousand, that's a~fact, similarly for~those other prefixes.
There's even a~correct terminology for~powers of~2 -- kibibyte (2\textsuperscript{10}), mebityte (2\textsuperscript{20}), gibibyte(2\textsuperscript{30}) etc.
When writing abbreviations, powers of~2 should~be denoted with the~first letter capital (KB,~MB,~GB) to~avoid confusion with powers of~10, which are generally written with first letter lowercase (kB,~mB,~gB).
But~not~many people know this terminology, not~even in~IT world, and~actually nobody uses~it.

When talking about this topic, always consider who~you're talking~to.
You~can~encounter an~IT~"expert", who~won't talk to~you again if~you~tell him that kilobyte has 1,000 bytes.
And~as~these "experts" otherwise tend to~be good programmers, they usually have high posts.
If~they don't talk to~you, you~can~have a~problem.

There's a~joke that a~normal human thinks that a~kilobyte has~1,000 bytes, and~a~programmer thinks that a~kilometer has~1,024 meters.
