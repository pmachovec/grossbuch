\newsection{Parameter VS Argument}
\index{Argument}
\index{Attribute}
\index{Parameter}
\label{parameterargument}
\begin{itemize}
    \itembf{Parameter} is a~\hyperref[variablefieldproperty]{variable} in~a~method \hyperref[declarationdefinition]{declaration}.
    \itembf{Argument} is a~concrete value of~a~parameter.
\end{itemize}
For~example consider a~method \mbitq{doSomething(String someVariable)}.
The~\mbit{someVariable} is~a~parameter.
When you call \mbitq{doSomething("someValue")}, then \mbit{someValue} is an~argument.

\warning Don't get confused by the~term \textit{attribute}.
That's either a~different name for~a~\hyperref[variablefieldproperty]{field} of~an~object or~for~a~\hyperref[relationaldatabase]{relational database} table column.
