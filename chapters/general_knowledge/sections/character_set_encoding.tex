\newsection{Character Set VS Character Encoding}
\index{Character set}
\index{Character encoding}
\index{Encoding scheme}
\index{Code point}
\label{charactersetencoding}
\begin{itemize}
    \itembf{Character set} is a~list of~characters, with which a~computer is~capable to~work.
            Each character in~the~list has~assigned a~number, so~called \textit{code point}.
            Various computers and~software support various character sets.
            \mbox{ASCII and Unicode} are~typical examples of~character~sets.
    \itembf{Character encoding}, also called \itq{encoding scheme} or~simply \itq{encoding}, is~a~way how code points (numbers) from a~character set are~stored in~\hyperref[bitsbytes]{bytes} -- it~describes how~characters are~\hyperref[encoding]{encoded} to~bytes.
            Old~character sets like ASCII have one encoding with the~same name (i.e.,~\mbitq{ASCII} is also name for~encoding).
            Newer character sets like Unicode have multiple encodings (see~\hyperref[utf]{further}).
\end{itemize}
