\newsection{Coroutines}
\index{Coroutine}
\index{Coprogram}
\label{coroutines}
A~\mbit{coroutine}, or~also a~\mbit{coprogram}[,] is~a~sequence of~tasks, which can~be run \hyperref[concurrentparallelasynchronous]{asynchronously} to~the~other code (the~whole sequence, individual tasks must~be executed one after another).
The~order of~execution of~tasks is guaranteed.

Coroutines provide a~\hyperref[preemption]{non--preemptive} multitasking environment as~an~alternative to~classic \hyperref[multithreading]{threads}.
On~each task start, the~coroutine execution can~be suspended and~taken over by another thread, or~even later picked up by the~same thread.
This~way, the~potentially time and~resource consuming computation of~the~coroutine (coroutines are supposed to~be used for~such computations) doesn't block the~main thread.
The~decision which~thread to~use must be done in~the~source code when~the~coroutine logic \mbox{is written.}

If~a~language has~a~good and~safe support for~coroutines (like, for~example, \hyperref[kotlincoroutine]{Kotlin's coroutines}), it's better to~use coroutines instead of threads as~it's much cheaper.
Threads consume \hyperref[systemmemory]{memory} and~their creation is \hyperref[processorcpucore]{CPU}--demanding.
That isn't the~case for~coroutines.
