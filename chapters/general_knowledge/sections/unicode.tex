\newsection{Unicode}
\index{Unicode}
\label{unicode}
It's~a~\hyperref[charactersetencoding]{character set} containing all known characters of~all~known languages, including such~specialities like coloured pictographs from chatting applications, Egyptian hieroglyphs, Klingon language or~Tengwar (Elve script).
Unicode code points are written as~numbers in~the~hexadecimal systems prefixed with~\mbitq{U+} or~with~\mbitq{\textbackslash{}u}, usually with four digits.
Higher code points for~very special characters can~have five and~theoretically even six digits.
For~example, code point for~lowercase letter~\mbitq{a} is \mbitq{U+0061} \mbox{(\itq{\textbackslash{}u0061}).}

Unicode is based on~the~old \hyperref[ascii]{ASCII} character set.
Because of~that, first 128 Unicode code points are identical to~those in~ASCII\@.
