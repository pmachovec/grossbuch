\newsection{Protocol VS Standard}
\index{Protocol}
\index{Standard}
\index{Communication protocol}
\index{Cryptographic protocol}
\index{Secured protocol}
\index{Standardised protocol}
\label{protocolstandard}
Both are~systems of~rules that developers should follow to~provide mutual compatibility and~interoperability of~their programs in~some ways depending on the~protocol or~standard.
The~difference is that protocols describe communication over a~network.
That's why the~term \itq{communication protocol} is~used sometimes.
There are no other protocols than communication.
You~can sometimes meet also the~term \itq{cryptographic protocol} or~\itq{secured protocol}, but~that denotes a~communication protocol securing the~communication by~incorporating some \hyperref[cryptography]{cryptography} techniques.

Standards have much broader range, they can~describe any software or~even hardware.
For~example, USB is a~standard that computers and external devices follow to~be able to~cooperate.

\warning It's~clear that not~all standards are protocols, but~also not~all protocols are standards.
They mostly are, like~\hyperref[http]{HTTP} or~\hyperref[ssh]{SSH}, but~anyone can~create own protocol not~bound by~any~rules.
Bigger companies do that for~their internal communication.
A~protocol, which~is also a~standard, is~called \itq{standardised protocol}.
That~holds, for~example, for~\hyperref[networkprotocols]{general network protocols}
