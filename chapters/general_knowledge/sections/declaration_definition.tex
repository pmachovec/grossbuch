\newsection{Declaration VS Definition}
\index{Declaration}
\index{Definition}
\label{declarationdefinition}
\begin{itemize}
    \itembf{Declaration of~a~variable} means specifying the~variable name, \hyperref[datatypes]{type} and~eventually \hyperref[javaaccessmodifiers]{access level}.
            It~doesn't include assigning a~value to~the~variable.
            Example: \mbitq{private int a;}.
    \itembf{Declaration of~a~method (or~a~function)} is the~method header.
            It~consists of~the~method name, return value \hyperref[datatypes]{type}, number, names and~\hyperref[datatypes]{types} of~arguments of~the~method and~eventually its \hyperref[javaaccessmodifiers]{access level}.
            Example: \mbitq{public void doSomething(int a);}.
    \itembf{Definition of~a~variable} means assigning a~value to~the~variable.
            It~can~immediately follow the~variable definition (\itq{private int a = 1;}) or~appear later as~a~separate command (\itq{private int a; \dots a = 1;}).
    \itembf{Definition of~a~method (or~a~function)} is the~method header with the~method body.
            Example: \mbitq{public void doSomething(int a) \{ \dots \}}".
\end{itemize}
\newpage

\noindent Generally a~declaration informs \hyperref[compiledinterpretedlanguages]{a~compiler or~an~interpreter} about some structure.
Definition allocates a~memory for~the~structure.
There can~be a~declaration without a~definition, but~not~reversed.
One~variable or~method (or~function) can~be declared multiple times, but~defined maximally once.
When you assign a~new value to~already defined variable, you~rewrite the~already allocated memory.
That isn't a~new definition.
