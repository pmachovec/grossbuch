\newsection{Data Types}
\index{Data type}
\label{datatypes}
Each~variable in~fact denotes a~sequence of~zeros and~ones in~the~\hyperref[systemmemory]{memory}, i.e.,~values of~memory \hyperref[bitsbytes]{bits} (grouped to~\hyperref[bitsbytes]{bytes}).
A~data type, usually written exactly before the~variable in~the~code, is~an~information for~\hyperref[compiledinterpretedlanguages]{a~compiler or~an~interpreter} how to~interpret that sequence, i.e.,~what value that sequence denotes.

Each~programming language can~theoretically interpret the~same value differently, but~basics like numbers corresponding to~characters are~usually the~same.
For~example consider the~value 01100001.
When treated as~an~integer number, it~represents the~value~97 (in~the~two's complement).
When treated as~a~character, it~represents lowercase letter~\itq{a} \mbox{((97)\textsubscript{10}\,=\,(61)\textsubscript{16})}.
