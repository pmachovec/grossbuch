\newsection{Bytecode}
\index{Bytecode}
\label{bytecode}
Bytecode is the result of compilation of Java code.
It's stored in \textit{.class} files.
For~people it's an unreadable mess (it's a type of \hyperref[bytecodebinarycode]{binary code}), however it isn't the machine code which a processor can read.
It must be read and further transformed to~the~machine code by JVM, i.e.,~it's a~middle step between written Java code and its execution.
The reason of this approach is that bytecode is transferable between platforms (that's why sometimes it's~called \textit{portable code})\,--\,no~matter on what \hyperref[platform]{platform} (platform\,=\,OS\,+\,hardware in~this~case, hardware\,=\,\hyperref[32bvs64b]{32b~VS~64b}) the original Java code was~written and~compiled, the~resulting bytecode will be executable on~any~other platform.
The~thing which is different for~each platform is~\hyperref[jdkjrejvm]{JRE}.

For~comparison consider C++.
In~this programming language the~result of~compilation is directly the~machine code, which must be different for~each platform, i.e.,~the~thing in C++ which is different for~each platform is the~compiler.

By~the~way the approach with middle--step code is used also by~C\#, which is compiled to~so~called CIL (\textit{Common Intermediate Language}).
The~main difference of~CIL from bytecode is that CIL is human--readable, although reading it is quite complicated (and~useless).

\newsubsection{Bytecode VS Binary Code}
\index{Binary code}
\label{bytecodebinarycode}
Binary code is generally any code in a~form readable for a~computer.
It can~be machine code read by processors, bytecode read by JVMs, stored database data read by database software etc.
The typical sign of binary code is that it can't~be read by humans.
By this definition the~C\#'s~CIL mentioned earlier is not a~binary code.

When working with Java, you can meet the~term \textit{Java~binaries}.
This refers to \textit{.class} files with bytecode, because the bytecode is~a~form of~binary code.
\newpage
