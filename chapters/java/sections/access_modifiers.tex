\newsection{Access Modifiers}
\index{Access modifier}
\label{javaaccessmodifiers}

\newsubsection{Public}
\index{Public}
\label{javapublic}

\newsubsection{Protected}
\index{Protected}
\label{javaprotected}

\newsubsection{Private}
\index{Private}
\label{javaprivate}

\newsubsection{Static}
\index{Static}
\label{javastatic}
Only \hyperref[variablefieldproperty]{fields} and~methods can~be static.
Such~members, if~not~blocked by~other modifiers, can~be accessed over the~dot notation without a~need of~instantiating the~class containing them.

\example[class with a~static method]
%! language = TEXT
\begin{lstlisting}[language=Java]
    public class ClassWithStaticMethod {
        public static String giveSomeString() {
            ...
        }
    }

    String someString = ClassWithStaticMethod.giveSomeString();
\end{lstlisting}

\warning A~special hidden instance of~the~class is~created for~static members.
This instance is~a~\hyperref[singletondp]{singleton}.
And~this can~be dangerous as~even non--static methods can~access static members.
A~static \hyperref[variablefieldproperty]{field} can~be overridden by~accident and~introduce hardly discoverable bugs.

\example[singleton behavior of~static members]
%! language = TEXT
\begin{lstlisting}[language=Java]
    public class ClassWithStaticMembers {
        private String simpleString;

        public static String getSimpleString() {
            return simpleString;
        }

        public static void setSimpleString(String simpleString) {
            this.simpleString = simpleString;
        }
    }

    ClassWithStaticMembers firstInstance = new ClassWithStaticMembers();
    firsInstance.setSimpleString("FIRST VALUE");
    ...MANY LINES OF CODE...
    ClassWithStaticMembers secondInstance = new ClassWithStaticMembers();
    secondInstance.setSimpleString("SECOND VALUE"); // Writes to the same instance
    ...MANY LINES OF CODE..
    System.out.println(firstInstance.getSimpleString()); // Will print "SECOND VALUE"
\end{lstlisting}

\newsubsection{Final}
\index{Final}
\label{javafinal}

\newsubsection{Synchronized}
\index{Synchronized}
\label{javasynchronized}

\newsubsection{None}
\label{noaccessmodifier}
