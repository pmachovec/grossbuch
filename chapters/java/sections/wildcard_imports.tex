\newsection{Imports with Wildcards}
\index{Import}
\index{Wildcard}
Some people say that when you import two or~more classes from one package, or~two or~more methods from one~class, you~should use the~wildcard import (i.e.,~something like \textit{import~package.*}).
It's~even written in~the~book \textit{Clean Code} by~Robert C.~Martin.
However, there are many objections to~this approach.

The~main (and~actually the~only) argument favouring imports with wildcards is that they make the~code shorter, therefore, more readable.
This~is somehow true, but~today's IDEs, even those for~free like Eclipse or~NetBeans, can~collapse import sections automatically.
If~someone is~writing code in~some editor not~capable of~this, then he's either a~beginner not~ready for~a~real IDE and~long imports, or~he's an~idiot.
Also, when a~class has~too~many imports, it's~too~dependent to~other classes, therefore, too~big and~more susceptible to~errors (if~a~dependant class breaks, my~class breaks,~too), and~therefore,~bad.
Furthermore, when~a~wildcard import is~used, the~class generally has~more imports than with explicit imports, and~the~previous problem with too~many imports is~even more serious.

Then there are only arguments against wildcard imports.
And~one of~them~is, funny enough, the~readability.
With wildcard imports it's~harder to~tell where an~imported feature comes from.
Consider the~following code:
%! language = TEXT
\begin{lstlisting}[language=Java]
    import firstPackage.*;
    import secondPackage.*;
    import thirdPackage.*;

    public class ExampleClass {
        private MysteryClass mysteryInstance = new MysteryClass();
    }
\end{lstlisting}

\noindent It~isn't clear what package the~\textit{MysteryClass} class comes from.
Consider it's~somehow broken and~compilation fails, and~\textit{Ctrl}\,+\,click also~doesn't work (let's~say the~package library is~completely missing).
And~now consider this code:
%! language = TEXT
\begin{lstlisting}[language=Java]
    import firstPackage.SomethingUseless;
    import secondPackage.SomethingEvenMoreUseless;
    import thirdPackage.MysteryClass;

    public class ExampleClass {
        private MysteryClass mysteryInstance = new MysteryClass();
    }
\end{lstlisting}

\noindent Here it's~clear that the~\textit{MysteryClass} class comes from the~third package.
So,~when observing problems, it's~clear where to~look.

Now~consider that the~previous code with wildcard imports works, i.e.,~there~is a~class called \textit{MysteryClass} in~the~third package.
And~now the~author of~the~second package, which lives on~the~other side of~the~world and~you don't know each~other, gets the~amazing idea to~create a~class called \textit{MysteryClass} in~the~second package.
Suddenly, there's an~ambiguity in~your code, compilation stops working and~you're screwed.
But~with explicit imports the~code still works without a~need of~change.

\note The~most favoured linter Ktlint for~\hyperref[kotlin]{Kotlin} language, which slowly starts to~push away Java, forbids wildcard imports in~the~default configuration.
