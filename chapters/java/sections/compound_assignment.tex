\newsection{Compound Assignment}
\index{Compound assignment}
\index{Augmented assignment}
Compound assignment operators, also called \textit{augmented assignment operators}, are shortened assignments with arithmetic operators.
All~arithmetic operators are~possible to~use in~this~way, \mbitqs[i.e.,]{+=}[,] \mbitqs{-=}[,] \mbitqs{$\star$=}[,] \mbitqs{/=} \mbitqs[and]{\%=}[.]
You~must~be extra careful when dealing with these operators because of the~computation order.
First, the~right side is~computed.
Then it's applied to~the~left side with the~used arithmetic operator.
And~that can~cause unintuitive results.
For~example consider the~following code:
%! language = TEXT
\begin{lstlisting}[language=Java, frame=no]
    int x = 5;
    x = x - 1 + 2 - 3; @>// 3
\end{lstlisting}

\noindent The~variable~$x$ is~evaluated to~$3$, because $5-1+2-3=3$.
And~now consider an~"equivalent" code with the~compound assignment subtraction:
%! language = TEXT
\begin{lstlisting}[language=Java, frame=no]
    int x = 5;
    x -= 1 + 2 - 3; @>// 5
\end{lstlisting}

\noindent The~variable~$x$ is~evaluated to~$5$, because $1+2-3=0$ (right side) and~$5-0=5$ (left side).
To~avoid falling to~this trap always treat the~right side to~be calculated first.
