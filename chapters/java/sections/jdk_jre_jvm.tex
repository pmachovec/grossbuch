\newsection{JDK VS JRE VS JVM}
\index{JDK}
\index{Java Development Kit}
\index{JRE}
\index{Java Runtime Environment}
\index{JVM}
\index{Java Virtual Machine}
\label{jdkjrejvm}
\begin{itemize}
    \itembf{JDK} stands for~\textit{Java Development Kit}.
            It's a set of tools for development and running Java applications.
    \itembf{JRE} stands for~\textit{Java Runtime Environment}.
            It's~the~most important part of~JDK, which serves for running already compiled code, i.e.,~it's the software transforming \hyperref[javabytecode]{bytecode} to machine code.
    \itembf{JVM} stands for~\textit{Java Virtual Machine}.
            It's~the~most important part of JRE, which actually runs the code (transfers \hyperref[javabytecode]{bytecode} to machine code), however it can't work alone -- it requires additional tools provided~by~JRE\@.
\end{itemize}

\noindent JDK,~JRE and~JVM are platform--dependent, i.e.,~you~need these things different for~Windows, different for~Linux etc.
The~thing which is~portable is~only the~\hyperref[javabytecode]{bytecode}.
When you only want to run Java programs, it's~OK to~install only~JRE, which is~smaller.
However, when you intend to develop Java applications, you of~course need to install~JDK\@.
be aware that when installing JDK (from Oracle), it~also installs a~separate JRE and computers tend to use this JRE by~default.
You should completely ignore this separate JRE\@.
To~do~it you~must set the~environment variable \textit{JAVA\_HOME} to~the~folder of~JDK (not~its~subfolder \textit{bin}, that's added to~the~\textit{Path} variable).

\newsubsection{Typical parts of JDK, that are not included in JRE}
\begin{itemize}
    \item Compiler (the \textit{javac} command)
    \item JavaDoc generator (the \textit{javadoc} command)
    \item Debugger (the \textit{jdb} command)
\end{itemize}

\newsubsection{Typical parts of JRE, that are not included in JVM}
\begin{itemize}
    \item Native \textit{.class} files, i.e.,~binaries of~all~default Java classes (ArrayList, Runtime, Thread,~\dots)
    \item Java execution command (the \textit{java} command)
    \item \hyperref[classloaders]{Classloader}
\end{itemize}
\newpage
