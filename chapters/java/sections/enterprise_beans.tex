\newsection{Enterprise Java Beans (EJB)}
\index{Enterprise Java Beans}
\index{EJB}
\label{ejb}
An~Enterprise Java Bean is a~server--side run component performing a~specific task.
EJBs~form huge \hyperref[distributedsystem]{distributed} server--side systems handling security, transaction processing, persistence etc.
One~EJB can~consist of~more classes and~interfaces (one~EJB\,=\,more files).
The~code uses special \hyperref[annotations]{annotations} provided by~Java~EE libraries.
EJBs must be deployed on a~specialized \hyperref[applicationserver]{application servers} like JBoss or~Glassfish.
EJB~specification was extensively enhanced in~last years, but~it still is quite a~challenging topic.

In~the~past there were no~annotations.
The~specification was~given by~interfaces.
This lead to~a~tight coupling between Java~EE libraries and~the~written code, lots of~useless methods enforced by~interfaces, hard implementation, hard maintenance and~overall mess in~the~code.
Furthermore, EJB systems were extremely slow.
That's why the~\hyperref[springframework]{Spring framework} was~introduced by~the~developers community.
Although there was the~mentioned extensive enhancement of~EJB specification, it~still caries the~stigma of~unusable crap and~the~\hyperref[springframework]{Spring framework} remains much more popular.

\warning EJB classes are \textbf{not} \hyperref[javabeans]{Java beans}.
However, with respect to~Java~EE, they're \hyperref[pojo]{POJOs} -- they use only Java~EE default annotations.
In~the~dark past they weren't even that.

\warning The~most popular \hyperref[applicationserver]{application server} -- \hyperref[tomcat]{Tomcat} -- doesn't support EJBs.
