\newsection{RequireJS}
\index{RequireJS}
\label{requirejs}
\href{https://requirejs.org}{RequireJS} is widely used implementation of~\hyperref[amd]{AMD}.
Modules syntax exactly follows the~\hyperref[amdsyntax]{AMD standard}.
Using modules from a~web page, however, is~specific.

You~should have one main file, which is referred from a~\mbit{script} element inside~a~page \mbit{header} element.
From this file you~should assign functionality to~elements in~the~corresponding web page.
And~from this file you can~also call \hyperref[amd]{AMD} modules.
If~the~page design forces you to~have the~\mbit{script} element somewhere else than in~the~\mbit{header} element or~to~have multiple \mbit{script} elements, then the~design is~bad and~should~be changed.

The~main file looks almost the~same as~a~standard \hyperref[amd]{AMD} module, only the~big enclosing function is~called \mbitq{requirejs}.
It~also doesn't have the~\mbit{return} statement.
It~can~contain RequireJS configuration (before the~main \mbit{requirejs} function), which usually specifies aliases for~paths to~third party libraries like \hyperref[jquery]{jQuery}.
The~code inside the~\mbit{requirejs} function is~executed like in~a~standard JavaScript file from the~top to~the~bottom.

Referring the~main file from a~page is~slightly different from \hyperref[javascriptmodularity]{standard JavaScript module referral}.
The~\mbit{src} attribute of~the~\mbit{script} element must contain path to~the~RequireJS library file (called \mbitq{require.js} by~default).
The~main script path is~specified in~the~\mbitq{data-main} element, and~it~can't contain the~\mbitq{.js} extension of~the~file.
The~extension is~expected, if~the~file doesn't have it, it~won't~work.

\note RequireJS is~usable even in~\hyperref[nodejs]{Node.js}.
Node.js already contains \mbox{the \hyperref[nodejscommonjs]{CommonJS}} technique for~modularization, but~the~performance is generally worse than with RequireJS\@.
Furthermore, when using RequireJS, you~can~develop a~functionality on~the~Node.js platform (server side), but~later use~it on~the~client side and~preserve the~effectiveness.

\warning You~can~encounter a~JavaScript file loading modules with a~keyword \mbitq{require}.
This is a \hyperref[nodejs]{NodeJS} feature called \hyperref[nodejscommonjs]{\mbit{CommonJS}}.
It~isn't supported in~JavaScript by~default.

However, to~make things even more confusing, RequireJS supports using the~\mbitq{require} keyword as~a~replacement of~the~standard \mbitq{requirejs} keyword, if~the~executing \hyperref[engine]{engine} doesn't use it on~its~own.
So~for~example, when~running RequireJS code in~a~web browser, you~can~use \mbitq{require} instead of \mbitq{requirejs}.
But~when~running the~same code in~\hyperref[nodejs]{NodeJS}, it~will fail.
Therefore, it's~advisable to~use the~default \itq{requirejs} keyword every time when working with~RequireJS.
\newpage

\example[assigning action to~a~button with RequireJS and jQuery]
%! language = TEXT
\begin{lstlisting}[language=JavaScript, title={Main JavaScript file \textit{\tmnbf{js3mainscript}{main}.js}}]
    requirejs.config({
      paths: {
        ...
        (*\tmnbf{js3mainjquery1}{jquery}*): 'SOME_PATH/jquery-X.Y.Z.js'
        ...
      }
    });

    requirejs(['(*\tmnbf{js3mainjquery2}{jquery}[ForestGreen]*)', 'SOME_PATH/amd_module'],
    function((*\$*), amdModuleHandler) {
      (*\$*)('#(*\tmnbf{js3mainbutton}{buttonId}[ForestGreen]*)').click(function() {
        ...
        amdModuleHandler.amdModuleFunction();
        ...
      });
    });
\end{lstlisting}
%! language = TEXT
\begin{lstlisting}[language=XML, title={Web page file loading the JavaScript file}]
    <!DOCTYPE html>
    <html lang="en">
      <head>
        ...
        <script data-main="SOME_PATH/(*\tmnbf{js3htmlscript}{main}[ForestGreen]*)" src="SOME_OTHER_PATH/require.js"></script>
        ...
      </head>
      <body>
        ...
        <button id="(*\tmnbf{js3htmlbutton}{buttonId}[ForestGreen]*)">BUTTON_LABEL</button>
        ...
      </body>
    </html>
\end{lstlisting}
\begin{tikzpicture}[remember picture, overlay]
    \drawarrow{js3mainscript.south}{js3htmlscript.north}
    \drawarrow{js3mainjquery1.south}{js3mainjquery2.north}[red]
    \drawarrow{js3mainbutton.south}{js3htmlbutton.north}[green]
\end{tikzpicture}
\newpage
