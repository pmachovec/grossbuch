\newsection{ECMAScript}
\label{ecmascript}
It's~a~\hyperref[protocolstandard]{standard} for~\hyperref[scriptinglanguages]{scripting languages}.
The~\mbitq{ECMA} stands for~\itq{European Computer Manufacturers Association}.
This is the~organization for~standards that created the~ECMAScript standard.

ECMAScript defines a~large number of~keywords and~features that a~\hyperref[scriptinglanguages]{scripting language} must provide.
JavaScript is~the~most well--known language following the~ECMAScript standard, but~there are others like ActionScript or~JScript.
Actually, there's no single JavaScript.
Each~\hyperref[internetweb]{web} browser or~\hyperref[compiledinterpretedlanguages]{interpeter} can~process the~JavaScript code in~its own way in~the~background.
But~because all~these implementations follow the~ECMAScript standard, the~syntax is~always the~same.

\warning TypeScript doesn't follow the~ECMAScript \hyperref[protocolstandard]{standard}, it's~only an~extension of~the~standard.
ECMAScript doesn't define data types used in~TypeScript.
In~other words, each~ECMAScript code is~a~valid TypeScript code, but~not~each TypeScript code is a~valid ECMAScript code.
