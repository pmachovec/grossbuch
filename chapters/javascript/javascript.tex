\newchapter{JavaScript}
\index{JavaScript}
\label{javascripts}

\newsection{ECMAScript}
\label{ecmascript}
It's~a~\hyperref[protocolstandard]{standard} for~\hyperref[scriptinglanguages]{scripting languages}.
The~\mbitq{ECMA} stands for~\itq{European Computer Manufacturers Association}.
This is the~organization for~standards that created the~ECMAScript standard.

ECMAScript defines a~large number of~keywords and~features that a~\hyperref[scriptinglanguages]{scripting language} must provide.
JavaScript is~the~most well--known language following the~ECMAScript standard, but~there are others like ActionScript or~JScript.
Actually, there's no single JavaScript.
Each~\hyperref[internetweb]{web} browser or~\hyperref[compiledinterpretedlanguages]{interpeter} can~process the~JavaScript code in~its own way in~the~background.
But~because all~these implementations follow the~ECMAScript standard, the~syntax is~always the~same.

\warning TypeScript doesn't follow the~ECMAScript \hyperref[protocolstandard]{standard}, it's~only an~extension of~the~standard.
ECMAScript doesn't define data types used in~TypeScript.
In~other words, each~ECMAScript code is~a~valid TypeScript code, but~not~each TypeScript code is a~valid ECMAScript code.

\newsection{Var VS Let VS Const}
\index{Const}
\index{Let}
\index{Var}
These are JavaScript key words starting a~variable \hyperref[declarationdefinition]{declaration}.
They define the~variable scope visibility and~changeability.

\begin{itemize}
    \item \mbitq{Var}~variables are~visible in the nearest function block, but~\mbitq{let} and~\mbitq{const} variables are visible in the nearest enclosing block, which can~be smaller than a function block.
          Accessing a~\mbitq{let} variable outside its enclosing block causes a~\mbit{ReferenceError} and~all the~following code isn't reached.
    \item \mbitq{Var}~variables can~be redeclared in~the~same (function) block, i.e.~you~can declare \mbitqls{var a =;;;...;} more times in~the~same (function) block.
          \mbitq{Let}~variables can~be only rewritten in~the~same (enclosing) block.
          \mbitq{Const}~variables can't~be even rewritten.
    \item \mbitq{Var}~variables can~be used even before their declaration, but~they're always \mbit{undefined}.
          Assigning a~value to~a~\mbitq{var}~variable before its declaration results in~an~error.
    \item Variables of~all~three types are~accessible from anywhere when declared in~the~executable code outside any~block.
          However, only \mbitq{var}~variables are~available in~the~\mbit{window} object.
\end{itemize}

\newsection{Modularity}
\label{javascriptmodularity}
JavaScript was initially created as~a~\hyperref[scriptinglanguages]{scripting language} for~writing only small scripts incorporated to~ordinary \hyperref[internetweb]{web} pages.
It~wasn't designed for~writing large code structures separated into multiple files.
However, with the~\hyperref[internetweb]{web} evolution a~need for~complex JavaScript implementations actually came onto scene.
And~the~JavaScript code modularity problem was~born\dots

The~only multiple file solution supported by~the~original JavaScript is to~load each file -- module --  in~a~separate \mbit{script} element.
Elements must~be specified in~the~correct order.
For~example, consider three modules \textit{A}, \textit{B} and~\textit{C}.
A~\hyperref[internetweb]{web} page uses some functionality \hyperref[declarationdefinition]{defined} in~the~module~\textit{A}.
But~this functionality uses some other functionality from~the~module~\textit{B} and~that one uses some functionality from the~module~\textit{C}.
Then you~must put exactly this into the~page:

%! language = TEXT
\begin{lstlisting}[language=XML,frame=no]
    <script src="C.js"></script>
    <script src="B.js"></script>
    <script src="A.js"></script>
\end{lstlisting}

\noindent If~you somehow change the~order, the~page loading will fail, and~exactly at~this point.
I.e.,~some mess can~be displayed in~the~browser.

Furthermore, this module loading is~synchronous.
Page loading waits for~each module until it's fully loaded and~only then it~continues with another~one.
When modules are~big, it~has of~course a~negative impact to~the~performance.

And~finally, this module loading doesn't check if~the~loaded module is~actually used.
It~just loads it.
Consider that you~remove the~dependency of~\textit{B} on~\textit{C} in~the~\textit{B}'s code in~the~previous example.
You~don't need to~load the~module~\textit{C} any~more.
But~until you~remove its \mbit{script} element from the~page source code, it~still will~be loaded and~affecting the~performance.

\newsection{Asynchronous Module Definition (AMD)}
\index{Asynchronous module definition}
\index{AMD}
\label{amd}
It's~a~\hyperref[protocolstandard]{standard} (sometimes also said to~be~\hyperref[api]{API}), which specifies writing JavaScript modules as~bodies of~large functions.
These modules can~be~treated as~classes and~the~code can~follow the~\hyperref[objectorientedprogramming]{object--oriented programming paradigm}.
Correct implementation of~the~standard should load modules asynchronously, in~the~correct order, only if~they're really used and~with minimal need of~adjusting \hyperref[internetweb]{web} pages source code.
I.e.,~it~should address all~problems of~the~\hyperref[javascriptmodularity]{default JavaScript modularity}.

\enlargethispage{20mm}
\thispagestyle{empty}
\newsubsection{Syntax}
\label{amdsyntax}
The~main function containing a~module -- equivalent of~a~class -- is~called \mbitq{define}.
It~gets an~array of~paths to~other modules (dependencies), and~a~function callback, usually anonymous, which contains the~actual module functional body.
\hyperref[parameterargument]{Parameters} of~the~callback are~handlers to~modules specified in~the~path array, the~order and~count must match.
If~a~module doesn't use any functionality from other modules (it~has no~dependencies), it~isn't necessary to~specify an~empty array.
The~\mbitq{define} function can~contain only the~callback without \hyperref[parameterargument]{parameters}.
Return statement of~the~callback contains handlers to~module members, which should~be available on~the~module.
Members with such handlers are~equivalent to~\hyperref[javapublic]{public} members of~a~class.
Other members are~equivalent to~\hyperref[javaprivate]{private} members.
\newpage

\example[AMD module]
%! language = TEXT
\begin{lstlisting}[language=JavaScript]
    define(['PATH_TO_OTHER_MODULE'], function((*\tmnbf{js1othmod1}{otherModuleHandler}*)) {
      let privateVariable = ...;
      let (*\tmnbf{js1modvar1}{publicVariable}*) = ...;

      function privateFunction() {
        ...
        (*\tmnbf{js1othmod2}{otherModuleHandler}*).publicFunctionFromOtherModule();
        ...
      }

      function (*\tmnbf{js1modfunc1}{publicFunction}*)() {
        ...
        privateFunction();
        ...
      }

      return {
        publicVariable: (*\tmnbf{js1modvar2}{publicVariable}*),
        publicFunction: (*\tmnbf{js1modfunc2}{publicFunction}*)
      }
    });
\end{lstlisting}
\begin{tikzpicture}[remember picture, overlay]
    \drawarrow{js1othmod1.south}{[xshift=9mm] js1othmod2.north}
    \drawarrow{[xshift=-3mm] js1modvar1.south}{[xshift=-9mm] js1modvar2.north}[red]
    \drawarrow{[xshift=6mm] js1modfunc1.south}{js1modfunc2.north}[green]
\end{tikzpicture}

\example[AMD module without dependencies]
\enlargethispage{10mm}
%! language = TEXT
\begin{lstlisting}[language=JavaScript]
    define(function() {
      let privateVariable = ...;
      let (*\tmnbf{js2modvar1}{publicVariable}*) = ...;

      function privateFunction() {
        ...
      }

      function (*\tmnbf{js2modfunc1}{publicFunction}*)() {
        ...
        privateFunction();
        ...
      }

      return {
        publicVariable: (*\tmnbf{js2modvar2}{publicVariable}*),
        publicFunction: (*\tmnbf{js2modfunc2}{publicFunction}*)
      }
    });
\end{lstlisting}
\begin{tikzpicture}[remember picture, overlay]
    \drawarrow{[xshift=-3mm] js2modvar1.south}{[xshift=-9mm] js2modvar2.north}
    \drawarrow{[xshift=6mm] js2modfunc1.south}{js2modfunc2.north}[red]
\end{tikzpicture}
\newpage

\newsubsection{Usage in web pages}
AMD~modules can't~be included into a~\hyperref[internetweb]{web} page directly.
They must~be called from another script and~this script can~be included into a~\hyperref[internetweb]{web} page.
Syntax of~this main script isn't standardised and~in~each AMD implementation it~can~be different.

\newsection{RequireJS}
\index{RequireJS}
\label{requirejs}
\href{https://requirejs.org}{RequireJS} is widely used implementation of~\hyperref[amd]{AMD}.
Modules syntax exactly follows the~\hyperref[amdsyntax]{AMD standard}.
Using modules from a~\hyperref[internetweb]{web} page, however, is~specific.

You~should have one main file, which is referred from a~\mbit{script} element inside~a~page \mbit{header} element.
From this file you~should assign functionality to~elements in~the~corresponding \hyperref[internetweb]{web} page.
And~from this file you can~also call \hyperref[amd]{AMD} modules.
If~the~page design forces you to~have the~\mbit{script} element somewhere else than in~the~\mbit{header} element or~to~have multiple \mbit{script} elements, then the~design is~bad and~should~be changed.

The~main file looks almost the~same as~a~standard \hyperref[amd]{AMD} module, only the~big enclosing function is~called \mbitq{requirejs}.
It~also doesn't have the~\mbit{return} statement.
It~can~contain RequireJS configuration (before the~main \mbit{requirejs} function), which usually specifies aliases for~paths to~third party libraries like \hyperref[jquery]{jQuery}.
The~code inside \mbox{the \textit{requirejs}} function is~executed like in~a~standard JavaScript file from the~top to~the~bottom.

Referring the~main file from a~page is~slightly different from \hyperref[javascriptmodularity]{standard JavaScript module referral}.
The~\mbit{src} attribute of~the~\mbit{script} element must contain path to~the~RequireJS library file (called \mbitq{require.js} by~default).
The~main script path is~specified in~the~\mbitq{data-main} element, and~it~can't contain the~\mbitq{.js} extension of~the~file.
The~extension is~expected, if~the~file doesn't have it, it~won't~work.

\note RequireJS is~usable even in~\hyperref[nodejs]{Node.js}.
Node.js already contains \mbox{the \hyperref[commonjs]{CommonJS}} technique for~modularization, but~the~performance is generally worse than with RequireJS\@.
Furthermore, when using RequireJS, you~can~develop a~functionality on~the~Node.js platform (server side), but~later use~it on~the~client side and~preserve the~effectiveness.

\warning You~can~encounter a~JavaScript file loading modules with a~keyword \mbitq{require}.
This is a \hyperref[nodejs]{NodeJS} feature called \hyperref[commonjs]{\mbit{CommonJS}}.
It~isn't supported in~JavaScript by~default.
\newpage

\example[assigning action to~a~button with RequireJS and jQuery]
%! language = TEXT
\begin{lstlisting}[language=JavaScript,title={Main JavaScript file \textit{\tmnbf{js3mainscript}{main}.js}}]
    requirejs.config({
      paths: {
        ...
        (*\tmnbf{js3mainjquery1}{jquery}*): 'SOME_PATH/jquery-X.Y.Z.js'
        ...
      }
    });

    requirejs(['(*\tmnbf{js3mainjquery2}{jquery}[ForestGreen]*)', 'SOME_PATH/amd_module'],
    function((*\$*), amdModuleHandler) {
      (*\$*)('#(*\tmnbf{js3mainbutton}{buttonId}[ForestGreen]*)').click(function() {
        ...
        amdModuleHandler.amdModuleFunction();
        ...
      });
    });
\end{lstlisting}
%! language = TEXT
\begin{lstlisting}[language=XML,title={Web page file loading the JavaScript file}]
    <!DOCTYPE html>
    <html lang="en">
      <head>
        ...
        <script data-main="SOME_PATH/(*\tmnbf{js3htmlscript}{main}[ForestGreen]*)" src="SOME_OTHER_PATH/require.js"></script>
        ...
      </head>
      <body>
        ...
        <button id="(*\tmnbf{js3htmlbutton}{buttonId}[ForestGreen]*)">BUTTON_LABEL</button>
        ...
      </body>
    </html>
\end{lstlisting}
\begin{tikzpicture}[remember picture, overlay]
    \drawarrow{js3mainscript.south}{js3htmlscript.north}
    \drawarrow{js3mainjquery1.south}{js3mainjquery2.north}[red]
    \drawarrow{js3mainbutton.south}{js3htmlbutton.north}[green]
\end{tikzpicture}
\newpage

\newsection{Native Modularity}
All~JavaScript implementations following the~\hyperref[ecmascript]{ECMAScript~6} (ES6, released in~2015) or~later provide modularity by~default.
There's no~need for~external libraries any~more.
The~major drawback is that the~syntax doesn't correspond to~any~existing implementation.
All~scripts written in~\hyperref[requirejs]{RequireJS}, \hyperref[commonjs]{CommonJS} or~any~other third-party library syntax still need their external libraries and~they don't use the~new native JavaScript modularity.

\todo

\newsection{JavaScript Object Notation (JSON)}
\index{JavaScript Object Notation}
\index{JSON}
\label{json}

\newsection{YAML}
\index{YAML}
\label{yaml}
% Maybe move with JSON somewhere else?
