\newsection{Enumeration}
\index{Enumeration}

\newsection{Annotations}
\index{Annotations}
\index{Java annotations}
\label{javaannotation}
Annotations are~special labels or~markers by~components of~a~Java code denoted by~the~symbol~\textquotesingle\textit{@}\textquotesingle. They carry additional information\,--\,metadata\,--\,about components they denote. Although annotations have no direct effect on~the~annotated code operation, they're processed, either during compilation or~during runtime.

The~most typical example is~the~method annotation \textit{@Override}, which tells the~compiler that the~annotated method should override a~method from an~interface or~a~superclass of~the~actual class. When the~compiler encounters a~method with this annotation, it~checks that the~\hyperref[definitiondeclaration]{definition} of~the~method in~the~class matches the~definition in~an~interface or~a~superclass and~if~not, it~stops the~compilation with complaining about missing override.

\todo How does it complain?

\newsection{String VS StringBuilder VS StringBuffer}
\index{String}
\index{StringBuilder}
\index{StringBuffer}