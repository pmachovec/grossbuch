\index{Git}
\label{git}
Git is a~popular \hyperref[distributedversioncontrolsystem]{distributed version vontrol system} and~probably the~most used version control system in~the~World.

\newsection{Create repository on a computer}
To~convert an~ordinary folder to~a~Git repository simply navigate to~that folder in a~shell and~run:
\begin{lstlisting}[frame=no]
    git init
\end{lstlisting}
\noindent You now have the~repository folder with the~master branch. Before adding anything new it's advisable to~create the~\textit{.gitignore} file and~commit all the~current contents of~the~folder to the~master branch, i.e.,~to~do an~initial commit, and~then to~connect the~local repository to your Git hub.\\

\noindent To~connect the~local repository to~your Git hub run:
\begin{lstlisting}[frame=no]
    git remote add origin GITHUB_ADDRESS:GITHUB_USER_NAME/REPOSITORY_NAME
\end{lstlisting}
\noindent Concrete example:
\begin{lstlisting}[frame=no]
    git remote add origin git@github.com:homer/donut-project
\end{lstlisting}
\noindent Then push the~master branch for~the~first time:
\begin{lstlisting}[frame=no]
    git push -u origin master
\end{lstlisting}

\newline\warning The~repository with the~specified name must already exist on~the~Git hub, i.e.,~it~must be created manually in~advance. Creating repositories by~pushing is~not~possible.