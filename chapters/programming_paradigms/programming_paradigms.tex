\newchapter{Programming Paradigms}

\newsection{Object--Oriented Programming (OOP)}
\index{Object--oriented programming}
\index{OOP}
\index{Object}
\label{objectorientedprogramming}
It's~a~programming paradigm in~which data are~\hyperref[encapsulation]{encapsulated} into objects together with tools for~manipulating these data. An~object holds data in~\hyperref[variablefieldproperty]{fields} (also called attributes) and~provides methods (also called functions) for~manipulating with these data (typically from other objects). Java and~most other languages using the~object paradigm like C\# or C++ are~not~pure object languages, because they contain \hyperref[javadatatypes]{primitive} \hyperref[datatypes]{data types} like \textit{integer} or~\textit{boolean}. In~pure object languages all~data (each variable) must be object. An~example of a~pure object language is Smalltalk (and~there is~probably no~other).

\warning Object--oriented programming is~not contradictory to~\hyperref[imperativeprogramming]{imperative programming} (or~even \hyperref[declarativeprogramming]{declarative programming}, see further).

\newsection{Basic Concepts of OOP}

\newsubsection{Encapsulation}
\index{Encapsulation}
\label{encapsulation}
Encapsulation is the~essence of~OOP. It~means having data (\hyperref[variablefieldproperty]{fields}) and~tools for~manipulating these data (methods, functions) in~a~same object (class). The~access to~data and~tools is~restricted, typically by \hyperref[javaaccessmodifiers]{access modifiers}. The~data can~also be objects, i.e.,~objects can~be encapsulated in~other objects. Purposes~are:
\begin{itemize}
    \item \textbf{Maintainability}\,--\,data and~tools for~manipulating them are~at~one place.
    \item \textbf{Security}\,--\,when the~access is~restricted, users can't read or~change data they aren't supposed to read or~change.
    \item \textbf{Changeability}\,--\,when needed, functionality of~encapsulated tools can~be changed without affecting users of~the~object.
\end{itemize}

\warning Having all \hyperref[variablefieldproperty]{fields} in~an~object private and~providing getters and~setters for~them is not an~encapsulation. This~is just a~convention that most of~OOP languages follow, but~for~example Python, although an~OOP language, accesses fields directly by the~dot notation.

\newsubsection{Inheritance}
\index{Inheritance}
\label{inheritance}
It's~a~mechanism for~avoiding writing a~code for~a~functionality common for~more objects, i.e.,~for~avoiding code duplicity. The~common code is~written only once in~one ancestor object (parent) and~descendant objects (children), which should contain the~code functionality, just inherit the~ancestor. Purposes~are:
\begin{itemize}
    \item \textbf{Simplicity}\,--\,no~need to~write common code more times. Users can~have multiple implementations in~variables of~the~same \hyperref[datatypes]{data type}, they don't have to~handle exact concrete type for~each place in~their code.
    \item \textbf{Changeability}\,--\,when common code needs a~change, there's only one place to~change.
    \item \textbf{Maintainability}\,--\,same as~changeability.
\end{itemize}

\warning When an~object implements an~\hyperref[javainterface]{interface}, it~isn't the~inheritance, that's the~\hyperref[abstraction]{abstraction}.

\newsubsection{Polymorphism}
\index{Polymorphism}
\index{Virtual method}
\label{polymorphism}
It~means that one~method or~function of~an~object can~get \hyperref[parameterargument]{arguments} of~various different \hyperref[datatypes]{data types} in~same \hyperref[parameterargument]{parameter} positions. For~example, basic numberic operations (addition, subtraction,~\dots) work with all available number types. Purposes are:
\begin{itemize}
    \item \textbf{Simplicity}\,--\,no~need to~have a~different method name for~each possible \hyperref[parameterargument]{argument} type combinations.
    \item \textbf{Maintainability}\,--\,same as~simplicity.
\end{itemize}

\noindent There are three types of~the~polymorphism:
\begin{itemize}
    \item \textbf{Ad hoc}\,--\,simple overloading based on~argument types. In~one~object there are more method implementations with the~same number of~parameters, but~in~each these parameters have different types. The~correct implementation is~decided by~types of provided arguments, usually during the~code compilation.
    \item \textbf{Parametrical}\,--\,method parameters have so~called \hyperref[javagenerics]{generic types}, which substitute concrete \hyperref[datatypes]{types}. Correct concrete types are~decided by~used values in~runtime.
    \item \textbf{Subtype}\,--\,connected with the~\hyperref[inheritance]{inheritance} and~\hyperref[javavirtualmethods]{virtual methods}. There can~be an~ancestor and~a~descendant and you can assign an~instance of~the~descendant to~the~variable of~the~\hyperref[datatype]{type} of~the~ancestor (\textit{Ancestor variable = new Descendant();}). When the~descendant \hyperref[javaoverride]{overrides} a~method \hyperref[declarationdefinition]{defined} in~the~ancestor and~the~method is declared as~\textit{virtual}, calling the~method on~the~variable (which has the~ancestor type) actually triggers the~implementation from the~descendant.
\end{itemize}

\warning In~Java or~C\# all methods are~virtual by~default and~the~subtype polymorphism is~happening automatically.

\newsubsection{Abstraction}
\index{Abstraction}
\index{Level of abstraction}
\label{abstraction}
It~means hiding implementation details of~objects and~providing only tools to~use objects (methods, functions). When you use some already existing library or~\hyperref[api]{API}, you~don't know and~don't care how it works inside. You~only have some object, call a~method and~get a~result. That's the~abstraction. Purposes~are:
\begin{itemize}
    \item \textbf{Simplicity}\,--\,users of~objects are~not~overwhelmed by internal implementation of~those objects, which can~be very complex. They can~simply incorporate objects to~their own implementations regardless objects complexity.
    \item \textbf{Changeability}\,--\,when needed, functionality of~provided tools can~be changed without affecting users of~the~object.
\end{itemize}

\warning The~abstraction is~broken by the~\hyperref[inheritance]{inheritance concept}\,--\,to~correctly implement a~descendant object you should know the~implementation of~the~ancestor. For~this reason it's~recommended to~avoid inheritance as~much as~possible, unless preventing the~code duplicity.

\warning \hyperref[javaabstractclasses]{Abstract classes}, \hyperref[javainterfaces]{interfaces} and~similar stuff are~based on~the~\hyperref[inheritance]{inheritance} and~the~\hyperref[polymorphism]{subtype polymorphism} concepts, but~they're sometimes considered to~belong under the~abstraction. Therefore, the~advantage of~having one \hyperref[datatypes]{data type} for~multiple implementations can~be also mentioned for~the~abstraction concept.

\warning Don't confuse the~abstraction with the~\hyperref[encapsulation]{encapsulation concept}. They're tightly coupled and~encapsulation doesn't make any use without the~abstraction. You~can quite often encounter a~definition of~one labeled as~a~definition of~the~other. But~they're different. Abstraction handles object manipulation tools, encapsulation handles (beside others) restriction of~access to~these tools. A~developer cares about the~abstraction during software \hyperref[architecturedesign]{design}, and~about the~encapsulation during implementation. The~abstraction (hiding implementation) is~achieved by method headers, \hyperref[javaabstractclasses]{abstract classes} and~\hyperref[javainterfaces]{interfaces} and~the~encapsulation (restricting access) is~achieved by \hyperref[javaaccessmodifiers]{access modifiers}.

Nevertheless, be~aware that deciding how~to~encapsulate \hyperref[javadatatypes]{primitive} variables and~"smaller" objects into "bigger" objects is a~part of~the~\hyperref[architecturedesign]{design}, i.e.,~it~belongs under the~abstraction concept. With varying depth of~\hyperref[variablefieldproperty]{fields} in~the~encapsulation tree we~talk about different levels of~abstraction.

\example[abstraction without encapsulation]
\begin{lstlisting}[language=Java]
    float PI = 3.1415926535897932384626433832795028841971693993751058;
    ...
    doSomethingWithFloat(PI);
\end{lstlisting}

\noindent The~complicated value is~not encapsulated in~an~object, but~it's~hidden behind a~variable. Something like this can actually happen in~real programming.
\newpage

\example[encapsulation without abstraction]
\begin{lstlisting}[language=Java]
    public class UselessClass {
        private int uselessVariable = 0;
    }
\end{lstlisting}

\noindent The~value is~encapsulated in~the~class, but~is~not~used inside any~method of~the~class. Something like this doesn't happen in~real programming.

\newsection{Object VS Data Structure}
\index{Data structure}
\label{objectdatastructure}
A~data structure is similar to~an~object. It's~also a~container for~data stored in~variables. The~difference is that data structures expose all its data and~contain no meaningful tools for~operating with data.

\example[simple data structure]
\begin{lstlisting}[language=Java]
    public class DataStructure {
        public int intVariable = 0;
        public String stringVariable = "abc";
    }
\end{lstlisting}

\noindent An~object with private variables, but~with only getters and~setters for~them, can~also be considered to~be rather a~data structure than an~object.

\example[data structure trying to look like an object]
\begin{lstlisting}[language=Java]
    public class WannaBeObject {
        private int intVariable = 0;
            private String stringVariable = "abc";

        public int getIntVariable() {
            return intVariable;
        }

        public void setIntVariable(int intVariable) {
            this.intVariable = intVariable;
        }

        public String getStringVariable() {
            return stringVariable;
        }

        public void setStringVariable(String stringVariable) {
            this.stringVariable = stringVariable;
        }
    }
\end{lstlisting}

\newsection{Imperative Programming}
\index{Imperative programming}
\index{Control flow}
\index{Flow of control}
\label{imperativeprogramming}
It's a~programming paradigm in which a~program behavior is changed by~specified statements. The~order of~executing statements is~called \textit{control flow}, or~also \textit{flow of~control}. It's~opposite of~the~\hyperref[declarativeprogramming]{declarative programming}.

\warning It is \textbf{not}~contradictory to~\hyperref[objectorientedprogramming]{object--oriented programming}, imperative languages can~be object--oriented. Actually, today's most used object--oriented languages (Java, C\#, C++) are~imperative.

\newsection{Procedural Programming}
\index{Procedural programming}
It's~a~variation of~the~\hyperref[imperativeprogramming]{imperative programming}. A~procedural program is~a~top--down sequence of~statements\,--\,a~procedure\,--\,that, based on~input \hyperref[parameterargument]{arguments}, performs a~certain action. The~main procedure can~break the~initial problem to~smaller subprocedures and~these can~be broken further until a~subprocedure is~simple enough to~be~solved.

The~\hyperref[imperativeprogramming]{imperative} \hyperref[objectorientedprogramming]{object--oriented} programming partially use the~procedural approach\,--\,a~sequence of~steps is~executed from~the~start to~the~end. But~the~difference is~the~data handling. In~real procedural programming tools for~data manipulation are~not~encapsulated together with data. Data are~passed through the~procedure in~a~form of~a~\hyperref[objectdatastructure]{data structure}, also called \textit{record}, and~adjusted at~various places.

The~procedural paradigm is easier to~grab than~the~object--oriented paradigm, but~its~main disadvantage is a~very tight coupling of~all~parts of~programs. When some step is changed at~the~beginning of~a~program procedure, it~can~affect data handling and~it~might be necessary to~adjust some following steps. The~bigger the~program~is, the~worse impact changes have.

Old~programming languages are procedural, for~example Pascal, C or~Fortran. Even in~modern languages, and~even in~those told to~be object--oriented, it's~possible to~use the~procedural paradigm. When a~program is~small, for~example, a~\hyperref[scriptinglanguages]{script}, it's~more convenient to~handle it as~a~procedure.

\newsection{Declarative Programming}
\index{Declarative programming}
\label{declarativeprogramming}
It's a~programming paradigm in which programs describe their desired results without explicitly listing commands or steps that must be performed. It's~opposite of~the~\hyperref[imperativeprogramming]{imperative programming}. Typical examples of~declarative languages are~SQL or~Prolog. Oracle's PL/SQL is a~kind of~hybrid language using imperative statements combined with declarative SQL. Even declarative languages can~be \hyperref[objectorientedprogramming]{object--oriented} (there~is an~object extension for~Prolog).
\newpage

\newsection{Functional Programming}
\index{Functional programming}
It's~a~variation of~the~\hyperref[declarativeprogramming]{declarative programming}. Functional programs can~be expressed as mathematical functions\,--\,for~a~set of~inputs values they produce an~output value. For~one~specific set of~inputs the~output of~a~functional program is always the~same. There are no state changes or~mutable data. An~output can~be changed only by changing the~input.

True functional languages, like Haskell or~Erlang, are used in the~academy and~science world rather than in~programming industry. When considering mainstream languages, JavaScript, Python or~Kotlin allow combining standalone functions (not~bound to~objects) with the~standard \hyperref[imperativeprogramming]{imperative paradigm}.

\example[functional \textit{Hello World} in JavaScript]
\begin{lstlisting}[language=JavaScript]
    var helloWorld = function() {
        console.log('Hello World!');
    }

    helloWorld();
\end{lstlisting}

\noindent The~function taking zero input values and~producing no~output (which is a~valid output) is defined, referenced by~the~variable and~called.
