\newsection{Functional Programming}
\index{Functional programming}
\label{functionalprogramming}
It's~a~variation of~the~\hyperref[declarativeprogramming]{declarative programming}.
Functional programs can~be expressed as mathematical functions -- for~a~set of~inputs values they produce an~output value.
For~one~specific set of~inputs the~output of~a~functional program is always the~same.
There are no state changes or~mutable data.
An~output can~be changed only by changing the~input.

True functional languages, like Haskell or~Erlang, are used in the~academy and~science world rather than in~programming industry.
When considering mainstream languages, JavaScript, Python or~Kotlin allow combining standalone functions (not~bound to~objects) with the~standard \hyperref[imperativeprogramming]{imperative paradigm}.

\example[functional \textit{Hello World} in~JavaScript]
%! language = TEXT
\begin{lstlisting}[language=JavaScript]
    var helloWorld = function() {
        console.log('Hello World!');
    }

    helloWorld();
\end{lstlisting}

\noindent The~function taking zero input values and~producing no~output (which is a~valid output) is defined, referenced by~the~variable and~called.
