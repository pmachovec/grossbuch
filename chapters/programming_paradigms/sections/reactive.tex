\newsection{Reactive Programming}
\index{Reactive programming}
\label{reactiveprogramming}
It's~a~programming paradigm a~which the~program is constructed as~a~bunch of~asynchronous events.
If~an~event happens (e.g.,~a~button is clicked), it~can~fire other events, these can~fire other events~etc.
I.e.,~the~program reacts on~events.

\newsubsection{React VS ReactiveX}
\index{React}
\index{ReactJS}
\index{ReactiveX}
\index{Rx}
\begin{itemize}
    \itembf{React}, also \textit{ReactJS}, is a~Javascript library for~dynamical rendering of~web pages.
    \itembf{ReactiveX}, also \textit{Rx}, is a~family of~libraries for~reactive programming.
            It~exists for~many languages, including JavaScript.
            That's why you can~encounter also names like \mbit{RxJS}, \mbit{RxJava}, \mbit{Rx.NET} \mbox{(for C\#)}, \mbit{RxCpp} \mbox{(for C++) etc.}
\end{itemize}
