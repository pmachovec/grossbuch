\newchapter{Java Programming Features}

\newsection{Abstract classes}
\index{Abstract class}
\index{Abstract method}
\label{javaabstractclasses}
Abstract classes are the~first Java tool to~achieve the~\hyperref[abstraction]{abstraction concept} of~\hyperref[objectorientedprogramming]{OOP}.
They're like standard classes, but~they can~contain also special methods without a~body, so~called \textit{abstract methods}.
A~variable can~have an~abstract class type, but~an~abstract class cannot be instantiated.
It~must~be inherited by~another class (the~keyword \mbitq{extends}), which must provide implementations for~all~abstract classes (or~be~also abstract and~further inherited).
An~instance of~such class can~be assigned to~a~variable of~the~abstract class type (\hyperref[polymorphism]{subtype polymorphism}).

As~abstract classes and~their usage are~based on~the~\hyperref[inheritance]{inheritance concept} (they use \hyperref[inheritance]{inheritance} to~achieve \hyperref[abstraction]{abstraction}), \mbox{non--abstract} members of~abstract classes are~normally available in~their descendants (if~not~\hyperref[javafinal]{final}).

Abstract classes and~abstract methods are~\hyperref[declarationdefinition]{declared} with the~keyword \mbitq{abstract}.
It's~advised (but~not~necessary) to~use the~\hyperref[javaoverride]{overriding} annotation for~methods providing implementations of~abstract methods from~predecessor abstract classes.

Because of~the~class inheritance mechanism in~Java, it's~possible to~inherit always only from~one abstract class.
If~you~manage to~end~up with an~abstract class, that contains only abstract members, it's~better to~change it to~an~\hyperref[javainterfaces]{interface}.

\enlargethispage{20mm}
\thispagestyle{empty}
\newsubsection{How (not) to use}
A~common mistake, that developers do with abstract classes, is to~\hyperref[declarationdefinition]{define} a~common \mbox{non--abstract} method in~an~abstract class and~use this method inside descendants of~the~abstract class.
This is wrong, because when the~common method is~changed for~whatever reason, all~descendants of~the~abstract class are~potentially affected.
When you need some common behavior for~descendants, it's~better to~implement some common \hyperref[dependencyinjection]{dependency} or~\hyperref[javastatic]{static} method (\hyperref[compositionoverinheritance]{composition over inheritance}).
The~danger of~a~change affecting multiple classes also exists in~this case, but~is smaller.

On~the~other hand, a~common method to~be~used on~descendants, not~in~them, is~correct and~encouraged with abstract classes.
I.e.,~when you call something like \mbitqls{descendant.;;commonMethod(\dots)} from outside the~descendant, then it's~correct.
Such~common method can~call abstract methods, whose implementations are~provided in~descendants.
This approach is actually the~\hyperref[templatedp]{template design pattern}.

A~general rule is that members of~an~abstract class must~be either abstract (must~be \hyperref[javaoverride]{overridden} in~descendants), private (aren't visible in~descendants) or~final (can't~be \hyperref[javaoverride]{overridden} in~descendants).
Final members furthermore can't~be called inside descendants.
Unfortunately, there's no~way how to~hide final members (which are always public) from descendants.
Not~using them inside is a~responsibility of~developers.
\newpage

\example[abstract class and its usage]
\begin{lstlisting}[language=Java, title={Abstract class}]
    public abstract class (*\tmnbf{abstract1class1}{AbstractClass}*) {
        (*\tmnbf{abstract1acmod1}{public}[blue]*) abstract (*\tmnbf{abstract1type1}{void}[blue]*) (*\tmnbf{abstract1method1}{abstractMethod}*)((*\tmnbf{abstract1paramtype1}{Object}*) (*\tmnbf{abstract1param1}{abstractMethodParam}*));

        public final void (*\tmnbf{abstract1concretemethod1}{concreteMethod}*)() {
            ...
        }
    }
\end{lstlisting}
\begin{lstlisting}[language=Java, title={Concrete class}]
    public class (*\tmnbf{abstract1concreteclass1}{ConcreteClass}*) extends (*\tmnbf{abstract1class2}{AbstractClass}*) {
        @>@Override
        (*\tmnbf{abstract1acmod2}{public}[blue]*) (*\tmnbf{abstract1type2}{void}[blue]*) (*\tmnbf{abstract1method2}{abstractMethod}*)((*\tmnbf{abstract1paramtype2}{Object}*) (*\tmnbf{abstract1param2}{abstractMethodParam}*)) {
            ...
        }
    }
\end{lstlisting}
\begin{lstlisting}[language=Java, title={Usage\,--\,concrete class instance in~the~variable of~the~abstract class type}]
    (*\tmnbf{abstract1class3}{AbstractClass}*) abstractClassVariable = new (*\tmnbf{abstract1concreteclass2}{ConcreteClass}*)();
    abstractClassVariable.(*\tmnbf{abstract1method3}{abstractMethod}*)((*\tmnbf{abstract1paramtype3}{Object}*) someObject);
    abstractClassVariable.(*\tmnbf{abstract1concretemethod2}{concreteMethod}*)();
\end{lstlisting}
\begin{tikzpicture}[remember picture, overlay]
    \drawarrow{abstract1class1.south}{[xshift=-3mm] abstract1class2.north}
    \drawarrow{abstract1class1.south}{[xshift=6mm] abstract1class3.north}
    \drawarrow{abstract1acmod1.south}{abstract1acmod2.north}[red]
    \drawarrow{abstract1type1.south}{abstract1type2.north}[green]
    \drawarrow{[xshift=3mm] abstract1method1.south}{[xshift=-3mm] abstract1method2.north}[blue]
    \drawarrow{[xshift=-3mm] abstract1method2.south}{[xshift=-3mm] abstract1method3.north}[blue]
    \drawarrow{abstract1paramtype1.south}{abstract1paramtype2.north}[Magenta]
    \drawarrow{abstract1paramtype2.south}{abstract1paramtype3.north}[Magenta]
    \drawarrow{abstract1param1.south}{abstract1param2.north}[yellow][.5]
    \drawarrow{[xshift=-3mm] abstract1concretemethod1.south}{[xshift=3mm] abstract1concretemethod2.north}[YellowOrange]
    \drawarrow{abstract1concreteclass1.south east}{abstract1concreteclass2.north}
\end{tikzpicture}

\newsection{Interfaces}
\index{Interface}
\label{javainterfaces}
Interfaces are newer Java tool to~achieve the~\hyperref[abstraction]{abstraction concept} of~\hyperref[objectorientedprogramming]{OOP}.
They're very similar to~\hyperref[javaabstractclasses]{abstract classes}.
An~interface is like an~abstract class with all methods abstract.
I.e.,~interfaces contain only method \hyperref[declarationdefinition]{declarations}.
These declarations don't have \hyperref[javaaccessmodifiers]{access modifiers}, all~methods are~treated as~\hyperref[javapublic]{public}.
When methods are~\hyperref[declarationdefinition]{defined} in~classes, they must~be \hyperref[javapublic]{public}.
In~terminology classes don't extend interfaces, but~implement.
The~keyword is~\mbitq{implements}.
Even methods from classes implementing interfaces should (but~don't have~to) use the~\hyperref[javaoverride]{overriding} annotation.

Opposite to~extending \hyperref[javaabstractclasses]{abstract classes}, a~class can~implement more interfaces.
Such class must then contain \hyperref[declarationdefinition]{definitions} of~all methods in~all interfaces it~implements.
If~more interfaces enforce the~same method with the~same \hyperref[declarationdefinition]{declaration}, only one method \hyperref[declarationdefinition]{definition} is~needed in~the~implementing class.
If~such methods have different return types in~interfaces, there will~be a~conflict and~the~implementing class won't~be compilable.
\newpage

\example[implementing multiple interfaces with a~common method]
\begin{lstlisting}[language=Java, title={First interface}]
    public interface (*\tmnbf{interface1interface1}{FirstInterface}*) {
        (*\tmnbf{interface1type1}{void}[blue]*) (*\tmnbf{interface1method1}{firstInterfaceMethod}*)();
        (*\tmnbf{interface1type2}{String}*) (*\tmnbf{interface1method2}{commonMethod}*)((*\tmnbf{interface1paramtype1}{Object}*) (*\tmnbf{interface1param1}{commonMethodParam}*));
    }
\end{lstlisting}
\begin{lstlisting}[language=Java, title={Second interface}]
    public interface (*\tmnbf{interface1interface2}{SecondInterface}*) {
        (*\tmnbf{interface1type3}{int}[blue]*) (*\tmnbf{interface1method3}{secondInterfaceMethod}*)();
        (*\tmnbf{interface1type4}{String}*) (*\tmnbf{interface1method4}{commonMethod}*)((*\tmnbf{interface1paramtype2}{Object}*) (*\tmnbf{interface1param2}{commonMethodParam}*));
    }
\end{lstlisting}
\begin{lstlisting}[language=Java, title={Implementing class}]
    public class ImplementingClass implements (*\tmnbf{interface1interface3}{FirstInterface}*), (*\tmnbf{interface1interface4}{SecondInterface}*) {
        public (*\tmnbf{interface1type5}{void}[blue]*) (*\tmnbf{interface1method5}{firstInterfaceMethod}*)() {
            ...
        }

        public (*\tmnbf{interface1type6}{int}[blue]*) (*\tmnbf{interface1method6}{secondInterfaceMethod}*)() {
            ...
        }

        public (*\tmnbf{interface1type7}{String}*) (*\tmnbf{interface1method7}{commonMethod}*)((*\tmnbf{interface1paramtype3}{Object}*) (*\tmnbf{interface1param3}{commonMethodParam}*)) {
            ...
        }
    }
\end{lstlisting}
\begin{tikzpicture}[remember picture, overlay]
    \drawarrow{[xshift=3mm] interface1interface1.south}{interface1interface3.north}
    \drawarrow{[xshift=3mm] interface1type1.south}{interface1type5.north}[red]
    \drawarrow{interface1method1.south}{interface1method5.north}[green]
    \drawarrow{[xshift=-3mm] interface1type2.south}{[xshift=-3mm] interface1type7.north}[blue]
    \drawarrow{[xshift=-3mm] interface1type4.south}{[xshift=-4mm] interface1type7.north}[blue]
    \drawarrow{interface1method2.south}{interface1method7.north}[Magenta]
    \drawarrow{[xshift=-3mm] interface1method4.south}{interface1method7.north}[Magenta]
    \drawarrow{interface1paramtype1.south}{interface1paramtype3.north}[yellow][.5]
    \drawarrow{interface1paramtype2.south}{interface1paramtype3.north}[yellow][.5]
    \drawarrow{interface1param1.south}{interface1param3.north}[YellowOrange]
    \drawarrow{interface1param2.south}{interface1param3.north}[YellowOrange]
    \drawarrow{interface1interface2.south}{[xshift=-3mm] interface1interface4.north east}
    \drawarrow{[xshift=1mm] interface1type3.south}{interface1type6.north}[red]
    \drawarrow{[xshift=-3mm] interface1method3.south}{interface1method6.north}[green]
\end{tikzpicture}

\noindent The~\hyperref[inheritance]{inheritance} with the~\mbitq{extends} keyword works even for~interfaces.
Opposite to~classes an~interface can~extend multiple other interfaces.
A~class implementing the~last interface in~the~inheritance line behaves like implementing all~interfaces in~the~line.

\warning Even interface \hyperref[inheritance]{inheritance} breaks \hyperref[abstraction]{abstraction} and~should~be rather avoided.

\example[an~interface extending two other interfaces]
\begin{lstlisting}[language=Java]
    public interface SomeInterface extends OtherInterface, OneMoreInterface {
        ...
    }
\end{lstlisting}
\newpage

\noindent Combining class extension (standard \hyperref[inheritance]{inheritance}) and~interface implementation is possible.
The~extension must~go first.

\example[a~class extending another class and~implementing an~interface]
\begin{lstlisting}[language=Java]
    public class SomeClass extends OtherClass implements SomeInterface {
        ...
    }
\end{lstlisting}
\newline

\noindent Interfaces support \hyperref[polymorphism]{subtype polymorphism}.
A~variable can~be of~an~interface type.
Any~instance of~a~class implementing that interface can~be assigned to~that variable.

\example[\hyperref[polymorphism]{subtype polymorphism} with an~interface]
\begin{lstlisting}[language=Java, title={Some interface}]
    public interface (*\tmnbf{interface2interface1}{SomeInterface}*) {
        ...
    }
\end{lstlisting}
\begin{lstlisting}[language=Java, title={Implementing class}]
    public class (*\tmnbf{interface2class1}{ImplementingClass}*) implements (*\tmnbf{interface2interface2}{SomeInterface}*) {
        ...
    }
\end{lstlisting}
\begin{lstlisting}[language=Java, title={Usage}]
    (*\tmnbf{interface2interface3}{SomeInterface}*) implementingClassVariable = new (*\tmnbf{interface2class2}{ImplementingClass}*)();
    ...
\end{lstlisting}
\begin{tikzpicture}[remember picture, overlay]
    \drawarrow{interface2interface1.south}{interface2interface2.north}
    \drawarrow{interface2interface1.south}{interface2interface3.north}
    \drawarrow{interface2class1.south}{interface2class2.north}[red]
\end{tikzpicture}

\noindent Interfaces can~contain \hyperref[variablefieldproperty]{fields}.
However, such~fields must have assigned a~value and~they're always public, even on~implementing classes.
Same as~methods, fields can't~have \hyperref[javaaccessmodofiers]{access modifiers}.
Although public fields aren't a~real problem, the~Java convention is having fields \hyperref[javaprivate]{private}.
Therefore, having fields in~interfaces isn't convenient.

\enlargethispage{20mm}
\thispagestyle{empty}
\example[a~field in~an~interface]
\begin{lstlisting}[language=Java, title={Interface with field}]
    public interface (*\tmnbf{interface3interface1}{InterfaceWithField}*) {
        int (*\tmnbf{interface3field1}{interfaceField}*) = 0;
    }
\end{lstlisting}
\begin{lstlisting}[language=Java, title={Implementing class}]
    public class (*\tmnbf{interface3class1}{ImplementingClass}*) implements (*\tmnbf{interface3interface2}{InterfaceWithField}*) {
        ...
    }
\end{lstlisting}
\begin{lstlisting}[language=Java, title={Usage}]
    (*\tmnbf{interface3interface3}{InterfaceWithField}*) implementingClassVariable = new (*\tmnbf{interface3class2}{ImplementingClass}*)();
    System.out.println(implementingClassVariable.(*\tmnbf{interface3field2}{interfaceField}*));
\end{lstlisting}
\begin{tikzpicture}[remember picture, overlay]
    \drawarrow{interface3interface1.south}{interface3interface2.north}
    \drawarrow{interface3interface1.south}{interface3interface3.north}
    \drawarrow{interface3field1.south}{interface3field2.north}[red]
    \drawarrow{interface3class1.south}{interface3class2.north}[green]
\end{tikzpicture}
\newpage

\newsubsection{Default Methods}
\index{Default method}
Imagine a~situation when you~have an~interface implemented by many classes and~you~want to~add a~new~method \hyperref[declarationdefinition]{declaration} to~that interface.
In~older versions of~Java, until you~had provided \hyperref[declarationdefinition]{definition} of~that method to~every class implementing the~interface, the~code was~uncompilable.
To~address this problem so~called default methods were~introduced in~Java~8.

Default methods are~marked by~the~keyword \mbitq{default}.
They contain full method \hyperref[declarationdefinition]{definition}, i.e.,~with the~method body, although they reside in~interfaces.
Classes implementing interfaces don't need to~implement these methods nor~have to~be abstract.
It's~very similar to~having a~non--abstract method in~an~\hyperref[javaabstractclasses]{abstract class}.
The~only difference is that default methods are~always public, they can't~have an~\hyperref[javaaccessmodifiers]{access modifier}.

\hyperref[javaprivate]{Private} and~\hyperref[javastatic]{static} methods are~also newly available in~interfaces.
Private to~be~called from~default methods, static to~be~called on~instances of~implementing classes stored in~variables of~the~interface type.
You~can~even have \hyperref[javaprivatestaticmethods]{private static methods} to~be~called from~other static methods.

\warning Don't use default methods in~new~interfaces.
Default methods serve purely for~the~described scenario when~extending an~existing interface implemented by many classes.

\example[various methods with bodies in~an~interface]
\begin{lstlisting}[language=Java]
    public interface SomeInterface {
        ...

        default void defaultMethod(Object defaultMethodParam) {
            ...
            privateMethod();
            ...
        }

        private void privateMethod() {
            ...
        }

        static int staticMethod() {
            ...
            return privateStaticMethod();
        }

        private static int privateStaticMethod() {
            ...
        }
    }
\end{lstlisting}

\note Private fields in~interfaces are~still forbidden.
Only methods can~be private.
\newpage

\newsection{Annotations}
\index{Annotations}
\index{Java annotations}
\label{javaannotation}
Annotations are~special labels by~components of~Java code denoted by~the~at~sign (\itq{@}).
They carry additional information\,--\,metadata\,--\,about components that they denote.
Although annotations have no direct effect on~the~annotated code operation, they're processed, either during compilation or~during runtime.
The~most typically used annotation is \hyperref[javaoverride]{\mbit{@Override}}.

\newsubsection{Custom Annotations}
\label{javacustomannotations}
Annotations aren't just fixed, developers can~create their own.
Annotation \hyperref[declarationdefinition]{definition} looks like a~simple class \hyperref[declarationdefinition]{definition}.
The~key word stating that an~annotation is~defined is \mbitq{@interface} (with the~at~sign at~the~beginning).
Annotation classes can~contain method \hyperref[declarationdefinition]{declarations} (without a~body), which in~fact define annotation \hyperref[parameterargument]{parameters}.
These method headers are~always public and~cant't~get an~\hyperref[javaaccessmodifiers]{access modifier}.
They can~be~followed by~the~keyword \mbitq{default} followed by~a~default value of~the~corresponding \hyperref[parameterargument]{parameter}.
The~default value type must match the~return type from the~method header.
If~a~parameter method doesn't have assigned default value, then the~parameter is~compulsory when using the~annotation.

Annotation usage and~behavior is~defined by~other, Java built--in annotations, sometimes called \textit{metaannotations}.
There are many of~them, but~only \mbitq{@Retention} and~\mbitq{@Target} are~necessary.

The~\mbit{@Retention} annotation defines if~the~custom annotation is~processed during compilation or~runtime.
It~gets one~of~values of~the~\mbitq{RetentionPolicy} enum.
There are only three possibilities:
\begin{itemize}
    \itembfd{RUNTIME} annotation is~processed during runtime.
    \itembfd{CLASS} annotation is~processed during compilation and~incorporated to~the~created \hyperref[bytecode]{bytecode}.
    \itembfd{SOURCE} annotation is~processed during compilation and~is~not~incorporated to~the~created \hyperref[bytecode]{bytecode}.
             I.e.,~the~annotation is~missing after eventual decompilation.
\end{itemize}

The~\mbit{@Target} annotation defines entities to~which the~custom annotation can~be applied.
It~gets one~or~more (enclosed in curly braces in~that~case) values of~the~\mbitq{ElementType} enum.
There are eleven possible values in~total.
They're quite self--explanatory, like \mbitq{METHOD}, \mbitq{FIELD} or~\mbitq{TYPE}.
\newpage

\newsubsection{Processing Annotations During Runtime}
This is relatively easy.
The~basic idea is to~get a~reference to~a~class with \hyperref[reflection]{reflection}, explore annotations of~its members or~the~class itself, and~when a~desired annotation is~detected, trigger some action.

\example
\begin{lstlisting}[language=Java, title={Annotation processed during runtime, applicable to fields and methods}]
    @@>@Retention<@@(RetentionPolicy.RUNTIME)
    @@>@Target<@@({ElementType.FIELD, ElementType.METHOD})
    public @interface (*\tmnbf{annot1annot}{ExampleAnnotation}*) {
        String (*\tmnbf{annot1param}{exampleParameter}*)() default "example default value";
    }
\end{lstlisting}
\begin{lstlisting}[language=Java, title={Annotation usage}]
    public class (*\tmnbf{annot1usageclass}{SimpleClass}*) {
        @>@(*\tmnbf{annot1usageannot1}{ExampleAnnotation}[LimeGreen]*)
        private String (*\tmnbf{annot1usagefield}{simpleField}*);

        @@>@(*\tmnbf{annot1usageannot2}{ExampleAnnotation}[LimeGreen]*)<@@((*\tmnbf{annot1usageparam}{exampleParameter}*) = "new value")
        public void simpleMethod() {
            ...
        }
    }
\end{lstlisting}
\begin{lstlisting}[language=Java, title={Runtime annotation processing}]
    Class<?> simpleClass = (*\tmnbf{annot1procclass}{SimpleClass}*).class;
    Field simpleField = simpleClass.getDeclaredField("(*\tmnbf{annot1procfield}{simpleField}[ForestGreen]*)");

    if (simpleField.isAnnotationPresent((*\tmnbf{annot1procannot1}{ExampleAnnotation}*).class)) {
        (*\tmnbf{annot1procannot2}{ExampleAnnotation}*) exampleAnnotation = simpleField.getAnnotation((*\tmnbf{annot1procannot3}{ExampleAnnotation}*).class);
        String exampleParameterValue = exampleAnnotation.(*\tmnbf{annot1procparam}{exampleParameter}*)();
        ...DO SOMETHING WITH THE PARAMETER VALUE...
    }
\end{lstlisting}
\begin{tikzpicture}[remember picture, overlay]
    \drawarrow{[xshift=3mm] annot1annot.south west}{annot1usageannot1.north}
    \drawarrow{[xshift=3mm] annot1annot.south west}{annot1usageannot2.north}
    \drawarrow{[xshift=3mm] annot1annot.south west}{[xshift=3mm] annot1procannot1.north west}
    \drawarrow{[xshift=3mm] annot1annot.south west}{annot1procannot2.north}
    \drawarrow{[xshift=3mm] annot1annot.south west}{[xshift=3mm] annot1procannot3.north west}
    \drawarrow{[xshift=-3mm] annot1param.south east}{[xshift=6mm] annot1usageparam.north}[red]
    \drawarrow{[xshift=-3mm] annot1param.south east}{[xshift=6mm] annot1procparam.north}[red]
    \drawarrow{[xshift=3mm] annot1usageclass.south}{[xshift=-3mm] annot1procclass.north}[green]
    \drawarrow{[xshift=6mm] annot1usagefield.south}{[xshift=3mm] annot1procfield.north west}[blue]
\end{tikzpicture}
\newpage

\newsubsection{Processing Annotations During Compilation}
That's much worse.
First, annotation must be already prepared in~the~compile time.
I.e.,~the~source code of~the~annotation class must be already compiled and~packed to~a~jar, which must be added to~classpath.
In~other words, you~must create the~annotation as~a~separate program (subproject is enough), pack it to~a~jar and~put it to~classpath.
Only then you~can~start writing the~code that uses the~annotation.

In~the~annotation program a~processor class must be created.
The~class must~be annotated with (repeatable) annotation \mbitq{@SupportedAnnotationTypes}, which specifies processed annotation class (or~classes), and~with annotation \mbitq{@SupportedSourceVersion}, which specifies minimal Java version under which the~processed annotation is~available during compilation.
The~class must also extend the~class \mbitq{AbstractProcessor} and~implement the~method \mbitq{process}.
The~method can~access annotated elements and~annotation values from its~\hyperref[parameterargument]{parameters}, namely from the~rounding environment, and~perform some action when a~desired annotation is~detected.

It's~desirable to~\hyperref[javaoverride]{override} the~\mbitq{init} method, retrieve the~processing environment messager in~it and~store the~messager to~a~global variable.
Interactions with the~compilation process (breaking compilation, printing warnings, \dots) in~the~\mbit{process} method are~then done through this messager, namely through its method \mbitq{printMessage}.
The~method can~take up~to~four \hyperref[parameterargument]{parameters}, but~only first two are~important:
\begin{itemize}
    \itembfd{Kind} a~member of~the~enum \mbitq{Diagnostic.Kind}.
             Specifies the~action to~perform.
             For~example, the~value \mbitq{ERROR} will~cause the~compilation to~fail.
    \itembfd{Message} character sequence (can~be string) with the~message to~display.
\end{itemize}
\noindent Nevertheless, it's~convenient to~use the~three--parameter version.
The~third parameter is~the~annotated element retrieved from the~\mbit{process} method parameters.
This~way the~element is~included to~printed information.

Additionally, the~jar must~be constructed so~that it~has a~file called exactly \mbitql{javax.;;annotation.;;processing.;;Processor} (without any further extension) inside the~\mbitql{META--INF/;;services} folder.
Inside this file there must~be package path to~the~processor class.
\newpage

\example
\begin{lstlisting}[language=Java, title={Annotation processed during compilation, included in~compiled code, applicable to~fields and~methods}]
    package (*\tmnbf{annot2annotpkg}{annotationpackage}*);

    @@>@Retention<@@(RetentionPolicy.SOURCE)
    @@>@Target<@@({ElementType.FIELD, ElementType.METHOD})
    public @interface (*\tmnbf{annot2annot}{ExampleAnnotation}*) {
        String (*\tmnbf{annot2param}{exampleParameter}*)() default "example default value";
    }
\end{lstlisting}
\begin{lstlisting}[language=Java, title={Annotation processor, working from Java~8, failing when annotated element is encountered}]
    package (*\tmnbf{annot2procpkg}{processorpackage}*);

    @@>@SupportedAnnotationTypes<@@("(*\tmnbf{annot2procannotpkg}{annotationpackage}[ForestGreen]*).(*\tmnbf{annot2procannot1}{ExampleAnnotation}[ForestGreen]*)")
    @@>@SupportedSourceVersion<@@(SourceVersion.RELEASE_8)
    public class (*\tmnbf{annot2proc}{ExampleProcessor}*) extends AbstractProcessor {
        private Messager messager;

        @>@Override
        public synchronized void init(ProcessingEnvironment processingEnv) {
            messager = processingEnv.getMessager();
        }

        @>@Override
        public boolean process(Set<? extends TypeElement> annotations, RoundEnvironment roundEnv) {
            for (Element element: roundEnv.getElementsAnnotatedWith((*\tmnbf{annot2procannot2}{ExampleAnnotation}*).class)) {
                messager.printMessage(Diagnostic.Kind.ERROR, element.getAnnotation((*\tmnbf{annot2procannot3}{ExampleAnnotation}*).class). (*\tmnbf{annot2procparam}{exampleParameter}*)(), element);
            }
        }
    }
\end{lstlisting}
\begin{lstlisting}[title={The file \textit{javax.annotation.processing.Processor}}]
    (*\tmnbf{annot2fileprocpkg}{processorpackage}*).(*\tmnbf{annot2fileproc}{ExampleProcessor}*)
\end{lstlisting}
\begin{tikzpicture}[remember picture, overlay]
    \drawarrow{annot2annotpkg.south}{[xshift=6mm] annot2procannotpkg.north west}
    \drawarrow{[xshift=-3mm] annot2annot.south east}{annot2procannot1.north}[red]
    \drawarrow{[xshift=-3mm] annot2annot.south east}{annot2procannot2.north}[red]
    \drawarrow{[xshift=-3mm] annot2annot.south east}{annot2procannot3.north}[red]
    \drawarrow{annot2param.south}{[xshift=-3mm] annot2procparam.north}[green]
    \drawarrow{annot2procpkg.south}{annot2fileprocpkg.north}[blue]
    \drawarrow{[xshift=3mm] annot2proc.south}{[xshift=3mm] annot2fileproc.north}[Magenta]
\end{tikzpicture}
\newpage

\newsection{Generic Types}
\index{Generic type}
\label{javagenerics}

\newsection{Threads}
\index{Thread}
\label{javathread}

\newsubsection{Thread safety}
\label{javathreadsafety}

\newsection{String VS StringBuilder VS StringBuffer}
\index{String}
\index{StringBuilder}
\index{StringBuffer}

\newsection{Reflection}
\index{Reflection}
\label{reflection}

\newsection{Imports with Wildcards}
\index{Import}
\index{Wildcard}
Some people say that when you import two or~more classes from one package, or~two or~more methods from one~class, you~should use the~wildcard import (i.e.,~something like \textit{import~package.*}).
It's~even written in~the~book \textit{Clean Code} by~Robert C.~Martin.
However, there are many objections to~this approach.

The~main (and~actually the~only) argument favouring imports with wildcards is that they make the~code shorter, therefore, more readable.
This~is somehow true, but~today's IDEs, even those for~free like Eclipse or~NetBeans, can~collapse import sections automatically.
If~someone is~writing code in~some editor not~capable of~this, then he's either a~beginner not~ready for~a~real IDE and~long imports, or~he's an~idiot.
Also, when a~class has~too~many imports, it's~too~dependent to~other classes, therefore, too~big and~more susceptible to~errors (if~a~dependant class breaks, my~class breaks,~too), and~therefore,~bad.
Furthermore, when~a~wildcard import is~used, the~class generally has~more imports than with explicit imports, and~the~previous problem with too~many imports is~even more serious.

Then there are only arguments against wildcard imports.
And~one of~them~is, funny enough, the~readability.
With wildcard imports it's~harder to~tell where an~imported feature comes from.
Consider the~following code:
\begin{lstlisting}[language=Java]
    import firstPackage.*;
    import secondPackage.*;
    import thirdPackage.*;

    public class ExampleClass {
        private MysteryClass mysteryInstance = new MysteryClass();
    }
\end{lstlisting}

\noindent It~isn't clear what package the~\textit{MysteryClass} class comes from.
Consider it's~somehow broken and~compilation fails, and~\textit{Ctrl}\,+\,click also~doesn't work (let's~say the~package library is~completely missing).
And~now consider this code:
\begin{lstlisting}[language=Java]
    import firstPackage.SomethingUseless;
    import secondPackage.SomethingEvenMoreUseless;
    import thirdPackage.MysteryClass;

    public class ExampleClass {
        private MysteryClass mysteryInstance = new MysteryClass();
    }
\end{lstlisting}

\noindent Here it's~clear that the~\textit{MysteryClass} class comes from the~third package.
So,~when observing problems, it's~clear where to~look.

Now~consider that the~previous code with wildcard imports works, i.e.,~there~is a~class called \textit{MysteryClass} in~the~third package.
And~now the~author of~the~second package, which lives on~the~other side of~the~world and~you don't know each~other, gets the~amazing idea to~create a~class called \textit{MysteryClass} in~the~second package.
Suddenly, there's an~ambiguity in~your code, compilation stops working and~you're screwed.
But~with explicit imports the~code still works without a~need of~change.

\note The~most favoured linter Ktlint for~\hyperref[kotlin]{Kotlin} language, which slowly starts to~push away Java, forbids wildcard imports in~the~default configuration.

\newsection{Compound Assignment}
\index{Compound assignment}
Compound assignment operators are those shortened assignments with arithmetic operators like \mbox{\textquotesingle\textit{+=}\textquotesingle} or~\mbox{\textquotesingle\textit{-=}\textquotesingle}.
You~must~be extra careful when dealing with the~subtraction compound operator (\mbox{\textquotesingle\textit{-=}\textquotesingle}).
It~first computes the~right side and~then subtracts it from~the~left side, and~that can~cause unintuitive results.
For~example consider the~following code:
\begin{lstlisting}[language=Java, frame=no]
    int x = 5;
    x = x - 1 + 2 - 3; @>// 3
\end{lstlisting}

\noindent The~variable~$x$ is~evaluated to~$3$, because $5-1+2-3=3$.
And~now consider an~"equivalent" code with the~compound assignment subtraction:
\begin{lstlisting}[language=Java, frame=no]
    int x = 5;
    x -= 1 + 2 - 3; @>// 5
\end{lstlisting}

\noindent The~variable~$x$ is~evaluated to~$5$, because $1+2-3=0$ (right side) and~$5-0=5$ (left side).
To~avoid falling to~this trap always treat the~right side to~be calculated first.

\newsection{Serialization}
\index{Serialization}
\index{Serializable}
\label{serialization}

\newsection{Servlets}
\index{Servlet}
\label{servlet}
A~Java program executed by~an~\hyperref[applicationserver]{application server} (most usually \hyperref[tomcat]{Tomcat}) based on~a~request (theoretically doesn't have to~be~\hyperref[http]{HTTP}, but~nowadays nothing else is~used) from a~client is~called \mbit{servlet}.
\hyperref[webserviceapplication]{Web services} implemented in~Java are~servlets.
The~implementation of~servlets is~said to~use the servlet~\hyperref[api]{API}

For~each servlet there is~a~class implementing the~interface \mbitq{javax.servlet.Servlet}.
This class is~an~entry point of~the~servlet program, sometimes the~class alone is (incorrectly) denoted as~\mbit{servlet}.
It~contains methods (enforced by its interface) that are~triggered when the~servlet is initiated (\itq{void init(ServletConfig config)}), when a~request comes to~it (\itq{void service(ServletRequest request, ServletResponse response)} and~when it's~terminated (\itq{void destroy()}).
Note that even the~\textit{service} method, from which some response is~expected, is~void.
The~response is~written to~a~writer instance of~the~\textit{response} object.

\enlargethispage{10mm}
\example
\begin{lstlisting}[language=Java]
    import java.io.PrintWriter;
    import javax.servlet.Servlet;
    import javax.servlet.ServletRequest;
    import javax.servlet.ServletResponse;

    public class ExampleServlet implements Servlet {
        ...
        protected void service(ServletRequest request, ServletResponse response) {
            ...
            PrintWriter out = response.getWriter();
            out.print("...EXAMPLE...");
            out.flush();
        }
        ...
    }
\end{lstlisting}

\newsubsection{Servlet Invocation}
Each servlet class must have~assigned some~URL\@.
When the~\hyperref[applicationserver]{application server} URL, correct port and~the~servlet URL are~invoked (for~example, in~a~web browser address bar), the~\textit{service} method of~the~class is~executed.
From there other classes and~methods can~be invoked.
The~content written to~the~writer of~the~response object is~sent back to~the~invoking client (a~web~browser displays it instead of~a~web~page).
One~servlet can~be configured as~the~default one, this is triggered when no~servlet URL part is~written to the~address~bar.

There already are some classes implementing the~\textit{Servlet} interface.
New~servlet classes are~implemented as~extending these classes.
With this approach a~developer can override only methods he needs, the~original \textit{Servlet} interface doesn't have to be implemented whole again.

\newsubsection{\textit{Service} method VS \textit{doGet} and \textit{doPost}}
\label{servicedopostdoget}
 Today's servlet classes usually extend the~\textit{HttpServlet} class and~the~logic of~servlets is~implemented by~overriding methods \textit{doGet} and~\textit{doPost} from that class, not~the~original \textit{service}.
That's because today's servlets are~triggered by web requests from web browsers, which use the~\hyperref[http]{HTTP protocol}.
This~\hyperref[protocolstandard]{protocol} enables (beside others) two~most common methods of~a~HTTP request\,--\,\textit{GET} and~\textit{POST}.
When an~\hyperref[applicationserver]{application server} gets a~\hyperref[http]{HTTP} request, it~actually triggers the~\textit{service} method, but~it~adds a~parameter with the~method type to~the~\textit{request} object.
Based on~this parameter the~implementation of~the\textit{service} method from the~\textit{HttpServlet} class delegates other parameters to~\textit{doGet} or~\textit{doPost} method, which are~expected to~be overridden by~the~developer (their body is empty in~the~\textit{HttpServlet} class).

When a~servlet is~invoked by~typing its URL to~a~web browser address bar, the~\hyperref[http]{HTTP} request has the~method~\textit{GET}, i.e., the~\textit{doGet} method is~triggered.
Eventual parameters available in~the~\textit{request} object can~be written as~a~part of~the~complete URL written to~the~address bar.
The~\textit{POST} method in~the~request, and~therefore even the~\textit{doPost} method in~the~servlet class, can~be triggered only by submitting a~HTML form configured to~use the~\textit{POST} method or~by~running a~\hyperref[javascript]{JavaScript} function on~the~client side.

\newsection{Java Server Pages (JSP)}
\index{JSP}
\index{Java server pages}
\label{jsp}
When a~\hyperref[http]{HTTP} request to~a~servlet is~sent from a~web browser, the~response is~displayed in~that~browser.
When some meaningful \hyperref[internetweb]{web} page should be displayed, its~whole source code must be written to~the~response, i.e.,~something like this:

\begin{lstlisting}[language=Java]
    import java.io.IOException;
    import java.io.PrintWriter;
    import javax.servlet.http.HttpServlet;
    import javax.servlet.http.HttpServletRequest;
    import javax.servlet.http.HttpServletResponse;

    public class ExampleServlet extends HttpServlet {
        ...
        @>@Override
        protected void doGet(HttpServletRequest request, HttpServletResponse response) throws IOException {
            ...
            PrintWriter out = response.getWriter();
            out.print("<html>");
            out.print("<head>");
            ...
            out.flush();
        }
        ...
    }
\end{lstlisting}

\noindent That sucks.
As~a~solution there are~JSP~pages.
A~JSP page is~basically a~HTML page with some advanced features for~interactions with request parameters, executing Java code and~dynamic context generation.
When a~\hyperref[http]{HTTP} request is~sent to~a~JSP page, its~content (dynamically adjusted based on~the~request) is~sent as~the~response (and~displayed in~the~initiating browser).

\warning JSP pages are~very dangerous for~code readability.
They enable mixing five syntaxes\,--\,HTML, CSS, Java, JavaScript and~\hyperref[el]{EL}\,--\,into one~file.
And~although this is considered and~known to~be a~VERY bad practice, you can~be sure that a~stressed and~tired developer, pressed by incompetent managers to~finish his job till yesterday (i.e.,~a~standard developer), will create a~real anarchy with this tool in~his~hands.

\newsubsection{Java Standard Tag Library (JSTL)}
\index{Java standard tag library}
\index{JSTL}
\label{jstl}

\newsubsection{Attribute VS Parameter}
\index{Attribute}
\index{Parameter}
\label{jspattributeparameter}

\newsubsection{Namespaces}
\index{Namespace}

\newsubsection{Expression Language (EL)}
\index{Expression language}
\index{EL}
\label{el}
