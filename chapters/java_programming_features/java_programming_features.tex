\newchapter{Java Programming Features}

\newsection{Abstract classes}
\index{Abstract class}
\label{javaabstractclasses}

\newsection{Interfaces}
\index{Interface}
\label{javainterfaces}

\newsection{Annotations}
\index{Annotations}
\index{Java annotations}
\label{javaannotation}
Annotations are~special labels by~components of~a~Java code denoted by~the~symbol~\textquotesingle\textit{@}\textquotesingle.
They carry additional information\,--\,metadata\,--\,about components they denote.
Although annotations have no direct effect on~the~annotated code operation, they're processed, either during compilation or~during runtime.
The~most typically used annotation is \hyperref[javaoverride]{\textit{@Override}}.

\newsubsection{Repeatable}
\label{annotationsrepeatable}

\newsection{Generic Types}
\index{Generic type}
\label{javagenerics}

\newsection{Threads}
\index{Thread}
\label{javathread}

\newsubsection{Thread safety}
\label{javathreadsafety}

\newsection{String VS StringBuilder VS StringBuffer}
\index{String}
\index{StringBuilder}
\index{StringBuffer}

\newsection{Reflection}
\index{Reflection}
\label{reflection}

\newsection{Imports with Wildcards}
\index{Import}
\index{Wildcard}
Some people say that when you import two or~more classes from one package, or~two or~more methods from one~class, you~should use the~wildcard import (i.e.,~something like \textit{import~package.*}).
It's~even written in~the~book \textit{Clean Code} by~Robert C.~Martin.
However, there are many objections to~this approach.

The~main (and~actually the~only) argument favouring imports with wildcards is that they make the~code shorter, therefore, more readable.
This~is somehow true, but~today's IDEs, even those for~free like Eclipse or~NetBeans, can~collapse import sections automatically.
If~someone is~writing code in~some editor not~capable of~this, then he's either a~beginner not~ready for~a~real IDE and~long imports, or~he's an~idiot.
Also, when a~class has~too~many imports, it's~too~dependent to~other classes, therefore, too~big and~more susceptible to~errors (if~a~dependant class breaks, my~class breaks,~too), and~therefore,~bad.
Furthermore, when~a~wildcard import is~used, the~class generally has~more imports than with explicit imports, and~the~previous problem with too~many imports is~even more serious.

Then there are only arguments against wildcard imports.
And~one of~them~is, funny enough, the~readability.
With wildcard imports it's~harder to~tell where an~imported feature comes from.
Consider the~following code:
\begin{lstlisting}[language=Java]
    import firstPackage.*;
    import secondPackage.*;
    import thirdPackage.*;

    public class ExampleClass {
        private MysteryClass mysteryInstance = new MysteryClass();
    }
\end{lstlisting}

\noindent It~isn't clear what package the~\textit{MysteryClass} class comes from.
Consider it's~somehow broken and~compilation fails, and~\textit{Ctrl}\,+\,click also~doesn't work (let's~say the~package library is~completely missing).
And~now consider this code:
\begin{lstlisting}[language=Java]
    import firstPackage.SomethingUseless;
    import secondPackage.SomethingEvenMoreUseless;
    import thirdPackage.MysteryClass;

    public class ExampleClass {
        private MysteryClass mysteryInstance = new MysteryClass();
    }
\end{lstlisting}

\noindent Here it's~clear that the~\textit{MysteryClass} class comes from the~third package.
So,~when observing problems, it's~clear where to~look.

Now~consider that the~previous code with wildcard imports works, i.e.,~there~is a~class called \textit{MysteryClass} in~the~third package.
And~now the~author of~the~second package, which lives on~the~other side of~the~world and~you don't know each~other, gets the~amazing idea to~create a~class called \textit{MysteryClass} in~the~second package.
Suddenly, there's an~ambiguity in~your code, compilation stops working and~you're screwed.
But~with explicit imports the~code still works without a~need of~change.

\note The~most favoured linter Ktlint for~Kotlin language, which slowly starts to~push away Java, forbids wildcard imports in~the~default configuration.

\newsection{Compound Assignment}
\index{Compound assignment}
Compound assignment operators are those shortened assignments with arithmetic operators like \mbox{\textquotesingle\textit{+=}\textquotesingle} or~\mbox{\textquotesingle\textit{-=}\textquotesingle}.
You~must~be extra careful when dealing with the~subtraction compound operator (\mbox{\textquotesingle\textit{-=}\textquotesingle}).
It~first computes the~right side and~then subtracts it from~the~left side, and~that can~cause unintuitive results.
For~example consider the~following code:
\begin{lstlisting}[language=Java, frame=no]
    int x = 5;
    x = x - 1 + 2 - 3; @>// 3
\end{lstlisting}

\noindent The~variable~$x$ is~evaluated to~$3$, because $5-1+2-3=3$.
And~now consider an~"equivalent" code with the~compound assignment subtraction:
\begin{lstlisting}[language=Java, frame=no]
    int x = 5;
    x -= 1 + 2 - 3; @>// 5
\end{lstlisting}

\noindent The~variable~$x$ is~evaluated to~$5$, because $1+2-3=0$ (right side) and~$5-0=5$ (left side).
To~avoid falling to~this trap always treat the~right side to~be calculated first.

\newsection{Serialization}
\index{Serialization}
\index{Serializable}
\label{serialization}

\newsection{Servlets}
\index{Servlet}
\label{servlet}
A~Java program executed by~an~\hyperref[applicationserver]{application server} (most usually \hyperref[tomcat]{Tomcat}) based on~a~request (theoretically doesn't have to~be~\hyperref[http]{HTTP}, but~nowadays nothing else is~used) from a~client is~called \mbit{servlet}.
\hyperref[webserviceapplication]{Web services} implemented in~Java are~servlets.
The~implementation of~servlets is~said to~use the servlet~\hyperref[api]{API}

For~each servlet there is~a~class implementing the~interface \mbitq{javax.servlet.Servlet}.
This class is~an~entry point of~the~servlet program, sometimes the~class alone is (incorrectly) denoted as~\mbit{servlet}.
It~contains methods (enforced by its interface) that are~triggered when the~servlet is initiated (\itq{void init(ServletConfig config)}), when a~request comes to~it (\itq{void service(ServletRequest request, ServletResponse response)} and~when it's~terminated (\itq{void destroy()}).
Note that even the~\textit{service} method, from which some response is~expected, is~void.
The~response is~written to~a~writer instance of~the~\textit{response} object.

\enlargethispage{10mm}
\example
\begin{lstlisting}[language=Java]
    import java.io.PrintWriter;
    import javax.servlet.Servlet;
    import javax.servlet.ServletRequest;
    import javax.servlet.ServletResponse;

    public class ExampleServlet implements Servlet {
        ...
        protected void service(ServletRequest request, ServletResponse response) {
            ...
            PrintWriter out = response.getWriter();
            out.print("...EXAMPLE...");
            out.flush();
        }
        ...
    }
\end{lstlisting}

\newsubsection{Servlet Invocation}
Each servlet class must have~assigned some~URL.
When the~\hyperref[applicationserver]{application server} URL, correct port and~the~servlet URL are~invoked (for~example, in~a~web browser address bar), the~\textit{service} method of~the~class is~executed.
From there other classes and~methods can~be invoked.
The~content written to~the~writer of~the~response object is~sent back to~the~invoking client (a~web~browser displays it instead of~a~web~page).
One~servlet can~be configured as~the~default one, this is triggered when no~servlet URL part is~written to the~address~bar.

There already are some classes implementing the~\textit{Servlet} interface.
New~servlet classes are~implemented as~extending these classes.
With this approach a~developer can override only methods he needs, the~original \textit{Servlet} interface doesn't have to be implemented whole again.

\newsubsection{\textit{Service} method VS \textit{doGet} and \textit{doPost}}
\label{servicedopostdoget}
 Today's servlet classes usually extend the~\textit{HttpServlet} class and~the~logic of~servlets is~implemented by~overriding methods \textit{doGet} and~\textit{doPost} from that class, not~the~original \textit{service}.
That's because today's servlets are~triggered by web requests from web browsers, which use the~\hyperref[http]{HTTP} protocol.
This~protocol enables (beside others) two~most common methods of~a~HTTP request\,--\,\textit{GET} and~\textit{POST}.
When an~\hyperref[applicationserver]{application server} gets a~\hyperref[http]{HTTP} request, it~actually triggers the~\textit{service} method, but~it~adds a~parameter with the~method type to~the~\textit{request} object.
Based on~this parameter the~implementation of~the\textit{service} method from the~\textit{HttpServlet} class delegates other parameters to~\textit{doGet} or~\textit{doPost} method, which are~expected to~be overridden by~the~developer (their body is empty in~the~\textit{HttpServlet} class).

When a~servlet is~invoked by~typing its URL to~a~web browser address bar, the~\hyperref[http]{HTTP} request has the~method~\textit{GET}, i.e., the~\textit{doGet} method is~triggered.
Eventual parameters available in~the~\textit{request} object can~be written as~a~part of~the~complete URL written to~the~address bar.
The~\textit{POST} method in~the~request, and~therefore even the~\textit{doPost} method in~the~servlet class, can~be triggered only by submitting a~HTML form configured to~use the~\textit{POST} method or~by~running a~\hyperref[javascript]{JavaScript} function on~the~client side.

\newsection{Java Server Pages (JSP)}
\index{JSP}
\index{Java server pages}
\label{jsp}
When a~\hyperref[http]{HTTP} request to~a~servlet is~sent from a~web browser, the~response is~displayed in~that~browser.
When some meaningful \hyperref[internetweb]{web} page should be displayed, its~whole source code must be written to~the~response, i.e.,~something like this:

\begin{lstlisting}[language=Java]
    import java.io.IOException;
    import java.io.PrintWriter;
    import javax.servlet.http.HttpServlet;
    import javax.servlet.http.HttpServletRequest;
    import javax.servlet.http.HttpServletResponse;

    public class ExampleServlet extends HttpServlet {
        ...
        @>@Override
        protected void doGet(HttpServletRequest request, HttpServletResponse response) throws IOException {
            ...
            PrintWriter out = response.getWriter();
            out.print("<html>");
            out.print("<head>");
            ...
            out.flush();
        }
        ...
    }
\end{lstlisting}

\noindent That sucks.
As~a~solution there are~JSP~pages.
A~JSP page is~basically a~HTML page with some advanced features for~interactions with request parameters, executing Java code and~dynamic context generation.
When a~\hyperref[http]{HTTP} request is~sent to~a~JSP page, its~content (dynamically adjusted based on~the~request) is~sent as~the~response (and~displayed in~the~initiating browser).

\warning JSP pages are~very dangerous for~code readability.
They enable mixing five syntaxes\,--\,HTML, CSS, Java, JavaScript and~\hyperref[el]{EL}\,--\,into one~file.
And~although this is considered and~known to~be a~VERY bad practice, you can~be sure that a~stressed and~tired developer, pressed by incompetent managers to~finish his job till yesterday (i.e.,~a~standard developer), will create a~real anarchy with this tool in~his~hands.

\newsubsection{Java Standard Tag Library (JSTL)}
\index{Java standard tag library}
\index{JSTL}
\label{jstl}

\newsubsection{Attribute VS Parameter}
\index{Attribute}
\index{Parameter}
\label{jspattributeparameter}

\newsubsection{Namespaces}
\index{Namespace}

\newsubsection{Expression Language (EL)}
\index{Expression language}
\index{EL}
\label{el}
