\newchapter{Linux}
\label{linux}

\newsection{Unix VS Linux}
\index{Unix}
\index{Linux}
\label{unixlinux}
\begin{itemize}
    \item \textbf{Unix},~often written in~capital letters (UNIX), was~originally a~\hyperref[shellcligui]{CLI}--controlled~\hyperref[os]{OS} developed in 1969 by~\href{https://en.wikipedia.org/wiki/Bell_Labs}{Bell~Labs} for~their internal use. In~late 1970s Bell~Labs owner \href{https://en.wikipedia.org/wiki/AT\%26T}{AT\&T} started to~sell Unix \hyperref[licencing]{licences} to~outside parties, which created their own enhanced mutations of~the~original~system. Nowadays there is so~called Unix family, which denotes all~\hyperref[os]{operating systems} derived this way from the~original Unix. The~most known Unix of~today is~probably macOS (although having \hyperref[hybridkernel]{hybrid kernel}). \warningnonl Unix systems are~commercial and~their owners don't release source codes (they also had~to~pay for~the~\hyperref[licencing]{licence}).
    \item \textbf{Linux} was~originally only an~\hyperref[kernel]{OS~kernel} developed in~1991 by~\href{https://en.wikipedia.org/wiki/Linus_Torvalds}{Linus Torvalds}. It~was~heavily inspired by~Unix (they both follow \hyperref[posix]{POSIX} standards), but~opposite to~Unix the~Linux source code was~publicly released under the~\hyperref[gpl]{GNU GPL} licence. \hyperref[os]{Operating systems} said to~be Linux use variations of~that original Linux kernel. Although claimed to~be free, there are~even commercial Linuxes containing some nontrivial enhancements or~specializations, for~example SUSE Linux Enterprise for~\hyperref[server]{server} computers.
\end{itemize}

\newsection{POSIX}
\index{POSIX}
\label{posix}
The~abbreviation stands for~\textit{Portable Operating System Interface} (yes,~\textit{X} is~missing). It's~a~set of~standards specifying \hyperref[api]{APIs}, command line \hyperref[shellcligui]{shells} and~utility interfaces. These standards serve for~maintaining compatibility between \hyperref[os]{operating systems}. Any~software working on~one~system following standards will~work even on~a~different system following standards. It's~typical for~Unix and~Linux systems.

\newsection{SH VS Bash}
\index{SH}
\index{Bash}
\index{Bourne}
\label{shbash}

\newsection{Apt VS Apt-get}
\index{Apt}
\label{linuxapt}

\newsection{Bashrc VS Bash Profile}
\index{Bashrc}
\index{Bash profile}
