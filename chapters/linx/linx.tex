\newchapter{Linux}
\label{linux}

\newsection{Unix VS Linux}
\index{Unix}
\index{Linux}
\label{unixlinux}
\begin{itemize}
    \itembf{Unix},~often written in~capital letters (UNIX), was~originally a~\hyperref[shellcligui]{CLI}--controlled~\hyperref[os]{OS} developed in 1969 by~\href{https://en.wikipedia.org/wiki/Bell_Labs}{Bell~Labs} for~their internal use.
            In~late 1970s Bell~Labs owner \href{https://en.wikipedia.org/wiki/AT\%26T}{AT\&T} started to~sell Unix \hyperref[licencing]{licences} to~outside parties, which created their own enhanced mutations of~the~original~system.
            Nowadays there is so~called Unix family, which denotes all~\hyperref[os]{operating systems} derived this way from the~original Unix.
            The~most known Unix of~today is~probably macOS (although having \hyperref[hybridkernel]{hybrid kernel}).
            \warningnonl Unix systems are~commercial and~their owners don't release source codes (they also had~to~pay for~the~\hyperref[licencing]{licence}).
    \itembf{Linux} was~originally only an~\hyperref[kernel]{OS~kernel} developed in~1991 by~\href{https://en.wikipedia.org/wiki/Linus_Torvalds}{Linus Torvalds}.
            It~was~heavily inspired by~Unix (they both follow \hyperref[posix]{POSIX} standards), but~opposite to~Unix the~Linux source code was~publicly released under the~\hyperref[gpl]{GNU GPL} licence.
            \hyperref[os]{Operating systems} said to~be Linux use variations of~that original Linux kernel.
            Although claimed to~be free, there are~even commercial Linuxes containing some nontrivial enhancements or~specializations, for~example SUSE Linux Enterprise for~\hyperref[server]{server} computers.
\end{itemize}

\newsection{POSIX}
\index{POSIX}
\label{posix}
The~abbreviation stands for~\textit{Portable Operating System Interface} (yes,~\textit{X} is~missing).
It's~a~set of~standards specifying \hyperref[api]{APIs}, command line \hyperref[shellcligui]{shells} and~utility interfaces.
These standards serve for~maintaining compatibility between \hyperref[os]{operating systems}.
Any~software working on~one~system following standards will~work even on~a~different system following standards.
It's~typical for~Unix and~Linux systems.
\newpage

\newsection{SH VS Bash}
\index{SH}
\index{Bash}
\index{Bourne}
\label{shbash}
\begin{itemize}
    \itembf{SH}, also \textit{Bourne Shell}, is~a~very old \hyperref[scriptinglanguages]{scripting language} specification (not~one concrete language) following \hyperref[posix]{POSIX} standards.
            Each~Unix has~its own implementation following the~specification.
    \itembf{Bash} (\textit{Bourne Again Shell}) is~a~concrete language initially fully following the~SH specification, but~nowadays containing many extensions not~included in \hyperref[posix]{POSIX} standards.
            All~SH scripts work when executed by~Bash, but~not~all Bash scripts necessarily work when~executed by~other SH language.
\end{itemize}

\noindent The~typical \hyperref[shebang]{shebang} \itq{\#!/bin/sh} is~a~symbolic link to~an~SH language of~a~specific Unix system.
When a~script with this shebang is~executed in~a~certain Unix, the~system automatically uses its SH language.

\warning Latest versions of~Debian or~Ubuntu Linuxes don't~use Bash as~their SH language, but~Dash, which still follows \hyperref[posix]{POSIX} standards.

\newsection{APT}
\index{APT}
\label{linuxapt}
APT is an~abbreviation of~\textit{Advanced Packaging Tool}.
It's~a~\hyperref[packagemanager]{package manager} for~Debian and~Ubuntu Linuxes.
The~\itq{apt} command serves for~handling packages of~APT\@.

\newsubsection{Apt VS Apt-get}
The~\itq{apt-get} is~a~predecessor of~the~\itq{apt} command.
There's a~possibility of~caching \itq{apt-get} commands, but~it~requires user's active participation.
The~newer \itq{apt} command handles caching automatically.

\newsection{Bashrc VS Bash Profile}
\index{Bashrc}
\index{Bash profile}
Both these scripts are a~sort of~initial scripts.
The~\itq{.bashrc} script is~executed any time when a~new terminal window is~opened.
The~\itq{.bash\_profile} script is~executed after an~user having the~script in~his home folder logins to~the~system, even remotely.

Generally, to~reload the~\itq{.bashrc} script you must open a~new terminal window.
To~reload the~\itq{.bash\_profile} script you~must reboot the~system.
However, this can~be substituted by~giving the~path to~the~script to~the~\itq{source} command, i.e.,~by~running \itq{source \textasciitilde/.bashrc} or~\itq{source \textasciitilde/.bash\_profile}.
