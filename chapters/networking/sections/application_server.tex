\newsection{Application Server}
\index{Application server}
\label{applicationserver}
Application servers are~very similar to web~servers, they're like web~servers with additional functionality.
They also get requests from clients and~send back responses and~they can~also retrieve static HTML document.
The~difference is that they can~furthermore trigger some more complex logic.
Application servers can, based on~incoming requests, execute programs written in~complex languages (Java, JavaScript, Python, C\# etc.).
These programs can then be called \hyperref[webserviceapplication]{web~services}.

\warning The~term \textit{web server} is~often used in~the~meaning of~\textit{application server}.
For~example, you~can~encounter claims that \hyperref[tomcat]{Tomcat}, a~typical example of~Java application server, is~a~web server.

\warning Do~not confuse \hyperref[tomcat]{Tomcat} with the~\href{https://en.wikipedia.org/wiki/Apache_HTTP_Server}{Apache HTTP Server}, often denoted as~\textit{Apache Server} or~simply \textit{Apache}.
That's a~typical \hyperref[webserver]{web server} designed for~static HTML documents handling.
It~supports some advanced logic using some \hyperref[scriptinglanguages]{scripting languages} like PHP or~Python, but~it~can't run Java programs.
On~the~other hand, \hyperref[tomcat]{Tomcat} serves mainly for~running Java programs (and~propagating their results to~web pages).
