\newsection{OAuth}
\index{OAuth}
\index{Open Authentication}
\index{Consumer}
\index{Token}
\index{Service provider}
The~acronym stands for~\itq{Open Authentication}.
It's~a~\hyperref[protocolstandard]{standard} for~\hyperref[authenticationauthorization]{\textbf{authorizing}} third party applications -- \textit{consumers} -- to~access certain resources at~a~\textit{service provider} on~behalf of~a~\textit{user}.
Resources are normally accessible only after \hyperref[authenticationauthorization]{\textbf{authentication}} to~the~provider with~the~user's credentials.
OAuth eliminates the~need of providing the~user's credentials to~the~consumer application, which is a~security risk.
Instead, the~user authenticates directly to~the service provider.
A~series of token exchanges between him, the~service provider and~the~consumer follows.
It~results in~assigning an~access token to~the~consumer.
Then, when the~consumer wants to~access resources, it~incorporates the~access token to~each request.
The~access token grants the~consumer to~use a~certain part of~all~resources available at~the~service provider for~the~corresponding user -- it~performs the~\hyperref[authenticationauthorization]{\textbf{authorization}}.

For~an~example consider your Facebook account.
That's the~\textit{service provider}.
A~dating application -- the~\textit{consumer} -- offers photo upload from~your Facebook account.
Instead~of giving your Facebook credentials to~the~dating application (and~allowing it to~do~anything to~your account), you~hit a~special button in~the~dating application that says \itq{Upload photos from Facebook}.
This redirects you to~Facebook login page where you provide your Facebook credentials.
Then it asks you if~you agree that the~dating application can~access your photos.
When you confirm, the~dating application can~access photos in~your Facebook account.

\textit{Token} is similar to~a~\hyperref[key]{cryptographic key}.
It's~just a~sequence of~\hyperref[bitsbytes]{bytes}.
Most~often it's~represented as~a~hexadecimal string, with lowercase letter characters and~without spaces between bytes.
However, other representations like \hyperref[base64]{base64} are also possible.

\newsubsection{Workflow}
\index{Consumer token}
\index{ID token}
\index{Request token}
\begin{itemize}
    \item User orders a~consumer to~access user's resources at~a~service provider.
    \item Consumer sends a~request to~the~service provider asking for~a~\textit{consumer token}, also called \itq{ID~token} or \itq{request token} (yes,~in~that case the~request asks for~a~request token).
    \item Service provider generates the~consumer token and~sends it in~a~response.
          This token identifies the~consumer.
    \item Consumer redirects the~user to~the~service provider's login page.
          The~consumer token and~wanted resources are~included in~the~background context of~this redirection.
    \item User \hyperref[authenticationauthorization]{authenticates} to~the~service provider and~is~forwarded to~an~approval page.
          A~good service provider should list an~overview of~all~resources demanded by the~consumer on~this~page.
    \item User approves -- \hyperref[authenticationauthorization]{authorizes} -- access to~demanded resources for~the~consumer.
          He's~redirected back to~the~consumer.
    \item Consumer automatically sends another request to~the~service provider asking for~an~\textit{access token}.
          He~includes the~consumer token in~the~request.
          And~because the~user previously \hyperref[authenticationauthorization]{authorized} the~consumer (identified by the~consumer token), the~consumer gets the~access token.
    \item Consumer can~request resources now.
          He~includes the~access token to~requests.
    \item When a~request for~resources with~an~access token comes to~the~service provider, the~provider decides if~the~owner of~the~access token (the~consumer) is \hyperref[authenticationauthorization]{authorized} to~use requested resources.
          If~yes, he~sends them in~a~response.
\end{itemize}

\newsubsection{Authentication with OAuth}
Although the~main purpose of~OAuth is consumer applications' \hyperref[authenticationauthorization]{authorization} to~service providers, consumers can~use it for~\hyperref[authenticationauthorization]{authentication} of~users to~themselves.
The~principle is still the~same -- consumer asks a~service provider for~user's resources.
The~service provider in~this case must~be some known system with big user numbers and~elaborated \hyperref[authenticationauthorization]{authentication} mechanism, like Facebook, Google~etc.
The~logic is that for~consumer's~\hyperref[authenticationauthorization]{authorization} the~user must \hyperref[authenticationauthorization]{authenticate} to~the~service provider.
If~the~consumer gets the~access token, it~means that the~user successfully \hyperref[authenticationauthorization]{authenticated} to~the~service provider.
And~as~the~service provider's authentication mechanism is known to~be~reliable, the~consumer accepts this as~an~authentication to~itself.
This way the~user can~use one authentication to~access multiple applications, without a~need to~create a~registration to~each separately.
Also consumers don't have to~implement own secure authentication mechanisms.
They exploit those of~service providers.

The~set of~requested resources can~be theoretically empty in~this~case, but~usually an~access to~at~least email or~user name is~requested.
Some applications even combine both usages of~OAuth -- they authenticate the~user by requesting "standard" resources.
For~example, dating applications like Tinder allow users to~authenticate with Facebook, Google or~Instagram, and~they automatically request access to~users' photos.
