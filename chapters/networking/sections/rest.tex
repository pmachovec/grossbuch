\newsection{REST}
\index{Representational State Transfer}
\index{REST}
\index{RESTful}
\label{rest}
The~abbreviation stands for \itq{Representational State Transfer}.
It's~a~technique for~transferring data -- for~exchanging messages -- between two \hyperref[applicationprocessprogramservicethread]{programs} using the~\hyperref[http]{HTTP protocol}.
Transferred data can~be theoretically in~any string form, but~the~only format used nowadays is~the~\hyperref[json]{JSON} format.

The~first program has~the~role of~a~client and~initiates a~standard \hyperref[http]{HTTP} request.
The~request can, but~doesn't have~to, already contain some JSON data in~the~body.
It's~responsibility of~the~first program to~create the~JSON string and~include it to~the~request.

The~second program is~a~standard \hyperref[webserviceapplication]{web service}.
It~gets the~request, parses the~incoming JSON (if~there is~any), creates (another) JSON (also doesn't have~to), includes it to~a~response body and~sends the~response to~the~client program.

\hyperref[webserviceapplication]{Web services} supporting this type of~data transfer are~said to~have \mbox{REST \hyperref[api]{API}}.
To~make~it more complicated the~\mbitq{RESTful} term is~sometimes used.
You~can~encounter many different combinations of~terms all~denoting a~\hyperref[webserviceapplication]{web service} supporting the~REST data transfer.
The~overview list follows:
\begin{itemize}
    \item Service with REST API
    \item Service with RESTful API
    \item REST web service
    \item RESTful web service
    \item REST service
    \item RESTful service
\end{itemize}

\noindent And~to~make it even more complicated you~can~also encounter a~term \itq{\mbox{RESTful} application} or~\itq{\mbox{RESTful} program}.
This can~denote both a~client side program or~a~server side web service supporting the~REST communication.

\warning Opposite to~\hyperref[soap]{SOAP}, REST isn't a~concrete \hyperref[protocolstandard]{protocol}, not~even a~\hyperref[protocolstandard]{standard}.
It's~just a~programming style -- sending structures (objects) encoded in~the~\hyperref[json]{JSON} format over~\hyperref[http]{HTTP}.
