\newsection{PGP and GPG}
\index{PGP}
\index{GPG}
\index{GnuPG}
\index{Pretty Good Privacy}
\index{GNU Privacy Guard}
\label{pgpgpg}
These are computer programs for~handling both \hyperref[symmetriccryptography]{symmetric} and~\hyperref[asymmetriccryptography]{asymmetric cryptography}.
They can~be integrated to~various communication software, for~example, Outlook, to~secure the~communication.
When messages are short or~electronic signatures are needed, the~asymmetric encryption can be used standalone.
For~longer messages a~combination of~both encryption types is used.
Messages are encrypted with symmetric encryption, and~the~symmetric key is encrypted by asymmetric encryption.
Users either send their public keys directly to~each other or~they can~upload them to~dedicated servers, from where other users can~download them.
These servers can~serve as~alternatives to~\hyperref[certificationauthority]{certification authorities}.

User's keys are~stored in~the~local program installation and~accessed by~passwords.
When using private keys (decrypting a message, signing), the~user doesn't have to~copy the~whole long key anywhere, he~only provides the~password.

\textit{PGP}~stands for~\itq{Pretty Good Privacy}.
It's~a~commercial software,~i.e.,~users need to~pay for~it.
\mbox{\textit{GPG}, or \textit{GnuPG}}, standing for~\itq{\hyperref[gnu]{GNU} Privacy Guard}, is~a~free alternative.

\newsubsection{OpenPGP}
This is a~\hyperref[protocolstandard]{standard} that~PGP, GPG and~their mutations follow.
Thanks to~this, they're mutually compatible -- a~message encrypted in~one program should~be decryptable in~another.
\newpage
