\newsection{Key VS Password}
\index{Key derivation function}
\index{KDF}
\index{Password}
\index{PBKDF}
\index{Password--Based Key Derivation Function}
\label{keypassword}
Fixed key lengths and~unprintability of~some byte values make direct usage of~keys difficult.
That's why the~concept of~passwords and~key derivation functions exists.
A~\textit{key derivation function} (KDF) is~an~one--way hashing function, that can~compute a~key of~a~desired byte length from a~string -- password -- of~any~length and~content.
Typical examples of~key derivation functions are~PBKDF1 and~PBKDF2 (\textit{Password--Based Key Derivation Function}), which are~mentioned in~the~\hyperref[pkcs]{PKCS~\#5} encryption standard.

\newsubsection{Usage in \hyperref[symmetriccryptography]{Symmetric Cryptography}}
When a~message is about to~be~encrypted, a~password is~provided by~the~sender and~a~key is~generated by~a~KDF\@.
When the~message is about to~be~decrypted, the~same password must~be provided by~the~receiver to~the~same KDF, which generates the~same key and~the~message can~be~decrypted.

\newsubsection{Usage in \hyperref[asymmetriccryptography]{Asymmetric Cryptography}}
In~\hyperref[asymmetriccryptography]{asymmetric cryptography} private keys are~sometimes protected by~passwords.
I.e.,~a~sender must provide a~password when \hyperref[electronicsignature]{signing} a~message and~a~receiver must provide a~password when \hyperref[asymmetriccryptography]{decrypting an~encrypted message}.
