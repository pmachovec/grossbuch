\newsection{Asymmetric Cryptography}
\label{asymmetriccryptography}
In~this type of~encryption there are two keys matching to~each other - public and~private.
Messages encrypted by a~public key can~be decrypted only~by a~matching private key.
That's why it's sometimes also called \itq{public key cryptography}.
It can work also in the~reversed way (encrypted by private key, decrypted by public key), which is used in \hyperref[electronicsignature]{electronic signatures}.
The~message processing flow~is:
\begin{itemize}
    \item A~sender owns the~public key.
          He~encrypts the~message with the~public key and~sends it to a~receiver.
    \item The~receiver has the~matching private key, which he~uses to~decrypt the~message.
\end{itemize}

\noindent The~main security benefit in this approach is that the~private key used for~discovering messages contents is never shared, only the receiver owns~it.
Nobody else can read those messages.
Typically the receiver generates a~pair of~keys and sends the~public key to~whoever he~wants to~receive messages from.
If someone else gets the~public key, there is no problem, he won't be able to~decrypt messages sent to~the~receiver.

\newsubsection{RSA}
\index{RSA}
\label{rsa}
\textit{RSA} is the~most well--known asymmetric encryption algorithm.
It's used for~both encrypting sent data and~\hyperref[electronicsignature]{electronic signatures}.
The abbreviation is nothing technical, it~consists of initial letters of~names of~authors of~the algorithm -- \textit{Rivest}, \textit{Shamir}, \textit{Adleman}.

Disadvantages of~this approach are message length limits and~performance demands.
In~RSA with a~2048 bits (256~bytes) key the~maximum message length is 245 bytes.
And~with too big keys the~computational time grows, the~maximum reasonable key length today is 4096 bits (512~bytes).
On~the~other hand with too short key the~security drops, so~the~minimum secure key length today is 2048 bits (256~bytes).

\newsubsection{RSA Key Length}
\index{Modulus}
The~length of a~key is in~fact only the~length of~so called \textit{modulus}, which is some hidden number that both public and~matching private key have in common.
There is also additional info encoded in keys, this info is needed for~the~RSA algorithm.
That's why both keys in the~pair contain more bits than the number specified during their generation.
There is more such info in private keys, that's why private keys are significantly longer than matching public keys.

\warning Although only some fixed lengths of the~key are~mentioned almost everywhere (512, 1024, 2048, 4096), there is no restriction, the~length can~be any number.
However, the~\textit{ssh-keygen} program provided with Git installation, which is usually used for~key generation, doesn't allow number lower than 1024.

\newsubsection{RSA Key Formats}
\index{OpenSSH}
RSA~keys are~most often represented as~\hyperref[base64]{\mbox{Base64}} encoded strings, even when stored in~files.
Files with RSA~keys are usually in~the~\hyperref[pem]{PEM} format.
Keys can~follow either \hyperref[pkcs]{PKCS~\#1} or~\hyperref[pkcs]{PKCS~\#8} standard.
In~the~first case they're usable only in~the~RSA~algorithm, in~the~second case they can~be used even in~different asymmetric cryptography algorithms.

Public keys are sometimes represented also in~the~\mbit{OpenSSH} format.
A~public key in~the~OpenSSH format incorporates also a~user email, which must~be provided during key~generation
This format is~used for~manual public key exchange combined with~\hyperref[authenticationauthorization]{authentication}, for~example on~Git~hubs.
Private keys are never represented in~this format.

\example[a~public key in~the~OpenSSH format]
%! language = TEXT
\begin{lstlisting}
    ssh-rsa AAAAB3NzaC1yc2... jbravo@gmail.com
\end{lstlisting}
\newline

\noindent \textbf{The same public key in PEM PKCS~\#1 format}
%! language = TEXT
\begin{lstlisting}
    -----BEGIN RSA PUBLIC KEY-----
    MIIBCgKCAQEAwNUeQzY4mlj1sAH2zLZFUSOf7yMGjpVKp7LgxnKQDUNXbA8WSt/E
    oM/IkHGq35yFOpXevsIF2Xp7oO4md/muTWcecqCG7k1MAJLz9ZUiMikVfEB9DnhH
    ...
    1W/oG/iF1qlnQJq7X/tjKUoX7yUhpZLu9wIDAQAB
    -----END RSA PUBLIC KEY-----
\end{lstlisting}
\newline

\noindent \textbf{The same public key in PEM PKCS~\#8 format}
%! language = TEXT
\begin{lstlisting}
    -----BEGIN PUBLIC KEY-----
    MIIBIjANBgkqhkiG9w0BAQEFAAOCAQ8AMIIBCgKCAQEAwNUeQzY4mlj1sAH2zLZF
    USOf7yMGjpVKp7LgxnKQDUNXbA8WSt/EoM/IkHGq35yFOpXevsIF2Xp7oO4md/mu
    ...
    9wIDAQAB
    -----END PUBLIC KEY-----
\end{lstlisting}
\noindent Note there's no \textit{RSA} word the~header and~the~footer.
