\newsection{Certification Authority}
\index{Certification authority}
\label{certificationauthority}
It's~something (typically some trusted and~state certified company) which can~verify that public keys used for~\hyperref[asymmetriccryptography]{asymmetric cryptography} and~\hyperref[electronicsignature]{electronic signatures} really belong to~users they claim (a~public key identifies a~user).
In~standard asymmetric encryption it~serves for~senders to~verify that they really encrypt messages with public keys they got from receivers.
In~electronic signature it~serves for~receivers to~verify that public keys they use for~the~signature verification really belong to~senders they get messages from.

Certification authorities are usually paid, that sucks.
An alternative can~be key sharing servers used~by some dedicated software like \hyperref[pgpgpg]{PGP} (this is commercial software though) or~\hyperref[pgpgpg]{GPG} (that's true free).
