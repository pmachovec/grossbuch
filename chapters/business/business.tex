\newchapter{Business}
\index{Business}
\index{Enterprise}
\label{business}
Business is an~organization or~economic system where goods and~services are~exchanged for~one~another or~for~money.
Every business requires some form of~investment and~enough customers to~whom its output can~be sold on~a~consistent basis in~order to~make a~profit.
Businesses can~be privately owned, state owned or~not~for~profit.
A~business created for~making profit can~be also called \textit{enterprise}.

\newsection{Line of Business (LOB)}
\index{Line of business}
\index{LOB}
This term denotes a~products or~services offered by a~\hyperref[business]{business} or~manufacturer.
For~example, a~line of~business of~a~power plant is~electricity.
A~line of~business of~a~brewery is~beer.
And~a~line of~business of~China is~everything.

A~line of~business application is one of~the~set of~critical computer applications that are~necessary for~running an~enterprise system.
LOB~applications are~usually large programs that contain a~number of~integrated capabilities and~tie~into databases and~database management systems.
For~example, if~a~power plant has~one central software controlling the~whole plant, the~software is a~LOB~application of~that~plant.

\newsection{Stakeholder}
\index{Stakeholder}
A~stakeholder is either an~individual, a~group or~an~organization who is impacted by the~outcome of~a~project.
They have an~interest in the~success of~the~project, and~can~be within or~outside the~organization that is sponsoring the~project.
Stakeholders are~all~people somehow participating in~the~project, from the~lowest workers, over bosses to~customers.
Stakeholders can~have a~positive or~negative influence on~the~project.
Usually the~more important a~stakeholder is~considered to~be in the~official hierarchy, the~more negative influence he~has.
\newpage

\newsection{Business Logic}
\index{Business logic}
\index{Domain logic}
\label{businesslogic}
Business logic, also called \textit{domain logic}, is~a~part of~a~\hyperref[applicationprocessprogramservicethread]{program} implementation that handles tasks the~program is actually created~for.
It's~basically the~code handling data transfers between a~user interface and~a~database, usually also performing some computations with these data.

For~example consider a~super modern wage management system.
Basic wages and~expenses are~inserted manually to~this system for~each employee.
Based on inserted values the~system does the~following:
\begin{itemize}
    \item For~each employee subtract expenses from the~basic wage.
    \item Subtract another 10\,\% of~the~previous result and~send it to~the~account of~the~boss.
    \item Send the~remaining money to~the~account of~the~employee.
    \item If~expenses are~higher than the~wage in~the~first step, send threatening email to~the~employee and~mark this employee as~unreliable.
    \item If~an~employee is~already marked as~unreliable, send him more vicious threatening email and~send even the~remaining money to~the~account of~the~boss.
\end{itemize}
\noindent The~code performing these points is the~business logic of~the~wage management \mbox{system}.

\newsection{Capital}
\index{Capital}
\label{capital}
Each~company needs money and/or~valuable tools to~do its business.
And~this money and~tools are called \textit{capital}.
Company aims to~use the~capital to~make profit.

When the~company's business is successful, the~company earns money.
When earned money is~higher than company's costs of~operation, the~company makes profit.
Earned money can~be eventually used to~buy more and/or~better tools to~do the~business.
Either way the~capital grows.
But~when the~company does bad business (which is often the~case), earned money is~lower than costs of~operation and the~capital decreases.

In~most of countries some minimal capital to~establish a~company is enforced by~law.
Also, some companies, and~even big and~successful, sometimes need more capital to~do the~business better and~make quicker and~higher profit.
And~instead of borrowing money, companies emit \hyperref[shares]{shares} to~gain capital.
\newpage

\newsection{Shares}
\index{Share}
\index{Dividend}
\label{share}
A~share is a unit of~a~company \hyperref[capital]{capital} owned by~someone else.
This someone, who~owns the~unit, is called \textit{shareholder}.
Simply said a~shareholder owns a~small part of~a~company.
By~buying a~share the~shareholder gives his money to~increase the~\hyperref[capital]{capital} of~the~company.
And~the~question is obvious.
Why~would anyone give his money to~a~company capital?

Companies motivate shareholders to~buy its shares by so~called \textit{dividends}.
A~dividend is an~amount of~money paid to~shareholders, typically once per~year.
When the~company's business has~been successful for~last year, the~company uses some part of~the~total profit for~dividends.
Each~shareholder will get a~portion of~this profit part -- a~dividend -- based on~the~number of~shares he~has.
If~the~dividend is higher than the~money he~had~paid for~his shares, then it~was a~successful share trade.

\newsubsection{Share VS Stock}
\index{Stock}
\index{Stockholder}
Share and~stock are very close synonyms.
In~modern English the~plural \itq{shares} is used almost interchangeably with \itq{stock}.
Also, the~term \itq{stockholder} is~used for~someone, who~owns stock.

A~slight difference between these two terms exists in~the~level of~generality.
When talking about a~concrete company, shares are used.
For~example:
"I'm~a~shareholder, I~own shares of~Microsoft."
When talking in~general, stock is used.
For~example:
"I'm~a~stockholder, I~own stock."

\newsection{Stock Market}
\index{Stock market}
\hyperref[share]{Dividends} aren't the~only way how~to~get money by \hyperref[share]{shares} ownership.
The~other way, and~actually much more known (from movies), is~(buying~and) selling owned shares in~stock markets.
When a~share has some price in~a~stock market, it~means that there's some idiot willing to~pay the~price for~the~share and~hoping to~get more from dividends or~to~sell it for~more to~another idiot.
That's the~whole mechanism of~stock markets.
Examples of~stock market idiots are crazy individuals, banks, all~sorts of~financial and~investment groups, mafia~etc.

There are many stock markets.
Each~has specific conditions for~companies and~shareholders to~fulfil to~be~allowed to~do share transactions.
The~bigger the~market is, the~better chance to~gain \hyperref[capital]{capital} for~companies and~to~earn money for~shareholders.
But~of~course, bigger markets usually have stricter conditions for~participants.
Many countries have a~big stock market named after the~capital city, like the~stock market of~London, Paris or~Beijing.
The~biggest and~most important stock market in~the~World is, of~course, the~one in~Wall~Street in~New~York.
\newpage

\newsection{ERP VS EPM}
\index{ERP}
\index{Enterprise Resource Planing}
\index{EPM}
\index{Enterprise planning management}
These abbreviations both denote a~type of~enterprise software.
Each~enterprise needs both nowadays to~be~able to~work properly.
\begin{itemize}
    \itembf{ERP} stands for~\itq{Enterprise Resource Planning}.
            It's~a~software for~managing transactions and~resources of~an~enterprise.
            Resources are all the~stuff an~enterprise needs to~work, like money (the~most important resource), available goods in~warehouses, employees etc.
            ERP~systems typically change records in~databases according to~transactions.
    \itembf{EPM} stands for~\itq{Enterprise Planning Management}.
            It's~a~software that serves for~automation of~financial processes in~an~enterprise, like budgeting, future financial planning, finance flow forecasting, creating financial reports, taxing and~many others.
            EPM~systems theoretically only read financial data from databases (they're written by an~ERP) and~perform complex computations with them.
            But~if~ERP and~EPM have separate databases, some data--write functionality is also necessary.
\end{itemize}

