\newsection{Business Logic}
\index{Business logic}
\index{Domain logic}
\label{businesslogic}
Business logic, also called \textit{domain logic}, is~a~part of~a~program implementation that handles tasks the~program is actually created~for.
It's~basically the~code handling data transfers between a~user interface and~a~database, usually also performing some computations with these data.

For~example consider a~super modern wage management system.
Basic wages and~expenses are~inserted manually to~this system for~each employee.
Based on inserted values the~system does the~following:
\begin{itemize}
    \item For~each employee subtract expenses from the~basic wage.
    \item Subtract another 10\,\% of~the~previous result and~send it to~the~account of~the~boss.
    \item Send the~remaining money to~the~account of~the~employee.
    \item If~expenses are~higher than the~wage in~the~first step, send threatening email to~the~employee and~mark this employee as~unreliable.
    \item If~an~employee is~already marked as~unreliable, send him more vicious threatening email and~send even the~remaining money to~the~account of~the~boss.
\end{itemize}
\noindent The~code performing these points is the~business logic of~the~wage management \mbox{system}.
