\newchapter{Object--Oriented Programming}
\index{Object--oriented programming}
\index{OOP}
\index{Object}
\label{objectorientedprogramming}

Object--Oriented Programming, often denoted by the~abbreviation \mbitq{OOP}, is~a~programming paradigm in~which data are~encapsulated into objects together with tools for~manipulating these data.
An~object holds data in~fields (also called attributes) and~provides methods (also called functions) for~manipulating with these data (typically from other objects).
\hyperref[java]{Java} and~most other languages using the~object paradigm like C\# or C++ are~not~pure object languages, because they contain \hyperref[javaprimitivetypes]{primitive} \hyperref[datatypes]{data types} like \textit{integer} or~\textit{boolean}.
In~pure object languages all~data (each variable) must be object.
An~example of a~pure object language is Smalltalk, and~there is~probably no~other.
\hyperref[kotlin]{Kotlin} doesn't allow primitives in~the~written code, but~the~compiled \hyperref[javabytecode]{bytecode} can~contain them.
Therefore, some "experts" may argue that Kotlin isn't a~pure object language.

\warning Object--oriented programming is~not contradictory to~\hyperref[imperativeprogramming]{imperative programming}, nor~even to~\hyperref[declarativeprogramming]{declarative programming}.

\inputsection{oop}{basic_concepts}
\inputsection{oop}{variance}
\inputsection{oop}{data_structures}
