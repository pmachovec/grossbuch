\newsection{Data Structure}
\index{Data structure}
\label{datastructure}
In~OOP, a~data structure is a~group of~values (data) somehow stored and~organized together, enabling to~access and~use stored data.
Theoretically, any~instance of any~class that has at~least one publicly accessible \hyperref[variablefieldproperty]{property}, is~a~data structure.
But~there's no exact definition.
It~always depends on~the~context and~stupidity of~the~author.
Sometimes, when talking about data structures, the~author means arrays and~collections, sometimes he means (instances~of) classes that have public \hyperref[variablefieldproperty]{property} variables or~getters and~setters for~all private \hyperref[variablefieldproperty]{property} variables, sometimes he means database tables~etc.

Some object--oriented programming languages have special syntax for~simpler defining of~classes that contain only publicly accessible \hyperref[variablefieldproperty]{property} and~no~other logic.
Instances of~these classes are~then considered to~be data structures.
For~example, there are data classes in~Kotlin, \mbox{or \textit{structs}} in~C++.
