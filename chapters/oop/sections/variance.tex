\newsection{Variance}
\index{Variance}
OOP~languages usually support more complex \hyperref[datatypes]{data types} (classes) that somehow contain simpler types.
For~example, Java supports \hyperref[javaarray]{arrays} or~\hyperref[javagenerics]{generic types}.
Variance describes how \hyperref[inheritance]{inheritance} and~subtype relations between two complex types are~affected by~relations between contained simpler types.

\newsubsection{Covariance}
\index{Covariance}
\label{covariance}
We~say that a~complex structure \mbox{is \textit{covariant}}, if~it~preserves the~subtype relation of~contained simpler types.
For example, in~Java, \hyperref[javaarray]{arrays} are covariant.
\mbox{I.e., when \itq{B}} is a~subclass \mbox{of \itq{A}}, then~also \mbitq{B[]} is a~subclass \mbox{of \itq{A[]}}.
In~\hyperref[javagenerics]{Java generics}, covariance is supported only when using \hyperref[javagenericswildcards]{wildcards} \mbox{and the \itq{extends}} keyword.
For~example, \itq{List<B>} is a~subclass \mbox{of \itq{List<? extends A>}}.

\newsubsection{Contravariance}
\index{Contravariance}
We~say that a~complex structure \mbox{is \textit{contravariant}}, if~it~reverses the~subtype relation of~contained simpler types.
This~is quite unintuitive and~an~example is harder to~find.
\hyperref[javagenerics]{Java generics} support contravariance when using \hyperref[javagenericswildcards]{wildcards} \mbox{and the \itq{super}} keyword.
For~example, \mbox{when \itq{B}} is a~subclass \mbox{of \itq{A}}, \mbox{then \itq{List<A>}} is a~subclass \mbox{of \itq{List<? super B>}}.

\newsubsection{Bivariance}
\index{Bivariance}
We~say that a~complex structure \mbox{is \textit{bivariant}}, if~it~both preserves and~reverses the~subtype relation of~contained simpler types.
This~is mostly a~theoretical thing not~occurring in~any~mainstream OOP language.
For~example, \hyperref[javaarray]{Java arrays} \highlight{would be} bivariant, \mbox{if with \itq{B}} being a~subclass \mbox{of \itq{A}}, \mbitq{B[]} was a~subclass \mbox{of \itq{A[]}} and~even \mbitq{A[]} was a~subclass \mbox{of \itq{B[]}}.

\newsubsection{Invariance}
\index{Invariance}
\label{invariance}
We~say that a~complex structure \mbox{is \textit{invariant}}, if~there's no subtype relation among two such structures containing simpler types, that~do~have a~subtype relation among them.
For~example, in~Java, simple \hyperref[javagenerics]{generics} without \hyperref[javagenericswildcards]{wildcards} are~invariant.
\mbox{When \itq{B}} is a~subclass \mbox{of \itq{A}}, \itq{List<B>} has~no~subtype relation \mbox{with \itq{List<A>}}.
