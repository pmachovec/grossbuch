\newsection{Deleting Packages from GitHub}
\label{githubdeletepackage}
GitHub can~be used as~a~package repository, also called \mbit{registry}, for~various \hyperref[packagemanager]{package managers}.
Package managers usually provide tools how~to~publish a~package to~GitHub.
But~they generally don't allow to~remove a~package once published.
Deleting a~published package is~considered to~be a~bad practice, therefore, GitHub makes it as~complicated as~possible.
You~must send special queries to~the~the~GitHub \hyperref[graphql]{GraphQL}~\hyperref[api]{API} -- \hyperref[https]{HTTPS} POST requests to~the~URL \mbitql{https://;;api.;;github.;;com/;graphql}.
The~best approach is to~use \href{https://www.getpostman.com/}{Postman}, you~can~also use~\hyperref[curl]{cURL}.

Each~package can~be published many times in~many versions.
At~first you~need GitHub internal ID of~the~version of~the~package you~want to~delete.
For~this you~need a~token with package reading right for~your GitHub account.
Put~the~following key--value pair \mbitqls{Authorization:;;;bearer;;;THE\_TOKEN\_HASH} to~the~request header and~write the~request body like this:
%! language = TEXT
\begin{lstlisting}
    query {
      repository(owner:"USER_NAME",name:"REPOSITORY_NAME") {
        registryPackages(first:1,name:"PACKAGE_NAME") {
          nodes {
            versions(first:10) {
              nodes {
                id,version
              }
            }
          }
        }
      }
    }
\end{lstlisting}

\noindent If~the~version to~delete is older than 10 newest, you~can~increase the~inner \mbitq{first} argument.
You~can~also use \mbitq{last} instead, in~that case the~listing goes from oldest versions.
The~response to~the~request contains similar \hyperref[json]{JSON} string, from which it's~possible to~read the~GitHub internal~ID\@.
It's~a~\hyperref[base64]{Base64} string.

With~the~ID it's~possible to~construct the~request for~actual package deletion.
For~this you need a~token with package deleting right.
Add~the~token to~the~request header as~previously (\mbitqls{Authorization:;;;bearer;;;THE\_TOKEN\_HASH}).
Add~another key--value pair \mbitqls{Accept:;;;application/;;vnd.;;github.;;package-;;deletes-;;preview+;;json} to~the~header and~write the~body like this:
%! language = TEXT
\begin{lstlisting}
    mutation {
      deletePackageVersion(input: {
        packageVersionId:"THE_ID_FROM_PREVIOUS_REQUEST"
      }) {
        success
    }
}
\end{lstlisting}

\noindent After submitting you~should~get a~response confirming successful package deletion.

\warning The~deleted version is blocked from another use and~can't~be unblocked.
I.e.,~you~will never be able to~publish a~package of~the~same name and~version to~the~same repository.
The~only possible workaround is~to~delete the~whole repository.
And~that can~be problematic if~you~have other stuff there.
