\newsection{Hooks}
\index{Git hooks}
\index{Hooks in Git}
Git hooks are~scripts that can~be run automatically before, after or~with a~certain Git event like commit or~push.
Hooks run before events (pre--commit, pre--push,~\dots) serve mainly for~processed (committed, pushed,~\dots) code verification.
They can, for~example, run unit tests.
If~a~pre--event hook ends with a~non--zero exit value (which happens when unit tests triggered by~it fail), the~corresponding event isn't executed.
Hooks run after events (post--commit, post--merge,~\dots) can~be used for~cleaning temporary data, automatic project publishing to~other places etc.
Hooks run with events (prepare--commit--msg, update,~\dots) are~used for~committed data automatic alteration (creating automatic commit message) and~further verification (that's basically useless).

Hook scripts exist separately for~each local copy of~a~repository (i.e.,~per~user).
They're files with names corresponding to~hook names without any~extension (\textit{pre--commit}, \textit{pre--push},~\dots).
They're by~default stored in~the~folder \textit{ROOT\_FOLDER/.git/hooks}, but~the~location can~be changed by~the~\textit{core.hooksPath} \hyperref[gitconfig]{configuration}, either relatively to~the~root folder (typically local configuration) or~absolutely (typically global configuration).
When you create a~repository (cloning, \hyperref[initrepo]{local initialization}), there's a~template for~each possible hook (called \textit{HOOK\_NAME.sample}) in~the~default folder.

Hooks can~be directly written in~any~\hyperref[scriptinglanguages]{scripting language}.
If~you~wanted to~use a~non-scripting language for~a~hook, you~would have~to implement it as~a~separate program and~then call~it from the~hook script.

A~hook language is~specified by~a~\hyperref[shebang]{shebang}.
For~the~hook to~work in~a~certain environment you~need an~\hyperref[compiledinterpretedlanguages]{interpreter} for~the~hook language installed in~that~environment.
As~this can't~be guaranteed for~all users, it's~a~good practice to~write hooks in~\hyperref[shbash]{Bash}, because that one is~understood even  by~non--\hyperref[linux]{Unix/Linux} Git installations (there's an~interpreter included).
To~specify that a~hook should~be executed in~Bash, put \textit{\#!/bin/sh} as~the~hook \hyperref[shebang]{shebang} to~the~first line (even in~non--Unix/Linux systems).

\newsubsection{Sharing Hooks}
This can~be tricky, as~the~\textit{ROOT\_FOLDER/.git} folder can't~be committed to~a~repository.
You~can~commit hooks in~a~different folder and~then configure the~\textit{core.hooksPath} \hyperref[gitconfig]{configuration}.
To~ensure that everybody will~have the~hooks path configured as~you~want, you~can~set the~building software (Gradle, Grunt) to~set~up the~hooks path in~the~first~run.
