\newsection{Configuration}
\index{Git configuration}
\label{gitconfig}
The~behavior of~Git can~be adjusted by~the~\textit{git~config} command.
Whenever you run the~command, some key--value configuration pair is~stored to~a~configuration file under~a~configuration section.
The~syntax of~the~command is \itq{git config section.key value}.
Values can~consist of~more words, in~that~case they must be enclosed in~apostrophes or~quotes.
Sections can~be further specified in~subsections, the~syntax is \itq{git config section.subsection.key value}.
Further separation to~subsubsections is~not~possible.
Everything between the~first and~the~last dot is~considered as~one~subsection name, even~if~it~contains other dots.

\example[simple configuration file]:
%! language = TEXT
\begin{lstlisting}
    [section1]
        key1 = value1
        key2 = value2
    [section2 "subsection"]
        key3 = value3
    [section2]
        key4 = value4
\end{lstlisting}
\newline

\enlargethispage{-8mm}
\noindent You~can set either a~local or~a~global configuration value.
The~only difference in~the~syntax is the~\mbitq{-{}-global} option just after the~command when setting a~global configuration (\textit{git config \mbox{-{}-global} section.key value}").
Local configuration values are~valid only in a~single repository folder.
The~command without the~\mbitq{-{}-global} option can~be run only from a~repository folder (not~necessarily the~root, it~works even from~its subfolders).
Local configuration entries are~stored to~the~file \mbit{ROOT\_FOLDER/.git/config}.
Global configuration values are~valid in all repository folders in~the~system where Git is~installed.
The~command with the~\mbitq{-{}-global} option can~be run from anywhere.
Global configuration entries are~stored to~the~file \mbit{HOME\_FOLDER/.gitconfig}.
You can also directly edit configuration files with a~text editor.
But~that should be done with caution, if~you~screw the~formatting somehow, your Git can~stop working.
When you want to~remove some configuration, use the~\mbitq{-{}-unset} option.

\warning Local configurations override global configurations.

\newsubsection{Basic configuration}
\label{gitbasicconfiguration}
When you install Git, you should set at~least following configurations:
\begin{itemize}
    \item User name (usually with apostrophes or~quotes):
        %! language = TEXT
\begin{lstlisting}[frame=no, gobble=12]
            git config --global user.name '...'
        \end{lstlisting}
    \item User email:
        %! language = TEXT
\begin{lstlisting}[frame=no, gobble=12]
            git config --global user.email ...
        \end{lstlisting}
    \item Default editor:
        %! language = TEXT
\begin{lstlisting}[frame=no, gobble=12]
            git config --global core.editor ...
        \end{lstlisting}
    \item GUI encoding (without this non--ASCII characters get screwed in~Gitk):
        %! language = TEXT
\begin{lstlisting}[frame=no, gobble=12]
            git config --global gui.encoding utf-8
        \end{lstlisting}
\end{itemize}
\noindent If~your repository requires \hyperref[gitgpg]{GPG~signing}, you~must set also this:
\begin{itemize}
    \item User sign key:
        %! language = TEXT
\begin{lstlisting}[frame=no, gobble=12]
            git config --global user.signingkey ...
        \end{lstlisting}
    \item Automatic signing:
        %! language = TEXT
\begin{lstlisting}[frame=no, gobble=12]
            git config --global commit.gpgsign true
        \end{lstlisting}
\end{itemize}
