\newsection{Aliases}
\index{Git Aliases}
\index{Aliases in Git}
When you tend to~use a~same long Git command with lots of~options many times, it's~convenient to~create an~alias for~it.
Aliases are~configurations, i.e., they're stored into configuration files and~can~be set by~the~\textit{git config} command.
The~syntax is \itq{git config \mbox{-{}-global} alias.name 'value'}.
The~value doesn't contain the~keyword \textit{git}.
The~global setting is not~necessary, but~aliases are~very rarely set as~local.
Also the~value doesn't have to~be enclosed in~apostrophes when~not~containing spaces, but~that's rare when creating aliases.

\example
\noindent Consider a~very long Git command:
%! language = TEXT
\begin{lstlisting}[frame=no]
    git VERY LONG COMMAND WITH MANY OPTIONS AND VALUES
\end{lstlisting}
\noindent You set~up an~alias like this:
%! language = TEXT
\begin{lstlisting}[frame=no]
    git config --global alias.short 'VERY LONG COMMAND WITH MANY OPTIONS AND VALUES'
\end{lstlisting}
\noindent And now instead of~calling the~long command you~can just call:
%! language = TEXT
\begin{lstlisting}[frame=no]
    git short
\end{lstlisting}
\noindent When you want to~remove the~alias, you~run:
%! language = TEXT
\begin{lstlisting}[frame=no]
    git config --global --unset alias.short
\end{lstlisting}

\warning Aliases can't override existing commands.
