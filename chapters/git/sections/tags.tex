\newsection{Tags}
\index{Git tags}
\index{Tags in Git}
Tags in~Git are like aliases for~commits.
A~tag can~be~used instead of~a~hash when referencing a~commit.
It's~visible with other information by~its commit in~a~Git history showing tool.
One~commit can~have more tags.
There can't~be two same tags in~one repository, not~even in~different branches.
Tags can't contain spaces and~usage of~special characters is~also restricted.

The~basic command for~tagging is~\textit{git tag}.
In~the~basic form without any~parameter or~switch it~displays a~list of~all~tags in~the~local repository, including those from other branches.

\newsubsection{Lightweight Tags}
A~lightweight tag is just a~commit alias.
To~create a~lightweight tag use the~\textit{git~tag} command in~the~format \itq{git tag TAG\_NAME COMMIT\_HASH}.

\example
%! language = TEXT
\begin{lstlisting}[frame=no]
    git tag exampleTag 1cra2y34ha5h6
\end{lstlisting}

\noindent Now you can~use the~alias \itq{simpleTag} instead of~the~commit hash, like for~showing differences, rebasing, cherry--picking etc.
You~will also see something like this when observing the~commit in~Git~log:
%! language = TEXT
\begin{lstlisting}[frame=no]
    commit 1cra2y34ha5h6 (tag: exampleTag, origin/...)
\end{lstlisting}

\newsubsection{Annotated Tags}
\noindent An~annotated tag contains, beside a~commit alias, also information about the~author, time of~creation and~a~nonempty message.
It's~also signed with~\hyperref[pgpgpg]{GPG} when \hyperref[gitbasicconfiguration]{configured}.
To~create an~annotated tag use the~\textit{git~tag} command with the~switch~\itq{-a}.
You~can also directly specify the~message with the~switch \itq{-m}.
If~you don't do it, a~text editor window will~appear, like when committing without the~switch.

\example
%! language = TEXT
\begin{lstlisting}[frame=no]
    git tag exampleTag 1cra2y34ha5h6 -a -m "Example message"
\end{lstlisting}

\noindent Additional information of~annotated tags isn't simply visible.
To~see~it you must use the~\hyperref[gitshow]{\textit{show}} command with the~tag name.
It~doesn't work with the~commit hash.

\example
%! language = TEXT
\begin{lstlisting}[frame=no]
    git show exampleTag
\end{lstlisting}

\noindent This will show the~standard info about the~commit, but~with preceding information from the~tag:
%! language = TEXT
\begin{lstlisting}[frame=no]
    tag exampleTag
    Tagger: TAGGER NAME <tagger@email.com>
    Date:   Sat Nov 16 19:58:59 2019 +0100

    Example message

    commit 1cra2y34ha5h6 (tag: exampleTag, origin/...)
    ...
\end{lstlisting}

\noindent Annotated tag information can~be also viewed in~some Git graphical tools like~Gitk.

\note When you want to~tag the~last commit in~the~actual branch, you~don't have to specify the~commit hash, not~even for~a~lightweight~tag.

\newsubsection{Pushing Tags}
The~standard \textit{git push} command doesn't publish tags to~remote repositories, not~even when you~push a~fresh commit (a~commit not~in~the~remote repository).
Tags must be pushed separately by the~command in~the~format \textit{git push origin TAG\_NAME}.
When you have more tags, you~can push them all at~once by~specifying the~option \mbitq{-{}-tags} instead of~a~concrete tag name.
This will~push all~tags in~all branches you~have in~your local repository.

\example[pushing a~concrete tag]
%! language = TEXT
\begin{lstlisting}[frame=no]
    git push origin exampleTag
\end{lstlisting}

\example[pushing all tags from a~local repository to~a~remote repository]
%! language = TEXT
\begin{lstlisting}[frame=no]
    git push origin --tags
\end{lstlisting}

\note Receiving new tags (not~deleted or~edited, see~\hyperref[gittagsynchronization]{further}) from~remote repositories by~fetching or~pulling doesn't require any special handling.
It's~performed automatically with standard operations.

\newsubsection{Deleting Tags}
To~delete a~tag from~a~local repository use the~switch~\itq{-d}.
To~delete a~tag from~a~remote repository act like when deleting a~branch, i.e.,~either push the~tag to~origin with the~\mbitq{-{}-delete} option or~with preceding colon (overriding by~nothing).

\example{deleting local tag}
%! language = TEXT
\begin{lstlisting}[frame=no]
    git tag exampleTag -d
\end{lstlisting}

\example{deleting remote tag}
%! language = TEXT
\begin{lstlisting}[frame=no]
    git push origin exampleTag --delete
\end{lstlisting}

\example{deleting remote tag by overriding}
%! language = TEXT
\begin{lstlisting}[frame=no]
    git push origin :exampleTag
\end{lstlisting}

\note Pushing deleted tags can't~be performed in~a~batch with the~\mbitq{--tags} switch.
You~must alwasy push deleted tags one~by~one.

\newsubsection{Editing Tags}
There's no option to~rename a~tag in~one command.
You~must tag the~corresponding commit with the~new name and~then delete the~old name.
The~order can~be even reversed, but~with creating the~new tag first you can~use the~old tag as~a~commit reference instead of~the~commit hash.

\example[expect that \textit{oldTag} already exists]
%! language = TEXT
\begin{lstlisting}[frame=no]
    git tag newTag oldTag
    git tag -d oldTag
\end{lstlisting}

\noindent And,~of~course, all~newly created and~deleted tags must~be explicitly pushed to~a~remote repository.

\newsubsection{Synchronizing Tags with Remote Repositories}
\label{gittagsynchronization}
As~already mentioned, freshly fetched commits come even with tags.
But~tags added later to, or~deleted from, existing commits, that you already have in~your local repository, must~be downloaded explicitly.
To~perform the~synchronization run the~command \itq{git fetch \mbox{-{}-prune} \mbox{-{}-prune-tags}}.

\newsubsection{Force Switch (-f)}
When any tag command would result in~two different commits with the~same tag (creating a~new tag, synchronizing with a~remote repository,~\dots), the~command fails.
But~when used with the~switch~\itq{-f}, it~deletes problematic tags from other locations so~that the~current command can~be performed.

For~example, consider that in~your local repository you~have a~tag on~a~commit~\textit{A}.
In~the~remote repository the~same tag exists on~a~different commit~\textit{B}, that you~also have in~the~repository.
When you~run the~\hyperref[gittagsynchronization]{synchronization command} in~the~basic form, it~will fail complaining that it~"would clobber existing tag".
But~when you~run it with the~switch~\itq{-f}, the~tag is~removed from the~commit~\textit{A} and~appears at~the~commit~\textit{B} in~your local repository.
