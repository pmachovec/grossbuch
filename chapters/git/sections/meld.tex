\newsection{Meld}
\index{Meld}
Viewing file differences and~resolving merge conflicts in~a~Shell isn't very comfortable.
It's~better to~use some graphical tool that can visualize file differences.
And~probably the~best tool for~this is~\href{https://meldmerge.org/}{Meld}, which can~be configured as~the~default difftool and mergetool for~Git.

\newsubsection{Configuration}
Following examples expect that you already installed meld and~added it to~the~\textit{PATH} system variable.

\begin{itemize}
    \item Configure Git to~use Meld as~the~difftool:
        %! language = TEXT
\begin{lstlisting}[frame=no, gobble=12]
            git config --global diff.tool meld
        \end{lstlisting}
    \item Configure git not~to~ask every time if~running Meld is~OK on~diff:
        %! language = TEXT
\begin{lstlisting}[frame=no, gobble=12]
            git config --global difftool.prompt false
        \end{lstlisting}
    \item Configure the~order of~displayed windows on~diff:
        %! language = TEXT
\begin{lstlisting}[frame=no, gobble=12]
            git config --global difftool.meld.cmd 'meld "$LOCAL" "$REMOTE"'
        \end{lstlisting}
    \item Configure Git to~use Meld as~the~mergetool:
        %! language = TEXT
\begin{lstlisting}[frame=no, gobble=12]
            git config --global merge.tool meld
        \end{lstlisting}
    \item Configure git not~to~ask every time if~running Meld is~OK on~merge:
        %! language = TEXT
\begin{lstlisting}[frame=no, gobble=12]
            git config --global mergetool.prompt false
        \end{lstlisting}
    \item Configure the~order of~displayed windows on~merge:
        %! language = TEXT
\begin{lstlisting}[frame=no, gobble=12]
            git config --global mergetool.meld.cmd 'meld "$LOCAL" "$BASE" "$REMOTE" --output "$MERGED"'
        \end{lstlisting}
\end{itemize}

\note For~the~last configuration the~only part you~should ever change is~the~middle window.
The~value \textit{\$BASE} means the~merged file as~it~was when~the~remote branch was~created.
Another possibility for~that~one is~\textit{\$MERGED}, which means the~messy merged file at~the~beginning of~merging, i.e., containing special symbols inserted by~Git, which mark the~conflict.
Using those values is~not~restricted, theoretically you can put everything everywhere.
\newpage

\newsubsection{Usage}
You~must use \textit{git difftool} for~viewing file differences and~\textit{git mergetool} for~merging.
Short commands \textit{diff} and~\textit{merge} will always use the~Shell console and~it~can't~be changed.

\begin{itemize}
    \item Show differences of~a~file from the~last committed version:
        %! language = TEXT
\begin{lstlisting}[frame=no, gobble=12]
            git difftool PATH_TO_THE_FILE
        \end{lstlisting}
    \item Show differences of~a~file from some old specific commit:
        %! language = TEXT
\begin{lstlisting}[frame=no, gobble=12]
            git difftool OLD_COMMIT_HASH PATH_TO_THE_FILE
        \end{lstlisting}
    \item Show differences of~a~file between two~commits:
        %! language = TEXT
\begin{lstlisting}[frame=no, gobble=12]
            git difftool FIRST_COMMIT_HASH SECOND_COMMIT_HASH PATH_TO_THE_FILE
        \end{lstlisting}
    \item Merge a~file in~a~conflict:
        %! language = TEXT
\begin{lstlisting}[frame=no, gobble=12]
            git mergetool PATH_TO_THE_FILE
        \end{lstlisting}
\end{itemize}

\warningnonl If~you use a~path to~a~folder instead of~a~file, the~triggered tool will start for~all~files in~the~folder and~its~subfolders one~by~one.
When~you~close one~file, another immediately opens.
This~is irritating when you~have many files in~the~folder.
You~can~fall to~this trap especially when you~accidentally trigger merging without specifying a~file.
When running difftool, Meld can graphically show differences among files in~both folder versions.
To~do~it run the~\textit{difftool} command with the~\mbitq{-{}-dir--diff} option.
When merging, there's nothing you~can~do about~it.
Just trigger the~\textit{mergetool} command with a~path to~a~specific file every time.

\warning For~the~\textit{difftool} command variables \textit{\$LOCAL} and~\textit{\$REMOTE} have a~different meaning from what can~seem obvious.
\textit{\$LOCAL} means "before the~change" and~\textit{\$REMOTE} means "after the~change".
Therefore, when you run the~command for~a~file that differs from a~repository version (the~first usage example), i.e.,~is~changed locally, you~will~see that local version in~the~right window, which is the~position of~\textit{\$REMOTE} according to~the~configuration.
The~same holds when you display differences of~a~file from some old specific commit(the~second usage example) -- the~actual version of~the~file will~be displayed in~the~right window.
Get~used to~it, don't switch those variables in~the~configuration, otherwise you can face unexpected behavior.
For~example, comparing a~file between two commits (the~third usage example) shows versions of~the~file in~the~order opposite of~the~order of~used commit hashes.

\note For~the~\textit{mergetool} command variables \textit{\$LOCAL} and~\textit{\$REMOTE} behave as~expected -- \textit{\$LOCAL} means your local branch version, \textit{\$REMOTE} means the~other merged branch.
Just remember for~the~malicious branching when \hyperref[gitrebase]{rebasing}.
\newpage
