\newsection{Special Types}

\newsubsection{Unit}
\index{Unit}
\label{kotlinunit}
This is actually an~\hyperref[kotlinobject]{object}.
It's~an~equivalent of Java's \textit{void}.
Any~function, that doesn't return any~value, actually returns the~one existing instance \mbox{of \textit{Unit}}.
Not~specifying a~return value type by a~function is equivalent of~specifying \textit{Unit}, \mbox{i.e., \textit{"fun}} \mbitls{(...);;;\{...\}"} is equivalent of \mbitqls{fun;;;(...) : Unit;;;\{...\}}.

\newsubsection{Nothing}
\index{Nothing}
\label{kotlinnothing}
This class represents the~type of~a~value that can~never exist.
For~example, when a~function return value type is \textit{Nothing}, this function must never end, it~must always throw an~exception or~trigger infinite loop (compilers can~detect this).
There's no equivalent \mbox{of \textit{Nothing}} class in~Java.

\newsubsection{Any}
\index{Any}
This is a~supertype of~any other class in~Kotlin.
It's~an~equivalent of Java's \textit{Object}.
\newpage
