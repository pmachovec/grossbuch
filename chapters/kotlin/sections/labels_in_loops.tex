\newsection{Labels in Loops}
\index{Label}
If~you~have nested loops and~want to~\textit{break} or~\textit{continue} the~outer one from~the~inner one, you~need to~use labels.
The~outer loop keyword (\textit{for}, \textit{while}, \dots) must~be preceded with~a~label in~the~syntax \mbitq{LABEL\_NAME@}, without any~space, the~keyword follows immediately after the~at~sign.
Then, inside the~inner loop, you~use \mbitq{break@LABEL\_NAME} or~\mbitq{continue@LABEL\_NAME}[.]
\newline

\example[break of outer loop from inner loop]
%! language = TEXT
\begin{lstlisting}[language=Kotlin]
    labelName@for (i in WHATEVER) {
        for (j in WHATEVER) {
            if (...SOME CONDITION...) {
                break@labelName
            }

            ...
        }
    }
\end{lstlisting}
\newline

\note Opposite to~\hyperref[kotlinreturnatlabel]{returning at~labels in~lambda functions} there are no implicit label names here.
The~label specification \mbitq{LABEL\_NAME@} must~be used every time.
\newline

\warning \textit{Break} and~\textit{continue} keywords don't work when looping over collections with a~\hyperref[kotlinlambda]{lambda function}.
For~such cases \hyperref[kotlinreturnatlabel]{returning at~labels} must~be used.
\newpage
