\newsection{Destructuring Declarations}
\index{Destructuring declaration}
\index{ComponentN}
\label{kotlindestdecl}
In~Kotlin it's~possible to~declare multiple variables and~fill them with~values in~one~command by~assigning a~class instance to~a~group of~those variables.
That's called \textit{destructuring declaration}.
The~syntax is \mbitqls{val/var (var1, var2, \dots);;;=;;;classInstance}.
\hyperref[datatypes]{Data types} of~individual variables are~inferred automatically.

To~make this possible, the~class must provide so~called \mbit{componentN} \hyperref[kotlinoperator]{operators} corresponding to~the~number of~variables available for~the~destructuring declaration.
For~example, if~a~three--variable destructuring declaration should~be possible for~a~class instance, the~class must provide operators \mbitq{component1}, \mbitq{component2} and~\mbitq{component3}.

By~default, only \hyperref[kotlindataclass]{data classes}, pairs and~triples provide \mbit{componentN} operators based on~their members (see~further).
To~enable destructuring declarations for~a~general class or~\hyperref[kotlinobject]{object}, you~must use \hyperref[kotlinoperatoroverload]{operators overloading}.
Overloaded operator functions don't~take parameters, but~they should return values.
Defining an~operator without a~return type doesn't make sense, but~it's possible.
The~corresponding variable in~the~destructuring declaration then~gets assigned the~\mbit{\hyperref[kotlinunit]{Unit}} instance.

\emten
\example[destructuring declaration with~a~general class]
%! language = TEXT
\begin{lstlisting}[language=Kotlin, title={Class supporting three \textit{componentN} operators}]
    class SimpleClass {
        operator fun component1(): String {
            return "simple string"
        }

        operator fun component2(): Int {
            return 1
        }

        operator fun component3() {
            // Makes no sense, but possible
            ...
        }
    }
\end{lstlisting}
%! language = TEXT
\begin{lstlisting}[language=Kotlin, title={Destructuring declaration behavior}]
    val simpleClass = SimpleClass()

    // Destructuring declaration of three variables
    val (one, two, three) = simpleClass

    // The same with direct instance creation
    val (four, five, six) = SimpleClass()

    println(one)   // Will print "simple string"
    println(two)   // Will print "1"
    println(three) // Will print "kotlin.Unit"
\end{lstlisting}

\example[destructuring declaration as~\hyperref[kotlinextension]{extensions} of~String]
%! language = TEXT
\begin{lstlisting}[language=Kotlin, title={Extending \textit{String} class with two \textit{componentN} operators}]
    package somepackage.subpackage

    operator fun String.component1(): Double {
        return 3.14
    }

    operator fun String.component2(): Boolean {
        return true
    }
\end{lstlisting}
%! language = TEXT
\begin{lstlisting}[language=Kotlin, title={Destructuring declaration behavior}]
    import somepackage.subpackage.component1
    import somepackage.subpackage.component2

    // Destructuring declaration of two variables from a string
    val (one, two) = "simple string"

    println(one) // Will print "3.14"
    println(two) // Will print "true"
\end{lstlisting}
\newline

\noindent You~don't have to~always use all \mbit{componentN} operators that~a~class is~providing.
If~you~want to~perform destructuring declaration only with~first few \mbit{componentN} operators, it's~possible to~declare only the~corresponding number of~variables.
If~you don't need all \mbit{componentN} operators, but~needed are "behind" not~needed, you~can~skip those not~needed by putting underscore instead of a~variable name.
In~following examples consider the~simple class from the~beginning.

\example[using first two \textit{componentN} operators of three available]
%! language = TEXT
\begin{lstlisting}[language=Kotlin]
    // Provides three componentN operators
    val simpleClass = SimpleClass()

    // Destructuring declaration of two variables
    val (one, two) = simpleClass

    println(one)   // Will print "simple string"
    println(two)   // Will print "1"
\end{lstlisting}

\example[using only third \textit{componentN} operator]
%! language = TEXT
\begin{lstlisting}[language=Kotlin]
    // Provides three componentN operators
    val simpleClass = SimpleClass()

    // Destructuring declaration of one variable
    val (_, _, three) = simpleClass

    println(three) // Will print "kotlin.Unit"
\end{lstlisting}
\newline

\noindent \mbit{ComponentN} operators ordering in~the~class body doesn't matter.
However, the~line of~numbers must~be consecutive.
If~you omit a~number in~the~line, destructuring declarations will~work only up to~the~last correctly declared operator.
For~example, if~you define operators \mbit{component1}, \mbit{component2} and~\mbit{component4}, but not~\mbit{component3}, only one--~or~two--variable destructuring declarations will~be possible.

\example
%! language = TEXT
\begin{lstlisting}[language=Kotlin, title={Class with \textit{componentN} operators jumbled and \textit{component3} missing}]
    class SimpleClass {
        operator fun component2(): ... {
            ...
            return ...
        }

        operator fun component4(): ... {
            ...
            return ...
        }

        operator fun component1(): ... {
            ...
            return ...
        }
    }
\end{lstlisting}
%! language = TEXT
\begin{lstlisting}[language=Kotlin, title={Destructuring declaration behavior}]
    // Provides only two componentN operators
    val simpleClass = SimpleClass()

    // Destructuring declaration of three variables - doesn't work
    // val (one, two, three) = simpleClass

    // Destructuring declaration of two variables
    val (one, two) = simpleClass
\end{lstlisting}

\newsubsection{With data classes}
\hyperref[kotlindataclass]{Data classes} support \mbit{componentN} operators based on~\hyperref[kotlinprimaryconstructor]{primary constructor} parameters.
For~example, when~the~constructor has~three parameters, then~the~data class automatically provides \mbit{component1} returning the~first constructor argument, \mbit{component2} returning the~second constructor argument and~\mbit{component3} returning the~third constructor argument.

As~in~many other features of~data classes, eventual other fields of~classes aren't supported by~default.
However, it's~possible to~define additional \mbit{componentN} operators in~the~class body and~these can~return other fields or~anything else.
The~numbering must start above operators provided automatically based on~the~primary constructor, otherwise, the~code isn't compilable.
For~example, when~the~constructor has~three parameters, the~first operator that~can~be defined is \mbit{component4}.

\example[destructuring declaration with~a~data class]
%! language = TEXT
\begin{lstlisting}[language=Kotlin, title={Data class with two parameters in the primary constructor and additional \mbit{component3}}]
    data class SimpleDataClass(
        stringParam: String,
        intParam: Int
    ) {
        val booleanProperty = true

        // Not allowed
        // operator fun component2(): ... { ... }

        operator fun component3(): Boolean {
            return booleanProperty
        }
    }
\end{lstlisting}

%! language = TEXT
\begin{lstlisting}[language=Kotlin, title={Destructuring declaration behavior}]
    val simpleDataClass = SimpleDataClass("simple string", 1)

    // Destructuring declaration of three variables
    val (one, two, three) = simpleDataClass

    println(one)   // Will print "simple string"
    println(two)   // Will print "1"
    println(three) // Will print "true"
\end{lstlisting}

\newsubsection{With pairs and triples}
Tuples support destructuring declaration in~quite intuitive and~expectable way.
\mbit{Pair} provides operators \mbit{component1} and~\mbit{component2} returning corresponding pair instance members, \mbit{Trippe} has the~same and~additionally provides \mbit{component3}.

\example[destructuring declaration with \textit{Pair} and \textit{Triple} instances]
%! language = TEXT
\begin{lstlisting}[language=Kotlin]
    val simplePair = Pair("pair one", "pair two")
    val (pairOne, pairTwo) = simplePair

    val simpleTriple = Triple("triple one", "triple two", "triple three")
    val (_, _, tripleThree) = simpleTriple
\end{lstlisting}

\newsubsection{In \textit{for} loops}
\label{kotlindestdeclforloop}
When you have an~iterable (array, collection, \hyperref[kotlinsequence]{sequence}), you~can~iterate over it in a~\mbit{for}~loop with~the~\mbitq{in} operator.
And~when the~iterable members type class supports destructuring declaration, you~can, instead of declaring one variable representing one whole member, provide destructuring declaration with~a~correct number of variables.
This way, in~each iteration step, you~have access to~\mbit{componentN} operators results for~the~actual member.
Remember that you can~use one x--or other.
If~you~need to~use something else than \textit{componentN} operators results on~the~actual iteration member, then you can't~use destructuring declaration.

\example[iterating over a list of pairs]
%! language = TEXT
\begin{lstlisting}[language=Kotlin]
    val pairList: List<Pair<CLASS1, CLASS2>> = ...INIT THE LIST...

    // Iterating over the list of pairs
    for ((_, second) in pairList) {
        // Accessing 2nd member of the current pair
        ...DO SOMETHING WITH second...

        // Current pair as a whole not available here
        // pair.toString()
    }
\end{lstlisting}
\newline

\noindent Iteration over a~map with destructuring declaration is~also possible.
Each~map is an~iterable collection of~\mbitq{Map.Entry<K, V>}[.]
This type provides \mbit{component1} operator returning the~key and~\mbit{component2} operator returning the~value.

\example[iterating over a map]
%! language = TEXT
\begin{lstlisting}[language=Kotlin]
    val theMap: Map<CLASS1, CLASS2> = ...INIT THE MAP...

    // Iterating over the list of pairs
    for ((key, value) in theMap) {
        // Accessing key and value of the current entry
        ...DO SOMETHING WITH key AND value...
    }
\end{lstlisting}

\newsubsection{In lambda functions}
When you have a~function \hyperref[kotlinanonymousfunction]{taking other function as~a~parameter}, then you can~specify the~other function as~a~\hyperref[kotlinlambda]{lambda function}.
And~when that other function has a~parameter of~a~type supporting destructuring declarations, you~can, instead of declaring one variable representing the~parameter value, provide destructuring declaration with~a~correct number of variables in~the~lambda parameters section (before the~arrow).
This~way you~have access to~\mbit{componentN} operators results of~the~parameter in~the~lambda body.
When the~lambda function has more parameters, it's possible to~use destructuring declaration for~one and~standard variables for~others at~any~place.
It~can~be also used on~iterable functions taking lambda like \mbit{forEach}[,] \mbit{filter}[,] \mbit{map}~etc.
Remember that even this is \mbox{x--or.}
If~you~need to~use something else than~\textit{componentN} operators results on~the~parameter, then~you can't~use destructuring declaration.

\example[destructuring declaration in lambda function parameters]
%! language = TEXT
\begin{lstlisting}[language=Kotlin]
    val thePair: Pair<CLASS1, CLASS2> = ...INIT THE PAIR...

    // Function with another function as parameter
    fun bigFunction(toBeLambda: (Int, Pair<CLASS1, CLASS2>, String) -> Unit) {
        toBeLambda(1, thePair, "something")
    }
    ...
    // Calling the big function with lambda
    bigFunction { theInt, (first, second), theString ->
        ...DO SOMETHING WITH first, second...

        // The pair as a whole not available here
        // pair.toString()
    }
\end{lstlisting}

\example[alternative to~the~mentioned \hyperref[kotlindestdeclforloop]{\textit{for} loop} with \textit{forEach}]
%! language = TEXT
\begin{lstlisting}[language=Kotlin]
    val pairList: List<Pair<CLASS1, CLASS2>> = ...INIT THE LIST...

    pairList.forEach { (_, second) ->
        // Accessing 2nd member of the current pair
        ...DO SOMETHING WITH second...

        // Current pair as a whole not available here
        // pair.toString()
    }
\end{lstlisting}

\example[destructuring declaration with \textit{forEach} on a map]
%! language = TEXT
\begin{lstlisting}[language=Kotlin]
    val theMap: Map<CLASS1, CLASS2> = ...INIT THE MAP...

    theMap.forEach { (key, value) ->
        // Accessing key and value of the current entry
        ...DO SOMETHING WITH key AND value...
    }
\end{lstlisting}

\warning Destructuring declaration is possible only with lambda functions parameters.
It~isn't allowed in~standard function headers.
I.e.,~you can't~declare anything like \mbitqls{fun doSomething(;;(first,;;;second):;;;Pair<;;CLASS1,;;;CLASS2>)}.
