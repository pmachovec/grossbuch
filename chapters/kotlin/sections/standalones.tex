\newsection{Standalone Functions and Variables}
\label{kotlinstandalone}
Kotlin enables to~\hyperref[declarationdefinition]{define} functions and~variables outside classes.
They just must be defined in~a~Kotlin file (with \mbox{the \itq{.kt}} extension) in~a~package folder.
Such~elements can~then be imported with the~package path and~used anywhere in~the~corresponding project.
This~is convenient especially for~very general functions that don't need a~class for~their work.
Having standalone variables can~occur for~some constants, in~that~case it's recommended to~\hyperref[declarationdefinition]{declare} them with \mbox{the \itq{const}} keyword.
Mutable standalone variables, although possible, are~a~bad practise and~shouldn't be used.

For~a~Kotlin file with standalone elements the~compilation process generates a~class with the~same name \mbox{and \itq{-Kt}} suffix.
If~the~file name starts with a~lowercase letter, it's~changed to~uppercase in~the~class name.
Eventual underscores in~the~file name are~preserved (even as~the~first character).
The~generated class contains static final functions corresponding to~standalone functions from~the~Kotlin file.
In~the~case of~a~standalone variable, if~it's defined with \mbox{the \itq{const}} keyword, a~public variable is generated for~it.
If~it's defined only with \mbox{the \itq{val}} keyword, a~private variable and~a~getter method are~generated for~it.
If~it's defined with \mbox{the \itq{var}} keyword, a~private variable, a~getter and a~setter methods are~generated for~it.

\example
%! language = TEXT
\begin{lstlisting}[language=Kotlin,title={Kotlin file \itq{Example.kt} with standalone elements}]
    package somepackage.subpackage

    const val constant = "constant"
    val withVal = "with val"
    var withVar = "with var"

    fun standaloneFunction(stringArg: String) {
        ...
    }
\end{lstlisting}

%! language = TEXT
\begin{lstlisting}[language=Java,title={Java equivalent}]
    package somepackage.subpackage

    public final class ExampleKt {
        public static final String constant = "constant";
        private static final String withVal = "with val";
        private static final String withVar = "with var";

        public static final void standaloneFunction(@@>@NotNull<@@ String stringArg) {
            ...
        }

        public static final String getWithVal() {
            return withVal;
        }

        public static final String getWithVar() {
            return withVar;
        }

        public static final void setWithVar(@@>@NotNull<@@ String var0) {
            this.withVar = withVar;
        }
    }
\end{lstlisting}
\newpage

%! language = TEXT
\begin{lstlisting}[language=Kotlin,title={Usage}]
    package otherpackage.othersubpackage

    import somepackage.subpackage.constant
    import somepackage.subpackage.withVal
    import somepackage.subpackage.withVar
    import somepackage.subpackage.standaloneFunction

    class SimpleClass {
        fun doSomething() {
            standaloneFunction("something")
            println(constant)
            println(withVal)
            withVar = "something else"
        }
    }
\end{lstlisting}
\noindent Notice that all standalone elements are~imported.
That's necessary no matter the~package structure.
Even if elements are~\hyperref[declarationdefinition]{defined} in~a~file directly above in~the~package path, they must be imported.

The~only possible exception is when a~standalone element is maliciously defined in~a~Kotlin file that contains also a~proper class or~\hyperref[kotlinobject]{object}.
In~that case the~compilation resulting \mbit{.class} file contains two classes -- the~one corresponding to~the~class or~object and~the~one with the~file name \mbox{and the \itq{-Kt}} suffix.
And~it's possible to~use the~standalone element in~the~class or~object.
