\newsection{Special Operators}
\newsubsection{Safe Call (?.)}
\index{Safe call}
\label{kotlinsafecall}
Alternative for~the~standard dot notation.
Can~be~used on~variables of~nullable types.
If~the~variable isn't null, the~safe call calls the~referenced function or~\hyperref[variablefieldproperty]{property} on~the~type instance stored in~the~variable.
Otherwise, it~returns null.
It~doesn't throw an~exception.

\example[variable is not null]
%! language = TEXT
\begin{lstlisting}[language=Kotlin]
    val first: String? = "first"
    val result: String? = first?.reversed() // result is "tsrif"
\end{lstlisting}

\example[variable is null]
%! language = TEXT
\begin{lstlisting}[language=Kotlin]
    val first: String? = null
    val result: String? = first?.reversed() // result is null
\end{lstlisting}
\newpage

\newsubsection{Non--null assertion (!!)}
Kotlin code aims to~avoid a~null pointer exception as~much as~possible.
Values must~be explicitly assigned to~non--nullable type variables, calls on~nullable type variables must~be done with the~\hyperref[kotlinsafecall]{safe call}~etc.
If~there's a~possibility to~hit a~null pointer, the~compilation fails.

\mbox{The~non--null} assertion operator can~be~used on~a~nullable type variable to~suppress the~null safety mechanism.
By~using the~operator the~code author asserts that he's aware of the~null possibility.
It~can~be~used in~an~assignment or~with the~dot notation.
When the~asserted variable is null, using the~operator results \mbox{in \textit{KotlinNullPointerException}.}

\example[usage in assignment]
%! language = TEXT
\begin{lstlisting}[language=Kotlin]
    val nullable: String? = "nullable"
    // val nonNullable String = nullable // Not compilable
    val nonNullable: String = nullable!!
\end{lstlisting}

\example[usage in dot notation]
%! language = TEXT
\begin{lstlisting}[language=Kotlin]
    val nullable: String? = "nullable"
    val result: String = nullable!!.reversed()
\end{lstlisting}

\example[causing exception]
%! language = TEXT
\begin{lstlisting}[language=Kotlin]
    val nullable: String? = null
    val nonNullable: String = nullable!! // KotlinNullPointerException
\end{lstlisting}

\newsubsection{Elvis Operator (?:)}
\index{Elvis operator}
Two--value operator written in~the~infix notation.
Returns the~left side \hyperref[parameterargument]{argument} if~it~isn't null.
Otherwise, returns the~right side \hyperref[parameterargument]{argument}, even when it's null.
I.e.,~it's coalesce operator for~two values.

\example[no~argument is null]
%! language = TEXT
\begin{lstlisting}[language=Kotlin]
    val first: String? = "first"
    val second: String? = "second"
    val result: String? = first ?: second // result is "first"
\end{lstlisting}

\example[first argument is null]
%! language = TEXT
\begin{lstlisting}[language=Kotlin]
    val first: String? = null
    val second: String? = "second"
    val result: String? = first ?: second // result is "second"
\end{lstlisting}
\newpage

\example[second argument is null]
%! language = TEXT
\begin{lstlisting}[language=Kotlin]
    val first: String? = "first"
    val second: String? = null
    val result: String? = first ?: second // result is "first"
\end{lstlisting}

\example[both arguments are null]
%! language = TEXT
\begin{lstlisting}[language=Kotlin]
    val first: String? = null
    val second: String? = null
    val result: String? = first ?: second // result is null
\end{lstlisting}
