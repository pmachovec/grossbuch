\newsection{Extensions}
\index{Extension}
\index{Receiver}
\index{Receiver type}
\label{kotlinextensions}
Extensions provide an~ability to~extend a~class with new functions without having to~change the~code of~the~class, inherit from the~class or~use design patterns such as \hyperref[decoratordp]{decorator}.
It's~possible to~extend even classes from libraries, i.e.,~those which aren't~written in~the~current project.
This is actually the~most usual usage of~extensions.
When extending a~class with~a~function, you~just prepend the~desired function name with the~extended class name and~a~dot.
The~class is said to~be \textit{receiver type} of~the~extending function.
Instance of~the~receiver type (on~which the~extension function is called) is called \textit{receiver} and~is accessible in~the~function body by~the~keyword \mbitq{this}.

\example[extending the \itq{String} class with a function]
%! language = TEXT
\begin{lstlisting}[language=Kotlin, title={Extension definition}]
    fun String.newFunction(...) {
        // "String" is receiver type of "newFunction"
        this.SOME_STRING_FUNCTION(...)
        ...
    }
\end{lstlisting}
\newpage

%! language = TEXT
\begin{lstlisting}[language=Kotlin, title={Usage}]
    val simpleString = "Simple string"
    simpleString.newFunction(...)
\end{lstlisting}

\note Receiver types can also appear in~\hyperref[kotlinanonymousfunction]{anonymous functions}.

\newsubsection{Extension property}
\index{Backing field}
\label{kotlinextensionproperty}
\noindent Extending a~class with~a~property is possible, but~slightly more complicated.
You~can't just assign values with~the~equals sign.
You~need to~specify at~least \hyperref[kotlingetset]{getter} (for~immutable property \mbox{with the \itq{val}} keyword).
Furthermore, \mbox{the \hyperref[kotlingetset]{\itq{field}}} keyword isn't available in~such getters and~setters.

If~you want the~property to~be mutable (\mbox{with the \itq{var}} keyword), it's~even more complicated.
You~of~course need to~specify also \hyperref[kotlingetset]{setter}, but~that's not enough.
A~publicly accessible field (i.e.,~another property) convenient for~this purpose must be declared in~the~class.
Such~field is~called \textit{backing field}.
If~there's none available and~you~can't change the~class code (which~is usually the~case with~classes from~libraries), you~can't create a~mutable extension property.

\example[extending a~class with properties]
%! language = TEXT
\begin{lstlisting}[language=Kotlin, title={Class with a~backing field}]
    class SimpleClass {
        var backingField: String? = null
    }
\end{lstlisting}
%! language = TEXT
\begin{lstlisting}[language=Kotlin, title={Extensions definition}]
    val SimpleClass.immutableProperty: String
        get() {
            "Immutable property value"
        }

    var SimpleClass.mutableProperty: String?
        get() {
            return backingField
        }
        set(value) {
            backingField = value
        }
\end{lstlisting}
%! language = TEXT
\begin{lstlisting}[language=Kotlin, title={Usage}]
    val simpleClassInstance = SimpleClass()
    simpleClassInstance.mutableProperty = "Something"
    println(simpleClassInstance.immutableProperty)
    println(simpleClassInstance.mutableProperty)
\end{lstlisting}
\newpage

\newsubsection{Object extensions}
Extending an~\hyperref[kotlinobject]{object} is completely the~same as~with classes.
You~just don't need an~instance to~call the~extension.

\example[object extension]
%! language = TEXT
\begin{lstlisting}[language=Kotlin, title={Object to be extended}]
    object SimpleObject {
        ...
    }
\end{lstlisting}
%! language = TEXT
\begin{lstlisting}[language=Kotlin, title={Object extension definition}]
    fun SimpleObject.newFunction(...) {
        ...
    }
\end{lstlisting}
%! language = TEXT
\begin{lstlisting}[language=Kotlin, title={Usage}]
    SimpleObject.newFunction(...)
\end{lstlisting}

\newsubsection{Companion object extensions}
When~a~class contains explicitly declared \hyperref[kotlincompanionobject]{companion object}, it's~possible to~create an~extension even for~the~companion object.
Such~extension is then accessible just with~the~class name, no~instance is needed.
The~companion object doesn't have to~have a~body.
Just those two keywords \mbitqls{companion;;;object} must appear in~the~class body.
The~syntax is similar to~all previous cases, just the~keyword \mbitq{Companion} (with~capital~C) is inserted between the~class name and~extension name in~the~dot notation chain.

\example[companion object extension]
%! language = TEXT
\begin{lstlisting}[language=Kotlin, title={Class with a~companion object without a~body}]
    class SimpleClass {
        ...
        companion object
    }
\end{lstlisting}
%! language = TEXT
\begin{lstlisting}[language=Kotlin, title={Companion object extension definition}]
    fun SimpleClass.Companion.newFunction(...) {
        ...
    }
\end{lstlisting}
%! language = TEXT
\begin{lstlisting}[language=Kotlin, title={Usage}]
    SimpleClass.newFunction(...)
\end{lstlisting}

\newsubsection{Scope}
\label{kotlinextensionscope}
Usually, extensions are~defined as~\hyperref[kotlinstandalone]{standalone elements}, i.e.,~outside classes.
And~in~that case they're handled as~standalone elements.
That means that special classes \mbox{with the \itq{-Kt}} suffix in~names and~elaborated combination of~static methods are~generated for~them during compilation.
Compiled code of~original classes or~objects (those being extended) isn't affected.
I.e.,~although extensions look like members of~extended structures, they're members of~other classes.
The~membership is just an~illusion created by generated names.
The~visibility of~standalone extensions is~the~same as~for~other \hyperref[kotlinstandalone]{standalone elements}.
No~matter where they're defined, they must be imported.
\newline

\example[extending the \itq{String} class with a function]
%! language = TEXT
\begin{lstlisting}[language=Kotlin, title={Extension definition}]
    package somepackage.subpackage

    fun String.newFunction(...) {
        ...
    }
\end{lstlisting}
\newline

%! language = TEXT
\begin{lstlisting}[language=Kotlin, title={Usage in a class}]
    package otherpackage.othersubpackage

    // The extended class (String) is not used in import
    import somepackage.subpackage.newFunction

    class SimpleClass {
        fun doSomething() {
            val simpleString = "Simple string"
            simpleString.newFunction(...)
        }
    }
\end{lstlisting}
\newline
\newline

\noindent It's~also possible to~define an~extension inside a~class, \hyperref[kotlinobject]{object} or~\hyperref[kotlincompanionobject]{companion object}.
Such~extension is visible only inside the~enclosing structure, and, if~not~declared as~private, in~its descendants.
It~isn't possible to~import or~call it from~outside.
\newpage

\example[extending the \itq{String} class inside another class]
%! language = TEXT
\begin{lstlisting}[language=Kotlin, title={Extension definition}]
    class SimpleClass {
        fun String.newFunction(...) {
            ...
        }

        fun doSomething() {
            val simpleString = "Simple string"
            simpleString.newFunction(...)
        }
    }
\end{lstlisting}

\newsubsection{Conflicting definitions}
Extensions can't override original members.
I.e.,~if~you define an~extension, but~the~extended structure (class/object) already contains a~member with the~same name, the~original implementation of~the~member will~be~used, even when~imported.
It~holds even for~\hyperref[kotlinextensionproperty]{extension properties}.

\enlargethispage{20mm}
\thispagestyle{empty}
\example
%! language = TEXT
\begin{lstlisting}[language=Kotlin, title={Class with some members}]
    package somepackage.subpackage

    class SimpleClass {
        val simpleProperty: String? = "Original property"

        fun simpleFunction() {
            println("Original function")
        }
    }
\end{lstlisting}
%! language = TEXT
\begin{lstlisting}[language=Kotlin, title={Extensions with same names}]
    package otherpackage.othersubpackage

    import somepackage.subpackage.SimpleClass

    fun SimpleClass.simpleFunction() {
        println("Extension function")
    }

    val SimpleClass.simpleProperty
        get() {
            return ("Extension property")
        }
\end{lstlisting}
\newpage

%! language = TEXT
\begin{lstlisting}[language=Kotlin, title={Behavior}]
    import otherpackage.othersubpackage.simpleFunction
    import otherpackage.othersubpackage.simpleProperty
    import somepackage.subpackage.SimpleClass

    val simpleClassInstance = SimpleClass()

    // Will print "Original function"
    simpleClass.simpleFunction()

    // Will print "Original property"
    println(simpleClassInstance.simpleProperty)
\end{lstlisting}
\newline

\noindent It's~possible to~define multiple extensions with~the~same name and~arguments for~different classes in~one~file.
When~\hyperref[kotlinextensionscope]{importing}, only one import is required (multiple would~be identical), but~it~imports all same named extensions.

\example
%! language = TEXT
\begin{lstlisting}[language=Kotlin, title={Same named extensions for different classes}]
    package somepackage.subpackage

    fun ClassOne.sameNamedFunction(...): ... {
        ...
    }

    fun ClassTwo.sameNamedFunction(...): ... {
        ...
    }
\end{lstlisting}
%! language = TEXT
\begin{lstlisting}[language=Kotlin, title={Usage}]
    import somepackage.subpackage.sameNamedFunction

    val classOne = ClassOne(...)
    val classTwo = ClassTwo(...)

    // Both calls will work
    classOne.sameNamedFunction(...)
    classTwo.sameNamedFunction(...)
\end{lstlisting}
\newpage
