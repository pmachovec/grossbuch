\newsection{Getters and Setters}
\index{Get}
\index{Set}
\label{kotlingetset}
In~Kotlin, basic getters and~setters (just retrieving from and~assigning to~a~property) are~created automatically.
The~code written in~the~direct property access style (retrieving by~dot notation, assigning with the~equals sign) is compiled to~\hyperref[javabytecode]{bytecode} corresponding to~Java code using getters and~setters.
When~you declare the~property with the~\mbitq{val} keyword, the~setter isn't~created.
\newline

\example
%! language = TEXT
\begin{lstlisting}[language=Kotlin, title={Direct property access in Kotlin}]
    class SimpleClass {
        var simpleMutable: String? = null
        val simpleImmutable: = "Immutable value"
    }
    ...
    val simpleClassInstance = SimpleClass()
    simpleClassInstance.simpleMutable = "Mutable value" // Calling setter
    println(simpleClassInstance.simpleMutable) // Calling getter
    println(simpleClassInstance.simpleImmutable) // Calling getter
\end{lstlisting}
\newpage

%! language = TEXT
\begin{lstlisting}[language=Java, title={Java equivalent}]
    public class SimpleClass {
        private String simpleMutable = null;
        private final String simpleImmutable = "Immutable value";

        public String getSimpleMutable() {
            return simpleMutable;
        }

        public void setSimpleMutable(String simpleMutable) {
            this.simpleMutable = simpleMutable;
        }

        public String getSimpleImmutable() {
            return simpleImmutable;
        }
    }
    ...
    private final SimpleClass simpleClassInstance = new SimpleClass();
    simpleClassInstance.setSimpleMutable("Mutable value");
    System.out.println(simpleClassInstance.getSimpleMutable());
    System.out.println(simpleClassInstance.getSimpleImmutable());
\end{lstlisting}
\newline

\noindent When~you~need some more sophisticated logic for~value retrieval and~assignment, in~Java you~have \mbit{get} \mbox{and \textit{set}} methods where you~can~write additional code.
Kotlin enables adding more logic \mbox{by \itq{get}} \mbox{and \itq{set}} keywords written under the~affected property declaration and~followed by~a~function body.
In~these function bodies the~keyword \mbitq{field} is available to~access the~current value of~the~corresponding field variable.
In~the~setter body you~must assign final desired value to~it -- setters don't return a~value (their return type \mbox{is \hyperref[kotlinunit]{\textit{Unit}}}).
Furthermore, the~setter gets one parameter of~the~variable type.
That represents the~set value.
Getters must return a~value of~the~corresponding field variable type.

\emtwen
\example[string property storing reversed value and returning uppercase]
%! language = TEXT
\begin{lstlisting}[language=Kotlin, title={Direct property access in Kotlin}]
    class SimpleClass {
        var complexValue: String? = null
            get() {
                return field.toUpperCase()
            }
            set(value) {
                field = value.reversed()
            }
    }
    ...
    val simpleClassInstance = SimpleClass()
    simpleClassInstance.complexValue = "abc" // 'complexValue' is 'cba'
    println(simpleClassInstance.complexValue) // Prints 'CBA'
\end{lstlisting}
\newpage

%! language = TEXT
\begin{lstlisting}[language=Java, title={Java equivalent}]
    public class SimpleClass {
        private String complexValue = null;

        public String getComplexValue() {
            return complexValue.toUpperCase();
        }

        public void setComplexValue(String value) {
            this.complexValue = value;
        }
    }
    ...
    private final SimpleClass simpleClassInstance = new SimpleClass();
    simpleClassInstance.setComplexValue("abc");
    System.out.println(simpleClassInstance.getComplexValue());
\end{lstlisting}
\newline
