\newsection{Data Classes}
\index{Data class}
\label{kotlindataclass}
Data classes serve for~situations when~you~need class instances just as~containers for~somehow related values, with~minimal, ideally none, additional logic.
A~data class is created by~the~keyword \mbitq{data}.
It~must have a~\hyperref[kotlinprimaryconstructor]{primary constructor} with~at~least one~parameter.
All~parameters of~the~primary constructor must have \hyperref[kotlinvalvar]{\textit{val} \mbox{or \textit{var}}} keyword.
Standard implementations \mbox{of \itq{\hyperref[kotlindataclasstostring]{toString}}}, \mbitq{\hyperref[kotlindataclassequals]{equals}}, \mbitq{\hyperref[kotlindataclasshashcode]{hashCode}} \mbox{and \itq{\hyperref[kotlindataclasscopy]{copy}}} functions \mbox{and \itq{componentN}} operators for~\hyperref[kotlindestdecl]{destructuring declarations}, are~automatically available on~data class instances (see~further for~examples).
These implementations work with~arguments of~the~primary constructor.
Eventual other fields of~the~class aren't used in~them.
It's~possible to~override implementations and~write some custom.

Although it's a~misuse, data classes can~have a~body with complex methods and~nested structures.
They~also can~extend other classes and~implement interfaces.
On~the~other hand, data classes can't~be declared as~open or~abstract and~therefore, they can't~be inherited.
Because of that, eventual declaring them as~\hyperref[kotlinsealedclass]{sealed} would make no sense and~is also forbidden.
Data classes can~be nested, but~not~\hyperref[kotlininnerclass]{inner}.

\newsubsection{\textit{toString}}
\label{kotlindataclasstostring}
It~returns a~string in~the~format \mbitqls{ClassName(;;firstParameterName;;=;;firstParameterToString,;;;secondParameterToString;;=secondParameterValue, \dots)}.
If~a~parameter type is a~class without proper implementation \mbox{of \itq{toString}}, it~uses the~default behavior for~the~parameter --~package path, at~sign and~an~address number.
Other fields not~included in~the~primary constructor aren't included in~the~returned value.

\example
%! language = TEXT
\begin{lstlisting}[language=Kotlin, title={Class without proper "toString" implementation}]
    package somepackage.subpackage

    class ClassWithoutToString(...) {
        ...
    }
\end{lstlisting}
%! language = TEXT
\begin{lstlisting}[language=Kotlin, title={Data class}]
    data class ExampleDataClass(
        val stringProperty: String,
        val intProperty: Int,
        val classWithoutToString: ClassWithoutToString
    ) {
        var otherProperty = ...
    }
\end{lstlisting}
\newpage
%! language = TEXT
\begin{lstlisting}[language=Kotlin, title={\itq{toString} behavior}]
    val classWithoutToString = ClassWithoutToString(...)
    val dataClass = ExampleDataClass("something", 1, classWithoutToString)

    // Will print "ExampleDataClass(stringProperty=something, intProperty=1, classWithoutToString= somepackage.subpackage.ClassWithoutToString@SOME_MESS)"
    println(dataClass)
\end{lstlisting}
\newline

\newsubsection{\textit{equals}}
\label{kotlindataclassequals}
It~compares two data class instances by comparing individual arguments of~the~primary constructor.
The~arguments comparison is done purely with \mbox{the \itq{equals}} function, \mbitq{hashcode} isn't used.
I.e.,~even when~\mbitq{hashcode} returns different values for~compared arguments, \mbox{if \itq{equals}} returns \textit{true}, arguments are~considered as~equal (but~in~such case \mbox{the \itq{equals}} implementation in~the~corresponding parameter type class is~obviously wrong).
If~a~parameter type is a~class without proper implementation \mbox{of \itq{equals}}, it~compares \mbox{the \hyperref[systemmemory]{memory}} address of~the~corresponding instance.
Other fields not~included in~the~primary constructor aren't used in~the~instances comparison.
\newline

\example[parameter type class without \itq{equals} implementation]
%! language = TEXT
\begin{lstlisting}[language=Kotlin, title={Simple class without proper "equals" implementation}]
    class ClassWithoutEquals(
        var simpleProperty: String
    )
\end{lstlisting}
\newline
%! language = TEXT
\begin{lstlisting}[language=Kotlin, title={Data class}]
    data class ExampleDataClass(
        val stringProperty: String,
        val intProperty: Int,
        var classWithoutEquals: ClassWithoutEquals
    )
\end{lstlisting}
\newpage
%! language = TEXT
\begin{lstlisting}[language=Kotlin, title={\itq{equals} behavior}]
    val classWithoutEquals = ClassWithoutEquals("first string")
    val dataClass1 = ExampleDataClass("second string", 1, classWithoutEquals)
    classWithoutEquals.simpleProperty = "third string" // Doesn't change instance address
    val dataClass2 = ExampleDataClass("second string", 1, classWithoutEquals)

    // Will print "true", address of 'classWithoutEquals' hasn't changed
    println(dataClass1 == dataClass2)

    // Assigning new instance with different address
    dataClass2.classWithoutEquals = ClassWithoutEquals("third string")

    // Will print "false"
    println(dataClass1 == dataClass2)
\end{lstlisting}

\example[parameter type class with \itq{equals} always returning false]
%! language = TEXT
\begin{lstlisting}[language=Kotlin, title={Simple class without proper "equals" implementation}]
    class ClassWithWrongEquals {
        override fun equals(other: Any?): Boolean {
            return false
        }
    }
\end{lstlisting}
%! language = TEXT
\begin{lstlisting}[language=Kotlin, title={Data class}]
    data class ExampleDataClass(
        val stringProperty: String,
        val intProperty: Int,
        var classWithWrongEquals: ClassWithWrongEquals
    )
\end{lstlisting}
%! language = TEXT
\begin{lstlisting}[language=Kotlin, title={\itq{equals} behavior}]
    val classWithWrongEquals = ClassWithWrongEquals()
    val dataClass1 = ExampleDataClass("something", 1, classWithWrongEquals)
    val dataClass2 = ExampleDataClass("something", 1, classWithWrongEquals)

    // Will print "false"
    println(dataClass1 == dataClass2)
\end{lstlisting}

\newsubsection{\textit{hashCode}}
\label{kotlindataclasshashcode}
It~computes the~instance hash code from~hash codes of~individual arguments of~the~primary constructor.
That~means the~result always depends on~the~implementation \mbox{of \itq{hashCode}} of~individual parameter types.
Other fields not~included in~the~primary constructor aren't used in~the~computation.
The~concrete algorithm computing the~data class hash code isn't so simple and~it~isn't necessary to~know~it.

\example
%! language = TEXT
\begin{lstlisting}[language=Kotlin, title={Data class}]
    data class ExampleDataClass(
        val stringProperty: String,
        val intProperty: Int,
        val boolProperty: Boolean
    )
\end{lstlisting}
%! language = TEXT
\begin{lstlisting}[language=Kotlin, title={\itq{hashCode} behavior}]
    val dataClass = ExampleDataClass("something", 1, true)

    // Will print "-1688972774"
    // It isn't important how this value was achieved
    println(dataClass.hashCode())
\end{lstlisting}
\newline

\newsubsection{\textit{copy}}
\label{kotlindataclasscopy}
It~creates a~copy of~the~current instance.
Only arguments of~the~primary constructor are~copied, other fields not~included in~the~primary constructor are~not.
All~copied values in~the~created copy are~links to~the~same object in~theory (Kotlin doesn't have \hyperref[javaprimitivetypes]{primitive types} like Java).
That~doesn't matter with booleans, numbers and~strings.
These types don't~have any mutable property to~be~maliciously altered.
However, it's~important fact to~consider when~the~data class constructor has~parameters of~other types.
\newline

\example
%! language = TEXT
\begin{lstlisting}[language=Kotlin, title={A class with mutable property}]
    class SimpleClass(
        var mutableProperty: String
    )
\end{lstlisting}
\newpage
%! language = TEXT
\begin{lstlisting}[language=Kotlin, title={Data class}]
    data class ExampleDataClass(
        val stringProperty: String,
        val intProperty: Int,
        val simpleClassProperty: SimpleClass
    )
\end{lstlisting}
%! language = TEXT
\begin{lstlisting}[language=Kotlin, title={\itq{copy} behavior}]
    val simpleClass = SimpleClass("one")
    val dataClass = ExampleDataClass("something", 1, simpleClass)
    val dataClassCopy = simpleClass.copy()

    // Will print "something"
    println(dataClassCopy.stringProperty)

    // Will print "1"
    println(dataClassCopy.intProperty)

    // Will print "one"
    println(dataClassCopy.simpleClassProperty.mutableProperty)

    // Change made to the original instance
    dataClass.simpleProperty.mutableProperty = "two"

    // Will print "two" - change propagated even to the copy
    println(dataClassCopy.simpleClassProperty.mutableProperty)
\end{lstlisting}

\warning Opposite to~other mentioned functions, \mbox{the \itq{copy}} function can~be overridden even without the~keyword \mbox{\itq{override}.}

\example{overriding functions in a~data class}
%! language = TEXT
\begin{lstlisting}[language=Kotlin]
    data class ExampleDataClass(
        ...
    ) {
        // Must be declared with 'override'
        override fun toString(): String {
            ...RETURN STRING...
        }

        // Can be declared without 'override'
        fun copy(): ExampleDataClass {
            ...RETURN NEW INSTANCE...
        }
    }
\end{lstlisting}
\newpage
