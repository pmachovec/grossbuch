\newsection{Inner Classes}
\index{Inner class}
\index{Outer class}
\index{Nested class}
\label{kotlininnerclass}
When you define a~class inside another class, it's~called \textit{nested class}.
\textit{Inner class} is a~nested class (i.e.,~it's~defined inside another class) with the~keyword \mbitq{inner}.
The~difference is that without the~keyword the~nested class doesn't see members of~its enclosing outer class, not~even public.

\example[a class with simple and inner nested classes]
%! language = TEXT
\begin{lstlisting}[language=Kotlin]
    class OuterClass {
        val outerClassMember = "..."
        private val privateMember = "..."

        fun outerClassFunction() {
            ...
        }

        class SimpleNestedClass {
            // Outer class members aren't available here
            // println(outerClassMember)
            // println(privateMember)
            // outerClassFunction()
        }

        inner class InnerClass {
            // Outer class members are available here, even private
            println(outerClassMember)
            println(privateMember)
            outerClassFunction()
        }
    }
\end{lstlisting}
\newline

\noindent Only classes can~be inner.
\hyperref[kotlinobject]{Objects} and~interfaces can~be nested, but~not inner.
I.e.,~you~can~define an~interface or~an~object inside a~class, but~you~can't access members of~the~outer class in~them.

And~similarly, \hyperref[kotlinobject]{objects} and~interfaces can~contain nested structures (classes, objects, interfaces), but~not~inner classes.
I.e.,~you~can~define a~class, an~object or~an~interface inside an~object or~an~interface,  but~you~can't access members of~the~outer structure in~them.
\newpage
