\newsection{Lambda Functions}
\index{Lambda function}
\label{kotlinlambda}
Lambda functions are an~alternative for~standard \hyperref[kotlinanonymousfunction]{anonymous functions} with simpler syntax.
The~syntax is \mbitqls{\{PARAMETERS;;;->;;;COMMANDS\}}.
Last line holds the~return value, \mbox{the \textit{return}} keyword isn't used.
Also, the~return value type isn't specified.
For~example, a~lambda equivalent of~the~function multiplying its parameter by two from the~\hyperref[kotlinanonymousfunction]{anonymous functions} example looks like this:
%! language = TEXT
\begin{lstlisting}[language=Kotlin]
    val multiplyByTwo = { a: Int ->
        a * 2
    }
\end{lstlisting}
\newline

\noindent When the~last line doesn't hold a~value, then~the~function returns the~\hyperref[kotlinunit]{Unit} instance.
That happens, for~example, when~the~last line is an~assignment.

\example[Lambda functions returning \hyperref[kotlinunit]{Unit} and fixed version]
%! language = TEXT
\begin{lstlisting}[language=Kotlin]
    val wrongMultiplyByTwo = { a: Int ->
        val b = a*2
    }

    val correctMultiplyByTwo = { a: Int ->
        val b = a * 2
        b
    }
\end{lstlisting}
\newline

\noindent So~far, it's quite understandable.
However, there are some features aiming to~reduce the~boilerplate code or~to~address some drawbacks.
And~these features can~cause confusions.

\newsubsection{No Parameters}
When a~lambda function has no~parameters, then only the~commands section is needed.
Even the~arrow is missing.
Such~function then looks like just a~sequence of~commands enclosed in~curly brackets.
\newpage

\example[Lambda function without parameters]
%! language = TEXT
\begin{lstlisting}[language=Kotlin]
    val noParamsFunction = {
        ...
        RETURN_VALUE
    }
\end{lstlisting}

\newsubsection{Last Parameter of Another Function}
Very often, a~standard function takes another function as~the~last parameter.
When calling such~standard function, it's~convenient to~write the~parameter function body as~a~part of~the~call in~the~lambda format.
And~the~lambda format can~stay outside parameter brackets of~the~holding function, it's~written after~them.

Furthermore, as~the~parameter function type has~to~be specified in~the~holding function header, parameter types of~the~parameter function can~be omitted.
This is true even when using normal \hyperref[kotlinanonymousfunction]{anonymous function}.
However, when the~parameter function isn't written directly as~a~part of~the~holding function call, but~is~specified earlier and~stored in~a~variable, then parameter types must~be used.
It~doesn't matter if~it's the~normal or~the~lambda format.

\example
%! language = TEXT
\begin{lstlisting}[language=Kotlin, title={Function taking other function as the~last parameter}]
    fun doSomething(
        value: Int,
        anotherValue: Int,
        twoArgsFunction: (Int, Int) -> Int,
    ) : Int {
        return twoArgsFunction(value, anotherValue)
    }
\end{lstlisting}
%! language = TEXT
\begin{lstlisting}[language=Kotlin, title={Usage}]
    // Result is 3
    val result = doSomething(1, 2) { a, b ->
        a + b
    }
\end{lstlisting}

%! language = TEXT
\begin{lstlisting}[language=Kotlin, title={Equivalent with directly written anonymous function -- parameter types not needed}]
    val result = doSomething(1, 2, fun(a, b) {
        return a + b
    })
\end{lstlisting}
\newpage

%! language = TEXT
\begin{lstlisting}[language=Kotlin, title={Equivalent with anonymous function in a variable -- parameter types must be used}]
    val plus = fun(a: Int, b: Int) : Int {
        return a + b
    }

    val result = doSomething(1, 2, plus)
\end{lstlisting}
%! language = TEXT
\begin{lstlisting}[language=Kotlin, title={Equivalent with lambda function in a variable -- parameter types must be used}]
    val plus = { a: Int, b: Int ->
        a + b
    }

    val result = doSomething(1, 2, plus)
\end{lstlisting}
\newline

\noindent If~the~holding function doesn't take any other parameter than the~parameter function, then empty brackets before the~lambda expression can~be omitted (but~don't necessarily have \mbox{to be).}

\example
%! language = TEXT
\begin{lstlisting}[language=Kotlin, title={Function taking other function as the~only parameter}]
    fun doSomething(
        twoArgsFunction: (Int, Int) -> Int,
    ) : Int {
        return twoArgsFunction(1, 2)
    }
\end{lstlisting}
%! language = TEXT
\begin{lstlisting}[language=Kotlin, title={Usage}]
    val result = doSomething { a, b ->
        a + b
    }
\end{lstlisting}
\newline

\noindent If~the~parameter function isn't \hyperref[declarationdefinition]{declared} as~the~last one in~the~holding function header, then the~parameter function must~be included in~the~holding function parameter brackets at~the~corresponding position.
If~there are multiple functions as~last parameters of~the~holding function, then only the~very last one can~be specified outside parameter brackets.
\newpage

\newsubsection{Implicit Parameter (\textit{it})}
\index{It}
This one is a~very tricky feature.
If~the~parameter function is specified as~the~last one in~the~holding function header and~if~it takes only one parameter, then, when~written in~the~lambda format, the~parameters section can~be omitted, including the~arrow.
I.e.,~the~lambda function looks like taking no~parameters at~the~first glance, but~it~takes one in~fact.
The~parameter is accessible in~the~commands section under \mbox{the keyword \itq{it}}.

\example
%! language = TEXT
\begin{lstlisting}[language=Kotlin, title={Function taking other function as the~last parameter, that one takes one parameter}]
    fun doSomething(
        value: Int,
        oneArgFunction: (Int) -> Int,
    ) : Int {
        return oneArgFunction(value)
    }
\end{lstlisting}
%! language = TEXT
\begin{lstlisting}[language=Kotlin, title={Usage}]
    // Result is 4
    val result = doSomething(2) {
        it * 2
    }
\end{lstlisting}
\newline

\noindent It's~used especially when iterating over collections.
Functions like \mbit{forEach} or~\mbit{filter} available on~collections take a~function as~the~last (and~often the~only) parameter and~that~function takes one~parameter.

\example[printing list members by \textit{forEach} function]
%! language = TEXT
\begin{lstlisting}[language=Kotlin]
    val theList = listOf(...SOMETHING...)

    theList.forEach {
        println(it)
    }
\end{lstlisting}

\newsubsection {\itq{Run} Function}
\index{Run}
\label{kotlinrun}
This is a~function to~use when~you~need to~immediately run a~(parameterless) lambda function and~eventually get a~result from~it.
It~takes a~parameterless function as~the~only parameter and~returns its return value.
It's~usable also with standard anonymous functions.
\newpage

\example[\textit{run} function]
%! language = TEXT
\begin{lstlisting}[language=Kotlin, title={\textit{Run} with lambda function}]
    // Result holds the return value
    val result = run {
        ...
        RETURN_VALUE
    }
\end{lstlisting}
%! language = TEXT
\begin{lstlisting}[language=Kotlin, title={Equivalent with standard anonymous function}]
    val result = run(fun() : RETURN_TYPE {
        ...
        return RETURN_VALUE
    })
\end{lstlisting}

\newsubsection{Return at Labels}
\index{Label}
\label{kotlinreturnatlabel}
You~may~have \mbit{if--then--else} branching in~a~lambda function and~you~may want to~exit the~function at~a~certain point before reaching the~end.
And~because lambda functions don't~support the~standard \mbit{return} statement, it~isn't so~straightforward.
You~must use return at~a~label.
In~the~basic version the~lambda function must~be preceded with~a~label in~the~syntax \mbitq{LABEL\_NAME@} (mind the~at~sign at~the~end).
Then, inside the~lambda function body, the~call \mbitqls{return@LABEL\_NAME;;;RETURN\_VALUE}, \mbox{or \itq{return@LABEL\_NAME}} without any~return value (returning \hyperref[kotlinunit]{Unit}), must~be made.

\example[return at label]
%! language = TEXT
\begin{lstlisting}[language=Kotlin, title={Function taking other function as the~only parameter}]
    fun doSomething(
        twoArgsFunction: (Int, Int) -> Int,
    ) : Int {
        return twoArgsFunction(1, 2)
    }
\end{lstlisting}
%! language = TEXT
\begin{lstlisting}[language=Kotlin, title={Usage}]
    val result = doSomething labelName@{ a, b ->
        if (...SOME CONDITION...) {
            return@labelName ANOTHER_RETURN_VALUE
        }

        a + b
    }
\end{lstlisting}
\newline

\noindent In~newer versions of~Kotlin the~name of~the~function that~gets the~lambda function as~a~parameter is~the~default label.
\newpage

\example[default label]
%! language = TEXT
\begin{lstlisting}[language=Kotlin, title={Function taking other function as the~only parameter}]
    fun doSomething(
        twoArgsFunction: (Int, Int) -> Int,
    ) : Int {
        return twoArgsFunction(1, 2)
    }
\end{lstlisting}
%! language = TEXT
\begin{lstlisting}[language=Kotlin, title={Usage}]
    val result = doSomething { a, b ->
        if (...SOME CONDITION...) {
            return@doSomething ANOTHER_RETURN_VALUE
        }

        a + b
    }
\end{lstlisting}
\newline

\noindent Returns at~labels can~be used to~simulate \textit{continue} when iterating over collections, i.e.,~to~stop processing current element and~start next iteration with the~next~one.

\example[return at label in \textit{forEach}]
%! language = TEXT
\begin{lstlisting}[language=Kotlin]
    val theList = listOf(...SOMETHING...)

    theList.forEach {
        if (...SOME CONDITION...) {
            return@forEach
        }

        ...
    }
\end{lstlisting}

\note Simulating \textit{break}, i.e.,~to~stop processing and~jump behind the~loop, is~also possible, but~more tricky.
As~returning at~labels always jumps at~the~start of~the~lambda function, you~must wrap the~holding function call to~another lambda function, run that with~the~\hyperref[kotlinrun]{\textit{run}} function and~return at~that \textit{run}'s label.
\newpage

\example[break of \textit{forEach} simulation]
%! language = TEXT
\begin{lstlisting}[language=Kotlin]
    val theList = listOf(...SOMETHING...)

    run {
        theList.forEach {
            if (...SOME CONDITION...) {
                return@run
            }

            ...
        }
    }
\end{lstlisting}
