\newsection{Anonymous Functions and Function Types}
\index{Anonymous function}
\index{Function type}
\label{kotlinanonymousfunction}
Kotlin supports syntax similar to~JavaScript, where anonymous functions can~be assigned to~variables.
Such~functions can~be then invoked by calling the~variable name or~passed to~another function as~\hyperref[parameterargument]{arguments}.
When specifying a~type of~a~variable holding a~function (which is the~case especially for~function \hyperref[parameterargument]{parameters}, standalone variables usually don't need type specification), you~must follow the~syntax \mbitqls{(TYPE\_1,;;;TYPE\_2,;;;...);;;->;;;RETURN\_TYPE}, where \mbitqls{(TYPE\_1,;;;TYPE\_2,;;;...)} are~types of~the~function parameters \mbox{and \itq{RETURN\_TYPE}} is the~function return value type.
Mind brackets by the~function parameter types.
They're compulsory even when the~function has only one parameter, or~even zero.
\hyperref[kotlinunit]{\textit{Unit}} (and~even \hyperref[kotlinnothing]{\textit{Nothing}}) type is available for~functions returning no value.

\example
%! language = TEXT
\begin{lstlisting}[language=Kotlin]
    // Function of type "(Int) -> Int"
    val multiplyByTwo = fun(a: Int) : Int {
        return a*2
    }

    // Function of type "() -> Int"
    val doNothing = fun() {}
\end{lstlisting}
%! language = TEXT
\begin{lstlisting}[language=Kotlin, title={Function taking other functions as parameters}]
    fun doSomething(
        value: Int,
        oneArgFunction: (Int) -> Int,
        nothingFunction: () -> Unit
    ) : Int {
        nothingFunction()
        return oneArgFunction(value)
    }
\end{lstlisting}
\newpage

%! language = TEXT
\begin{lstlisting}[language=Kotlin, title={Usage}]
    // Standalone invocation
    val result1 = multiplyByTwo(2) // Result is 4
    doNothing() // Doesn't return a value

    // Passing to another function
    val result2 = doSomething(3, multiplyByTwo, doNothing) // Result is 6
\end{lstlisting}
