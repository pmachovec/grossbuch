\newsection{Objects}
\index{Object}
\label{kotlinobject}
Kotlin has the~ability to~provide the~\hyperref[singletondp]{singleton design pattern} without any~special implementation or~framework.
Instead of~a~standard class it's possible to~\hyperref[declarationdefinition]{define} an~\textit{object}.
Objects are almost (but~not~completely) same as~classes.
They can~be instantiated and~assigned to~variables, they can~implement interfaces and~extend classes.
When an~object is~called for~the~first time, it's~instantiated.
Each~further call refers to~the~same instance.
It~isn't necessary to~assign an~object instance to~a~variable, it~can~be referred by~the~object name.

\example[object definition and~usage, the~object implements an~interface]
%! language = TEXT
\begin{lstlisting}[language=Kotlin]
    object SimpleObject: SimpleInterface {
        lateinit var simpleValue: String
    }

    // Object instantiated, notice there are no brackets
    val simpleVariable = SimpleObject
    simpleVariable.value = "one"
    SimpleObject.value = "two"
    println(simpleVariable.value) // Will print "two"
    println(SimpleObject.value)   // Will print "two
\end{lstlisting}
\newline

\noindent Objects can't~be inherited and~don't allow to~be~declared with the~keyword \itq{open}.
They also can't~have \hyperref[kotlinconstructor]{constructors}, not~even \hyperref[kotlinprimaryconstructor]{primary}.
However, this can~be hacked by~\hyperref[kotlininvokeoverload]{overloading \mbox{the \textit{invoke}} operator}.

\example[object looking like having a~constructor]
%! language = TEXT
\begin{lstlisting}[language=Kotlin]
    object ObjectWithConstructor {
        lateinit var value: String

        operator fun invoke(value: String): ObjectWithConstructor {
            this.value = value
            return this
        }
    }

    val objectWithConstructor = ObjectWithConstructor("SOMETHING")
    println(objectWithConstructor.value) // Will print "SOMETHING"
\end{lstlisting}
\newline

\noindent Java equivalent of~a~Kotlin object is~a~final class implementing the~\hyperref[singletondp]{singleton design pattern}.
The~instance variable in~such class is~\hyperref[javapublic]{public}, \hyperref[javastatic]{static} and~\hyperref[javafinal]{final}.
This enables the~described behavior.

\example[Java equivalent of~a~Kotlin object]
%! language = TEXT
\begin{lstlisting}[language=Java]
    public final class SimpleObject {
        @>@NotNull
        public static String simpleValue;
        public static final SimpleObject INSTANCE;

        private SimpleObject(){}

        static {
            SimpleObject var0 = new SimpleObject();
            INSTANCE = var0;
        }

        ...COMPLEX GETTER AND SETTER FOR simpleValue...
    }
\end{lstlisting}

\warning Calling object name with~the~\hyperref[kotlinreflection]{reflection operator} doesn't create the~singleton instance.
This can~be important when dealing with \hyperref[objectinitblock]{initializer blocks of~objects}.
