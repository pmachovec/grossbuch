\newsection{Constructors}
\label{kotlinconstructor}
\newsubsection{Primary Constructor}
\index{Primary constructor}
\label{kotlinprimaryconstructor}
Typical constructor usage in~Java is just assigning received arguments to~class variables (\itq{this.someVariable = someVariable;}).
In~Kotlin there's a~syntax called \itq{primary constructor} for~simplifying this.
It~works by~specifying variables with types in~brackets directly in~a~class header (see~the~example).

When a~variable has \hyperref[kotlinvalvar]{\itq{val} or~\mbitq{var}} keyword, it~becomes \mbox{a property} of~the~class.
It~isn't necessary (and~even possible) to~specify that~property variable separately (you~still can~specify other properties separately).
When a~variable doesn't have \itq{val} nor~\mbitq{var} keyword, it's~just an~immutable parameter of~the~constructor.
It~can~be used only for~setting other class properties (typically at~the~beginning of~the~class) or~inside \hyperref[kotlininitblock]{initializer blocks}.

\example[Kotlin class with a primary constructor]
%! language = TEXT
\begin{lstlisting}[language=Kotlin]
    class SimpleClass(
        var mutableProperty: SomeClass,
        val immutableProperty: SomeOtherClass,
        constructorParameter: ThirdClass
    ) {
        var anotherProperty = constructorParameter

        init {
            ....READ constructorParameter...
        }

        fun doSomething() {
            ...DO WHATEVER WITH mutableProperty...
            ...READ immutableProperty...
        }
    }
\end{lstlisting}
\newpage

\newsubsection{Secondary Constructor}
\index{Secondary constructor}
\label{kotlinsecondaryconstructor}
A~constructor for~a~Kotlin class can~be also created by~the~keyword \itq{constructor}.
Such~constructor is~called \itq{secondary constructor}.
This~is equivalent to~constructors in~Java.
Overloading of~secondary constructors by~different parameters is~possible.

\example[Kotlin class with secondary constructors overloading]
%! language = TEXT
\begin{lstlisting}[language=Kotlin]
    class SimpleClass {
        constructor() {
            ...
        }

        constructor(someVariable: SomeClass) {
            ...
        }
    }
\end{lstlisting}

\noindent A~class can~have both primary and~secondary constructors, but~secondary must always call the~primary~one with the~keyword \mbitq{this}.
Because of~that, it's~guaranteed that whenever an~instance of~the~class is created, the~primary constructor is used no~matter what.
Also, secondary constructors must have different parameters, it's~still constructor \mbox{overloading.}

\example[Kotlin class with primary and secondary constructors]
%! language = TEXT
\begin{lstlisting}[language=Kotlin]
    class SimpleClass(val immutableProperty: SomeClass) {
        // Calling primary constructor from secondary
        constructor() : this(SOME_VALUE) {
            ...
        }

        // Conflict with primary constructor
        // constructor(someVariable: SomeClass) : this(SOME_VALUE) {}

        // Calling primary constructor from secondary
        constructor(someVariable: SomeClass, otherVariable: OtherClass) : this(SOME_VALUE) {
            ...
        }
    }
\end{lstlisting}
\newline

\emten
\noindent Primary constructor can~be empty.
That happens when you specify empty brackets in~a~class header.
It's~equivalent to~specifying a~public parameterless constructor with empty body in~Java.
Remember that, similarly to~Java, such~constructor exists by~default in~Kotlin classes, and~when you define another constructor (with the~\textit{constructor} keyword), the~default one disappears.
When you're in~a~situation where you~need both the~default and~a~non--default constructor (for~example, in~\hyperref[hibernate]{Hibernate} entities), you~must use empty brackets for~the~default constructor.
\newpage

\example[empty primary constructor]
%! language = TEXT
\begin{lstlisting}[language=Kotlin]
    class SimpleClass() {
        // Conflict with primary constructor
        // constructor() : this(SOME_VALUE) {}

        // Calling primary constructor from secondary
        constructor(someVariable: SomeClass) : this() {
            ...
        }
    }
\end{lstlisting}
