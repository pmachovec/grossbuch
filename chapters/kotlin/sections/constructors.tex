\newsection{Constructors}
\label{kotlinconstructor}
A~standard constructor for~a~Kotlin class is~created by~the~keyword \itq{constructor}.
This~is equivalent to~constructors in~Java.
Overloading of~constructors by~different \hyperref[parameterargument]{parameters} is~possible.

\example[Kotlin class with constructor overloading]
%! language = TEXT
\begin{lstlisting}[language=Kotlin]
    class simpleClass {
        constructor() {
            ...
        }

        constructor(someVariable: SomeClass) {
            ...
        }
    }
\end{lstlisting}

\newsubsection{Primary Constructor}
\index{Primary constructor}
\label{kotlinprimaryconstructor}
Typical constructor usage in~Java is just assigning received \hyperref[parameterargument]{arguments} to~class variables (\itq{this.someVariable = someVariable;}).
In~Kotlin there's a~syntax called \itq{primary constructor} for~simplifying this.
It~works by~specifying variables with types in~brackets directly in~a~class header (see~the~example).
It's~possible to~have both primary and~standard constructors, but~standard must always call the~primary~one.
Also, standard must have different \hyperref[parameterargument]{parameters}, it's~still constructor overloading.
\newpage

\example[Kotlin class with primary and standard constructors]
%! language = TEXT
\begin{lstlisting}[language=Kotlin]
    class simpleClass(someVariable: SomeClass) {
        // Calling primary constructor from the standard one
        constructor() : this(SOME_VALUE) {
            ...
        }

        // Conflict with primary constructor
        // constructor(someVariable: SomeClass) : this(SOME_VALUE) {}

        // Calling primary constructor from the standard one
        constructor(someVariable: SomeClass, otherVariable: OtherClass) : this(SOME_VALUE) {
            ...
        }
    }
\end{lstlisting}
\newline

\noindent Primary constructor can~be empty.
That happens when you specify empty brackets in~a~class header.
It's~equivalent to~specifying a~public parameterless constructor with empty body in~Java.
Remember that, similarly to~Java, such~constructor exists by~default in~Kotlin classes, and~when you \hyperref[declarationdefinition]{define} another constructor (with the~\textit{constructor} keyword), the~default one disappears.
When you're in~a~situation where you~need both the~default and~a~non--default constructor (for~example, in~\hyperref[hibernate]{Hibernate} entities), you~must use empty brackets for~the~default constructor.

\example[empty primary constructor]
%! language = TEXT
\begin{lstlisting}[language=Kotlin]
    class simpleClass() {
        // Conflict with primary constructor
        // constructor() : this(SOME_VALUE) {}

        // Calling primary constructor from the standard one
        constructor(someVariable: SomeClass) : this() {
            ...
        }
    }
\end{lstlisting}
\newpage
