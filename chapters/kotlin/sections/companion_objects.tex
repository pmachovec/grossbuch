\newsection{Companion Objects}
\index{Companion object}
\label{kotlincompanionobject}
It's~an~\hyperref[kotlinobject]{object} \hyperref[declarationdefinition]{defined} inside a~standard class.
It~doesn't have own name, it's~just a~code block labeled as \itq{companion object}.
\hyperref[variablefieldproperty]{Fields} and~functions \hyperref[declarationdefinition]{defined} inside a~companion object can~be~accessed over the~main class, like \hyperref[javastatic][static] members in~Java.
Companion objects basically represent those special instances created by~Java for~\hyperref[javastatic][static] members.

\example[class with companion object]
%! language = TEXT
\begin{lstlisting}[language=Kotlin]
    class ClassWithCompanionObject {
        companion object {
            lateinit var simpleValue: String
        }
    }

    ClassWithCompanionObject.simpleValue = "SOMETHING"
\end{lstlisting}

\note Class members outside a~companion object, and~therefore even instances of~the~class, can't~access members of~the~companion object.
This prevents unwanted overriding of~the~singleton instance \hyperref[variablefieldproperty]{fields}.
