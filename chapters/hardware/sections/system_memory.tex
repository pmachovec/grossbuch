\newsection{System Memory}
\index{System memory}
\index{Computer memory}
\index{Operational memory}
\index{Memory}
\label{systemmemory}
System memory, also \textit{computer memory}, \textit{operational memory} or~maliciously simply \textit{memory}, is a~memory from which a~\hyperref[processorcpucore]{processor} reads a~\hyperref[applicationprocessprogramservicethread]{process} instructions and~other data that the~process requires.
I.e.,~to~run a~process the~computer \hyperref[os]{OS} must load its instructions and~required data to~the~system memory, from which the~computer processor can~read~them and~subsequently process~them.

System memories are~almost every time \hyperref[ram]{RAMs} because of~the~fast access to~any of~stored data.
That's why you can often meet the~"state--of--the--art" claim that a~computer has~16~GB of~RAM, but~that isn't accurate.
It~means that the~computer memory can hold up~to~16~GB of~data at~a~time and~the~memory is the~\hyperref[ram]{RAM} type.

\warning Memory doesn't mean \hyperref[harddiskdrive]{hard disk} (permanent computer memory).
That's called \hyperref[harddiskdrive]{\mbit{storage}}.
\newpage
