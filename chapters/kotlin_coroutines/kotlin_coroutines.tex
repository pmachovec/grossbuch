\newchapter{Kotlin Coroutines}
\index{Coroutine}
\label{kotlincoroutine}
Although this belongs under~Kotlin, it's~so~complex topic that it~deserves its own chapter.
Kotlin provides a~library (not~a~part of~the~core) for~handling \hyperref[concurrentparallelasynchronous]{asynchronous programming} as~\hyperref[coroutines]{coroutines}.
It's~an~alternative to~"good" old \hyperref[coroutines]{threads}, which~are expensive and~hard to~maintain, and~also to~various \hyperref[reactiveprogramming]{reactive programming} approaches (Future, Promise, RxKotlin), which~are~generally good, but~with~large amount of~functions for~various situations and~therefore hard to~use.
Also, reactive programming tools tend to~have poor to~no documentation, making their usage even harder.

\note Some sources, including official Kotlin web, claim that passing functions as~parameters to~other functions (i.e.,~callbacks) provides asynchronous behavior.
This~is somehow true.
If~you pass a~function as~callback, the~function will~be~executed inside the~wrapping function, not~at~the~place of~the~pass.
I.e.,~it~will~be executed at~a~different position than~it's~written.
And~that's asynchronicity.

\warning It's~important to~realize that, opposite to~threads, you~can't~hold a~reference to~a~coroutine, there's nothing like \mbit{Coroutine} class.
You~only can~have references to~various objects related to~a~concrete coroutine.

\inputsection{kotlin_coroutines}{scope}
\inputsection{kotlin_coroutines}{context}
\inputsection{kotlin_coroutines}{builders}
\inputsection{kotlin_coroutines}{suspended_functions}
\inputsection{kotlin_coroutines}{dispatchers}
\inputsection{kotlin_coroutines}{coroutinescope}
\inputsection{kotlin_coroutines}{withcontext}
\inputsection{kotlin_coroutines}{jobs}
