\newsection{\textit{coroutineScope}}
\index{coroutineScope}
\label{kotlincoroutinescopefun}
It's~a~\hyperref[kotlinsuspendfunction]{suspended function} to~be~used inside a~coroutine when~you~need some code block to~be~completed before proceeding with~the~coroutine computation.
The~code block is~best to~be~passed in~the~\hyperref[kotlinlambda]{lambda format}, it~can~even return a~value, which~is~returned \mbox{by the \textit{withContext}} function itself, similarly like when~returning a~value from~\mbit{\hyperref[kotlincoroutinerunblocking]{runBlocking}}[.]

At~the~first glance, it~may seem identical to~\mbit{runBlocking}[.]
The~difference is (beside that~\mbit{coroutineScope} being suspended, and~therefore, impossible to~call from~outside a~coroutine) is that~\mbit{runBlocking} waits for~all waiting suspended code to~be~completed before executing its callback.
The~\mbit{coroutineScope}, on~the~other hand, executes its callback immediately.
It~only releases the~thread for~waiting code on~the~same level when~there's a~suspended code \mbox{(e.g.,}~\mbit{launch}[)] inside its own callback.
When~the~waiting code contains another suspended code, the~suspended code from~\mbit{coroutineScope} is~executed first.

\example[\textit{coroutineScope} behavior]
%! language = TEXT
\begin{lstlisting}[language=Kotlin, title={Numbers represent the~order of~execution}]
    runBlocking {
        launch {
            ...4...

            launch {
                // Suspended code from the same level in "coroutineScope" is executed first.
                ...7...
            }

            ...5...
        }

        ...1...

        // Executes its callback immediately.
        coroutineScope {
            ...2...

            // Waits for suspended code only from higher levels.
            launch {
                ...6...
            }

            ...3...
        }

        // Is executed after "coroutineScope" and everything before it finishes.
        ...8...
    }
\end{lstlisting}

\example[same situation, but \textit{coroutineScope} is replaced with \textit{runBlocking}]
%! language = TEXT
\begin{lstlisting}[language=Kotlin, title={Numbers represent the~order of~execution}]
    runBlocking {
        launch {
            ...2...

            launch {
                ...4...
            }

            ...3...
        }

        ...1...

        // Is executed after everything preceding finishes.
        runBlocking {
            ...5...

             launch {
                ...7...
            }

            ...6...
        }

        // Is executed after "runBlocking" and everything before it finishes.
        ...8...
    }
\end{lstlisting}
