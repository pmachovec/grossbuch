\newsection{Scope}
\index{Scope}
\index{Coroutine scope}
\label{kotlincoroutinescope}
Kotlin coroutine is just a~sequence of~commands.
We~say that these commands run in~the~scope of~that~coroutine.
If~a~command in~a~coroutine scope creates another coroutine, then the~new coroutine also runs in~the~scope of~the~original coroutine.
This~way, coroutines are~organized in~a~parent--child hierarchy.
When~a~coroutine is terminated (reaches the~end, fails~etc.), even its descendant coroutines are terminated.

There's an~interface \mbit{CoroutineScope}.
Each~coroutine has~assigned an~instance of~a~specific implementation of~this interface based on the~coroutine type, allowing to~manipulate with~the~scope programmatically.
The~scope of~a~coroutine is typically created automatically and~made accessible inside the~coroutine's \hyperref[kotlincoroutinebuilder]{builder}.
However, it's~possible to~create one in~advance with~the~factory function \mbit{CoroutineScope} and~use a~\hyperref[kotlincoroutinebuilder]{builder} with~it (see~further for~examples).
