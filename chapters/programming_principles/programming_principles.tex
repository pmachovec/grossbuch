\newchapter{Programming Principles}
\index{Programming Principles}
\index{Principle}
A~principle means a~basic idea or~rule that explains or~controls how something happens or~works. In~programming there exist such rules for~writing easily readable, easily maintainable and~robust code.

\newsection{Keep It Simple, Stupid (KISS)}
\index{Keep it simple, stupid}
\index{KISS}
Simplicity of~a~system is~one of~main goals during the~system development. When developing a~system, create it as~simple as~possible to~perform the~intended goal. If~the~system is complex and~can't~be simple as~a~whole (which is typical), separate it to~a~big number of~simple components\,--\,objects and~methods.

For~example, many (wannabe) skilled developers are~crazy about the~\textit{Rule~30}, which says that a~method shouldn't have more than 30 lines. Also, some "experts" say to~avoid more than three arguments for~a~method or~constructor. If~you need more, wrap them to~a~new class, instantiate it (which is kind of a~paradox as~you need to~pass all those arguments to~that instance somehow) and~use that single instance.

\newsection{Inversion of~Control (IoC)}
\index{Inversion of control}
\index{IoC}
\label{inversionofcontrol}
Use a~general reusable framework to~call custom code (code written by a~programmer to~achieve a~concrete goal). The~custom code must exist in a~form of~components matching templates accepted by the~framework. This approach is reversed to~the~traditional \hyperref[imperativeprogramming]{imperative programming} control flow, where the~custom code calls reusable frameworks. In~IoC you of~course needed even some small amount of~the~custom code to~initiate the~framework to~do~something (for~example, you~still need a~main method in~Java).

The~purpose is to~achieve better modularity (less \hyperref[loosetightcoupling]{coupling}), better maintainability and~simpler source code of~the~implemented program. This is possible because the~program exists in~the~form of~only very weakly dependent components.

The~most typical example is~the~\hyperref[springinversionofcontrol]{inversion of~control from the~Java Spring framework}. In~this framework classes are~instantiated by~calling methods of~Spring containers instead~of directly calling constructors.

\newsection{Dependency Injection}
\index{Dependency injection}
\index{Dependency}
\label{dependencyinjection}
An~object used by another object is~called \textit{dependency}. References to~dependencies are~stored in~\hyperref[variablefieldproperty]{fields}. And~setting concrete references to~these \hyperref[variablefieldproperty]{fields} from outside is the~dependency injection. I.e.,~when you instantiate a~class and~set an~instance of~another class to~a~\hyperref[variablefieldproperty]{field} inside the~first class instance by~a~constructor or~by~a~setter, you~perform the~dependency injection.

The~purpose is to~separate construction and~use of~objects, which leads to~better readability and~reusability of~the~code. Dependency injection is a~special type of~the~\hyperref[inversionofcontrol]{inversion of~control}\,--\,the~responsibility for~instantiating of~a~dependency is~delegated to~an~injector, which can~be seen as~the~reusable framework from the~original \hyperref[inversionofcontrol]{IoC} definition.

The~\hyperref[springframework]{Java Spring framework} provides an~\hyperref[springdependencyinjection]{automated and~highly configurable dependecy injection} implementation.