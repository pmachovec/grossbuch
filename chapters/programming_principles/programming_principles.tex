\newchapter{Programming Principles}
\index{Programming Principles}
\index{Principle}
A~principle means a~basic idea or~rule that explains or~controls how something happens or~works.
In~programming there exist such rules for~writing easily readable, easily maintainable and~robust code.

\newsection{Keep It Simple, Stupid (KISS)}
\index{Keep it simple, stupid}
\index{KISS}
Simplicity of~a~system is~one of~main goals during the~system development.
When developing a~system, create it as~simple as~possible to~perform the~intended goal.
If~the~system is complex and~can't~be simple as~a~whole (which is typical), separate it to~a~big number of~simple components\,--\,objects and~methods.

For~example, many (wannabe) skilled developers are~crazy about the~\textit{Rule~30}, which says that a~method shouldn't have more than 30 lines.
Also, some "experts" say to~avoid more than three arguments for~a~method or~constructor.
If~you need more, wrap them to~a~new class, instantiate it (which is a~paradox as~you need to~pass all those arguments to~that instance somehow) and~use that single instance.

\newsection{Inversion of~Control (IoC)}
\index{Inversion of control}
\index{IoC}
\label{inversionofcontrol}
Use a~general reusable framework to~call custom code (code written by a~programmer to~achieve a~concrete goal).
The~custom code must exist in a~form of~components matching templates accepted by the~framework.
This approach is reversed to~the~traditional \hyperref[imperativeprogramming]{imperative programming} control flow, where the~custom code calls reusable frameworks.
In~IoC you of~course need at~least some small amount of~the~custom code to~initiate the~framework to~do~something (for~example, you~still need a~main method in~Java).

The~purpose is to~achieve better modularity (less \hyperref[loosetightcoupling]{coupling}), better maintainability and~simpler source code of~the~implemented program.
This is possible because the~program exists in~the~form of~only very weakly dependent components.

The~most typical example is~the~\hyperref[springinversionofcontrol]{inversion of~control from the~Java Spring framework}.
In~this framework classes are~instantiated by~calling methods of~Spring containers instead of~directly calling constructors.

\newsection{Dependency Injection}
\index{Dependency injection}
\index{Dependency}
\label{dependencyinjection}
An~object used by another object is~called \textit{dependency}.
References to~dependencies are~stored in~\hyperref[variablefieldproperty]{fields}.
And~setting concrete references to~these \hyperref[variablefieldproperty]{fields} from outside is the~dependency injection.
I.e.,~when you instantiate a~class and~set an~instance of~another class to~a~\hyperref[variablefieldproperty]{field} inside the~first class instance by~a~constructor or~by~a~setter, you~perform the~dependency injection.

The~purpose is to~separate construction and~use of~objects, which leads to~better readability and~reusability of~the~code.
Dependency injection is a~special type of~the~\hyperref[inversionofcontrol]{inversion of~control}\,--\,the~responsibility for~instantiating of~a~dependency is~delegated to~an~injector, which can~be seen as~the~reusable framework from the~original \hyperref[inversionofcontrol]{IoC} definition.

The~\hyperref[springframework]{Java Spring framework} provides an~\hyperref[springdependencyinjection]{automated and~highly configurable dependency injection} implementation.

\newsection{Composition Over Inheritance}
\index{Composition}
\index{Composite reuse principle}
\label{compositionoverinheritance}
\hyperref[inheritance]{Inheritance}, although a~fundamental concept of~\hyperref[objectorientedprogramming]{OOP}, brings more problems than~use.
Many developers (allegedly even \href{https://en.wikipedia.org/wiki/James_Gosling}{James Gosling}, the~author of~Java) repeat the~mantra saying that \itq{inheritance is evil}.
To~address these problems the~principle of~composition over inheritance, also called \itq{composite reuse principle}, is~preferred.
Creating common functionality by this principle consists of~following steps:
\begin{itemize}
    \item Enforce the~common functionality by an~\hyperref[javainterface]{interface}.
    \item Make classes, which should contain the~common functionality, implementing the~interface.
    \item In~affected classes, delegate the~implementation to~some common descendant.
          The~descendant itself doesn't implement the~interface.
          It~should~be \hyperref[singletondp]{singleton}.
\end{itemize}

\noindent This approach deals with all~problems brought by inheritance:
\begin{itemize}
    \item \hyperref[abstraction]{Abstraction} isn't broken, because there's no ancestor implementation necessary to~know.
          Only an~interface is implemented, and~that generally doesn't contain an~implementation.
    \item When some common functionality isn't wanted in~a~class any~more, the~class can~simply stop implementing the~interface.
          The~unwanted common functionality then stops being enforced and~can~be removed.
          The~rest of~common functionality, which is still wanted, remains in~the~class.
    \item When there's a~problem with the~common code call, investigating downwards to~a~correct descendant is~much easier as~the~descendant class is clearly identifiable.
          Furthermore, logs and~stack traces identify the~problematic place more clearly.
\end{itemize}

\warning The~composite reuse principle doesn't follow the~\hyperref[compositedp]{composite design pattern}.
The~described structure is different.
\newpage

\enlargethispage{20mm}
\thispagestyle{empty}
\example[composition over inheritance in~Java]
\begin{lstlisting}[language=Java, title={Common functionality enforcing interface}]
    public interface EnforcingInterface {
        void commonFunctionality();
    }
\end{lstlisting}
\begin{lstlisting}[language=Java, title={First class to~contain common functionality}]
    public class FirstClass implements EnforcingInterface {
        private final CommonFunctionalityDescendant theDescendant;

        public FirstClass() {
            theDescendant = CommonFunctionalityDescendant.getDescendant();
        }

        @>@Override
        public void commonFunctionality() {
            theDescendant.commonFunctionalityImplementation()
        }
    }
\end{lstlisting}
\begin{lstlisting}[language=Java, title={Second class to~contain common functionality}]
    public class SecondClass implements EnforcingInterface {
        private final CommonFunctionalityDescendant theDescendant;

        public SecondClass() {
            theDescendant = CommonFunctionalityDescendant.getDescendant();
        }

        @>@Override
        public void commonFunctionality() {
            theDescendant.commonFunctionalityImplementation()
        }
    }
\end{lstlisting}
\begin{lstlisting}[language=Java, title={\hyperref[singletondp]{Singleton} descendant containing the~common functionality implementation}]
    public class CommonFunctionalityDescendant {
        ...SINGLETON STUFF...

        public static CommonFunctionalityDescendant getDescendant() {
            ...RETURN THE SINGLETON INSTANCE...
        }

        public void commonFunctionalityImplementation() {
            ...
        }
    }
\end{lstlisting}
