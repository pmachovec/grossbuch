\newchapter{Programming Principles}
\index{Programming Principles}
\index{Principle}
A~principle means a~basic idea or~rule that explains or~controls how something happens or~works. In~programming there exist such rules for~writing easily readable, easily maintainable and~robust code.

\newsection{Keep It Simple, Stupid (KISS)}
\index{Keep it simple, stupid}
\index{KISS}

\newsection{Inversion of~Control (IoC)}
\index{Inversion of control}
\index{IoC}
\label{inversionofcontrol}
Use a~general reusable framework to~call custom code (code written by a~programmer to~achieve a~concrete goal). The~custom code must exist in a~form of~components matching templates accepted by the~framework. This approach is reversed to~the~traditional \hyperref[imperativeprogramming]{imperative programming} control flow, where the~custom code calls reusable frameworks. In~IoC you of~course needed even some small amount of~the~custom code to~initiate the~framework to~do~something (for~example, you~still need a~main method in~Java).

The~purpose is to~achieve better modularity (less \hyperref[loosetightcoupling]{coupling}), better maintainability and~simpler source code of~the~implemented program. This is possible because the~program exists in~the~form of~only very weakly dependent components.

The~most typical example is~the~\hyperref[springinversionofcontrol]{inversion of~control from the~Java Spring framework}. In~this framework classes are~instantiated by~calling methods of~Spring containers instead~of directly calling constructors.

\newsection{Dependency Injection}
\index{Dependency injection}