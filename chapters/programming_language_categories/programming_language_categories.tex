\newchapter{Programming Language Categories}

\newsection{Compiled VS Interpreted Languages}
\index{Compiled language}
\index{Interpreted language}
\index{Compilation}
\index{Compiler}
\index{Interpreter}
\label{compiledinterpretedlanguages}
Each~source code in~any~language must~be somehow transferred to~a~machine code to~be~executed.
This can~happen either as~a~separate step, or~directly as~a~part of~the~execution.

The~process of~transforming a~source code to~some more machine--readable code before its execution is~called \textit{compilation}.
There's a~software called \textit{compiler} performing the~transformation.
A~language processed by~a~compiler is~called \textit{compiled language}.
The~result of~the~compilation can~be either directly the~machine code, or~some other, further processed form.
For~example, \mbox{C~or~C++} are~compiled directly to~the~machine code.
Because of~that a~special compiler is~needed for~each~\hyperref[platform]{platform} for~these languages.
On~the~other hand, Java or~C\# are~compiled to~an~intermediate form, which must~be further transferred to~the~machine code, but~the~intermediate form can~be the~same for~all~\hyperref[platform]{platforms} (see~\hyperref[bytecode]{Java bytecode} for~example).

A~source code of~some languages can~be seemingly executed directly on~the~fly, without any compilation.
Such languages are~called \textit{interpreted languages}.
There's a~software called \textit{interpreter}, which transfers the~source code to~the~machine code and~executes~it immediately.
For~each such language a~special interpreter is~needed for~each~\hyperref[platform]{platform}.
Developing a~working \hyperref[applicationprocessprogramservicethread]{program} in~an~interpreted language is generally simpler than in~a~compiled language, but~programs written in~interpreted languages are~usually slower than~equivalents written in~compiled languages.
Interpreted languages are~therefore more suitable for~writing smaller and~simpler programs, for~example, \hyperref[scriptinglanguages]{scripts}.
Examples of~typical interpreted languages are JavaScript, \hyperref[powershell]{PowerShell} or~\hyperref[shbash]{Bash}.

\warning There is no pure compiled or~interpreted language.
Compilation and~interpretation are just ways of~transforming some text following some syntax (i.e.,~a~source code) to~a~machine code.
There can~theoretically exist both a~compiler and~an~interpreter for~any language.
\newpage

\newsection{Scripting Languages}
\index{Scripting language}
\label{scriptinglanguages}
There isn't an~exact definition what a~scripting language is, but~usually the~term denotes a~language, whose source code files\,--\,scripts\,--\,are~immediately available for~execution.
I.e.,~it's~a~synonym for~an~\hyperref[compiledinterpretedlanguages]{interpreted language}.

A~scripting language always has a~special run--time environment (an~\hyperref[compiledinterpretedlanguages]{interpreter}) capable of~real time execution of~scripts.
Scripting languages aren't limited to~only small scripts, even large and~complex programs can~be written in~them.
Examples of~languages generally considered to~be scripting are~JavaScript, \hyperref[powershell]{PowerShell}, Python, Groovy, \hyperref[shbash]{Bash} or~VisualBasic.

Scripting languages are~popular for~tasks automation.
A~developer writes a~script triggering multiple tasks.
Then, instead~of executing each task separately, only the~script is~executed.

\newsubsection{Shebang}
\index{Shebang}
\index{Hashbang}
\label{shebang}
Scripts in~Linux systems sometimes contain a~special first line starting with hash sign and~exclamation mark (\itq{\#!}), following a~name~of, or~full path~to, the~corresponding executing program.
This line is~called \textit{shebang}, or~\textit{hashbang}.
When the~script is~executed without specifying the~executor directly, \hyperref[os]{operating system} decides by~the~shebang what executor to~use.
Shebang can~directly point to~a~specific program, or~can~tell the~system to~resolve the~path (the~executing program must~be included in~the~\textit{PATH} system variable).
It~can~even contain switches.
\newline

\noindent \textbf{Examples:}
\begin{itemize}
    \item \textit{\#!/bin/sh} = \hyperref[shbash]{SH} symbolic link
    \item \textit{\#!/bin/bash -v} = \hyperref[shbash]{Bash}, direct path, verbose output
    \item \textit{\#!/usr/bin/env node} = \hyperref[nodejs]{Node.js}, resolved path
\end{itemize}

\newsection{Language Typing Systems}
Based on~how a~programming languages work with \hyperref[datatypes]{data types} of~their variables, they can~be divided to~statically and~dynamically typed and~to~strongly and~weakly typed.
That makes four categories in~total.
A~language exists for~each of~these categories.

\newsubsection{Statically VS Dynamically Typed Languages}
\begin{itemize}
    \itembf{Statically typed language} checks correctness of~data types before a~program can~be~executed, typically (but~not necessarily, see~further) during \hyperref[compiledinterpretedlanguages]{compilation}.
            As~a~compulsory part of~the~source code for~each variable it~must~be stated what type it~has.
            For~example, if~you~\hyperref[declarationdefinition]{declare} a~variable as~string and~then assign a~number to~it, your~program won't even start.
            Statically typed languages are~more time--consuming during development (the~code is~more verbose), but~programs are~generally faster, more stable and~easier to~debug.
    \itembf{Dynamically typed language} doesn't check correctness of~data.
            A~type of~a~variable is~decided at~program runtime based on~the~first assignment of~a~value to~that~variable.
            When later in~the~program some other type value is~assigned to~the~variable, the~program can (but~doesn't always have~to) fail.
            For~example, if~you~assign a~string to~a~variable and~then try to~multiply~it by~a~number, the~(running) program will~fail at~the~multiplication step.
            Source code is~easier to~write in~dynamically typed languages, but~programs are~generally slower, more unstable and~harder to~debug.
\end{itemize}

\warning Usual definition of~statically typed languages says that the~type check is~performed during compilation.
This implies that languages, for~which an~\hyperref[compiledinterpretedlanguages]{interpreter} exist, can't~be statically typed, because there's no~compilation for~them.
However, this is not~necessarily true.
Although it's extremely rare, there can~be a~type check before interpreting, i.e.,~directly executing, a~source code.
Some variations of~Haskell do~this.

\newsubsection{Strongly VS Weakly Typed Languages}
\begin{itemize}
    \itembf{Strongly typed language} doesn't allow changes of~a~variable type.
            Once a~variable type is~decided, either before or~at~runtime, it's~fixed until the~program ends.
            For~example, consider a~dynamically typed language and~a~variable.
            At~first a~string is~assigned to~the~variable.
            When you~then try to~assign a~number to~it, the~program will~fail (at~runtime).
            Strongly typed languages are~more time--consuming during development, but~programs are~generally more stable and~easier to~debug.
    \itembf{Weakly typed language} allows variable type changes.
            Various values of~various types can~be assigned to~one~variable through a~program.
            Consider the~same example\,--\,a~dynamically typed language and~a~variable.
            At~first a~string is~assigned to~the~variable.
            This time, when you try to~assign a~number to~it, the~original string will~be overridden by~the~number and~program will~continue running.
            Source code is~easier to~write in~weakly typed languages, but~programs can~behave unexpectedly and~are~harder to~debug.
\end{itemize}

\warning Although it~may~seem impossible, statically weakly typed languages do~exist, namely C and~C++.
In~these lower level languages each value is~treated as~a~bunch of~bytes.
Therefore, it's~possible to~assign for~example a~string to~a~variable \hyperref[declarationdefinition]{declared} as~integer.
Byte representation of~the~string is~then treated as~an~integer.
\newpage

\begin{table}[ht]
    \begin{adjustbox}{center}
        \begin{tabu}{|[2pt]c|[2pt]l|l|[2pt]}
            \tabucline[2pt]{2-}
            \multicolumn{1}{r|[2pt]}{}& \textbf{Statically typed languages} & \textbf{Dynamically typed languages}\\
            \tabucline[2pt]{-}
            \multirow{4}{*}{\textbf{Strongly typed languages}} & Java & Groovy\\
                & Kotlin & Python\\
                & C\# & Ruby\\
                & Haskell & Erlang\\
            \tabucline{-}
            \multirow{4}{*}{\textbf{Weakly typed languages}} & C & JavaScript\\
                & C++ & PHP\\
                & & VisualBasic\\
                & & Perl\\
            \tabucline[2pt]{-}
        \end{tabu}
    \end{adjustbox}
    \captit{Language examples for each typing system category}
\end{table}
