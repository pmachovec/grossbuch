\newsection{Registries}
A~repository with packages for~NPM is~called \textit{registry}.
By~default NPM uses the~registry with the~url \href{https://registry.npmjs.org/}{\textit{https://registry.npmjs.org/}} available on~the~web.
Another registry can~be configured by~the~key \mbitq{registry}.
There can~be only~one (main) registry configured at~a~time.
If~you~need more, you~must use \hyperref[npmscope]{scopes} or~switch \hyperref[npmprofile]{profiles}.

\newsubsection{Scopes}
\index{Scope}
\label{npmscope}
To~set a~scope for~a~registry simply add \mbitq{@SCOPE\_NAME:} before the~key \mbitq{registry} when configuring the~registry (mind the~colon).
You~should have something like \mbitql{@SCOPE\_NAME:registry=;;REGISTRY\_URL} in~the~\mbit{.npmrc} file.
With (unique) scopes you~can~configure an~unlimited number of~registries.
However, remember that packages installed from scoped registries have limitations, see~\hyperref[nsmscopedpackages]~further.

\newsubsection{Installed Packages Usage}
\label{npmpackageusage}
Installed package are~available as~standard \hyperref[nodejscommonjs]{CommonJS} modules by~their names.
Specifying a~whole path to~a~package is~not~needed.
They can~be loaded even \hyperref[amd]{asynchronously}.

\example[standard package loading]
%! language = TEXT
\begin{lstlisting}[language=JavaScript]
    let packageHandler = require('PACKAGE_NAME');
    packageHandler.doSomething(...);
\end{lstlisting}

\example[loading package with \hyperref[requirejs]{\textit{RequireJS}}]
%! language = TEXT
\begin{lstlisting}[language=JavaScript]
    let requirejs = require('requirejs');
    ...
    requirejs(['PACKAGE_NAME'], function(packageHandler) {
      packageHandler.doSomething(...);
    });
\end{lstlisting}

\newsubsection{Scoped Packages}
\label{nsmscopedpackages}
To~install a~package from~a~scoped registry use \mbitqls{npm;;;install;;;@SCOPE\_NAME/;;PACKAGE\_NAME}.
Such~package is~stored into a~subfolder \mbitq{@SCOPE} of~the~usual package folder (project--specific or~global).
The~scope must~be used even in~the~code when working with the~package, i.e.,~something like \mbitqls{let;;;packageHandler =;;;require('@SCOPE\_NAME/;;PACKAGE\_NAME')}.

\warning Avoid any~global usage of~scopes, both in~\hyperref[npmconfiguration]{configuration} and~installation.
Always configure scoped registries in~project--level \mbit{.npmrc} files and~install scoped packages per~project.
If~you~use scopes globally and~someone else works with your code in~a~different system, his~global scopes configuration will~have to~be exactly the~same as~yours.
If~any~scope name or~registry URL is~different, the~code will~not~work.

\enlargethispage{10mm}
\thispagestyle{empty}
\newsection{Configuration Profiles}
\label{npmprofile}
This~is the~only way how~to at~least somehow have multiple main registries.
The~principle~is changing the~global \mbit{.npmrc} file and~switching between various global configurations.
The~standard text version of~the~file is~replaced by~a~symbolic link with the~same name.
The~link points to~one of~files in~the~folder \mbitq{HOME/.npmrcs}.
These files have the~\mbit{key=value} structure and~each represents one~configuration -- one~profile.
Their names define profile names, they don't have extensions.

There's a~Node.js implementation of~this behavior called \mbitq{npmrc} (yes,~like the~configuration file) available in~the~default NPM~registry.
Install~it globally as~any~other package.
A~default profile called \mbitq{default} will~be created from~the~actual configuration.
To~create a~new~profile run \mbitqls{npmrc -c;;;PROFILE\_NAME}.
To~switch to~a~different profile run \mbitqls{npmrc;;;PROFILE\_NAME}.
To~see a~list of~available profiles run \mbitq{npmrc} without any parameter.
To~delete a~profile delete the~corresponding file under the~\mbitq{HOME/.npmrcs} folder.
\newpage

\warning You~need a~right to~create symbolic links in~the~home folder.
After each profile switch the~old \mbit{.npmrc} link is~deleted and~new is~created.

\warning There's no~way to~have some configuration common to~all profiles.
If~you~want to~have some common configuration, you~must set the~corresponding \mbit{key=value} pair for~each profile -- to~each profile file.
And~you sure want to~have at~least the~global packages installation folder common as~you~use a~globally installed package -- the~\mbit{npmrc}.
