\newchapter{Node Package Manager (NPM)}
\index{NPM}
\index{Node Package Manager}
\label{npm}
It's~a~\hyperref[packagemanager]{package manager} for~the~\hyperref[nodejs]{Node.js} \hyperref[platform]{platform}.
It's~provided together with a~Node.js installation, it~isn't needed to~install it separately.
The~official web site is~\href{https://www.npmjs.com/}{\textit{www.npmjs.com}}.

\newsection{Configuration}
\label{npmconfig}
NPM~configuration is~stored in~\mbitq{key=value} pairs.
There's a~lot of~built--in pairs, but~you~can (and~often must) specify others.
Pairs can~be defined in~a~special file called \mbitq{.npmrc} (with the~dot at~the~beginning).
It~can~be created either in~user home directory (global configuration) or~per~project in~a~project root folder.

\newsubsection{Configuration with CLI}
Configurations can~be set even from~\hyperref[shellcligui]{CLI} by~the~command \mbitqls{npm;;;config;;;set;;;KEY;;;VALUE}.
This will add the~pair to the~user--specific configuration file (global).
Storing pairs to~a~project--specific \mbit{.npmrc} file is~possible only manually.

\note System variables can~be used in~NPM configurations with the~syntax \mbitq{\$\{VARIABLE\}}.

\newsection{Registries}
A~repository with packages for~NPM is~called \textit{registry}.
By~default NPM uses the~registry with the~url \href{https://registry.npmjs.org/}{\textit{https://registry.npmjs.org/}} available on~the~\hyperref[internetweb]{web}.
Another registry can~be configured by~the~key \mbitq{registry}.
There can~be only~one (main) registry configured at~a~time.
If~you~need more, you~must use \hyperref[npmscope]{scopes} or~switch \hyperref[npmprofile]{profiles}.

\newsubsection{Scopes}
\index{Scope}
\label{npmscope}
To~set a~scope for~a~registry simply add \mbitq{@SCOPE\_NAME:} before the~key \mbitq{registry} when configuring the~registry (mind the~colon).
You~should have something like \mbitql{@SCOPE\_NAME:registry=;;REGISTRY\_URL} in~the~\mbit{.npmrc} file.
With (unique) scopes you~can~configure an~unlimited number of~registries.
However, remember that packages installed from scoped registries have limitations, see~\hyperref[nsmscopedpackages]~further.

\newsubsection{Installed Packages Usage}
\label{npmpackageusage}
Installed package are~available as~standard \hyperref[commonjs]{CommonJS} modules by~their names.
Specifying a~whole path to~a~package is~not~needed.
They can~be loaded even \hyperref[amd]{asynchronously}.

\example[standard package loading]
%! language = TEXT
\begin{lstlisting}[language=JavaScript]
    let packageHandler = require('PACKAGE_NAME');
    packageHandler.doSomething(...);
\end{lstlisting}

\example[loading package with \hyperref[requirejs]{\textit{RequireJS}}]
%! language = TEXT
\begin{lstlisting}[language=JavaScript]
    let requirejs = require('requirejs');
    ...
    requirejs(['PACKAGE_NAME'], function(packageHandler) {
      packageHandler.doSomething(...);
    });
\end{lstlisting}

\newsubsection{Scoped Packages}
\label{nsmscopedpackages}
To~install a~package from~a~scoped registry use \mbitqls{npm;;;install;;;@SCOPE\_NAME/;;PACKAGE\_NAME}.
Such~package is~stored into a~subfolder \mbitq{@SCOPE} of~the~usual package folder (project--specific or~global).
The~scope must~be used even in~the~code when working with the~package, i.e.,~something like \mbitqls{let;;;packageHandler =;;;require('@SCOPE\_NAME/;;PACKAGE\_NAME')}.

\warning Avoid any~global usage of~scopes, both in~\hyperref[npmconfig]{configuration} and~installation.
Always configure scoped registries in~project--level \mbit{.npmrc} files and~install scoped packages per~project.
If~you~use scopes globally and~someone else works with your code in~a~different system, his~global scopes configuration will~have to~be exactly the~same as~yours.
If~any~scope name or~registry URL is~different, the~code will~not~work.

\enlargethispage{10mm}
\thispagestyle{empty}
\newsection{Configuration Profiles}
\label{npmprofile}
This~is the~only way how~to at~least somehow have multiple main registries.
The~principle~is changing the~global \mbit{.npmrc} file and~switching between various global configurations.
The~standard text version of~the~file is~replaced by~a~symbolic link with the~same name.
The~link points to~one of~files in~the~folder \mbitq{HOME/.npmrcs}.
These files have the~\mbit{key=value} structure and~each represents one~configuration -- one~profile.
Their names define profile names, they don't have extensions.

There's a~Node.js implementation of~this behavior called \mbitq{npmrc} (yes,~like the~configuration file) available in~the~default NPM~registry.
Install~it globally as~any~other package.
A~default profile called \mbitq{default} will~be created from~the~actual configuration.
To~create a~new~profile run \mbitqls{npmrc -c;;;PROFILE\_NAME}.
To~switch to~a~different profile run \mbitqls{npmrc;;;PROFILE\_NAME}.
To~see a~list of~available profiles run \mbitq{npmrc} without any parameter.
To~delete a~profile delete the~corresponding file under the~\mbitq{HOME/.npmrcs} folder.
\newpage

\warning You~need a~right to~create symbolic links in~the~home folder.
After each profile switch the~old \mbit{.npmrc} link is~deleted and~new is~created.

\warning There's no~way to~have some configuration common to~all profiles.
If~you~want to~have some common configuration, you~must set the~corresponding \mbit{key=value} pair for~each profile -- to~each profile file.
And~you sure want to~have at~least the~global packages installation folder common as~you~use a~globally installed package -- the~\mbit{npmrc}.

\newsection{Creating Custom Registry}
Similarly to~\hyperref[gradle]{Gradle} repositories, creating a~real NPM registry is~complicated and~must~be done with \href{https://jfrog.com/artifactory/}{Artifactory}.
But~similarly to~\hyperref[gradle]{Gradle} it's also possible to~use a~local folder or~Github.

\newsubsection{Local Folder}
\label{npmpublishlocal}
Simply choose a~folder and~copy the~project folder there.
It~should contain the~\mbit{package.json} file and~source files.
The~\mbit{node\_modules} doesn't have to~be~included.
When the~project is~installed by~NPM, all~dependencies will~be~downloaded automatically.
The~\mbit{package.json} file must contain package name, three--number version and~the~path to~the~root file.
And~of~course dependencies must~be mentioned to~be~automatically downloaded.

\example[\mbit{package.json} file of~a~distributable project]
%! language = TEXT
\begin{lstlisting}
    {
      "name": "simplepackage",
      "main": "src/main/js/mainFile.js",
      "version": "1.0.0",
      "dependencies": {
        "requirejs": "^2.3.6"
      }
    }
\end{lstlisting}

\note The~package name can~differ from the~copied project folder name.
\newline

\noindent To~install the~project as~an~NPM package use the~path to~the~copied project folder prefixed with \mbitq{file:} (with the~colon at~the~end).
I.e.,~to~install it in~a~\hyperref[shellcligui]{shell} run \mbitqls{npm;;;install;;;file:FOLDER\_PATH} (you~don't use the~package name in~this case).
To~include it to~\mbit{package.json} as~a~dependency use \mbitqls{"PACKAGE\_NAME":;;;"file:FOLDER\_PATH"}.
The~path can~be relative to~the~\mbit{package.json} file location.
It~doesn't copy the~downloaded project folder to~\mbit{node\_modules}, but~only creates symbolic link to~it there.
If~the~downloaded project has~configured some dependencies in~its \mbit{package.json}, they're downloaded to~the~project folder when they're missing there.

\newsubsection{Publishing to GitHub}
\label{npmpublishgithub}
First you need a~token with writing rights to~your GitHub account.
You should store hash of~this token to~the~global \mbit{.npmrc} configuration (eventually to~a~\hyperref[npmprofile]{profile}) in~your system.
You don't want to~distribute it with projects.
The~hash must~be stored in~the~pattern \mbitql{//npm.pkg.github.com/:;;\_authToken=;;THE\_HASH}.

Then you~need to~configure the~project you~want to~publish in~its \mbit{package.json} file.
The~basic structure is~the~same as~\hyperref[npmpublishlocal]{when publishing to~a~local folder}, but~the~package name must~be prefixed with your GitHub user name in~the~pattern \mbitq{@USER\_NAME/} (at~sign at~the~beginning, slash at~the~end).
Additionally you~must configure registry and~repository~URLs.
The~registry URL is~a~GitHub global URL used for~the~NPM registry functionality.
The~repository URL is~the~SSH URL to~a~concrete Git repository which serves as~your~package registry.
If~the~repository has~the~same name as~the~project without the~user name prefix, you~don't have~to specify the~repository.
See~the~following example for~the~exact syntax.

\example[\mbit{package.json} file of~a~project distributable to~GitHub]
%! language = TEXT
\begin{lstlisting}
    {
      "name": "@USER_NAME/simplepackage",
      "main": "src/main/js/mainFile.js",
      "version": "1.0.0",
      "dependencies": {
        "requirejs": "^2.3.6"
      },
      "publishConfig": {
        "registry": "https://npm.pkg.github.com/"
      },
      "repository": {
        "type": "git",
        "url": "ssh://git@github.com/USER_NAME/REPO_NAME.git"
      }
    }
\end{lstlisting}

\noindent With the~\mbit{package.json} ready in~\hyperref[shellcligui]{CLI} go to~the~project root folder (where the~\mbit{package.json} should~be located) and~simply run \mbitq{npm publish}.
\newline

\noindent To~install the~package in~a~different project you must configure the~main NPM registry (the~one without a~\hyperref[npmscope]{scope}) to~the~user--specific GitHub NPM registry~URL\@.
I.e.,~there must~be something like \mbitqls{registry=;;https://;;npm.pkg.github.com/;;USER\_NAME} in~global or~project--specific \mbit{.npmrc} file.
GitHub registry is~connected with the~default NPM registry and~NPM~installation of~other features will still work.
However, if~you~already have the~main registry configured to~some other URL,  you're screwed, see~\hyperref[multipleregistries]{further}.

Additionally you~need a~token with reading rights for~the~account, even when the~repository is~public.
It~doesn't have to~be the~same token with which the~package was~published (but~can~be, especially on~the~same computer).
It~can~be even a~token only for~reading without writing rights, that~can~be distributed to~a~wide audience of~users.

Packages installation is~the~same as~usual.
Just use the~GitHub user name as~the~\hyperref[nsmscopedpackages]{scope}.
For~example, \mbitqls{npm;;;install;;;@USER\_NAME/;;simplepackage}.
I.e.,~you~don't have configured a~scope, but~you~act like you~do.

\warning It~may~seem acceptable to~create a~read--only token and~commit~it in~a~project--specific \mbit{.npmrc} file, or~even commit~it to~the~repository which you~use as~NPM registry.
However, don't do any~of~that.
Committing tokens is~considered to~be a~security violation, no~matter what rights they guarantee.
GitHub contains an~automatic security tool that can~discover a~committed token.
And~if it does, it~will~revoke the~token without any~possibility to~take it back.

\newsubsection{Multiple Registries}
\label{multipleregistries}
Only one main registry can~be configured at~a~time and~there's no way around.
And~as~GitHub registries need to~be~configured as~main (can't~be hidden behind a~\hyperref[npmscope]{scope}), you~can't use a~different main registry than the~default~one (for~example, inside a~company), which~is connected with GitHub.
The~only option here is switching \hyperref[npmprofile]{profiles}.

What is possible is to~refer to~multiple GitHub user registries.
Configure a~first user registry as~described above, but~for~others use user names as~scopes, not~as~final parts of~URLs.
You~should~have something like this in \mbit{.npmrc} file:
%! language = TEXT
\begin{lstlisting}[frame=no]
    registry=https://npm.pkg.github.com/USER1
    @USER2:registry=//npm.pkg.github.com
    @USER3:registry=//npm.pkg.github.com
\end{lstlisting}

\newsubsection{Deleting Published Packages}
When a~package is~\hyperref[npmpublishlocal]{published to~a~local folder}, it's~simple.
Just delete the~folder with the~package.
But~when a~package is~\hyperref[npmpublishgithub]{published to~GitHub}, deleting it is complicated and~restricted in~certain ways.
You~can't delete a~package with NPM at~all.
You~must use the~GitHub \hyperref[graphql]{GraphQL}~\hyperref[api]{API} and~send queries for~\hyperref[githubdeletepackage]{general package removal}.
