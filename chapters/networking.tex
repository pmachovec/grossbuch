\newsection{Client--Server Architecture}
\index{Client--server architecture}
\label{clientserverarchitecture}
\todo It's a\dots Client is\dots Server is\dots

\newsection{Network protocols}

\newsubsection{TCP}
\index{IP}
\label{tcp}

\newsubsection{IP}
\index{IP}
\label{ip}

\newsubsection{HTTP}
\index{HTTP}
\label{http}

\newsubsection{SMTP}
\index{SMTP}
\label{smtp}

\newsubsection{SSH}
\index{SSH}
\label{ssh}
It's a~protocol for~network communication that includes cryptographic features to~encrypt the~communication. It~enables secure communication over a~network in~a~\hyperref[clientserverarchitecture]{client--server architecture} that generally is~not secured. The abbreviation stands for \textit{Secure Shell}. It's typically used for~remote command line login and~remote commands execution, but~any network service can be secured with SSH (e.g., file transfer or~tunneling). It~uses \hyperref[asymmetricencryption]{asymmetric encryption} approach to~secure the~network communication.

\newsubsection{SFTP}
\index{SFTP}
\label{sftp}
It's a~protocol for~secured network file access, file transfer, and~file management. It's an~extension of~the \hyperref[ssh]{SSH protocol}, therefore, when working with SFTP, you often have to~deal with documentation, terms and~when programming also object names that contain the text \textit{SSH}. For~example, one~of the~current Java libraries for~SFTP file management is called \textit{SSHJ}.

\warning It has nothing to~do with the~FTP protocol. That is completely different, very old and~unsecured protocol for network file transferring.

\newsection{Internet VS Web}
\index{Internet}
\index{Web}
\label{internetweb}
\begin{itemize}
    \item \textbf{Internet} is a~huge computer network connecting smaller networks, computers, smartphones and~other devices. Simply it's one big computer network constructed all over the world. Sometimes it's referred as a~network of~networks. The~word \textit{Internet} therefore denotes the~connecting infrastructure, i.e.,~hardware.
    \item \textbf{Web}, more precisely \textit{World Wide Web}, is~one of~more software systems using the~Internet infrastructure. It~mainly serves for~sharing visualised and mutualy linked HTML documents accessed by web browsers, but~can also provide \hyperref[webservice]{web services}. Information exchange on~the~Web is~performed exclusively with the~\hyperref[http]{HTTP}~protocol. I.e.,~other systems like email (\hyperref[smtp]{SMTP}~protocol), remote shell (\hyperref[ssh]{SSH}~protocol) or instant messaging (many different protocols), which also run on the~Internet infrastructure, are~not a~part of~Web.
\end{itemize}

\newsection{Namespaces}
\index{Namespace}
\label{namespaces}

\newsection{Web Server}
\index{Web server}
\label{webserver}
It's typically a~software, but~can be even a~physical computer, that stores, processes and sends HTML documents to clients over a~network, usually the Internet. Using dedicated software\,--\,a~web browser\,--\,clients see these documents as web pages. The~communication with clients is mainly performed over the~\hyperref[http]{HTTP protocol}. Based on~incoming request from a~client (initiated by a~web browser) the~server picks a~stored HTML document and sends it back to the~client in a~response.

A~web address entered to a~browser's address bar is separated to the~part identifying the~server's physical computer (\textit{www.something.com}) and the~part identifying the~wanted HTML document location (everything that follows). The~location is referred from the~server's configured home folder. For~example, consider that \textit{www.something.com} refers to~some computer with running Apache HTTP Server (a~concrete example of a~web server). The~default home folder of~Apache HTTP servers is \textit{/home/www} and hasn't been configured otherwise in~this example. You~enter \textit{www.something.com/one/two/three.html} to your web browser's address bar. The~computer is reached over DNS, the~server gets the~request and it looks for the~file \textit{/home/www/one/two/three.html}. If~the~file is found, its contents are returned to your web browser and displayed. If~not, you get the~dinosaur.

\newsection{Application Server}
\index{Application server}
\label{applicationserver}
Application servers are very similar to \hyperref[webserver]{web~servers}. They also get requests from clients and send back responses. The~difference is that they trigger some more complex logic. While web servers basically only return static HTML documents, application servers can, based on incoming requests, execute programs written in complex languages (Java, JavaScript, Python, C\# etc.). These programs can then be~called \hyperref[webservice]{web~services}.